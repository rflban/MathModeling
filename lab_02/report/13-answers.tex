\chapter{Ответы на контрольные вопросы}

\section{Какие способы тестирования программы можно предложить?}
Программу можно тестировать на разных значениях входных параметров.

Например, на разных значениях шага. Уменьшая его, получится обнаружить определённое значение, дальнейшее уменьшение которого не будет менять результат. Это будет точный результат.

Так же можно проверять случай, когда $R_k + R_p = 0$. В этом случае колебания тока должны быть незатухающими.

\section{Получите систему разностных уравнений для решения сформулированной задачи неявным методом трапеций. Опишите  алгоритм реализации полученных уравнений.}

Неявный метод трапеций записывается следующим образом:
\begin{equation*}
y_{n + 1} = y_n + \frac{h}{2} \cdot \Big( f(x_n, y_n) + \alpha \cdot f( x_{n + 1}, y_{n + 1}) \Big)
\end{equation*}
Тогда для системы
\begin{equation*}
\begin{dcases}
    \frac{dI}{dT} = \frac{U - (R_k + R_p(I))I}{L_k}, \\
    \frac{dU}{dt} = -\frac{I}{C_k}. \\
\end{dcases}
\end{equation*}
имеем следующее:
\begin{flalign*}
    &
    I_{n+1} = I_n + \frac{h}{2 L_k} \cdot \Big( U_{n} - (R_k + R_p(I_{n})) I_{n} + U_{n+1} - (R_k + R_p(I_{n+1})) I_{n+1} \Big) \\&
    U_{n+1} = U_n - \frac{h}{2 C_k} \cdot \Big( I_{n} + I_{n+1} \Big)
    &
\end{flalign*}

Таким образом:
\begin{flalign*}
    &
    I_{n+1} = I_n + \frac{h}{2 L_k} \cdot \Big( U_{n} - (R_k + R_p(I_{n})) I_{n} + U_n - \\&
    - \frac{h}{2 C_k} \cdot \Big( I_{n} + I_{n+1} \Big) - (R_k + R_p(I_{n+1})) I_{n+1} \Big) \\&
\end{flalign*}
Здесь $I_{n+1}$ присутствует в обеих частях уравнения. При этом $I_{n+1}$ нельзя выразить, так как $R_p(I)$ вычисляется посредством интерполяции и численного интегрирования. Тогда воспользуемся методом простых итераций:
\begin{enumerate}
    \item Сначала в $I_{n+1}$ в правой части уравнения подставим $I_n$.
    \item \label{itm:step2} Затем вычислим значение выражения, которое так же подставим в правую часть уравнения на место $I_{n+1}$.
    \item Будем повторять шаг~\ref{itm:step2} пока $|\frac{I_{n+1} - I_{n}}{I_{n+1}}| > \epsilon$.
\end{enumerate}

Зная $I_{n+1}$ можем найти $U_{n+1}$. Так как нам известны $I_0$, $U_0$ и шаг, используя полученные формулы и метод простых итераций мы можем получить решение системы уравнений.

\section{Из каких соображений проводится выбор того или иного метода, учитывая, что чем выше порядок точности метода, тем он более сложен?}

Выбор того или иного метода обусловлен достижимостью теоретического порядка точности этого метода при решении конкретного ОДУ. В случае когда, менее точный и, соответственно, более простой метод не может обеспечить сходящийся к точному результат, следует выбрать более сложный метод.

