\chapter{Теоретическая часть}

\section{Исходные данные}
Уравнение функции $T(x)$:
\begin{equation}\label{eq:T}
    \frac{d}{dx}\bigg(K(x)\frac{dT}{dx}\bigg)-\frac{2}{R}\alpha(x)T+\frac{2T_0}{R}\alpha(x)=0
\end{equation}

Краевые условия:
\begin{equation*}
 \begin{dcases}
   x=0,~-k(0)\frac{dT}{dx}=F_0
   \\
   x =l,~-k(l)\frac{dT}{dx}=\alpha_N(T(l)-T_0)
\end{dcases}
\end{equation*}

Функции $k(x)$ и $\alpha(x)$ заданы своими константами:
$$
k(x)=\frac{a}{x-b}
$$
\begin{equation*}
    \alpha(x) = \frac{c}{x - d}
\end{equation*}
где константы $a$, $b$ следует найти из условий $k(0) = k_0$, $k(l) = k_N$, а константы $c$, $d$ из условий $\alpha(0) = \alpha_0$, $\alpha(l) = \alpha_N$. Величины $k_0$, $k_N$, $\alpha_0$, $\alpha_N$ задает пользователь. Тогда
\begin{flalign*}
    &
    b = \frac{k_N l}{k_N - k_0}
    \\&
    a =  -b k_0
    \\&
    c =  -d \alpha_0
    \\&
    d = \frac{\alpha_N l}{\alpha_N - \alpha_0}
\end{flalign*}

Разностная схема с разностными краевыми условиями при $x = 0$.
\begin{flalign}
    & \label{eq:subsheme}
    A_n y_{n-1}-B_n y_n+C_n y_{n+1}=-D_n,\quad 1\le n\le N-1
  \\& \nonumber
  A_n = \frac{\chi_{n+1/2}}{h}
  \\& \nonumber
  C_n = \frac{\chi_{n+1/2}}{h}
  \\& \nonumber
  B_n = A_n + C_n + p_n h
  \\& \nonumber
  D_n = f_n h
  \\& \nonumber
\end{flalign}

Система~\ref{eq:subsheme} совместно с краевыми условиями решается методом прогонки. Для величин $\chi_{n+1/2}$ можно получить различные приближенные выражения интеграл методом трапеций или методом средних:
\begin{equation*}
    \chi_{n \pm 1/2} = \frac{k_n + k_{n \pm 1}}{2}
\end{equation*}

Разностный аналог краевого условия при $x = 0$:
\begin{equation*}
    y_0\cdot\bigg(x_{\frac{1}{2}} + \frac{h^2}{8}p_{\frac{1}{2}}+\frac{h^2}{4}p_0\bigg)-y_1\cdot\bigg(x_{\frac{1}{2}}-\frac{h^2}{8}p_{\frac{1}{2}}\bigg)=\bigg(hF_0+\frac{h^2}{4}(f_{\frac{1}{2}}+f_0)\bigg)
\end{equation*}
где
\begin{flalign*}
    & F=-k(x)\frac{dT}{dx}\\
    & p(x)=\frac{2}{R}\alpha(x)\\
    & f(x)=\frac{2T_0}{R}\alpha(x)\\
    & p_n=p(x_n),~f_n=f(x_n)
\end{flalign*}

Простая аппроксимация:
\begin{equation*}
    p_{\frac{1}{2}}=\frac{p_0+p_1}{2}
\end{equation*}
\begin{equation*}
    f_{\frac{1}{2}}=\frac{f_0+f_1}{2}
\end{equation*}

Так же необходимо учесть, что
\begin{equation*}
    F_N = \alpha_N(y_N - T_0)
\end{equation*}
\begin{equation}\label{eq:FNSubHalf}
    F_{N - \frac{1}{2}} = \chi_{N - \frac{1}{2}} \frac{y_{N - 1} - y_N}{h}
\end{equation}

Значения параметров для отладки:
\begin{flalign*}
& K_0=0.4 \text{~Вт/см К} \\
& K_N=0.1 \text{~Вт/см К} \\
& \alpha_0=0.05 \text{~Вт/см}^2\text{~К} \\
& \alpha_N=0.01 \text{~Вт/см}^2\text{~К} \\
& l=10 \text{~см} \\
& T_0=300 \text{~К} \\
& R=0.5 \text{~см} \\
& F_0=50 \text{~Вт/см}^2
&
\end{flalign*}

\section{Физическое содержание задачи}
Сформулированная математическая модель описывает температурное поле $T(x)$ вдоль цилиндрического стержня радиуса $R$ и длиной $l$, причем $R \ll l$ и температуру можно принять постоянной по радиусу цилиндра. Ось $x$ направлена вдоль оси цилиндра и начало координат совпадает с левым торцем стержня. Слева при $x = 0$ цилиндр нагружается тепловым потоком $F_0$. Стержень обдувается воздухом, температура которого равна $T_0$. В результате происходит съем тепла с цилиндрической поверхности и поверхности правого торца при $x = l$. Функции $k(x)$,$\alpha(x)$ являются, соответственно, коэффициентами теплопроводности материала стержня и теплоотдачи при обдуве.

