\chapter{Практическая часть}

\section{Реализация}
\begin{lstlisting}[caption={Теплоёмкость стержня}]
def c(T):
    return a2 + b2 * T ** m2 - c2 / T ** 2
\end{lstlisting}

\begin{lstlisting}[caption={Коэффициенты теплопроводности материала стержня и теплоотдачи при обдуве}]
def k(T):
    return a1 * (b1 + c1 * T ** m1)


def alpha(x):
    d = (alphaN * l) / (alphaN - alpha0)
    c = -alpha0 * d
    return c / (x - d)
\end{lstlisting}

\begin{lstlisting}[caption={Замены p и f}]
def p(x):
    return 2 * alpha(x) / R


def f(x):
    return 2 * alpha(x) * T0 / R
\end{lstlisting}

\begin{lstlisting}[caption={Метод средних}]
def FAddHalf(x, h, F):
    return (F(x) + F(x + h)) / 2


def FSubHalf(x, h, F):
    return (F(x) + F(x - h)) / 2
\end{lstlisting}

\begin{lstlisting}[caption={Параметры разностной схемы}]
def A(T):
    return FSubHalf(T, t, k) * t / h


def D(T):
    return FAddHalf(T, t, k) * t / h


def B(x, T):
    return A(T) + D(T) + c(T) * h + p(x) * h * t


def F(x, T):
    return f(x) * h * t + c(T) * T * h
\end{lstlisting}

\begin{lstlisting}[caption={Краевые условия}]
def bounds_left(prevT):
    K0 = h / 8 * FAddHalf(
            prevT[0], t, c) + h / 4 * c(prevT[0]) + \
            FAddHalf(prevT[0], t, k) * t / h + \
            t * h / 8 * p(h / 2) + t * h / 4 * p(0)

    M0 = h / 8 * FAddHalf(
            prevT[0], t, c) - FAddHalf(prevT[0], t, k) * \
            t / h + t * h * p(h / 2) / 8

    P0 = h / 8 * FAddHalf(
            prevT[0], t, c) * (prevT[0] + prevT[1]) + \
            h / 4 * c(prevT[0]) * prevT[0] + F0 * t + \
            t * h / 8 * (3 * f(0) + f(h))

    return K0, M0, P0


def bounds_right(prevT):
    KN = h / 8 * FSubHalf(
            prevT[-1], t, c) + h / 4 * c(prevT[-1]) + \
            FSubHalf(prevT[-1], t, k) * t / h + t * \
            alphaN + t * h / 8 * p(l - h / 2) + t * h / 4 * p(l)

    MN = h / 8 * FSubHalf(
            prevT[-1], t, c) - FSubHalf(prevT[-1], t, k) * \
            t / h + t * h * p(l - h / 2) / 8

    PN = h / 8 * FSubHalf(
            prevT[-1], t, c) * (prevT[-1] + prevT[-2]) + h / \
            4 * c(prevT[-1]) * prevT[-1] + t * alphaN * T0 + \
            t * h / 4 * (f(l) + f(l - h / 2))

    return KN, MN, PN
\end{lstlisting}

\begin{lstlisting}[caption={Метод прогонки}]
def thomas(prevT):
    K0, M0, P0 = bounds_left(prevT)
    KN, MN, PN = bounds_right(prevT)

    # Прямой ход
    eps = [0, -M0 / K0]
    eta = [0, P0 / K0]

    x = h
    n = 1
    while (x + h < l):
        eps.append(D(prevT[n]) / (B(x, prevT[n]) - A(prevT[n]) * eps[n]))
        eta.append((
            F(x, prevT[n]) + A(prevT[n]) * eta[n]) /
            (B(x, prevT[n]) - A(prevT[n]) * eps[n]))

        n += 1
        x += h

    # Обратный ход
    y = [0] * (n + 1)
    y[n] = (PN - MN * eta[n]) / (KN + MN * eps[n])

    for i in range(n - 1, -1, -1):
        y[i] = eps[i + 1] * y[i + 1] + eta[i + 1]

    return y
\end{lstlisting}

\begin{lstlisting}[caption={Метод простых итераций}]
def termtest1(T, newT):
    max = fabs((T[0] - newT[0]) / newT[0])

    for i, j in zip(T, newT):
        d = fabs(i - j) / j

        if d > max:
            max = d

    return max < 1


def termtest2(T, newT):
    for i, j in zip(T, newT):
        if fabs((i - j) / j) > 1e-2:
            return True

    return False


def fixed_point_iteration():
    n = int(l / h)
    ti = 0

    result = []
    newT = [0] * (n + 1)
    T = [T0] * (n + 1)

    result.append(T)

    while (True):
        buf = T

        while True:
            newT = thomas(buf)
            if termtest1(buf, newT):
                break
            buf = newT

        result.append(newT)
        ti += t

        if (termtest2(T, newT) == False):
            break

        T = newT

    return result, ti
\end{lstlisting}

\section{Результаты работы}

\subsection{Представить разностный аналог краевого условия при $x = l$ и его краткий вывод интегро-интерполяционным методом}
См. раздел~\ref{theory}.

\subsection{График зависимости температуры от координаты при нескольких фиксированных значениях времени}
На рисунке представлены графики зависимости температуры от координаты при нескольких фиксированных $t$. Последняя~--– синяя кривая соответствует установившемуся режиму, когда поле перестает меняться с точностью $1\text{e--}3$.
\begin{figure}[H]
    \caption{Зависимость температуры от координаты стержня}
    % This file was created by tikzplotlib v0.9.2.
\begin{tikzpicture}[scale=0.875]

\definecolor{color0}{rgb}{0.83921568627451,0.152941176470588,0.156862745098039}

\begin{groupplot}[group style={group size=2 by 3, vertical sep=2.5cm, horizontal sep=2.5cm}]
\nextgroupplot[
    x label style={at={(axis description cs:0.5,-0.05)},anchor=north},
    y label style={at={(axis description cs:-0.005,.5)},rotate=0,anchor=south},
tick align=outside,
tick pos=left,
x grid style={white!69.0196078431373!black},
xlabel={Время},
xmin=-2.995e-05, xmax=0.00062895,
xtick style={color=black},
y grid style={white!69.0196078431373!black},
ylabel={$I$},
ymin=-39.0265960035922, ymax=830.558516075437,
ytick style={color=black}
]
\addplot [semithick, color0]
table {%
0 0.5
1e-06 6.81415305462132
2e-06 13.510922328665
3e-06 20.1818708210294
4e-06 26.7970402968579
5e-06 33.3629781150593
6e-06 39.8843789519145
7e-06 46.3650672954001
8e-06 52.7991924336229
9e-06 59.1704751310961
1e-05 65.4730036997804
1.1e-05 71.7123249881364
1.2e-05 77.8933263894551
1.3e-05 84.019917080479
1.4e-05 90.0952246610569
1.5e-05 96.1222363654308
1.6e-05 102.103441027124
1.7e-05 108.040873737754
1.8e-05 113.936635709444
1.9e-05 119.79283484951
2e-05 125.611275593647
2.1e-05 131.393215588199
2.2e-05 137.13991966128
2.3e-05 142.852636557955
2.4e-05 148.532330524176
2.5e-05 154.17980851507
2.6e-05 159.796021888008
2.7e-05 165.381880440116
2.8e-05 170.938034791426
2.9e-05 176.465525193228
3e-05 181.967705909288
3.1e-05 187.447653685535
3.2e-05 192.906172884135
3.3e-05 198.343821031476
3.4e-05 203.754371876694
3.5e-05 209.12895417711
3.6e-05 214.465380948387
3.7e-05 219.764350439942
3.8e-05 225.026515919966
3.9e-05 230.252485844627
4e-05 235.442901305064
4.1e-05 240.598441503502
4.2e-05 245.719700653368
4.3e-05 250.80718375781
4.4e-05 255.861373993542
4.5e-05 260.882749437068
4.6e-05 265.871779974733
4.7e-05 270.828901368291
4.8e-05 275.754520037197
4.9e-05 280.649025823707
5e-05 285.512792846645
5.1e-05 290.346188268139
5.2e-05 295.14965102139
5.3e-05 299.923622721845
5.4e-05 304.668444902132
5.5e-05 309.38442106926
5.6e-05 314.071843116169
5.7e-05 318.731021678669
5.8e-05 323.362258717018
5.9e-05 327.965837280642
6e-05 332.54203882077
6.1e-05 337.091114451901
6.2e-05 341.613309634771
6.3e-05 346.1088731406
6.4e-05 350.578031237783
6.5e-05 355.020990357557
6.6e-05 359.437949639585
6.7e-05 363.829101245414
6.8e-05 368.194630655657
6.9e-05 372.534716951922
7e-05 376.849533084424
7.1e-05 381.139246126181
7.2e-05 385.404017514584
7.3e-05 389.644003281122
7.4e-05 393.859354269953
7.5e-05 398.050319395914
7.6e-05 402.216461620676
7.7e-05 406.355938658766
7.8e-05 410.467625186938
7.9e-05 414.551744552501
8e-05 418.60848766461
8.1e-05 422.638027510278
8.2e-05 426.640543141027
8.3e-05 430.616212888999
8.4e-05 434.565192849587
8.5e-05 438.487631777464
8.6e-05 442.383685673948
8.7e-05 446.253506920038
8.8e-05 450.097233752107
8.9e-05 453.915001107796
9e-05 457.706940792752
9.1e-05 461.47318158035
9.2e-05 465.213866760858
9.3e-05 468.929140013123
9.4e-05 472.61912479486
9.5e-05 476.283938241974
9.6e-05 479.923694906455
9.7e-05 483.538506836862
9.8e-05 487.128483656035
9.9e-05 490.693732636151
0.0001 494.234358771222
0.000101 497.750464847157
0.000102 501.242158444869
0.000103 504.709547854966
0.000104 508.152732398641
0.000105 511.571806468801
0.000106 514.966862567937
0.000107 518.337991365461
0.000108 521.685281753208
0.000109 525.008820899158
0.00011 528.308694299453
0.000111 531.584985828784
0.000112 534.83777778919
0.000113 538.067150957344
0.000114 541.273184630379
0.000115 544.455956670303
0.000116 547.615543547062
0.000117 550.752020380294
0.000118 553.865460979827
0.000119 556.95594327755
0.00012 560.023555779659
0.000121 563.068380305183
0.000122 566.090493154296
0.000123 569.08997687885
0.000124 572.066905124041
0.000125 575.021342809928
0.000126 577.953353834614
0.000127 580.863001105245
0.000128 583.750346568127
0.000129 586.615451237991
0.00013 589.458375226426
0.000131 592.279177769521
0.000132 595.077917254741
0.000133 597.854651247045
0.000134 600.609436514297
0.000135 603.342329051971
0.000136 606.053384107183
0.000137 608.742656202077
0.000138 611.410199156569
0.000139 614.056066110486
0.00014 616.680309545112
0.000141 619.282981304159
0.000142 621.864132614182
0.000143 624.423814104457
0.000144 626.962075826337
0.000145 629.478973465416
0.000146 631.974562300411
0.000147 634.44889088732
0.000148 636.902006985675
0.000149 639.333957867563
0.00015 641.744790334156
0.000151 644.134550731829
0.000152 646.503286061557
0.000153 648.85104824581
0.000154 651.177887661983
0.000155 653.483848819395
0.000156 655.768975848289
0.000157 658.033312513651
0.000158 660.2769022287
0.000159 662.499788068028
0.00016 664.702012780434
0.000161 666.883618801431
0.000162 669.044648265453
0.000163 671.185143017763
0.000164 673.305144626077
0.000165 675.404694391897
0.000166 677.483833361585
0.000167 679.542602337157
0.000168 681.581041886827
0.000169 683.599192355299
0.00017 685.59709387381
0.000171 687.574786369939
0.000172 689.53230957718
0.000173 691.469703044292
0.000174 693.387006144429
0.000175 695.284258084052
0.000176 697.161497911638
0.000177 699.01876452618
0.000178 700.856096685492
0.000179 702.673533014322
0.00018 704.471112012272
0.000181 706.248872061541
0.000182 708.006851434483
0.000183 709.745088300993
0.000184 711.463620735727
0.000185 713.162486725148
0.000186 714.841724174418
0.000187 716.501370914127
0.000188 718.141466304533
0.000189 719.762051254786
0.00019 721.363165095428
0.000191 722.944848440156
0.000192 724.507141751751
0.000193 726.050082343909
0.000194 727.5737074999
0.000195 729.078054478163
0.000196 730.56316051776
0.000197 732.029062843718
0.000198 733.475798672243
0.000199 734.903405215816
0.0002 736.311919688179
0.000201 737.701379309201
0.000202 739.071821309639
0.000203 740.423282935787
0.000204 741.755801454025
0.000205 743.069414155259
0.000206 744.364158359264
0.000207 745.640071418929
0.000208 746.897190724402
0.000209 748.135553707143
0.00021 749.355197843887
0.000211 750.556173978148
0.000212 751.738663794062
0.000213 752.902842510208
0.000214 754.048753372243
0.000215 755.176434161252
0.000216 756.28592154927
0.000217 757.377250121625
0.000218 758.450454009179
0.000219 759.505566907101
0.00022 760.54262081277
0.000221 761.561649532973
0.000222 762.56268957021
0.000223 763.545775842333
0.000224 764.510940651948
0.000225 765.458216431826
0.000226 766.387635747318
0.000227 767.29923129871
0.000228 768.193037284144
0.000229 769.069088556436
0.00023 769.927418745112
0.000231 770.768061022569
0.000232 771.591048707601
0.000233 772.396415267431
0.000234 773.184194319682
0.000235 773.954419634298
0.000236 774.707125135425
0.000237 775.442344903237
0.000238 776.160113175716
0.000239 776.860464350385
0.00024 777.543432986006
0.000241 778.209056274008
0.000242 778.857371882493
0.000243 779.488415172466
0.000244 780.102221194601
0.000245 780.69882516955
0.000246 781.278262489376
0.000247 781.840568718945
0.000248 782.385779597275
0.000249 782.913931038852
0.00025 783.425059134908
0.000251 783.919200154661
0.000252 784.396390546522
0.000253 784.856666939262
0.000254 785.300066143155
0.000255 785.726625151071
0.000256 786.136381139554
0.000257 786.529371469853
0.000258 786.905633688931
0.000259 787.265205530434
0.00026 787.608124915636
0.000261 787.934429954354
0.000262 788.244158945824
0.000263 788.537350379561
0.000264 788.814042936182
0.000265 789.074275488197
0.000266 789.318087100787
0.000267 789.545517032539
0.000268 789.756604736163
0.000269 789.951389859181
0.00027 790.129912244587
0.000271 790.292211931488
0.000272 790.438329155716
0.000273 790.568304350411
0.000274 790.682178146591
0.000275 790.779991373687
0.000276 790.861785418052
0.000277 790.927601804492
0.000278 790.97748190022
0.000279 791.01146716724
0.00028 791.029599269209
0.000281 791.031920071844
0.000282 791.018471643298
0.000283 790.989296254519
0.000284 790.944436379591
0.000285 790.883934696055
0.000286 790.807834085205
0.000287 790.716178028626
0.000288 790.609010408368
0.000289 790.48637491352
0.00029 790.348315399534
0.000291 790.194875925987
0.000292 790.026100756773
0.000293 789.842034360274
0.000294 789.64272140952
0.000295 789.428206782326
0.000296 789.198535561411
0.000297 788.953753034508
0.000298 788.693904694449
0.000299 788.419036239241
0.0003 788.129193572118
0.000301 787.824422801589
0.000302 787.504770241458
0.000303 787.170282410837
0.000304 786.821006034144
0.000305 786.45698804108
0.000306 786.078275566596
0.000307 785.684915950849
0.000308 785.276956739135
0.000309 784.85444568182
0.00031 784.417430734243
0.000311 783.965960056619
0.000312 783.50008201392
0.000313 783.019845175747
0.000314 782.525298316188
0.000315 782.016490413663
0.000316 781.493470650756
0.000317 780.956288414036
0.000318 780.404993293868
0.000319 779.839635084203
0.00032 779.260263782369
0.000321 778.666929588839
0.000322 778.059682906994
0.000323 777.438575250101
0.000324 776.8036602304
0.000325 776.154990746054
0.000326 775.492618117547
0.000327 774.816593866248
0.000328 774.126969714098
0.000329 773.423797583302
0.00033 772.707129595998
0.000331 771.977018073922
0.000332 771.233515538064
0.000333 770.476674708305
0.000334 769.706548503058
0.000335 768.923190038887
0.000336 768.126652630123
0.000337 767.316990337037
0.000338 766.494258669216
0.000339 765.658512778521
0.00034 764.80980681554
0.000341 763.948195126223
0.000342 763.073732251435
0.000343 762.18647292651
0.000344 761.286472080789
0.000345 760.373788281219
0.000346 759.448480846721
0.000347 758.510605844477
0.000348 757.5602190676
0.000349 756.597377635116
0.00035 755.622139812768
0.000351 754.634562917419
0.000352 753.634707567271
0.000353 752.622635611595
0.000354 751.598408119451
0.000355 750.562092293956
0.000356 749.513867376674
0.000357 748.453921021028
0.000358 747.382326486004
0.000359 746.299144256784
0.00036 745.204434977972
0.000361 744.098259452882
0.000362 742.980678642814
0.000363 741.851753666315
0.000364 740.711545798438
0.000365 739.560116469987
0.000366 738.397527266752
0.000367 737.223839928737
0.000368 736.039116349376
0.000369 734.843418574739
0.00037 733.636808802733
0.000371 732.419349382289
0.000372 731.191102812541
0.000373 729.952131741996
0.000374 728.702498967692
0.000375 727.442267434354
0.000376 726.171500233532
0.000377 724.890260602734
0.000378 723.59861192455
0.000379 722.296617725767
0.00038 720.984342350919
0.000381 719.661852245842
0.000382 718.329213312278
0.000383 716.986490392014
0.000384 715.633750456672
0.000385 714.271059883989
0.000386 712.898483235156
0.000387 711.51608520722
0.000388 710.123930632113
0.000389 708.722084475668
0.00039 707.310611836626
0.000391 705.88957794564
0.000392 704.459048164256
0.000393 703.019087983906
0.000394 701.569763024867
0.000395 700.111139035233
0.000396 698.643281889865
0.000397 697.166257589333
0.000398 695.680132258852
0.000399 694.184972147211
0.0004 692.680843625683
0.000401 691.167813186938
0.000402 689.645947443935
0.000403 688.115313128816
0.000404 686.57597709178
0.000405 685.028006299957
0.000406 683.471467836263
0.000407 681.906428898255
0.000408 680.33295679697
0.000409 678.75111895576
0.00041 677.160982909107
0.000411 675.562616301444
0.000412 673.956086885951
0.000413 672.341462523353
0.000414 670.7188111807
0.000415 669.088200930144
0.000416 667.4496999477
0.000417 665.803376512001
0.000418 664.149299003044
0.000419 662.487535900921
0.00042 660.818155784545
0.000421 659.141227330364
0.000422 657.456819311065
0.000423 655.765000594267
0.000424 654.065840141204
0.000425 652.359407005403
0.000426 650.645770331337
0.000427 648.924999353089
0.000428 647.197163392984
0.000429 645.462333658929
0.00043 643.720584452906
0.000431 641.971988344365
0.000432 640.216615005506
0.000433 638.454534189024
0.000434 636.685815726676
0.000435 634.910529527846
0.000436 633.128745578095
0.000437 631.340533937697
0.000438 629.545964740169
0.000439 627.745108190786
0.00044 625.938037478293
0.000441 624.124827592022
0.000442 622.305550665296
0.000443 620.480277178658
0.000444 618.649077676019
0.000445 616.812022763103
0.000446 614.969183105872
0.000447 613.120629428946
0.000448 611.266432514006
0.000449 609.406663198189
0.00045 607.541392372469
0.000451 605.670690980018
0.000452 603.794630014571
0.000453 601.913280518761
0.000454 600.026713582453
0.000455 598.135000341057
0.000456 596.238211973836
0.000457 594.336419702194
0.000458 592.429694787952
0.000459 590.518108531618
0.00046 588.601732270629
0.000461 586.680637377597
0.000462 584.754895258527
0.000463 582.824577351025
0.000464 580.889755122502
0.000465 578.950500068345
0.000466 577.006883710095
0.000467 575.058977593592
0.000468 573.106853287119
0.000469 571.150582379526
0.00047 569.190236478338
0.000471 567.225887207853
0.000472 565.25760620722
0.000473 563.285471685733
0.000474 561.309565318396
0.000475 559.32996220736
0.000476 557.346733997792
0.000477 555.35995600726
0.000478 553.369711102834
0.000479 551.376078425607
0.00048 549.379129578672
0.000481 547.378936158006
0.000482 545.375569750379
0.000483 543.369101931243
0.000484 541.35960426261
0.000485 539.34714829091
0.000486 537.331805544832
0.000487 535.313647533153
0.000488 533.292745742548
0.000489 531.269171635382
0.00049 529.242996647487
0.000491 527.214292185923
0.000492 525.183129626721
0.000493 523.149580312609
0.000494 521.113715550721
0.000495 519.075606610289
0.000496 517.035324720317
0.000497 514.992941067243
0.000498 512.948526792571
0.000499 510.9021529905
0.0005 508.853890705528
0.000501 506.803810930036
0.000502 504.75198460186
0.000503 502.698482601842
0.000504 500.643375751364
0.000505 498.586735420222
0.000506 496.528638123465
0.000507 494.469159654459
0.000508 492.408370529274
0.000509 490.346341190741
0.00051 488.283142005871
0.000511 486.218843263262
0.000512 484.153515170481
0.000513 482.08722785143
0.000514 480.020051343696
0.000515 477.952055595877
0.000516 475.883310464895
0.000517 473.813885713288
0.000518 471.743851006484
0.000519 469.673275910057
0.00052 467.602229886959
0.000521 465.530782294746
0.000522 463.459002382769
0.000523 461.3869634278
0.000524 459.314746064858
0.000525 457.242426601169
0.000526 455.170073711131
0.000527 453.097755944221
0.000528 451.025541722073
0.000529 448.953499335551
0.00053 446.881696941792
0.000531 444.810202561244
0.000532 442.739084074672
0.000533 440.668409220158
0.000534 438.598245590079
0.000535 436.528666138476
0.000536 434.459743854229
0.000537 432.391546053276
0.000538 430.324139647953
0.000539 428.257591388281
0.00054 426.191967858845
0.000541 424.127335475645
0.000542 422.063767290814
0.000543 420.001340628406
0.000544 417.94012577155
0.000545 415.880188286611
0.000546 413.821593552979
0.000547 411.764411536446
0.000548 409.70871275375
0.000549 407.654562733361
0.00055 405.602044919927
0.000551 403.551245409493
0.000552 401.502235756726
0.000553 399.454707171718
0.000554 397.407361063272
0.000555 395.359708007014
0.000556 393.312244636008
0.000557 391.265041794368
0.000558 389.218170309375
0.000559 387.171700989089
0.00056 385.125704619926
0.000561 383.080251964197
0.000562 381.035413757616
0.000563 378.991260706765
0.000564 376.947863486527
0.000565 374.905292737477
0.000566 372.863619063238
0.000567 370.822913027796
0.000568 368.783245152773
0.000569 366.744685914664
0.00057 364.70730574203
0.000571 362.671175012651
0.000572 360.636364050632
0.000573 358.602943123474
0.000574 356.570982439098
0.000575 354.540552142816
0.000576 352.511722314275
0.000577 350.484562964337
0.000578 348.459144031929
0.000579 346.43553538083
0.00058 344.413806796427
0.000581 342.394027982409
0.000582 340.376268557423
0.000583 338.360602837381
0.000584 336.347112008755
0.000585 334.3358723354
0.000586 332.326952972352
0.000587 330.3204229688
0.000588 328.316351264409
0.000589 326.314806685582
0.00059 324.315862512452
0.000591 322.319601230072
0.000592 320.32610055663
0.000593 318.335428690883
0.000594 316.347653694763
0.000595 314.362844040472
0.000596 312.381081217553
0.000597 310.402445891412
0.000598 308.42700536431
0.000599 306.454826776704
};

\nextgroupplot[
    x label style={at={(axis description cs:0.5,-0.05)},anchor=north},
    y label style={at={(axis description cs:-0.005,.5)},rotate=0,anchor=south},
tick align=outside,
tick pos=left,
x grid style={white!69.0196078431373!black},
xlabel={Время},
xmin=-2.995e-05, xmax=0.00062895,
xtick style={color=black},
y grid style={white!69.0196078431373!black},
ylabel={$U$},
ymin=51.8468447060338, ymax=1464.19776929971,
ytick style={color=black}
,ytick={250, 500, ..., 1250}
]
\addplot [semithick, color0]
table {%
0 1400
1e-06 1399.98704108981
2e-06 1399.94912917103
3e-06 1399.88621351832
4e-06 1399.79851785438
5e-06 1399.68623537965
6e-06 1399.54954023774
7e-06 1399.38859063703
8e-06 1399.20353218967
9e-06 1398.99456634499
1e-05 1398.76196087896
1.1e-05 1398.50596135265
1.2e-05 1398.22679386515
1.3e-05 1397.92466753899
1.4e-05 1397.59977927143
1.5e-05 1397.25231487641
1.6e-05 1396.88244894038
1.7e-05 1396.4903487175
1.8e-05 1396.07617361086
1.9e-05 1395.64007518392
2e-05 1395.18219712249
2.1e-05 1394.70267751115
2.2e-05 1394.20165065424
2.3e-05 1393.67924520629
2.4e-05 1393.1355861441
2.5e-05 1392.57079492745
2.6e-05 1391.98499051391
2.7e-05 1391.37828726763
2.8e-05 1390.7507973471
2.9e-05 1390.10263027017
3e-05 1389.43389044966
3.1e-05 1388.7446623471
3.2e-05 1388.03502739883
3.3e-05 1387.30506408253
3.4e-05 1386.55484967208
3.5e-05 1385.78450896814
3.6e-05 1384.99418570693
3.7e-05 1384.18402093004
3.8e-05 1383.35415312705
3.9e-05 1382.50471846459
4e-05 1381.63585089175
4.1e-05 1380.74768179316
4.2e-05 1379.84034004545
4.3e-05 1378.9139525997
4.4e-05 1377.96864456458
4.5e-05 1377.00453928542
4.6e-05 1376.02175830365
4.7e-05 1375.02042145714
4.8e-05 1374.00064703563
4.9e-05 1372.96255184414
5e-05 1371.90625126316
5.1e-05 1370.83185930579
5.2e-05 1369.73948861336
5.3e-05 1368.62924992309
5.4e-05 1367.50125257888
5.5e-05 1366.3556047702
5.6e-05 1365.19241357621
5.7e-05 1364.01178500808
5.8e-05 1362.81382382743
5.9e-05 1361.59863379077
6e-05 1360.36631754853
6.1e-05 1359.11697675589
6.2e-05 1357.85071218964
6.3e-05 1356.56762367368
6.4e-05 1355.26781012758
6.5e-05 1353.95136968687
6.6e-05 1352.61839973072
6.7e-05 1351.26899690855
6.8e-05 1349.90325716546
6.9e-05 1348.5212757667
7e-05 1347.12314732096
7.1e-05 1345.70896580281
7.2e-05 1344.27882457413
7.3e-05 1342.83281640472
7.4e-05 1341.37103349207
7.5e-05 1339.89356748028
7.6e-05 1338.40050872016
7.7e-05 1336.89195205864
7.8e-05 1335.36800166802
7.9e-05 1333.8287607806
8e-05 1332.27433188852
8.1e-05 1330.70481679981
8.2e-05 1329.12031671577
8.3e-05 1327.52093210875
8.4e-05 1325.90676284837
8.5e-05 1324.27790824246
8.6e-05 1322.63446705056
8.7e-05 1320.97653741654
8.8e-05 1319.30421696233
8.9e-05 1317.61760280041
9e-05 1315.9167915457
9.1e-05 1314.20187932706
9.2e-05 1312.47296179842
9.3e-05 1310.73013401972
9.4e-05 1308.97349057188
9.5e-05 1307.20312559385
9.6e-05 1305.41913279242
9.7e-05 1303.62160545171
9.8e-05 1301.81063644232
9.9e-05 1299.98631823027
0.0001 1298.14874288563
0.000101 1296.29800209089
0.000102 1294.43418714906
0.000103 1292.55738893996
0.000104 1290.66769795813
0.000105 1288.76520434232
0.000106 1286.84999788268
0.000107 1284.92216802763
0.000108 1282.98180389068
0.000109 1281.02899425689
0.00011 1279.06382758926
0.000111 1277.08639203483
0.000112 1275.09677543064
0.000113 1273.09506530956
0.000114 1271.08134890582
0.000115 1269.05571316051
0.000116 1267.01824472681
0.000117 1264.96902997513
0.000118 1262.90815499806
0.000119 1260.83570561519
0.00012 1258.75176733781
0.000121 1256.65642532455
0.000122 1254.5497644762
0.000123 1252.43186938235
0.000124 1250.30282432727
0.000125 1248.16271335108
0.000126 1246.01162025364
0.000127 1243.84962859825
0.000128 1241.67682171527
0.000129 1239.49328270573
0.00013 1237.29909444468
0.000131 1235.09433958452
0.000132 1232.87910055826
0.000133 1230.65345958257
0.000134 1228.41749866086
0.000135 1226.17129958618
0.000136 1223.91494394405
0.000137 1221.64851311524
0.000138 1219.37208827837
0.000139 1217.08575041257
0.00014 1214.78958029991
0.000141 1212.48365852786
0.000142 1210.16806549161
0.000143 1207.84288139632
0.000144 1205.50818625936
0.000145 1203.16405991238
0.000146 1200.81058195732
0.000147 1198.44783181301
0.000148 1196.07588871918
0.000149 1193.69483173827
0.00015 1191.3047397572
0.000151 1188.90569148915
0.000152 1186.4977654752
0.000153 1184.08104007786
0.000154 1181.65559345055
0.000155 1179.22150357988
0.000156 1176.77884828712
0.000157 1174.32770522958
0.000158 1171.86815190191
0.000159 1169.40026563744
0.00016 1166.92412360939
0.000161 1164.43980283204
0.000162 1161.94738016193
0.000163 1159.44693229892
0.000164 1156.93853578726
0.000165 1154.42226701661
0.000166 1151.898202223
0.000167 1149.36641748977
0.000168 1146.82698874846
0.000169 1144.27999177966
0.00017 1141.72550221384
0.000171 1139.16359553208
0.000172 1136.5943470669
0.000173 1134.01783200287
0.000174 1131.43412537735
0.000175 1128.84330208109
0.000176 1126.24543685887
0.000177 1123.64060431002
0.000178 1121.02887888904
0.000179 1118.41033490602
0.00018 1115.78504652721
0.000181 1113.15308777541
0.000182 1110.51453253041
0.000183 1107.8694545294
0.000184 1105.21792736733
0.000185 1102.56002449725
0.000186 1099.89581923063
0.000187 1097.22538473764
0.000188 1094.54879404742
0.000189 1091.86612003651
0.00019 1089.17743542861
0.000191 1086.48281280686
0.000192 1083.78232459175
0.000193 1081.07604306452
0.000194 1078.36404036722
0.000195 1075.64638850284
0.000196 1072.92315933538
0.000197 1070.19442458994
0.000198 1067.46025585271
0.000199 1064.72072457106
0.0002 1061.97590205348
0.000201 1059.22585946963
0.000202 1056.47066785023
0.000203 1053.71039808708
0.000204 1050.94512093298
0.000205 1048.17490700162
0.000206 1045.39982676748
0.000207 1042.61995056576
0.000208 1039.83534859219
0.000209 1037.04609090293
0.00021 1034.25224741439
0.000211 1031.45388790303
0.000212 1028.65108190984
0.000213 1025.84389784861
0.000214 1023.03240395453
0.000215 1020.21666831958
0.000216 1017.39675889717
0.000217 1014.57274350541
0.000218 1011.74468983979
0.000219 1008.91266546359
0.00022 1006.07673782091
0.000221 1003.23697423624
0.000222 1000.39344190157
0.000223 997.546207868672
0.000224 994.695339069068
0.000225 991.840902313501
0.000226 988.982964291449
0.000227 986.12159157063
0.000228 983.256850596492
0.000229 980.388807681621
0.00023 977.517529011213
0.000231 974.643080646962
0.000232 971.765528526512
0.000233 968.884938462908
0.000234 966.001376144039
0.000235 963.114907132074
0.000236 960.225596862886
0.000237 957.333510645478
0.000238 954.438713661395
0.000239 951.541270964132
0.00024 948.641247478536
0.000241 945.738708000201
0.000242 942.83371717658
0.000243 939.926339521855
0.000244 937.01663941974
0.000245 934.10468112285
0.000246 931.190528752057
0.000247 928.274246295855
0.000248 925.355897609708
0.000249 922.4355464154
0.00025 919.513256300377
0.000251 916.589090717086
0.000252 913.663112982306
0.000253 910.73538627648
0.000254 907.805973643041
0.000255 904.874937987727
0.000256 901.942342077904
0.000257 899.008248541875
0.000258 896.072719868189
0.000259 893.135818404948
0.00026 890.197606359107
0.000261 887.258145795773
0.000262 884.317498637501
0.000263 881.375726663581
0.000264 878.432891509333
0.000265 875.489054665388
0.000266 872.54427747697
0.000267 869.598621143178
0.000268 866.652146716264
0.000269 863.704915100904
0.00027 860.756987053472
0.000271 857.808423181307
0.000272 854.859283941985
0.000273 851.909629642578
0.000274 848.95952043892
0.000275 846.00901633487
0.000276 843.058177181563
0.000277 840.107062674159
0.000278 837.155732353596
0.000279 834.204245606479
0.00028 831.252661664329
0.000281 828.301039602827
0.000282 825.349438341063
0.000283 822.397916640779
0.000284 819.446533105608
0.000285 816.495346180322
0.000286 813.544414150065
0.000287 810.593795139598
0.000288 807.643547109429
0.000289 804.693727857139
0.00029 801.744395017047
0.000291 798.795606059454
0.000292 795.847418289878
0.000293 792.899888848293
0.000294 789.953074708362
0.000295 787.007032676677
0.000296 784.061819391995
0.000297 781.117491324468
0.000298 778.174104774885
0.000299 775.231715873899
0.0003 772.29038058127
0.000301 769.350154685089
0.000302 766.411093801022
0.000303 763.473253371536
0.000304 760.536688665139
0.000305 757.601454775607
0.000306 754.667606621228
0.000307 751.735198944025
0.000308 748.804286308999
0.000309 745.874923103359
0.00031 742.94716353576
0.000311 740.021061635535
0.000312 737.096671251934
0.000313 734.174046053359
0.000314 731.253239526598
0.000315 728.334304976069
0.000316 725.417295523049
0.000317 722.50226410492
0.000318 719.589263474406
0.000319 716.67834619881
0.00032 713.769564659258
0.000321 710.86297104994
0.000322 707.958617377352
0.000323 705.056555459539
0.000324 702.15683691846
0.000325 699.259513172296
0.000326 696.364635448118
0.000327 693.472254781125
0.000328 690.582422013901
0.000329 687.695187795662
0.00033 684.810602581511
0.000331 681.928716631692
0.000332 679.049580010849
0.000333 676.173242587279
0.000334 673.299754032193
0.000335 670.429163818973
0.000336 667.561521222439
0.000337 664.696875318107
0.000338 661.835274977308
0.000339 658.976768861417
0.00034 656.121405430121
0.000341 653.26923294069
0.000342 650.420299447247
0.000343 647.57465280004
0.000344 644.732340644718
0.000345 641.893410421609
0.000346 639.057909339212
0.000347 636.225884395348
0.000348 633.397382380013
0.000349 630.572449874665
0.00035 627.751133242995
0.000351 624.933478631735
0.000352 622.119531976783
0.000353 619.309338972426
0.000354 616.502945093111
0.000355 613.700395577316
0.000356 610.901735397822
0.000357 608.107008454051
0.000358 605.316258321983
0.000359 602.529528352278
0.00036 599.746861669606
0.000361 596.968301172057
0.000362 594.193889530545
0.000363 591.423669188224
0.000364 588.6576823599
0.000365 585.895971031451
0.000366 583.138576959243
0.000367 580.385541669559
0.000368 577.63690645802
0.000369 574.892712389016
0.00037 572.153000295138
0.000371 569.417810776613
0.000372 566.687184200741
0.000373 563.961160701338
0.000374 561.239780178177
0.000375 558.523082296441
0.000376 555.811106486165
0.000377 553.1038919417
0.000378 550.401477621159
0.000379 547.703902245888
0.00038 545.011204299921
0.000381 542.323422024386
0.000382 539.640593407533
0.000383 536.962756198627
0.000384 534.289947902004
0.000385 531.62220576722
0.000386 528.959566803196
0.000387 526.3020677777
0.000388 523.64974521684
0.000389 521.002635404563
0.00039 518.360774382158
0.000391 515.72419794776
0.000392 513.092941655862
0.000393 510.467040816826
0.000394 507.846530496404
0.000395 505.231445515254
0.000396 502.62182044847
0.000397 500.017689625108
0.000398 497.419087127719
0.000399 494.826046791887
0.0004 492.23860220577
0.000401 489.656786709645
0.000402 487.080633395454
0.000403 484.510175106363
0.000404 481.945444436313
0.000405 479.386473729589
0.000406 476.833295080376
0.000407 474.28594033234
0.000408 471.744441078194
0.000409 469.20882865928
0.00041 466.679134165152
0.000411 464.155388433165
0.000412 461.637622048064
0.000413 459.125865341581
0.000414 456.620148392039
0.000415 454.120501023954
0.000416 451.626952807646
0.000417 449.139533058857
0.000418 446.658270838363
0.000419 444.183194951607
0.00042 441.714333948323
0.000421 439.251716122169
0.000422 436.795369510366
0.000423 434.345321893346
0.000424 431.901600794393
0.000425 429.4642334793
0.000426 427.033246956028
0.000427 424.608667974367
0.000428 422.190523025605
0.000429 419.778838342201
0.00043 417.373639884016
0.000431 414.974953329086
0.000432 412.582804095645
0.000433 410.197217341807
0.000434 407.818217965263
0.000435 405.445830602995
0.000436 403.080079630983
0.000437 400.720989163927
0.000438 398.368583054969
0.000439 396.022884895425
0.00044 393.683918014518
0.000441 391.351705457368
0.000442 389.026269993609
0.000443 386.707634130019
0.000444 384.39582011027
0.000445 382.090849914692
0.000446 379.792745260049
0.000447 377.501527599312
0.000448 375.21721812145
0.000449 372.939837751214
0.00045 370.669407148936
0.000451 368.40594671033
0.000452 366.149476566301
0.000453 363.90001658276
0.000454 361.657586360443
0.000455 359.422205234739
0.000456 357.193892275524
0.000457 354.972666286996
0.000458 352.75854580753
0.000459 350.55154910952
0.00046 348.351694199245
0.000461 346.158998816733
0.000462 343.97348043563
0.000463 341.795156263083
0.000464 339.624043239622
0.000465 337.460158039056
0.000466 335.303517068367
0.000467 333.154136467619
0.000468 331.012032109872
0.000469 328.8772196011
0.00047 326.749714280114
0.000471 324.629531218505
0.000472 322.516685220576
0.000473 320.411190823295
0.000474 318.313062247325
0.000475 316.222313420124
0.000476 314.138958001852
0.000477 312.063009385325
0.000478 309.994480668621
0.000479 307.933384626223
0.00048 305.879733765482
0.000481 303.833540326585
0.000482 301.794816282584
0.000483 299.763573339437
0.000484 297.739822936051
0.000485 295.723576244337
0.000486 293.714844169275
0.000487 291.713637348981
0.000488 289.719966154786
0.000489 287.733840691323
0.00049 285.755270796622
0.000491 283.784266042214
0.000492 281.820835733239
0.000493 279.864988908571
0.000494 277.916734340944
0.000495 275.976080537087
0.000496 274.043035737873
0.000497 272.117607918474
0.000498 270.19980478852
0.000499 268.289633792275
0.0005 266.387102108817
0.000501 264.492216652226
0.000502 262.604984071786
0.000503 260.725410752192
0.000504 258.853502813765
0.000505 256.989266112682
0.000506 255.13270623665
0.000507 253.283828470887
0.000508 251.442637837394
0.000509 249.609139095187
0.00051 247.783336740576
0.000511 245.965235007451
0.000512 244.154837867579
0.000513 242.352149030913
0.000514 240.557171945908
0.000515 238.769909799847
0.000516 236.990365519182
0.000517 235.218541769875
0.000518 233.454440957762
0.000519 231.698065228914
0.00052 229.949416470021
0.000521 228.208496308774
0.000522 226.475306114269
0.000523 224.74984699741
0.000524 223.032119780483
0.000525 221.322124972639
0.000526 219.619862826613
0.000527 217.925333339124
0.000528 216.238536251343
0.000529 214.559471049379
0.00053 212.888136964769
0.000531 211.224532974988
0.000532 209.568657803958
0.000533 207.920509922581
0.000534 206.28008754927
0.000535 204.647388650505
0.000536 203.022410900319
0.000537 201.4051517204
0.000538 199.795608282468
0.000539 198.193777508869
0.00054 196.599656073187
0.000541 195.013240400867
0.000542 193.434526669854
0.000543 191.863510760497
0.000544 190.300188273594
0.000545 188.744554564986
0.000546 187.196604746208
0.000547 185.656333685195
0.000548 184.123735971398
0.000549 182.598805946418
0.00055 181.081537710722
0.000551 179.571924983397
0.000552 178.069961222029
0.000553 176.575639611911
0.000554 175.088955910096
0.000555 173.609912995807
0.000556 172.138510290332
0.000557 170.674746953706
0.000558 169.218621881648
0.000559 167.770133705631
0.00056 166.329280792957
0.000561 164.896061246841
0.000562 163.470472906508
0.000563 162.052513347295
0.000564 160.642179880766
0.000565 159.239469554834
0.000566 157.844379153894
0.000567 156.456905198967
0.000568 155.077043947849
0.000569 153.704791395282
0.00057 152.340143273118
0.000571 150.983095050512
0.000572 149.63364193411
0.000573 148.291778868259
0.000574 146.957500535221
0.000575 145.630801355403
0.000576 144.311675487595
0.000577 143.000116829222
0.000578 141.696119016605
0.000579 140.399675425235
0.00058 139.110779170062
0.000581 137.829423105793
0.000582 136.555599827199
0.000583 135.289301669445
0.000584 134.030520672769
0.000585 132.779248566399
0.000586 131.535476821387
0.000587 130.299196650938
0.000588 129.070399010809
0.000589 127.849074599729
0.00059 126.635213859828
0.000591 125.428806943041
0.000592 124.229843675872
0.000593 123.038313630181
0.000594 121.854206123607
0.000595 120.677510220085
0.000596 119.50821472631
0.000597 118.346308098117
0.000598 117.191778540031
0.000599 116.044614005746
};

\nextgroupplot[
    x label style={at={(axis description cs:0.5,-0.05)},anchor=north},
    y label style={at={(axis description cs:-0.005,.5)},rotate=0,anchor=south},
tick align=outside,
tick pos=left,
x grid style={white!69.0196078431373!black},
xlabel={Время},
xmin=-3e-05, xmax=0.00063,
xtick style={color=black},
y grid style={white!69.0196078431373!black},
ylabel={$R_p$},
ymin=-27.959405762658, ymax=604.712856859887,
ytick={0,100,...,500},
ytick style={color=black}
]
\addplot [semithick, color0]
table {%
0 575.95502674068
1e-06 23.0827554578555
1e-06 21.5418264555666
2e-06 10.6234782932276
2e-06 10.6206403750673
3e-06 7.56920474967348
3e-06 7.57851666227927
4e-06 5.9994722442285
4e-06 6.00421986361188
5e-06 5.02885305628313
5e-06 5.03168038963841
6e-06 4.36166851384855
6e-06 4.36350769857728
7e-06 3.87132258067227
7e-06 3.87261805388201
8e-06 3.55674138647501
8e-06 3.55757638225747
9e-06 3.36976074171787
9e-06 3.37074487237578
1e-05 3.20588616800197
1e-05 3.20670446317028
1.1e-05 3.05977915833017
1.1e-05 3.06044767032009
1.2e-05 2.9281996032452
1.2e-05 2.92876962175676
1.3e-05 2.81037058723001
1.3e-05 2.81085546325094
1.4e-05 2.7038104367761
1.4e-05 2.70423605218895
1.5e-05 2.60606604951521
1.5e-05 2.60642766368552
1.6e-05 2.51726267644521
1.6e-05 2.51758403197157
1.7e-05 2.43552757089088
1.7e-05 2.4358153620215
1.8e-05 2.35962281645352
1.8e-05 2.35987519686576
1.9e-05 2.28854392786006
1.9e-05 2.28877000611181
2e-05 2.22242177245108
2e-05 2.22262503807913
2.1e-05 2.16121354484771
2.1e-05 2.16139837220693
2.2e-05 2.10355067073013
2.2e-05 2.10371834432639
2.3e-05 2.04962973296673
2.3e-05 2.04978392168387
2.4e-05 1.99902605189081
2.4e-05 1.99916671847969
2.5e-05 1.95165717019602
2.5e-05 1.95178767397657
2.6e-05 1.9064880918662
2.6e-05 1.90660810284281
2.7e-05 1.86397671190075
2.7e-05 1.86408805079168
2.8e-05 1.82370727775549
2.8e-05 1.8238120407647
2.9e-05 1.78473470648658
2.9e-05 1.78484143455765
3e-05 1.74282235737563
3e-05 1.74290705640239
3.1e-05 1.70266414721836
3.1e-05 1.70274253038295
3.2e-05 1.66412536233039
3.2e-05 1.6641977765066
3.3e-05 1.62749665850255
3.3e-05 1.62756461866695
3.4e-05 1.60463334571167
3.4e-05 1.60468655671028
3.5e-05 1.58771633509466
3.5e-05 1.58777638920819
3.6e-05 1.57126521190328
3.6e-05 1.57132305490478
3.7e-05 1.55524837722282
3.7e-05 1.55530397826559
3.8e-05 1.539673408813
3.8e-05 1.53972694722198
3.9e-05 1.52452148340933
3.9e-05 1.5245730804373
4e-05 1.50967936088897
4e-05 1.50972963198791
4.1e-05 1.4951281740969
4.1e-05 1.4951765036202
4.2e-05 1.48095788371131
4.2e-05 1.48100455970739
4.3e-05 1.46715331582969
4.3e-05 1.46719842969002
4.4e-05 1.45369995976573
4.4e-05 1.45374359637196
4.5e-05 1.44056184605999
4.5e-05 1.44060416977333
4.6e-05 1.42773269434805
4.6e-05 1.42777365805653
4.7e-05 1.41521726846373
4.7e-05 1.41525697193273
4.8e-05 1.40300387004636
4.8e-05 1.40304237749793
4.9e-05 1.39108137460227
4.9e-05 1.39111874601789
5e-05 1.37943922302891
5e-05 1.37947551448817
5.1e-05 1.36805712266617
5.1e-05 1.36809263206676
5.2e-05 1.35682682027335
5.2e-05 1.35686135220443
5.3e-05 1.34582057392676
5.3e-05 1.34585410098365
5.4e-05 1.33506256139506
5.4e-05 1.33509518809282
5.5e-05 1.32454437683385
5.5e-05 1.32457614465384
5.6e-05 1.31425795299471
5.6e-05 1.31428890092316
5.7e-05 1.30416062612899
5.7e-05 1.30419083953802
5.8e-05 1.29427825301339
5.8e-05 1.29430770971412
5.9e-05 1.28458281754714
5.9e-05 1.2846116203291
6e-05 1.27507986313123
6e-05 1.27510795234221
6.1e-05 1.26577573604567
6.1e-05 1.26580316452751
6.2e-05 1.25664858717496
6.2e-05 1.25667543880699
6.3e-05 1.24769564315522
6.3e-05 1.24772186146373
6.4e-05 1.23892459577628
6.4e-05 1.23895023016912
6.5e-05 1.23032998413663
6.5e-05 1.23035505807593
6.6e-05 1.22190655211687
6.6e-05 1.22193108782017
6.7e-05 1.21364926085822
6.7e-05 1.213673279379
6.8e-05 1.20555327740147
6.8e-05 1.20557679870361
6.9e-05 1.19761396404275
6.9e-05 1.19763700706933
7e-05 1.18982686835313
7e-05 1.18984945109025
7.1e-05 1.182187713814
7.1e-05 1.18220985334983
7.2e-05 1.17469239102361
7.2e-05 1.17471410360324
7.3e-05 1.16733694943426
7.3e-05 1.1673582505106
7.4e-05 1.16011758958277
7.4e-05 1.16013849386379
7.5e-05 1.15293383510117
7.5e-05 1.15295673687256
7.6e-05 1.14651058843997
7.6e-05 1.14652627555363
7.7e-05 1.14141685426902
7.7e-05 1.14143397234692
7.8e-05 1.13637057155248
7.8e-05 1.13638752274707
7.9e-05 1.13138987473712
7.9e-05 1.13140659061869
8e-05 1.12647819696973
8e-05 1.12649465733749
8.1e-05 1.12164142499459
8.1e-05 1.12165764700893
8.2e-05 1.11686184546819
8.2e-05 1.11687787854358
8.3e-05 1.11215131607251
8.3e-05 1.1121671187249
8.4e-05 1.10751152338202
8.4e-05 1.10752710627895
8.5e-05 1.10294102839465
8.5e-05 1.10295639760429
8.6e-05 1.09842928599954
8.6e-05 1.09844446563027
8.7e-05 1.09398422941152
8.7e-05 1.09399920632211
8.8e-05 1.08960456741051
8.8e-05 1.08961934711546
8.9e-05 1.08528901181514
8.9e-05 1.08530359955927
9e-05 1.08103631096072
9e-05 1.08105071180116
9.1e-05 1.07684524845563
9.1e-05 1.07685946727003
9.2e-05 1.07270061127773
9.2e-05 1.07271469083777
9.3e-05 1.06861269103779
9.3e-05 1.06862659304384
9.4e-05 1.06458327203818
9.4e-05 1.06459700495516
9.5e-05 1.06061126271503
9.5e-05 1.06062483080224
9.6e-05 1.05669559722106
9.6e-05 1.05670900459255
9.7e-05 1.05283523845584
9.7e-05 1.05284848908648
9.8e-05 1.04902917708491
9.8e-05 1.04904227481618
9.9e-05 1.04527643059949
9.9e-05 1.045289379145
0.0001 1.04157604241503
0.0001 1.04158884536573
0.000101 1.03792708100649
0.000101 1.0379397418357
0.000102 1.03432346438028
0.000102 1.03433600492653
0.000103 1.03076740233046
0.000103 1.03077980401906
0.000104 1.02726024050781
0.000104 1.02727250957363
0.000105 1.02380113962975
0.000105 1.0238132791294
0.000106 1.02038927866204
0.000106 1.02040129155571
0.000107 1.01702385721555
0.000107 1.01703574637077
0.000108 1.01370409489159
0.000108 1.01371586308681
0.000109 1.01042923065254
0.000109 1.01044088058056
0.00011 1.00719852221659
0.00011 1.00721005648779
0.000111 1.0040112454755
0.000111 1.00402266662098
0.000112 1.00086669393445
0.000112 1.000878004409
0.000113 0.997764178172811
0.000113 0.997775380357731
0.000114 0.994703025325134
0.000114 0.994714121530981
0.000115 0.991682578581307
0.000115 0.991693571050478
0.000116 0.988702196705122
0.000116 0.988713087614343
0.000117 0.985761253570444
0.000117 0.985772045033151
0.000118 0.982859137714226
0.000118 0.982869831782834
0.000119 0.97999163081815
0.000119 0.980002267871781
0.00012 0.977153840267181
0.00012 0.977164372225946
0.000121 0.974353300601483
0.000121 0.974363740641028
0.000122 0.971584339219052
0.000122 0.971594720095835
0.000123 0.968846641763684
0.000123 0.968856927011945
0.000124 0.966144712124074
0.000124 0.966154910537735
0.000125 0.963478030407515
0.000125 0.9634881436844
0.000126 0.960846081474098
0.000126 0.96085611126554
0.000127 0.958248361654699
0.000127 0.958258309567121
0.000128 0.95568437843229
0.000128 0.955694246028716
0.000129 0.953153650134055
0.000129 0.953163438935568
0.00013 0.950655705633909
0.00013 0.950665417121043
0.000131 0.948190084064982
0.000131 0.948199719679063
0.000132 0.945756334541699
0.000132 0.945765895686137
0.000133 0.943354015891064
0.000133 0.943363503932583
0.000134 0.940982696392793
0.000134 0.940992112662629
0.000135 0.938641953527967
0.000135 0.938651299323006
0.000136 0.936331373735852
0.000136 0.936340650319731
0.000137 0.934050552178603
0.000137 0.934059760782769
0.000138 0.931799092513527
0.000138 0.931808234338255
0.000139 0.929576606672643
0.000139 0.929585682888011
0.00014 0.927382714649235
0.00014 0.927391726396071
0.000141 0.925217044291173
0.000141 0.925225992681961
0.000142 0.92307923110072
0.000142 0.923088117220475
0.000143 0.920968918040605
0.000143 0.920977742947713
0.000144 0.918885755346122
0.000144 0.918894520073151
0.000145 0.916825720694781
0.000145 0.9168344374012
0.000146 0.914792051088934
0.000146 0.914800709278672
0.000147 0.9127845787835
0.000147 0.912793179657686
0.000148 0.910802980618407
0.000148 0.910811525115894
0.000149 0.908846939633361
0.000149 0.908855428670764
0.00015 0.90691614515051
0.00015 0.906924579622859
0.000151 0.905010292622454
0.000151 0.905018673403823
0.000152 0.903128450795226
0.000152 0.90313679463434
0.000153 0.901267859715374
0.000153 0.90127614710569
0.000154 0.899431381163139
0.000154 0.899439617283905
0.000155 0.897618737153179
0.000155 0.897626922802343
0.000156 0.895829650793581
0.000156 0.895837786750927
0.000157 0.894063850520802
0.000157 0.894071937548453
0.000158 0.892321069975691
0.000158 0.892329108818613
0.000159 0.890601047883075
0.000159 0.890609039269547
0.00016 0.888903527934778
0.00016 0.888911472576861
0.000161 0.887228258675975
0.000161 0.887236157269955
0.000162 0.88557499339475
0.000162 0.885582846621567
0.000163 0.883943490014766
0.000163 0.883951298540434
0.000164 0.882333510990937
0.000164 0.882341275466948
0.000165 0.880744823208008
0.000165 0.880752544271722
0.000166 0.879177197881944
0.000166 0.879184876156966
0.000167 0.877630410464043
0.000167 0.877638046560587
0.000168 0.876104240547669
0.000168 0.876111835062916
0.000169 0.87459847177755
0.000169 0.874606025295984
0.00017 0.873112891761522
0.00017 0.873120404855267
0.000171 0.871647291984674
0.000171 0.871654765213811
0.000172 0.870201467725796
0.000172 0.870208901638676
0.000173 0.868775217976067
0.000173 0.868782613109611
0.000174 0.867368345359916
0.000174 0.867375702239906
0.000175 0.865980656057972
0.000175 0.865987975199333
0.000176 0.864611959732057
0.000176 0.864619241639132
0.000177 0.863262069452152
0.000177 0.863269314618965
0.000178 0.861930801625264
0.000178 0.86193801053578
0.000179 0.860617975926155
0.000179 0.860625149054527
0.00018 0.859323415229865
0.00018 0.859330553040676
0.000181 0.85804694554597
0.000181 0.858054048494468
0.000182 0.85678839595454
0.000182 0.856795464486862
0.000183 0.855547598543718
0.000183 0.855554633097115
0.000184 0.854324388348904
0.000184 0.854331389351953
0.000185 0.853118603293474
0.000185 0.853125571166283
0.000186 0.85193008413099
0.000186 0.851937019285402
0.000187 0.850758674388869
0.000187 0.850765577228655
0.000188 0.849603388279666
0.000188 0.849610268238603
0.000189 0.848464054961031
0.000189 0.848470902157881
0.00019 0.847341355839527
0.00019 0.847348181816482
0.000191 0.846233611835489
0.000191 0.846240404788377
0.000192 0.845142279941423
0.000192 0.845149042427507
0.000193 0.844067220846792
0.000193 0.844073953226357
0.000194 0.843008295479932
0.000194 0.84301499810639
0.000195 0.84196536715159
0.000195 0.841972040371632
0.000196 0.840938301509645
0.000196 0.84094494566339
0.000197 0.839926966494909
0.000197 0.839933581916053
0.000198 0.838931232298
0.000198 0.838937819313958
0.000199 0.837950971317237
0.000199 0.83795753024928
0.0002 0.836986058117537
0.0002 0.836992589280929
0.000201 0.836036369390292
0.000201 0.836042873094413
0.000202 0.835101783914173
0.000202 0.835108260462657
0.000203 0.83418218251688
0.000203 0.834188632207729
0.000204 0.833277448037766
0.000204 0.833283871163477
0.000205 0.832387465291341
0.000205 0.832393862139019
0.000206 0.831512121031623
0.000206 0.831518491883091
0.000207 0.830651303917302
0.000207 0.830657649049222
0.000208 0.829804904477722
0.000208 0.829811224161692
0.000209 0.828972815079631
0.000209 0.828979109582295
0.00021 0.828154929894689
0.00021 0.828161199477837
0.000211 0.82734450881035
0.000211 0.827351268041861
0.000212 0.826482451739875
0.000212 0.826489150809064
0.000213 0.825631442521699
0.000213 0.825638129054933
0.000214 0.824793866462801
0.000214 0.824800534938203
0.000215 0.82396993737747
0.000215 0.823976598103923
0.000216 0.823159806474294
0.000216 0.823166440930056
0.000217 0.822364219011392
0.000217 0.822370838311741
0.000218 0.82158247239298
0.000218 0.821589066271906
0.000219 0.820815347809539
0.000219 0.820821917841067
0.00022 0.820062745779514
0.00022 0.820069292207181
0.000221 0.819323736991369
0.000221 0.819330277896793
0.000222 0.818597766820429
0.000222 0.818604282572165
0.000223 0.817886064295994
0.000223 0.817892557122291
0.000224 0.817188537145508
0.000224 0.817195007274108
0.000225 0.816505092753642
0.000225 0.816511540408343
0.000226 0.815835640131776
0.000226 0.815842065532507
0.000227 0.815180089891592
0.000227 0.815186493254487
0.000228 0.814537691799331
0.000228 0.814544081760387
0.000229 0.813908757083129
0.000229 0.813915124982636
0.00023 0.813293484290827
0.00023 0.813299830767601
0.000231 0.81269179047873
0.000231 0.812698115734732
0.000232 0.81210359375017
0.000232 0.812109897983895
0.000233 0.811528813665661
0.000233 0.811535097072196
0.000234 0.810967371220228
0.000234 0.810973633991317
0.000235 0.810419188821239
0.000235 0.810425431145334
0.000236 0.8098841902667
0.000236 0.809890412329024
0.000237 0.80936230072403
0.000237 0.809368502706631
0.000238 0.808853446709284
0.000238 0.808859628791087
0.000239 0.808357556066828
0.000239 0.808363718423688
0.00024 0.807874557949438
0.00024 0.807880700754193
0.000241 0.807403195852768
0.000241 0.807409335735879
0.000242 0.806944410735741
0.000242 0.806950531048517
0.000243 0.806498348194879
0.000243 0.806504449441966
0.000244 0.806064942577186
0.000244 0.806071024920421
0.000245 0.805644129110858
0.000245 0.805650192709295
0.000246 0.805235844227686
0.000246 0.805241889237634
0.000247 0.80484002554584
0.000247 0.804846052120911
0.000248 0.804456611853022
0.000248 0.80446262014417
0.000249 0.80408554308997
0.000249 0.804091533245534
0.00025 0.803726760334303
0.00025 0.803732732500042
0.000251 0.8033802057847
0.000251 0.803386160103837
0.000252 0.80304582274541
0.000252 0.803051759358667
0.000253 0.802723555611083
0.000253 0.802729474656719
0.000254 0.802413349851911
0.000254 0.802419251465756
0.000255 0.802115151999073
0.000255 0.802121036314565
0.000256 0.801828909630486
0.000256 0.801834776778705
0.000257 0.801554571356839
0.000257 0.801560421466538
0.000258 0.80129208680792
0.000258 0.801297920005558
0.000259 0.801041406619213
0.000259 0.80104722302899
0.00026 0.800802482418779
0.00026 0.800808282162663
0.000261 0.800575266814397
0.000261 0.800581050012152
0.000262 0.800359713380957
0.000262 0.800365480150177
0.000263 0.800155776648127
0.000263 0.800161527104263
0.000264 0.799963412088254
0.000264 0.799969146344639
0.000265 0.799782576104511
0.000265 0.79978829427239
0.000266 0.799613226019287
0.000266 0.799618928207841
0.000267 0.799455320062806
0.000267 0.799461006379179
0.000268 0.799308817361972
0.000268 0.799314487911296
0.000269 0.799173677929444
0.000269 0.799179332814862
0.00027 0.79904986265292
0.00027 0.799055501975609
0.000271 0.79893733328464
0.000271 0.798942957143837
0.000272 0.798836052431096
0.000272 0.798841660924115
0.000273 0.798745983542944
0.000273 0.798751576765205
0.000274 0.798667090905119
0.000274 0.798672668950163
0.000275 0.798599339627142
0.000275 0.798604902586654
0.000276 0.798542526341764
0.000276 0.798548093643519
0.000277 0.798496794802926
0.000277 0.798502347126343
0.000278 0.798462143310196
0.000278 0.798467680816207
0.000279 0.79843854024601
0.000279 0.798444063019581
0.00028 0.798425954702807
0.00028 0.798431462827137
0.000281 0.798424356548575
0.000281 0.798429850105114
0.000282 0.798433716418395
0.000282 0.798439195486861
0.000283 0.798454005706141
0.000283 0.798459470364537
0.000284 0.79848519655635
0.000284 0.798490646880979
0.000285 0.798527261856243
0.000285 0.798532697921725
0.000286 0.798580175227911
0.000286 0.798585597107195
0.000287 0.79864372359691
0.000287 0.798649112057372
0.000288 0.798717993703469
0.000288 0.798723368083644
0.000289 0.79880299826189
0.000289 0.798808358655419
0.00029 0.798898713734355
0.00029 0.798904060207183
0.000291 0.799005117249897
0.000291 0.79901044986637
0.000292 0.799122186623544
0.000292 0.799127505446425
0.000293 0.799249900349515
0.000293 0.79925520543999
0.000294 0.799388237594537
0.000294 0.799393529012231
0.000295 0.799537178191298
0.000295 0.799542455994287
0.000296 0.799696702632024
0.000296 0.799701966876843
0.000297 0.799866792062171
0.000297 0.799872042803824
0.000298 0.800047428274243
0.000298 0.800052665566215
0.000299 0.800238593701727
0.000299 0.800243817595992
0.0003 0.80044027141314
0.0003 0.800445481960173
0.000301 0.80065244510619
0.000301 0.800657642354971
0.000302 0.800875099102045
0.000302 0.800880283100071
0.000303 0.801108218339714
0.000303 0.801113389133008
0.000304 0.801351788370529
0.000304 0.801356946003643
0.000305 0.80160579535273
0.000305 0.801610939868758
0.000306 0.801870226046158
0.000306 0.801875357486741
0.000307 0.802145067807036
0.000307 0.802150186212368
0.000308 0.802430308582857
0.000308 0.802435413991692
0.000309 0.802725936907362
0.000309 0.802731029357021
0.00031 0.80303194189561
0.00031 0.803037021421984
0.000311 0.803348313239142
0.000311 0.803353379876702
0.000312 0.80367504120123
0.000312 0.803680094983029
0.000313 0.804012116612219
0.000313 0.804017157569897
0.000314 0.804359530864947
0.000314 0.804364559028738
0.000315 0.804717275910256
0.000315 0.80472229130899
0.000316 0.805085344252576
0.000316 0.805090346913685
0.000317 0.805463728945599
0.000317 0.805468718895118
0.000318 0.805852423588021
0.000318 0.805857400850593
0.000319 0.806251422319367
0.000319 0.806256386918247
0.00032 0.80666071981589
0.00032 0.806665671772947
0.000321 0.807080311286539
0.000321 0.807085250622261
0.000322 0.807510192469006
0.000322 0.807515119202497
0.000323 0.807949923192583
0.000323 0.807954822110493
0.000324 0.808399033054582
0.000324 0.808403918123654
0.000325 0.808858385664748
0.000325 0.808863258160606
0.000326 0.809327979910152
0.000326 0.809332839846122
0.000327 0.809807813922588
0.000327 0.809812661310625
0.000328 0.810297886335921
0.000328 0.810302721186613
0.000329 0.810798196282575
0.000329 0.810803018605142
0.00033 0.811308743390073
0.00033 0.811313553192369
0.000331 0.81182952777764
0.000331 0.811834325066151
0.000332 0.812360550052859
0.000332 0.812365334832704
0.000333 0.81290181130838
0.000333 0.81290658358331
0.000334 0.813453313118684
0.000334 0.813458072891083
0.000335 0.814015057536907
0.000335 0.814019804807786
0.000336 0.814587047091694
0.000336 0.814591781860699
0.000337 0.815169017405787
0.000337 0.815173732319715
0.000338 0.815760631961362
0.000338 0.815765333494626
0.000339 0.81636248218068
0.000339 0.816367171191188
0.00034 0.816974573205772
0.00034 0.816979249687562
0.000341 0.8175969097745
0.000341 0.817601573720234
0.000342 0.81822949708018
0.000342 0.818234148481141
0.000343 0.818872340768763
0.000343 0.818876979614851
0.000344 0.819525446936046
0.000344 0.819530073215775
0.000345 0.820187128128946
0.000345 0.820191725854792
0.000346 0.82085880243767
0.000346 0.820863387207404
0.000347 0.821540720633139
0.000347 0.821545292769449
0.000348 0.82223289065153
0.000348 0.822237450137187
0.000349 0.822934759468657
0.000349 0.82293929887666
0.00035 0.823646422734667
0.00035 0.823650948777571
0.000351 0.824368342753525
0.000351 0.824372856073483
0.000352 0.825098534349996
0.000352 0.825103019081967
0.000353 0.825838471439718
0.000353 0.825842936330246
0.000354 0.826587138438691
0.000354 0.826591575799575
0.000355 0.827342599338917
0.000355 0.827346978715933
0.000356 0.828050962530844
0.000356 0.828054911905714
0.000357 0.82876152537645
0.000357 0.828765452618751
0.000358 0.829481126202232
0.000358 0.829485039969022
0.000359 0.830209776866417
0.000359 0.83021367712412
0.00036 0.830947480809085
0.00036 0.830951367522909
0.000361 0.831694241738668
0.000361 0.831698114872604
0.000362 0.832450063629034
0.000362 0.832453923145857
0.000363 0.833214950716582
0.000363 0.833218796577845
0.000364 0.833988907497346
0.000364 0.833992739663377
0.000365 0.834771938724101
0.000365 0.834775757154001
0.000366 0.835564049403486
0.000366 0.835567854055125
0.000367 0.836365244793121
0.000367 0.836369035623137
0.000368 0.837175530398741
0.000368 0.837179307362536
0.000369 0.837994911971327
0.000369 0.837998675023059
0.00037 0.83882339550424
0.00037 0.838827144596826
0.000371 0.839660987230364
0.000371 0.839664722315473
0.000372 0.840507693619243
0.000372 0.840511414647292
0.000373 0.841363521374227
0.000373 0.841367228294379
0.000374 0.842228477429611
0.000374 0.84223217018977
0.000375 0.84310256894778
0.000375 0.843106247494585
0.000376 0.843985803316345
0.000376 0.84398946759517
0.000377 0.844878188145286
0.000377 0.844881838100231
0.000378 0.84577973126408
0.000378 0.845783366837972
0.000379 0.846690440718843
0.000379 0.846694061853225
0.00038 0.847609974911941
0.00038 0.847613575776818
0.000381 0.848537674502781
0.000381 0.848541259453954
0.000382 0.849474544376263
0.000382 0.849478114701737
0.000383 0.850420218622597
0.000383 0.850423768490182
0.000384 0.851374041734419
0.000384 0.851377575464803
0.000385 0.85233704136083
0.000385 0.85234056026815
0.000386 0.853309228069519
0.000386 0.853312732086491
0.000387 0.854290611238211
0.000387 0.854294100296241
0.000388 0.85528120043184
0.000388 0.855284674461019
0.000389 0.856281005399586
0.000389 0.856284464328687
0.00039 0.857290036071905
0.00039 0.857293479828377
0.000391 0.858308302557546
0.000391 0.858311731067506
0.000392 0.859335815140547
0.000392 0.85933922832878
0.000393 0.860372584277228
0.000393 0.860375982067176
0.000394 0.861418620593153
0.000394 0.861422002906914
0.000395 0.862473934880092
0.000395 0.862477301638411
0.000396 0.863538538092952
0.000396 0.863541889215217
0.000397 0.864612441346701
0.000397 0.864615776750938
0.000398 0.865695655913262
0.000398 0.865698975516129
0.000399 0.866788193218401
0.000399 0.86679149693518
0.0004 0.867890064838583
0.0004 0.867893352583177
0.000401 0.869001282497813
0.000401 0.869004554182737
0.000402 0.870121858064453
0.000402 0.870125113600833
0.000403 0.871251803548018
0.000403 0.87125504284558
0.000404 0.872391131095949
0.000404 0.872394354063016
0.000405 0.873539852990358
0.000405 0.873543059533843
0.000406 0.874697981644756
0.000406 0.874701171670154
0.000407 0.875865529600744
0.000407 0.875868703012131
0.000408 0.877042509524691
0.000408 0.877045666224711
0.000409 0.878228934204372
0.000409 0.878232074094236
0.00041 0.879424816545588
0.00041 0.879427939525067
0.000411 0.880630169568753
0.000411 0.880633275536169
0.000412 0.88184500640545
0.000412 0.881848095257673
0.000413 0.88306934029496
0.000413 0.8830724119274
0.000414 0.884303184580761
0.000414 0.884306238887363
0.000415 0.885546552706991
0.000415 0.885549589580226
0.000416 0.88679945821488
0.000416 0.88680247754574
0.000417 0.888061914739153
0.000417 0.888064916417147
0.000418 0.88933393600439
0.000418 0.889336919917535
0.000419 0.890615535821366
0.000419 0.890618501856179
0.00042 0.891906728083335
0.00042 0.891909676124833
0.000421 0.893207526762301
0.000421 0.893210456693988
0.000422 0.89451794590523
0.000422 0.894520857609093
0.000423 0.89583799963024
0.000423 0.895840892986746
0.000424 0.89716770212275
0.000424 0.897170577010833
0.000425 0.898507067631579
0.000425 0.898509923928639
0.000426 0.899856110465017
0.000426 0.899858948046914
0.000427 0.901214844986854
0.000427 0.901217663727897
0.000428 0.902583285612361
0.000428 0.902586085385307
0.000429 0.90396040458937
0.000429 0.903963178241023
0.00043 0.905345499124765
0.00043 0.905348251180899
0.000431 0.906740304983008
0.000431 0.906743037685752
0.000432 0.908144838828262
0.000432 0.908147552044633
0.000433 0.909559114979548
0.000433 0.909561808574993
0.000434 0.910983147780025
0.000434 0.910985821618414
0.000435 0.912416951592661
0.000435 0.912419605536279
0.000436 0.913860540795866
0.000436 0.913863174705411
0.000437 0.915313929779076
0.000437 0.915316543513652
0.000438 0.916777132938292
0.000438 0.916779726355402
0.000439 0.918250164671561
0.000439 0.918252737627104
0.00044 0.919731298733609
0.00044 0.919733843918711
0.000441 0.921221227549539
0.000441 0.92122375057542
0.000442 0.922720991529058
0.000442 0.922723493661442
0.000443 0.92423060616386
0.000443 0.924233087252789
0.000444 0.925750085508227
0.000444 0.925752545402123
0.000445 0.927279443588277
0.000445 0.927281882133942
0.000446 0.928818694397107
0.000446 0.928821111439716
0.000447 0.930367851889868
0.000447 0.930370247272969
0.000448 0.931926929978801
0.000448 0.931929303544305
0.000449 0.933495942528197
0.000449 0.933498294116378
0.00045 0.935074903349325
0.00045 0.935077232798813
0.000451 0.936663826195285
0.000451 0.936666133343066
0.000452 0.938262724755816
0.000452 0.938265009437225
0.000453 0.939871612652043
0.000453 0.939873874700762
0.000454 0.941490503431163
0.000454 0.941492742679219
0.000455 0.943119410561081
0.000455 0.943121626838839
0.000456 0.944758347424979
0.000456 0.944760540561144
0.000457 0.946407327315828
0.000457 0.946409497137438
0.000458 0.948066363430844
0.000458 0.94806850976327
0.000459 0.949735468865879
0.000459 0.949737591532822
0.00046 0.951414656609752
0.00046 0.95141675543324
0.000461 0.953103939538522
0.000461 0.953106014338909
0.000462 0.954803330409693
0.000462 0.954805381005658
0.000463 0.956512841856363
0.000463 0.956514868064908
0.000464 0.958232486381308
0.000464 0.958234488017756
0.000465 0.959962276351
0.000465 0.959964253228994
0.000466 0.961702223989567
0.000466 0.961704175921068
0.000467 0.963452341372682
0.000467 0.963454268167974
0.000468 0.965212640421397
0.000468 0.965214541889081
0.000469 0.9669831328959
0.000469 0.966985008842898
0.00047 0.968763830389225
0.00047 0.968765680620775
0.000471 0.970554744320874
0.000471 0.970556568640536
0.000472 0.97235588593039
0.000472 0.972357684140047
0.000473 0.974162912538909
0.000473 0.974164671909296
0.000474 0.975977811461493
0.000474 0.975979541188493
0.000475 0.977802886497725
0.000475 0.977804589566229
0.000476 0.979638150658251
0.000476 0.979639826865451
0.000477 0.981481142561248
0.000477 0.981482781558123
0.000478 0.983329200631232
0.000478 0.983330805413569
0.000479 0.985187377459646
0.000479 0.985188954830307
0.00048 0.987055688267349
0.00048 0.987057238019381
0.000481 0.988934141020392
0.000481 0.988935662945267
0.000482 0.990822743431287
0.000482 0.99082423731891
0.000483 0.992721502951984
0.000483 0.992722968590698
0.000484 0.99463042676678
0.000484 0.994631863943375
0.000485 0.996549521785168
0.000485 0.996550930284883
0.000486 0.99847879463462
0.000486 0.998480174241153
0.000487 1.00041825165331
0.000487 1.00041960214882
0.000488 1.00236789888276
0.000488 1.0023692200479
0.000489 1.00432774206048
0.000489 1.00432903367435
0.00049 1.00629778661242
0.00049 1.00629904845264
0.000491 1.00827803764554
0.000491 1.00827926948822
0.000492 1.01026849994015
0.000492 1.01026970155991
0.000493 1.01226917794228
0.000493 1.01227034911227
0.000494 1.01428007575598
0.000494 1.01428121624787
0.000495 1.01630119713554
0.000495 1.01630230671954
0.000496 1.01833254547764
0.000496 1.01833362392253
0.000497 1.02037412381349
0.000497 1.02037517088662
0.000498 1.02242593480085
0.000498 1.02242695026814
0.000499 1.02448798071605
0.000499 1.02448896434201
0.0005 1.0265602634459
0.0005 1.02656121499366
0.000501 1.02864278447956
0.000501 1.02864370371087
0.000502 1.03073554490039
0.000502 1.03073643157564
0.000503 1.03283854537767
0.000503 1.0328393992559
0.000504 1.03495178615835
0.000504 1.03495260699729
0.000505 1.03707480921606
0.000505 1.03707559452566
0.000506 1.0392041397913
0.000506 1.0392048862339
0.000507 1.04134365216838
0.000507 1.04134436489734
0.000508 1.0434933498574
0.000508 1.04349402862699
0.000509 1.0456532305057
0.000509 1.04565387506896
0.00051 1.0478232912872
0.00051 1.04782390139599
0.000511 1.05000352889381
0.000511 1.05000410429882
0.000512 1.05219393952679
0.000512 1.05219447997757
0.000513 1.05439451888809
0.000513 1.05439502413307
0.000514 1.05660526217163
0.000514 1.05660573195814
0.000515 1.05882616405454
0.000515 1.05882659812884
0.000516 1.06105721868834
0.000516 1.06105761679565
0.000517 1.06329841969015
0.000517 1.06329878157466
0.000518 1.06554976013374
0.000518 1.06555008553866
0.000519 1.06781123254068
0.000519 1.06781152120824
0.00052 1.07008282887138
0.00052 1.07008308054288
0.000521 1.07236454051606
0.000521 1.07236475493189
0.000522 1.07465635828582
0.000522 1.07465653518548
0.000523 1.07695491778923
0.000523 1.07695505175266
0.000524 1.07925740609422
0.000524 1.07925749381404
0.000525 1.08156985288611
0.000525 1.08156990246711
0.000526 1.08389225424364
0.000526 1.08389226542613
0.000527 1.08622459725897
0.000527 1.08622456978261
0.000528 1.08856686839246
0.000528 1.08856680199628
0.000529 1.09091905346366
0.000529 1.09091894788612
0.00053 1.09328113764233
0.00053 1.09328099262131
0.000531 1.09565310543939
0.000531 1.09565292071228
0.000532 1.09803494069792
0.000532 1.09803471600164
0.000533 1.10042662658411
0.000533 1.10042636165513
0.000534 1.10282814557826
0.000534 1.10282784015266
0.000535 1.10523475836159
0.000535 1.1052344064632
0.000536 1.1076508984173
0.000536 1.10765050530336
0.000537 1.11007675915229
0.000537 1.11007632484622
0.000538 1.1125123201795
0.000538 1.11251184441846
0.000539 1.11495756009666
0.000539 1.11495704261765
0.00054 1.11741245676343
0.00054 1.11741189730337
0.000541 1.11987698729257
0.000541 1.11987638558832
0.000542 1.12234509542721
0.000542 1.12234444568529
0.000543 1.12481868684974
0.000543 1.12481798909749
0.000544 1.12730172250051
0.000544 1.12730098192933
0.000545 1.12979418127168
0.000545 1.1297933976233
0.000546 1.13229603581956
0.000546 1.13229520883601
0.000547 1.13480291945835
0.000547 1.13480204480926
0.000548 1.13731835820036
0.000548 1.13731743882679
0.000549 1.13984304963098
0.000549 1.13984208626426
0.00055 1.14235951695141
0.00055 1.14235849419894
0.000551 1.14488210526861
0.000551 1.1448810346533
0.000552 1.14740929920395
0.000552 1.14740818078644
0.000553 1.15029829592003
0.000553 1.1502972340352
0.000554 1.15413187649012
0.000554 1.15413254315053
0.000555 1.1575910267802
0.000555 1.15759091325284
0.000556 1.16107052657387
0.000556 1.16107035183322
0.000557 1.16457117048714
0.000557 1.16457093378646
0.000558 1.16809303289457
0.000558 1.16809273347878
0.000559 1.17163618761525
0.000559 1.17163582472089
0.00056 1.1752007078855
0.00056 1.1752002807407
0.000561 1.17878666633088
0.000561 1.17878617415533
0.000562 1.18239413493776
0.000562 1.18239357694266
0.000563 1.18602318502425
0.000563 1.18602256041228
0.000564 1.18967388721064
0.000564 1.18967319517596
0.000565 1.19334631138925
0.000565 1.19334555111745
0.000566 1.19704052669371
0.000566 1.19703969736178
0.000567 1.2007566014677
0.000567 1.20075570224402
0.000568 1.20449460323305
0.000568 1.20449363327737
0.000569 1.20825459865733
0.000569 1.20825355712076
0.00057 1.21203665352082
0.00057 1.21203553954579
0.000571 1.21584083268285
0.000571 1.21583964540314
0.000572 1.21966720004767
0.000572 1.21966593858836
0.000573 1.22351581852956
0.000573 1.22351448200705
0.000574 1.22738675001747
0.000574 1.22738533753948
0.000575 1.23128005533899
0.000575 1.23127856600453
0.000576 1.23519579422373
0.000576 1.23519422712316
0.000577 1.23913402526611
0.000577 1.2391323794811
0.000578 1.24309480588748
0.000578 1.24309308049104
0.000579 1.24707819229771
0.000579 1.2470763863542
0.00058 1.25108423945609
0.00058 1.25108235202125
0.000581 1.25511300103162
0.000581 1.25511103115259
0.000582 1.25916452936276
0.000582 1.25916247607811
0.000583 1.26323358561356
0.000583 1.2632314451145
0.000584 1.26731762058146
0.000584 1.26731538075087
0.000585 1.2714243392026
0.000585 1.27142201329129
0.000586 1.27555379975228
0.000586 1.27555138677582
0.000587 1.27970604472321
0.000587 1.27970354368906
0.000588 1.28388111500903
0.000588 1.28387852491661
0.000589 1.28807904986017
0.000589 1.28807636970093
0.00059 1.29229461580768
0.00059 1.29229184547192
0.000591 1.29652214484933
0.000591 1.29651926286747
0.000592 1.30077236662407
0.000592 1.30076939187989
0.000593 1.30504533042862
0.000593 1.30504226189661
0.000594 1.3093410637751
0.000594 1.3093379004225
0.000595 1.31365893649393
0.000595 1.31365568838795
0.000596 1.31798383865537
0.000596 1.31798046520385
0.000597 1.32233130494336
0.000597 1.32232783396301
0.000598 1.32670138125204
0.000598 1.32669781169666
0.000599 1.33109408249291
0.000599 1.33109041330995
0.0006 1.33550942140247
};

\nextgroupplot[
    x label style={at={(axis description cs:0.5,-0.05)},anchor=north},
    y label style={at={(axis description cs:-0.005,.5)},rotate=0,anchor=south},
tick align=outside,
tick pos=left,
x grid style={white!69.0196078431373!black},
xlabel={Время},
xmin=-3e-05, xmax=0.00063,
xtick style={color=black},
y grid style={white!69.0196078431373!black},
ylabel={$I \cdot R_p$},
ymin=119.082547295159, ymax=655.990364623057,
ytick style={color=black}
]
\addplot [semithick, color0]
table {%
0 287.97751337034
1e-06 148.790884563562
1e-06 146.789302544322
2e-06 143.487448082791
2e-06 143.494647188219
3e-06 152.987763657292
3e-06 152.948644293139
4e-06 160.923874326811
4e-06 160.895321636402
5e-06 167.895129494809
5e-06 167.87184272148
6e-06 174.055077314755
6e-06 174.035794609653
7e-06 179.569776596813
7e-06 179.554196677621
8e-06 187.88934697265
8e-06 187.837160004124
9e-06 199.511740420672
9e-06 199.448575644181
1e-05 210.004954555098
1e-05 209.95257318125
1.1e-05 219.516947954216
1.1e-05 219.471817943179
1.2e-05 228.169554040884
1.2e-05 228.13160806702
1.3e-05 236.201486774399
1.3e-05 236.167842947556
1.4e-05 243.667921988841
1.4e-05 243.638754658493
1.5e-05 250.56202130436
1.5e-05 250.535655958177
1.6e-05 257.077729831861
1.6e-05 257.053992739239
1.7e-05 263.188658510476
1.7e-05 263.167619976647
1.8e-05 268.895498168595
1.8e-05 268.876240625048
1.9e-05 274.195302301168
1.9e-05 274.178247350665
2e-05 279.202416289343
2e-05 279.186766199497
2.1e-05 284.007743373718
2.1e-05 283.993082291367
2.2e-05 288.517077380564
2.2e-05 288.503764730882
2.3e-05 292.8293710772
2.3e-05 292.817037586647
2.4e-05 296.952562900528
2.4e-05 296.940891802156
2.5e-05 300.937263746026
2.5e-05 300.926249835781
2.6e-05 304.678449333415
2.6e-05 304.668390133722
2.7e-05 308.296003591107
2.7e-05 308.286387145879
2.8e-05 311.767714079765
2.8e-05 311.758846077257
2.9e-05 314.96912154666
2.9e-05 314.962981135851
3e-05 317.157102196529
3e-05 317.152798666652
3.1e-05 319.178964196465
3.1e-05 319.175092150855
3.2e-05 321.037537654954
3.2e-05 321.034023988176
3.3e-05 322.820740167704
3.3e-05 322.817385442041
3.4e-05 326.977906725648
3.4e-05 326.961901421478
3.5e-05 332.068085732502
3.5e-05 332.050015742215
3.6e-05 337.011740816588
3.6e-05 336.994397563136
3.7e-05 341.817059946699
3.7e-05 341.800368520194
3.8e-05 346.495481977185
3.8e-05 346.479390401447
3.9e-05 351.052268080889
3.9e-05 351.036741622489
4e-05 355.469907275356
4e-05 355.45512474146
4.1e-05 359.751365328036
4.1e-05 359.737136543675
4.2e-05 363.925755419849
4.2e-05 363.911997077572
4.3e-05 367.997219667232
4.3e-05 367.983906164435
4.4e-05 371.969726464927
4.4e-05 371.956834002043
4.5e-05 375.841225221723
4.5e-05 375.828776660972
4.6e-05 379.616786171254
4.6e-05 379.604723868525
4.7e-05 383.30419523438
4.7e-05 383.292490862356
4.8e-05 386.906641796118
4.8e-05 386.89527739879
4.9e-05 390.427162081724
4.9e-05 390.416120875019
5e-05 393.868640526046
5e-05 393.85790680508
5.1e-05 397.230839553228
5.1e-05 397.220480918308
5.2e-05 400.48711389227
5.2e-05 400.477154587547
5.3e-05 403.663117504516
5.3e-05 403.653437622068
5.4e-05 406.770804677984
5.4e-05 406.761374752558
5.5e-05 409.812414979935
5.5e-05 409.803223675879
5.6e-05 412.790100844483
5.6e-05 412.781137500063
5.7e-05 415.694769285771
5.7e-05 415.686078749914
5.8e-05 418.538745860101
5.8e-05 418.530264488009
5.9e-05 421.316959866366
5.9e-05 421.308725641676
6e-05 424.035036348197
6e-05 424.026998188457
6.1e-05 426.69885614787
6.1e-05 426.690999407321
6.2e-05 429.304701168525
6.2e-05 429.297055787583
6.3e-05 431.855079743641
6.3e-05 431.847607464102
6.4e-05 434.356045340414
6.4e-05 434.348732494287
6.5e-05 436.809030928399
6.5e-05 436.801871209545
6.6e-05 439.215417374009
6.6e-05 439.208404806949
6.7e-05 441.576529523318
6.7e-05 441.569658442037
6.8e-05 443.893639097356
6.8e-05 443.886904125705
6.9e-05 446.167967406166
6.9e-05 446.16136343972
7e-05 448.400687894878
7e-05 448.39421008412
7.1e-05 450.592928533968
7.1e-05 450.586572268699
7.2e-05 452.745774064863
7.2e-05 452.739534959732
7.3e-05 454.860268111136
7.3e-05 454.854141992195
7.4e-05 456.937415164708
7.4e-05 456.931398056909
7.5e-05 458.939641781362
7.5e-05 458.934797361792
7.6e-05 461.160009296014
7.6e-05 461.151741708313
7.7e-05 463.83746314954
7.7e-05 463.828473250039
7.8e-05 466.459063485846
7.8e-05 466.450287754058
7.9e-05 469.035198331276
7.9e-05 469.026575939176
8e-05 471.56871528612
8e-05 471.56022487031
8.1e-05 474.06354277059
8.1e-05 474.055175473674
8.2e-05 476.513593399517
8.2e-05 476.505384724031
8.3e-05 478.925283299969
8.3e-05 478.917192764987
8.4e-05 481.300708387584
8.4e-05 481.292730526256
8.5e-05 483.640607038819
8.5e-05 483.632738739308
8.6e-05 485.941652866441
8.6e-05 485.933911213667
8.7e-05 488.208620497398
8.7e-05 488.20098238898
8.8e-05 490.442191360599
8.8e-05 490.434653979445
8.9e-05 492.643023929515
8.9e-05 492.635584596241
9e-05 494.811758009212
9e-05 494.804414140336
9.1e-05 496.949015372292
9.1e-05 496.941764476023
9.2e-05 499.048872973696
9.2e-05 499.041749255819
9.3e-05 501.117183011228
9.3e-05 501.110149271201
9.4e-05 503.155852790259
9.4e-05 503.148904741136
9.5e-05 505.165435928855
9.5e-05 505.158571411719
9.6e-05 507.146472995407
9.6e-05 507.139689924979
9.7e-05 509.099489976773
9.7e-05 509.092786338325
9.8e-05 511.024998775865
9.8e-05 511.018372622284
9.9e-05 512.923497688585
9.9e-05 512.916947137583
0.0001 514.795471861137
0.0001 514.788995092588
0.000101 516.641393728635
0.000101 516.634988982056
0.000102 518.459129595834
0.000102 518.452811666619
0.000103 520.250656675007
0.000103 520.244408824492
0.000104 522.017513573949
0.000104 522.011332657851
0.000105 523.760124215615
0.000105 523.754008690974
0.000106 525.478903300188
0.000106 525.472851672714
0.000107 527.174255098517
0.000107 527.168265920696
0.000108 528.846573782855
0.000108 528.84064565214
0.000109 530.496243744814
0.000109 530.490375301906
0.00011 532.123639901147
0.00011 532.117829828341
0.000111 533.729127987873
0.000111 533.723375007494
0.000112 535.31306484329
0.000112 535.307367716186
0.000113 536.875798680353
0.000113 536.870156204464
0.000114 538.417669348903
0.000114 538.412080357883
0.000115 539.939008588165
0.000115 539.933471950077
0.000116 541.440140269968
0.000116 541.434654886022
0.000117 542.921380633062
0.000117 542.915945436422
0.000118 544.383038508939
0.000118 544.377652463565
0.000119 545.823398711775
0.000119 545.818087516667
0.00012 547.240324938505
0.00012 547.235066315172
0.000121 548.638625903341
0.000121 548.633413270843
0.000122 550.015676712694
0.000122 550.010534245162
0.000123 551.371861188884
0.000123 551.366766192142
0.000124 552.710301426189
0.000124 552.705249541717
0.000125 554.03125577825
0.000125 554.026246162849
0.000126 555.334980224848
0.000126 555.330012058403
0.000127 556.621725044493
0.000127 556.616797529196
0.000128 557.891734972838
0.000128 557.886847332428
0.000129 559.145249355695
0.000129 559.140400834744
0.00013 560.38250229691
0.00013 560.377692160122
0.000131 561.603722801254
0.000131 561.598950332806
0.000132 562.809134912574
0.000132 562.804399415471
0.000133 563.998957847342
0.000133 563.994258642805
0.000134 565.173406123824
0.000134 565.1687425507
0.000135 566.332689687009
0.000135 566.328061101201
0.000136 567.477014029474
0.000136 567.472419803394
0.000137 568.606580308332
0.000137 568.60201983038
0.000138 569.721585458429
0.000138 569.717058132483
0.000139 570.82222230191
0.000139 570.817727546842
0.00014 571.908679654298
0.00014 571.904216903505
0.000141 572.981142427232
0.000141 572.976711128185
0.000142 574.039791727958
0.000142 574.035391341769
0.000143 575.084804955721
0.000143 575.080434956725
0.000144 576.116355895157
0.000144 576.112015770508
0.000145 577.132298506315
0.000145 577.12800049305
0.000146 578.135046925113
0.000146 578.130777838494
0.000147 579.124861197026
0.000147 579.120620443329
0.000148 580.101901217144
0.000148 580.097688331997
0.000149 581.066323811474
0.000149 581.0621383415
0.00015 582.018282696466
0.00015 582.014124198964
0.000151 582.957928554256
0.000151 582.953796596887
0.000152 583.885000063142
0.000152 583.880905494203
0.000153 584.798039608894
0.000153 584.793972808471
0.000154 585.699231637286
0.000154 585.695190062435
0.000155 586.588713058946
0.000155 586.584696316786
0.000156 587.466620237264
0.000156 587.462627943853
0.000157 588.33308691042
0.000157 588.329118690507
0.000158 589.188244252468
0.000158 589.18429973925
0.000159 590.032220932663
0.000159 590.028299767545
0.00016 590.865143173086
0.00016 590.86124500546
0.000161 591.687134804644
0.000161 591.683259291663
0.000162 592.498317321475
0.000162 592.494464127844
0.000163 593.298809933826
0.000163 593.294978731599
0.000164 594.088729619456
0.000164 594.084920087831
0.000165 594.868191173597
0.000165 594.864402998729
0.000166 595.637307257539
0.000166 595.633540132352
0.000167 596.396188445872
0.000167 596.39244206988
0.000168 597.144943272428
0.000168 597.141217351562
0.000169 597.883678274977
0.000169 597.879972521413
0.00017 598.612498038696
0.00017 598.608812170696
0.000171 599.331505238475
0.000171 599.327838980226
0.000172 600.040800680076
0.000172 600.037153761538
0.000173 600.740483340188
0.000173 600.736855496947
0.000174 601.430650405425
0.000174 601.42704137855
0.000175 602.111397310277
0.000175 602.107806846179
0.000176 602.782817774064
0.000176 602.779245624362
0.000177 603.445003836921
0.000177 603.441449758311
0.000178 604.098045894835
0.000178 604.094509648965
0.000179 604.742032733775
0.000179 604.738514087122
0.00018 605.377051562936
0.00018 605.373550286686
0.000181 606.003188047123
0.000181 605.999703917057
0.000182 606.620526338307
0.000182 606.617059134688
0.000183 607.22914910637
0.000183 607.225698613836
0.000184 607.829137569067
0.000184 607.825703576525
0.000185 608.420571521237
0.000185 608.417153821759
0.000186 609.003529363263
0.000186 609.000127753991
0.000187 609.578088128833
0.000187 609.57470241088
0.000188 610.143725985897
0.000188 610.140363820258
0.000189 610.70050308097
0.000189 610.697156967155
0.00019 611.249089950664
0.00019 611.245766372994
0.000191 611.788448640333
0.000191 611.785141183669
0.000192 612.319809665015
0.000192 612.316517083382
0.000193 612.8432411272
0.000193 612.839963244345
0.000194 613.358811007257
0.000194 613.355547650287
0.000195 613.866586119852
0.000195 613.863337119159
0.000196 614.366632136097
0.000196 614.363397325282
0.000197 614.859013605172
0.000197 614.855792820976
0.000198 615.343793975422
0.000198 615.340587057655
0.000199 615.821035614956
0.000199 615.817842406431
0.0002 616.290799831754
0.0002 616.287620178221
0.000201 616.753146893298
0.000201 616.749980643376
0.000202 617.208136045736
0.000202 617.20498305086
0.000203 617.65582553261
0.000203 617.652685646961
0.000204 618.096272613125
0.000204 618.093145693577
0.000205 618.529533580011
0.000205 618.526419486074
0.000206 618.955663776962
0.000206 618.952562370722
0.000207 619.374717615671
0.000207 619.371628761742
0.000208 619.786748592479
0.000208 619.783672157945
0.000209 620.191809304636
0.000209 620.188745159005
0.00021 620.589951466196
0.00021 620.586899481346
0.000211 620.976245127822
0.000211 620.973602277469
0.000212 621.306396549967
0.000212 621.303849869494
0.000213 621.627814892705
0.000213 621.625294250269
0.000214 621.942318273822
0.000214 621.939815150912
0.000215 622.250187835019
0.000215 622.24770918844
0.000216 622.551659213801
0.000216 622.54919036722
0.000217 622.847416475602
0.000217 622.844964100762
0.000218 623.137043256426
0.000218 623.134600322905
0.000219 623.421250115548
0.000219 623.418816039653
0.00022 623.700074052351
0.00022 623.697648743323
0.000221 623.97291959801
0.000221 623.970517947391
0.000222 624.239475805629
0.000222 624.237083411923
0.000223 624.500791039071
0.000223 624.498407083609
0.000224 624.756899308548
0.000224 624.754523707203
0.000225 625.007834746874
0.000225 625.005467416972
0.000226 625.253630886373
0.000226 625.251271746666
0.000227 625.494320668618
0.000227 625.491969639259
0.000228 625.729427581704
0.000228 625.727092169336
0.000229 625.959290663861
0.000229 625.956963332693
0.00023 626.184158852654
0.00023 626.181839368735
0.000231 626.404062572509
0.000231 626.401750861555
0.000232 626.61903186222
0.000232 626.616727851216
0.000233 626.829096225086
0.000233 626.826799842271
0.000234 627.03428463724
0.000234 627.03199581208
0.000235 627.234625555795
0.000235 627.232344218962
0.000236 627.43014692681
0.000236 627.427873010163
0.000237 627.620876193091
0.000237 627.618609629652
0.000238 627.806840301817
0.000238 627.804581025757
0.000239 627.988065712015
0.000239 627.985813658631
0.00024 628.164578401861
0.00024 628.162333507555
0.000241 628.335480173366
0.000241 628.333257189842
0.000242 628.501585736666
0.000242 628.49936985163
0.000243 628.663084094735
0.000243 628.66087512506
0.000244 628.819999094353
0.000244 628.817796981029
0.000245 628.972354260736
0.000245 628.970158945768
0.000246 629.120172680853
0.000246 629.117984107241
0.000247 629.263477009683
0.000247 629.261295121411
0.000248 629.402289476346
0.000248 629.400114218363
0.000249 629.536631890112
0.000249 629.534463208319
0.00025 629.666525646273
0.00025 629.664363487506
0.000251 629.791991731903
0.000251 629.789836043924
0.000252 629.913050731494
0.000252 629.910901462972
0.000253 630.029722832476
0.000253 630.027579932977
0.000254 630.142027830626
0.000254 630.139891250599
0.000255 630.24998513536
0.000255 630.247854826123
0.000256 630.353613774923
0.000256 630.351489688653
0.000257 630.452932401467
0.000257 630.450814491187
0.000258 630.54795929603
0.000258 630.545847515596
0.000259 630.63871237341
0.000259 630.636606677501
0.00026 630.72520918694
0.00026 630.723109531046
0.000261 630.807466933171
0.000261 630.805373273583
0.000262 630.885502456448
0.000262 630.883414750247
0.000263 630.959332253413
0.000263 630.957250458459
0.000264 631.028972477397
0.000264 631.026896552321
0.000265 631.094438942735
0.000265 631.092368846927
0.000266 631.155747128995
0.000266 631.153682822595
0.000267 631.212912185114
0.000267 631.210853629003
0.000268 631.265948933459
0.000268 631.26389608925
0.000269 631.314871873802
0.000269 631.312824703833
0.00027 631.359695187221
0.00027 631.357653654542
0.000271 631.400432739918
0.000271 631.398396808287
0.000272 631.437098086961
0.000272 631.435067720835
0.000273 631.469704475962
0.000273 631.467679640486
0.000274 631.498264850663
0.000274 631.496245511666
0.000275 631.522791854472
0.000275 631.520777978459
0.000276 631.543163946099
0.000276 631.541171081095
0.000277 631.559533950041
0.000277 631.557546447897
0.000278 631.571937747908
0.000278 631.569955550712
0.000279 631.580386663053
0.000279 631.578409740091
0.00028 631.584891762885
0.00028 631.582920084078
0.000281 631.585463835426
0.000281 631.583497371323
0.000282 631.582113392346
0.000282 631.580152114121
0.000283 631.574850671959
0.000283 631.572894551401
0.000284 631.56368564214
0.000284 631.561734651651
0.000285 631.548628003203
0.000285 631.546682115791
0.000286 631.52968719071
0.000286 631.527746379981
0.000287 631.506724177794
0.000287 631.504773471961
0.000288 631.47983724109
0.000288 631.477891630648
0.000289 631.449064233356
0.000289 631.447123684142
0.00029 631.414413662714
0.00029 631.412478150595
0.000291 631.375893794297
0.000291 631.373963295724
0.000292 631.333512643328
0.000292 631.331587135326
0.000293 631.287277977593
0.000293 631.28535743776
0.000294 631.237197319875
0.000294 631.235281726379
0.000295 631.183277950337
0.000295 631.181367281907
0.000296 631.125526908856
0.000296 631.123621144784
0.000297 631.063950997321
0.000297 631.062050117455
0.000298 630.998556781878
0.000298 630.99666076662
0.000299 630.929350595144
0.000299 630.927459425443
0.0003 630.856338538362
0.0003 630.854452195716
0.000301 630.779526483538
0.000301 630.777644949986
0.000302 630.698920075514
0.000302 630.697043333635
0.000303 630.614524734021
0.000303 630.612652766933
0.000304 630.526345655684
0.000304 630.524478447036
0.000305 630.434387815989
0.000305 630.432525349963
0.000306 630.338655971218
0.000306 630.336798232525
0.000307 630.239154660344
0.000307 630.237301634222
0.000308 630.135888206895
0.000308 630.134039879104
0.000309 630.028860720778
0.000309 630.027017077601
0.00031 629.918076100072
0.00031 629.916237128313
0.000311 629.803538032791
0.000311 629.801703719769
0.000312 629.685249998608
0.000312 629.683420332158
0.000313 629.563215270553
0.000313 629.561390239025
0.000314 629.437436916676
0.000314 629.435616508933
0.000315 629.307917801684
0.000315 629.306102007098
0.000316 629.174660588544
0.000316 629.172849396996
0.000317 629.037667740059
0.000317 629.03586114194
0.000318 628.896941520415
0.000318 628.895139506621
0.000319 628.752483996702
0.000319 628.750686558634
0.00032 628.604297040404
0.00032 628.602504169969
0.000321 628.452382328868
0.000321 628.450594018474
0.000322 628.296741346737
0.000322 628.294957589298
0.000323 628.137036084304
0.000323 628.135245768031
0.000324 627.972907808375
0.000324 627.971122543052
0.000325 627.805035333182
0.000325 627.803254652468
0.000326 627.633419001913
0.000326 627.631642900779
0.000327 627.45805943254
0.000327 627.456287906459
0.000328 627.278957058331
0.000328 627.27719010328
0.000329 627.096112129136
0.000329 627.094349741593
0.00033 626.909524712659
0.00033 626.907766889605
0.000331 626.719194695704
0.000331 626.717441434623
0.000332 626.525121785406
0.000332 626.523373084282
0.000333 626.327305510435
0.000333 626.325561367757
0.000334 626.125745222185
0.000334 626.124005636944
0.000335 625.920440095945
0.000335 625.918705067635
0.000336 625.711389132046
0.000336 625.709658660666
0.000337 625.498385991298
0.000337 625.496654885374
0.000338 625.281170750723
0.000338 625.279444545009
0.000339 625.060195796066
0.000339 625.058474175453
0.00034 624.835459159295
0.00034 624.83374212585
0.000341 624.606959019635
0.000341 624.605246575932
0.000342 624.374693387954
0.000342 624.372985537079
0.000343 624.13866010781
0.000343 624.136956853357
0.000344 623.898856856473
0.000344 623.897158202548
0.000345 623.653993065062
0.000345 623.652289705119
0.000346 623.405150971126
0.000346 623.403452397357
0.000347 623.15251165448
0.000347 623.150817747233
0.000348 622.896072085953
0.000348 622.894382851512
0.000349 622.635404705404
0.000349 622.633715482962
0.00035 622.370576637443
0.00035 622.368892374125
0.000351 622.101929467244
0.000351 622.100249923997
0.000352 621.827955923365
0.000352 621.82627249871
0.000353 621.549770489342
0.000353 621.548087342089
0.000354 621.266599420138
0.000354 621.264912535909
0.000355 620.976989596815
0.000355 620.975279398114
0.000356 620.640561945576
0.000356 620.638639422703
0.000357 620.294664384544
0.000357 620.292752619271
0.000358 619.9443641793
0.000358 619.942458957384
0.000359 619.589655450275
0.000359 619.587756788009
0.00036 619.230536414812
0.00036 619.228644329082
0.000361 618.86700515866
0.000361 618.865119666948
0.000362 618.499059637397
0.000362 618.497180757782
0.000363 618.126697677852
0.000363 618.124825429011
0.000364 617.749916979527
0.000364 617.748051380734
0.000365 617.368715116001
0.000365 617.366856187134
0.000366 616.983089536352
0.000366 616.98123729789
0.000367 616.593037566556
0.000367 616.591192039584
0.000368 616.198556410896
0.000368 616.196717617103
0.000369 615.799643153365
0.000369 615.797811115046
0.00037 615.396294759063
0.00037 615.394469499124
0.000371 614.988508075601
0.000371 614.986689617559
0.000372 614.576279834499
0.000372 614.574468202483
0.000373 614.159606652582
0.000373 614.157801871337
0.000374 613.738485033383
0.000374 613.736687128268
0.000375 613.312911368537
0.000375 613.31112036553
0.000376 612.882881939181
0.000376 612.881097864884
0.000377 612.448392917357
0.000377 612.446615798993
0.000378 612.009440367414
0.000378 612.007670232829
0.000379 611.566020247406
0.000379 611.564257125076
0.00038 611.117876166927
0.00038 611.11611649916
0.000381 610.664526394395
0.000381 610.662774485652
0.000382 610.206690638812
0.000382 610.204945859696
0.000383 609.744094255318
0.000383 609.742353115726
0.000384 609.276260626477
0.000384 609.274527384586
0.000385 608.803921353562
0.000385 608.802195364845
0.000386 608.32707113221
0.000386 608.325352429706
0.000387 607.845705241013
0.000387 607.843993858405
0.000388 607.359818866046
0.000388 607.358114837666
0.000389 606.869407102328
0.000389 606.867710463157
0.00039 606.374464955285
0.00039 606.37277574096
0.000391 605.874987342223
0.000391 605.873305589033
0.000392 605.370969093802
0.000392 605.369294838699
0.000393 604.862404955529
0.000393 604.860738236123
0.000394 604.349289589244
0.000394 604.34763044381
0.000395 603.831617574624
0.000395 603.829966042102
0.000396 603.309383410691
0.000396 603.307739530694
0.000397 602.782581517336
0.000397 602.780945330145
0.000398 602.251206236838
0.000398 602.249577783413
0.000399 601.715251835411
0.000399 601.713631157387
0.0004 601.174712504743
0.0004 601.173099644437
0.000401 600.62958236356
0.000401 600.627977363972
0.000402 600.079855459189
0.000402 600.078258364008
0.000403 599.525525769144
0.000403 599.523936622746
0.000404 598.966587202714
0.000404 598.965006050168
0.000405 598.403033602561
0.000405 598.401460489633
0.000406 597.834858746347
0.000406 597.833293719499
0.000407 597.262056348349
0.000407 597.260499454748
0.000408 596.684620061112
0.000408 596.683071348626
0.000409 596.102543477092
0.000409 596.101002994301
0.00041 595.515820130335
0.00041 595.514287926525
0.000411 594.924443498149
0.000411 594.922919623325
0.000412 594.328407002811
0.000412 594.32689150769
0.000413 593.72770401327
0.000413 593.726196949293
0.000414 593.122327846878
0.000414 593.120829266208
0.000415 592.512271771134
0.000415 592.510781726661
0.000416 591.89752900544
0.000416 591.896047550781
0.000417 591.278092722877
0.000417 591.276619912384
0.000418 590.653956051997
0.000418 590.652491940757
0.000419 590.02511207863
0.000419 590.02365672247
0.00042 589.391553847716
0.00042 589.390107303203
0.000421 588.753274365142
0.000421 588.75183668959
0.000422 588.110266599609
0.000422 588.10883785108
0.000423 587.462523484515
0.000423 587.461103721822
0.000424 586.810037919849
0.000424 586.80862720256
0.000425 586.152802774118
0.000425 586.151401162556
0.000426 585.490810886277
0.000426 585.489418441531
0.000427 584.8240550677
0.000427 584.822671851618
0.000428 584.152528104149
0.000428 584.151154179349
0.000429 583.475550043704
0.000429 583.474182569193
0.00043 582.792662227428
0.00043 582.791305383585
0.000431 582.104978138945
0.000431 582.103630820531
0.000432 581.412489422782
0.000432 581.411151695551
0.000433 580.715188879389
0.000433 580.713860809873
0.000434 580.013069297043
0.000434 580.011750952556
0.000435 579.306123453988
0.000435 579.304814902627
0.000436 578.5943441206
0.000436 578.593045431252
0.000437 577.877724061575
0.000437 577.876435303916
0.000438 577.156256038141
0.000438 577.154977282639
0.000439 576.42993281029
0.000439 576.428664128212
0.00044 575.697657600354
0.00044 575.696397264844
0.000441 574.959863879348
0.000441 574.95861450156
0.000442 574.217190881403
0.000442 574.21595183479
0.000443 573.469630797254
0.000443 573.468402156297
0.000444 572.717176529266
0.000444 572.715958369251
0.000445 571.959820993594
0.000445 571.958613390614
0.000446 571.197557122614
0.000446 571.196360153571
0.000447 570.430377867359
0.000447 570.429191609967
0.000448 569.658276200002
0.000448 569.65710073279
0.000449 568.881245116354
0.000449 568.880080518664
0.00045 568.099277638393
0.00045 568.098123990386
0.000451 567.312366816819
0.000451 567.311224199476
0.000452 566.520505733643
0.000452 566.519374228767
0.000453 565.723687504799
0.000453 565.722567195015
0.000454 564.921905282784
0.000454 564.920796251542
0.000455 564.115152259332
0.000455 564.114054590908
0.000456 563.303421668113
0.000456 563.302335447612
0.000457 562.486706787464
0.000457 562.485632100818
0.000458 561.665000943148
0.000458 561.663937877123
0.000459 560.838297511146
0.000459 560.837246153336
0.00046 560.006589920473
0.00046 560.005550359307
0.000461 559.169871656035
0.000461 559.168843980773
0.000462 558.328136261501
0.000462 558.327120562241
0.000463 557.481377342228
0.000463 557.480373709902
0.000464 556.629588568194
0.000464 556.62859709457
0.000465 555.77276367698
0.000465 555.771784454661
0.000466 554.910896476773
0.000466 554.9099295992
0.000467 554.043980849405
0.000467 554.043026410857
0.000468 553.172010753427
0.000468 553.171068849019
0.000469 552.294980227208
0.000469 552.294050952892
0.00047 551.412883392071
0.00047 551.411966844637
0.000471 550.525714455465
0.000471 550.524810732537
0.000472 549.633467714156
0.000472 549.632576914199
0.000473 548.733685155161
0.000473 548.732806716005
0.000474 547.827515672884
0.000474 547.826652024161
0.000475 546.916254476571
0.000475 546.915404128262
0.000476 545.999894943941
0.000476 545.999057997622
0.000477 545.077059858163
0.000477 545.076234388002
0.000478 544.146291957872
0.000478 544.145483710223
0.000479 543.210416873592
0.000479 543.209622422557
0.00048 542.269426817282
0.00048 542.268646267416
0.000481 541.323317655041
0.000481 541.322551111693
0.000482 540.372085380971
0.000482 540.371332950285
0.000483 539.415726120734
0.000483 539.414987909646
0.000484 538.454236135136
0.000484 538.453512251368
0.000485 537.487611823746
0.000485 537.486902375805
0.000486 536.515849728542
0.000486 536.515154825717
0.000487 535.538946537598
0.000487 535.538266289952
0.000488 534.556899088788
0.000488 534.556233607159
0.000489 533.569704373539
0.000489 533.569053769534
0.00049 532.577359540602
0.00049 532.576723926591
0.000491 531.579861899861
0.000491 531.579241388973
0.000492 530.577208926169
0.000492 530.576603632288
0.000493 529.569398263222
0.000493 529.568808300982
0.000494 528.556427727454
0.000494 528.555853212232
0.000495 527.538295311971
0.000495 527.537736359883
0.000496 526.514999190514
0.000496 526.514455918405
0.000497 525.486537721446
0.000497 525.486010246889
0.000498 524.452909451776
0.000498 524.452397893061
0.000499 523.414113121211
0.000499 523.413617597342
0.0005 522.370147666233
0.0005 522.369668296918
0.000501 521.321012224214
0.000501 521.320549129857
0.000502 520.266706137546
0.000502 520.266259439242
0.000503 519.207228957816
0.000503 519.206798777341
0.000504 518.142580449994
0.000504 518.142166909796
0.000505 517.072532322057
0.000505 517.072135058532
0.000506 515.995364499229
0.000506 515.994986892968
0.000507 514.913033577225
0.000507 514.912673021696
0.000508 513.825537673325
0.000508 513.825194293242
0.000509 512.732877871292
0.000509 512.732551791983
0.00051 511.635055495083
0.00051 511.634746842483
0.000511 510.532072113212
0.000511 510.531781013848
0.000512 509.423929543133
0.000512 509.423656124114
0.000513 508.310629855645
0.000513 508.310374244651
0.000514 507.192175379319
0.000514 507.19193770459
0.000515 506.06856870495
0.000515 506.068349095268
0.000516 504.939812690023
0.000516 504.939611274708
0.000517 503.805910463208
0.000517 503.805727372096
0.000518 502.666865428861
0.000518 502.666700792296
0.000519 501.522681271559
0.000519 501.522535220377
0.00052 500.37336196064
0.00052 500.373234626156
0.000521 499.218911754765
0.000521 499.218803268756
0.000522 498.059335206497
0.000522 498.059245701188
0.000523 496.893089391187
0.000523 496.893021076389
0.000524 495.718926443111
0.000524 495.718881709793
0.000525 494.539672027295
0.000525 494.539646742849
0.000526 493.355328051649
0.000526 493.355322348935
0.000527 492.165901007756
0.000527 492.165915019977
0.000528 490.971397710361
0.000528 490.971431571038
0.000529 489.77182530197
0.000529 489.771879144929
0.00053 488.567191257468
0.00053 488.567265216819
0.000531 487.357503388733
0.000531 487.357597598849
0.000532 486.142769849256
0.000532 486.142884444757
0.000533 484.922999138767
0.000533 484.923134254494
0.000534 483.698200107856
0.000534 483.698355878853
0.000535 482.466321066748
0.000535 482.466501223732
0.000536 481.229353554327
0.000536 481.229554814105
0.000537 479.987395986366
0.000537 479.987618337394
0.000538 478.74045872023
0.000538 478.740702297529
0.000539 477.488552634168
0.000539 477.488817572836
0.00054 476.231688985422
0.00054 476.231975420609
0.000541 474.969879414783
0.000541 474.970187481671
0.000542 473.70058980669
0.000542 473.700924943853
0.000543 472.424703479156
0.000543 472.425063383895
0.000544 471.143932174876
0.000544 471.144314169935
0.000545 469.858286709504
0.000545 469.858690928549
0.000546 468.56778111635
0.000546 468.56820769292
0.000547 467.270643383314
0.000547 467.27109619124
0.000548 465.96838788722
0.000548 465.968863854115
0.000549 464.66132851342
0.000549 464.661827261138
0.00055 463.342404157919
0.00055 463.34294127874
0.000551 462.017605117387
0.000551 462.018167380047
0.000552 460.68636046908
0.000552 460.686949911312
0.000553 459.491424899115
0.000553 459.491644781969
0.000554 458.660960319054
0.000554 458.660768290694
0.000555 457.664772767597
0.000555 457.664805455214
0.000556 456.663135967921
0.000556 456.663186259842
0.000557 455.655826983955
0.000557 455.655895080467
0.000558 454.642830372813
0.000558 454.642916476285
0.000559 453.624130882049
0.000559 453.624235196941
0.00056 452.599713456928
0.00056 452.599836189797
0.000561 451.569563247834
0.000561 451.569704607336
0.000562 450.533665617793
0.000562 450.533825814693
0.000563 449.492006150142
0.000563 449.492185397317
0.000564 448.444570656326
0.000564 448.444769168769
0.000565 447.39134518383
0.000565 447.391563178652
0.000566 446.332316024243
0.000566 446.332553720677
0.000567 445.267469721462
0.000567 445.267727340865
0.000568 444.19679308003
0.000568 444.197070845884
0.000569 443.120273173616
0.000569 443.12057131153
0.00057 442.037897353628
0.00057 442.038216091334
0.000571 440.949653257966
0.000571 440.949992825322
0.000572 439.855528819919
0.000572 439.855889448908
0.000573 438.755512277201
0.000573 438.755894201922
0.000574 437.649592181123
0.000574 437.649995637794
0.000575 436.537757405911
0.000575 436.538182632863
0.000576 435.419997158166
0.000576 435.420444395836
0.000577 434.296300986468
0.000577 434.296770477393
0.000578 433.166658791111
0.000578 433.167150779922
0.000579 432.031060833996
0.000579 432.031575567409
0.00058 430.889497748658
0.00058 430.890035475465
0.000581 429.741960550434
0.000581 429.74252152149
0.000582 428.588440646777
0.000582 428.589025114992
0.000583 427.427139992413
0.000583 427.427753292076
0.000584 426.25722683551
0.000584 426.257868319831
0.000585 425.081322069418
0.000585 425.081987920176
0.000586 423.899415249585
0.000586 423.900105726865
0.000587 422.711500460225
0.000587 422.712215826041
0.000588 421.517572249087
0.000588 421.518312767353
0.000589 420.317625638706
0.000589 420.318391575224
0.00059 419.109946665635
0.00059 419.110744482035
0.000591 417.892742182948
0.000591 417.893571794548
0.000592 416.669531079146
0.000592 416.670387024303
0.000593 415.440305350322
0.000593 415.441187900578
0.000594 414.20506326371
0.000594 414.205972692287
0.000595 412.963597495057
0.000595 412.96453829158
0.000596 411.7111860676
0.000596 411.712162743993
0.000597 410.452789458094
0.000597 410.453793932412
0.000598 409.188400535113
0.000598 409.189433084983
0.000599 407.918021130549
0.000599 407.919082035033
0.0006 406.641653693917
};
\end{groupplot}

\end{tikzpicture}

\end{figure}

\sloppy\subsection{График зависимости температуры от времени при нескольких фиксированных значениях координаты}
На рисунке представлены графики зависимости температуры от времени при фиксированных $x = 0, 0.2, 0.4, \ldots, 3.2$.
\begin{figure}[H]
    \caption{Зависимость температуры от времени}
    % This file was created by tikzplotlib v0.9.2.
\begin{tikzpicture}[scale=1.75]

\definecolor{color0}{rgb}{0.12156862745098,0.466666666666667,0.705882352941177}
\definecolor{color1}{rgb}{1,0.498039215686275,0.0549019607843137}
\definecolor{color2}{rgb}{0.172549019607843,0.627450980392157,0.172549019607843}
\definecolor{color3}{rgb}{0.83921568627451,0.152941176470588,0.156862745098039}
\definecolor{color4}{rgb}{0.580392156862745,0.403921568627451,0.741176470588235}
\definecolor{color5}{rgb}{0.549019607843137,0.337254901960784,0.294117647058824}
\definecolor{color6}{rgb}{0.890196078431372,0.466666666666667,0.76078431372549}
\definecolor{color7}{rgb}{0.737254901960784,0.741176470588235,0.133333333333333}
\definecolor{color8}{rgb}{0.0901960784313725,0.745098039215686,0.811764705882353}

\begin{axis}[
    x label style={at={(axis description cs:0.5,-0.05)},anchor=north},
    y label style={at={(axis description cs:-0.005,.5)},rotate=0,anchor=south},
    tick label style={font=\scriptsize},
tick align=outside,
tick pos=left,
x grid style={white!69.0196078431373!black},
xlabel={\footnotesize Время, c},
xmin=-0.9, xmax=18.9,
xtick style={color=black},
y grid style={white!69.0196078431373!black},
ylabel={\footnotesize Температура, K},
ymin=263.698013348176, ymax=1062.34171968815,
ytick style={color=black}
]
\addplot [semithick, color0]
table {%
0 300
1 599.930543206081
2 691.9724624112
3 755.219922274379
4 802.585678547953
5 839.990634877678
6 870.491502282562
7 895.903060044576
8 917.403896652301
9 935.808788017152
10 951.707409780728
11 965.541691793125
12 977.652040966931
13 988.30650388706
14 997.719990901908
15 1006.06741519652
16 1013.49294918595
17 1020.11671568309
18 1026.03973303634
};
\addplot [semithick, color1]
table {%
0 300
1 333.394658561331
2 373.621175240765
3 411.607515582236
4 445.372628097261
5 474.865027797009
6 500.532928665687
7 522.903946782287
8 542.463832064905
9 559.629009203146
10 574.74807143863
11 588.111018799887
12 599.959371789716
13 610.495128531151
14 619.888160968252
15 628.282161800075
16 635.799383793026
17 642.544414154532
18 648.607190570679
};
\addplot [semithick, color2]
table {%
0 300
1 303.748155297563
2 311.837753721511
3 322.88949514572
4 335.428999399817
5 348.423411651856
6 361.244372935233
7 373.53798589785
8 385.121248827281
9 395.914212973978
10 405.897875124974
11 415.088441535676
12 423.521657315748
13 431.24328137508
14 438.303322283714
15 444.752587470037
16 450.640662431259
17 456.014776662865
18 460.91921931379
};
\addplot [semithick, color3]
table {%
0 300
1 300.423637009955
2 301.754098259009
3 304.195142744808
4 307.669333593863
5 311.954261235521
6 316.795947421975
7 321.96631863137
8 327.282093776014
9 332.605048411659
10 337.835465279259
11 342.904212149289
12 347.765525586863
13 352.391084430106
14 356.765372344197
15 360.88213939271
16 364.741740056593
17 368.349147764878
18 371.712482827168
};
\addplot [semithick, color4]
table {%
0 300
1 300.048176909111
2 300.247178965463
3 300.708916019098
4 301.509810345145
5 302.674558766991
6 304.183404153429
7 305.988412528113
8 308.028860705083
9 310.242114190161
10 312.569973073417
11 314.961684614216
12 317.374869883059
13 319.775294668952
14 322.136082089848
15 324.436716251647
16 326.66202564864
17 328.80123893999
18 330.847151480283
};
\addplot [semithick, color5]
table {%
0 300
1 300.005508692328
2 300.033735156366
3 300.113107857459
4 300.276253870814
5 300.551552015656
6 300.957456062218
7 301.500724772732
8 302.177610051874
9 302.976499554865
10 303.880829309814
11 304.871611467812
12 305.929340539841
13 307.035282316754
14 308.172253209834
15 309.325021458159
16 310.480448475988
17 311.62746339775
18 312.756938484995
};
\addplot [semithick, color6]
table {%
0 300
1 300.000632947548
2 300.004506352159
3 300.017314923332
4 300.047811587871
5 300.1065686021
6 300.204149303033
7 300.3494204995
8 300.548452558754
9 300.804106182089
10 301.116176561166
11 301.481883409194
12 301.896510343444
13 302.354052686887
14 302.84779183658
15 303.370760463483
16 303.916092684948
17 304.477269671099
18 305.048278128466
};
\addplot [semithick, white!49.8039215686275!black]
table {%
0 300
1 300.000073043363
2 300.000592912477
3 300.002570284441
4 300.007927287313
5 300.019548710994
6 300.041063794374
7 300.076433174691
8 300.129472428116
9 300.203432294587
10 300.300707679604
11 300.422695786813
12 300.569786682182
13 300.74145131431
14 300.936388585274
15 301.15269831369
16 301.388055619189
17 301.63987105218
18 301.905428073458
};
\addplot [semithick, color7]
table {%
0 300
1 300.000009424732
2 300.000085451141
3 300.000410611901
4 300.001393451331
5 300.003754304339
6 300.008558428194
7 300.017178503708
8 300.031193763164
9 300.052250814681
10 300.081916721543
11 300.121550197918
12 300.172206987134
13 300.234585243203
14 300.309008790416
15 300.39544137121
16 300.493523088033
17 300.602620368568
18 300.721882061609
};
\addplot [semithick, color8]
table {%
0 300
1 300.000001095637
2 300.000011030567
3 300.000058483705
4 300.000217637626
5 300.000639169485
6 300.001579190726
7 300.003416682157
8 300.006652933153
9 300.011891582925
10 300.019802732067
11 300.031077655875
12 300.046381473912
13 300.066310153056
14 300.091356249845
15 300.121885597125
16 300.158125230584
17 300.200161480095
18 300.247946354757
};
\addplot [semithick, color0]
table {%
0 300
1 300.000000127781
2 300.000001414624
3 300.000008205117
4 300.000033236718
5 300.000105735827
6 300.000281656647
7 300.000654040095
8 300.001360963928
9 300.002588897624
10 300.004570269784
11 300.007575289226
12 300.01189913321
13 300.017846282818
14 300.025713987382
15 300.035776655747
16 300.048272546157
17 300.063393605829
18 300.08127881056
};
\addplot [semithick, color1]
table {%
0 300
1 300.000000014943
2 300.000000180487
3 300.000001137038
4 300.000004981881
5 300.00001707383
6 300.000048804499
7 300.000121151036
8 300.00026851188
9 300.000542126315
10 300.001012364649
11 300.001769345747
12 300.002921627709
13 300.004593037093
14 300.006917974041
15 300.010035707484
16 300.01408424304
17 300.019194319523
18 300.025483995864
};
\addplot [semithick, color2]
table {%
0 300
1 300.000000001569
2 300.000000020678
3 300.000000141175
4 300.000000667762
5 300.000002461801
6 300.000007543417
7 300.000020005856
8 300.000047216726
9 300.000101196155
10 300.000199991421
11 300.000368833643
12 300.000640874176
13 300.001057351238
14 300.001667115872
15 300.002525531442
16 300.003692835969
17 300.005232110343
18 300.007207023387
};
\addplot [semithick, color3]
table {%
0 300
1 300.000000000178
2 300.000000002611
3 300.000000019182
4 300.000000097151
5 300.000000382279
6 300.000001246696
7 300.000003509274
8 300.000008767194
9 300.000019838339
10 300.00004128929
11 300.00007999953
12 300.000145695377
13 300.000251381613
14 300.000413605889
15 300.000652508645
16 300.000991635493
17 300.001457515163
18 300.002079029548
};
\addplot [semithick, color4]
table {%
0 300
1 300.000000000015
2 300.000000000323
3 300.000000002582
4 300.000000013968
5 300.000000058458
6 300.000000202227
7 300.000000602344
8 300.000001588624
9 300.000003786269
10 300.000008281914
11 300.000016828244
12 300.000032074254
13 300.000057801122
14 300.00009913927
15 300.000162741324
16 300.000256888487
17 300.000391513759
18 300.00057813321
};
\addplot [semithick, color5]
table {%
0 300
1 299.999999999995
2 300.000000000035
3 300.000000000341
4 300.000000001983
5 300.000000008816
6 300.000000032263
7 300.000000101415
8 300.000000281669
9 300.000000705526
10 300.000001618692
11 300.000003443339
12 300.000006858116
13 300.000012891889
14 300.000023025445
15 300.00003929324
16 300.000064375988
17 300.00010167474
18 300.000155358162
};
\addplot [semithick, color6]
table {%
0 300
1 299.999999999993
2 299.999999999997
3 300.000000000039
4 300.000000000273
5 300.000000001308
6 300.000000005067
7 300.000000016779
8 300.000000048974
9 300.000000128661
10 300.000000309052
11 300.000000687129
12 300.000001428023
13 300.000002796544
14 300.000005195316
15 300.000009208003
16 300.000015645212
17 300.000025589865
18 300.000040438506
};
\end{axis}

\end{tikzpicture}

\end{figure}

