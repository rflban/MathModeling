\chapter{Практическая часть}

\section{Реализация}
\lstset{language=python}
\begin{lstlisting}[caption={Методы Рунге-Кутта второго и четвёртого порядков точности},label={lst:rungekutta}]
def RungeKutta2Order(x0, y0, z0, h):
    alpha = 0.5
    nh = h / (2 * alpha)

    k1 = F1(x0, y0, z0)
    g1 = F2(x0, y0, z0)
    k2 = F1(x0 + nh, y0 + nh * k1, z0 + nh * g1)
    g2 = F2(x0 + nh, y0 + nh * k1, z0 + nh * g1)

    y1 = y0 + h * ((1 - alpha) * k1 + alpha * k2)
    z1 = z0 + h * ((1 - alpha) * g1 + alpha * g2)

    return y1, z1


def RungeKutta4Order(xn, yn, zn, hn):
    k1 = hn * F1(xn, yn, zn)
    g1 = hn * F2(xn, yn, zn)

    k2 = hn * F1(xn + hn / 2, yn + k1 / 2, zn + g1 / 2)
    g2 = hn * F2(xn + hn / 2, yn + k1 / 2, zn + g1 / 2)

    k3 = hn * F1(xn + hn / 2, yn + k2 / 2, zn + g2 / 2)
    g3 = hn * F2(xn + hn / 2, yn + k2 / 2, zn + g2 / 2)

    k4 = hn * F1(xn + hn, yn + k3, zn + g3)
    g4 = hn * F2(xn + hn, yn + k3, zn + g3)

    yn_1 = yn + (k1 + 2 * k2 + 2 * k3 + k4) / 6
    zn_1 = zn + (g1 + 2 * g2 + 2 * g3 + g4) / 6

    return yn_1, zn_1
\end{lstlisting}
\begin{lstlisting}[caption={Интерполяция},label={lst:interpolation}]
def interpolate(table, x):
    x1 = y1 = 0
    x2 = y2 = 0

    flag = False
    for i in range(len(table) - 1):
        if (table[i][0] <= x <= table[i + 1][0]):
            flag = True

            y1 = table[i][1]
            y2 = table[i + 1][1]
            x1 = table[i][0]
            x2 = table[i + 1][0]

    if (not flag):
        if (x < table[0][0]):
            return table[0][1]
        if (x> table[-1][0]):
            return table[-1][1]

    return y1 + ((x- x1) / (x2 - x1)) * (y2 - y1)
\end{lstlisting}
\begin{lstlisting}[caption={Метод Симпосона},label={lst:simpson}]
def simpson(I, a=0, b=1, n=50):
    res = 0
    h = (b - a) / n

    for i in range(n):
        x1 = a + h * (i)
        x2 = a + h * (i + 1)

        res += (x2 - x1) / 6.0 * (Sigma(I, x1) +
               4.0 * Sigma(I, 0.5 * (x1 + x2)) + Sigma(I, x2))

    return res
\end{lstlisting}
\begin{lstlisting}[caption={Вычисление параметров},label={lst:params}]
def Tz(T0, m, r):
    return (params['Tw'] - T0) * math.pow(r, m) + T0


def Sigma(I, z):
    T0 = interpolate(IT, I)
    m  = interpolate(IM, I)

    return z * interpolate(TS, Tz(T0, m, z))


def Rp(I):
    return params['Le'] / (2 * math.pi * params['R']**2 * simpson(I))


def F1(t, I, U):
    return ((U - (params['Rk'] + Rp(I)) * I) / params['Lk'])


def F2(t, I, U):
    return -1 / params['Ck'] * I
\end{lstlisting}

\section{Примеры работы}
Графики зависимости от времени импульса
$I(t)$, $U(t)$, $R_p(t)$, $I(t) \cdot R_p(t)$, $T_0(t)$
при
\begin{itemize}
    \item $R   =$ 0.35 см;
    \item $L_{\text{э}}  =$ 12 см;
    \item $L_k  =$ 187 10--6 Гн;
    \item $C_k  =$ 268 10--6 Ф;
    \item $R_k  =$ 0.25 Ом;
    \item $U_0 =$ 1400 В;
    \item $I_0  =$ 0.3 A;
    \item $T_w  =$ 2000 K.
\end{itemize}

\begin{figure}[H]
    \caption{Метод Рунге-Кутта второго порядка}\label{img:plot01}
    % This file was created by tikzplotlib v0.9.2.
\begin{tikzpicture}[scale=0.875]

\definecolor{color0}{rgb}{0.83921568627451,0.152941176470588,0.156862745098039}

\begin{groupplot}[group style={group size=2 by 3, vertical sep=2.5cm, horizontal sep=2.5cm}]
\nextgroupplot[
    x label style={at={(axis description cs:0.5,-0.05)},anchor=north},
    y label style={at={(axis description cs:-0.005,.5)},rotate=0,anchor=south},
tick align=outside,
tick pos=left,
x grid style={white!69.0196078431373!black},
xlabel={Время},
xmin=-2.995e-05, xmax=0.00062895,
xtick style={color=black},
y grid style={white!69.0196078431373!black},
ylabel={$I$},
ymin=-39.0265960035922, ymax=830.558516075437,
ytick style={color=black}
]
\addplot [semithick, color0]
table {%
0 0.5
1e-06 6.81415305462132
2e-06 13.510922328665
3e-06 20.1818708210294
4e-06 26.7970402968579
5e-06 33.3629781150593
6e-06 39.8843789519145
7e-06 46.3650672954001
8e-06 52.7991924336229
9e-06 59.1704751310961
1e-05 65.4730036997804
1.1e-05 71.7123249881364
1.2e-05 77.8933263894551
1.3e-05 84.019917080479
1.4e-05 90.0952246610569
1.5e-05 96.1222363654308
1.6e-05 102.103441027124
1.7e-05 108.040873737754
1.8e-05 113.936635709444
1.9e-05 119.79283484951
2e-05 125.611275593647
2.1e-05 131.393215588199
2.2e-05 137.13991966128
2.3e-05 142.852636557955
2.4e-05 148.532330524176
2.5e-05 154.17980851507
2.6e-05 159.796021888008
2.7e-05 165.381880440116
2.8e-05 170.938034791426
2.9e-05 176.465525193228
3e-05 181.967705909288
3.1e-05 187.447653685535
3.2e-05 192.906172884135
3.3e-05 198.343821031476
3.4e-05 203.754371876694
3.5e-05 209.12895417711
3.6e-05 214.465380948387
3.7e-05 219.764350439942
3.8e-05 225.026515919966
3.9e-05 230.252485844627
4e-05 235.442901305064
4.1e-05 240.598441503502
4.2e-05 245.719700653368
4.3e-05 250.80718375781
4.4e-05 255.861373993542
4.5e-05 260.882749437068
4.6e-05 265.871779974733
4.7e-05 270.828901368291
4.8e-05 275.754520037197
4.9e-05 280.649025823707
5e-05 285.512792846645
5.1e-05 290.346188268139
5.2e-05 295.14965102139
5.3e-05 299.923622721845
5.4e-05 304.668444902132
5.5e-05 309.38442106926
5.6e-05 314.071843116169
5.7e-05 318.731021678669
5.8e-05 323.362258717018
5.9e-05 327.965837280642
6e-05 332.54203882077
6.1e-05 337.091114451901
6.2e-05 341.613309634771
6.3e-05 346.1088731406
6.4e-05 350.578031237783
6.5e-05 355.020990357557
6.6e-05 359.437949639585
6.7e-05 363.829101245414
6.8e-05 368.194630655657
6.9e-05 372.534716951922
7e-05 376.849533084424
7.1e-05 381.139246126181
7.2e-05 385.404017514584
7.3e-05 389.644003281122
7.4e-05 393.859354269953
7.5e-05 398.050319395914
7.6e-05 402.216461620676
7.7e-05 406.355938658766
7.8e-05 410.467625186938
7.9e-05 414.551744552501
8e-05 418.60848766461
8.1e-05 422.638027510278
8.2e-05 426.640543141027
8.3e-05 430.616212888999
8.4e-05 434.565192849587
8.5e-05 438.487631777464
8.6e-05 442.383685673948
8.7e-05 446.253506920038
8.8e-05 450.097233752107
8.9e-05 453.915001107796
9e-05 457.706940792752
9.1e-05 461.47318158035
9.2e-05 465.213866760858
9.3e-05 468.929140013123
9.4e-05 472.61912479486
9.5e-05 476.283938241974
9.6e-05 479.923694906455
9.7e-05 483.538506836862
9.8e-05 487.128483656035
9.9e-05 490.693732636151
0.0001 494.234358771222
0.000101 497.750464847157
0.000102 501.242158444869
0.000103 504.709547854966
0.000104 508.152732398641
0.000105 511.571806468801
0.000106 514.966862567937
0.000107 518.337991365461
0.000108 521.685281753208
0.000109 525.008820899158
0.00011 528.308694299453
0.000111 531.584985828784
0.000112 534.83777778919
0.000113 538.067150957344
0.000114 541.273184630379
0.000115 544.455956670303
0.000116 547.615543547062
0.000117 550.752020380294
0.000118 553.865460979827
0.000119 556.95594327755
0.00012 560.023555779659
0.000121 563.068380305183
0.000122 566.090493154296
0.000123 569.08997687885
0.000124 572.066905124041
0.000125 575.021342809928
0.000126 577.953353834614
0.000127 580.863001105245
0.000128 583.750346568127
0.000129 586.615451237991
0.00013 589.458375226426
0.000131 592.279177769521
0.000132 595.077917254741
0.000133 597.854651247045
0.000134 600.609436514297
0.000135 603.342329051971
0.000136 606.053384107183
0.000137 608.742656202077
0.000138 611.410199156569
0.000139 614.056066110486
0.00014 616.680309545112
0.000141 619.282981304159
0.000142 621.864132614182
0.000143 624.423814104457
0.000144 626.962075826337
0.000145 629.478973465416
0.000146 631.974562300411
0.000147 634.44889088732
0.000148 636.902006985675
0.000149 639.333957867563
0.00015 641.744790334156
0.000151 644.134550731829
0.000152 646.503286061557
0.000153 648.85104824581
0.000154 651.177887661983
0.000155 653.483848819395
0.000156 655.768975848289
0.000157 658.033312513651
0.000158 660.2769022287
0.000159 662.499788068028
0.00016 664.702012780434
0.000161 666.883618801431
0.000162 669.044648265453
0.000163 671.185143017763
0.000164 673.305144626077
0.000165 675.404694391897
0.000166 677.483833361585
0.000167 679.542602337157
0.000168 681.581041886827
0.000169 683.599192355299
0.00017 685.59709387381
0.000171 687.574786369939
0.000172 689.53230957718
0.000173 691.469703044292
0.000174 693.387006144429
0.000175 695.284258084052
0.000176 697.161497911638
0.000177 699.01876452618
0.000178 700.856096685492
0.000179 702.673533014322
0.00018 704.471112012272
0.000181 706.248872061541
0.000182 708.006851434483
0.000183 709.745088300993
0.000184 711.463620735727
0.000185 713.162486725148
0.000186 714.841724174418
0.000187 716.501370914127
0.000188 718.141466304533
0.000189 719.762051254786
0.00019 721.363165095428
0.000191 722.944848440156
0.000192 724.507141751751
0.000193 726.050082343909
0.000194 727.5737074999
0.000195 729.078054478163
0.000196 730.56316051776
0.000197 732.029062843718
0.000198 733.475798672243
0.000199 734.903405215816
0.0002 736.311919688179
0.000201 737.701379309201
0.000202 739.071821309639
0.000203 740.423282935787
0.000204 741.755801454025
0.000205 743.069414155259
0.000206 744.364158359264
0.000207 745.640071418929
0.000208 746.897190724402
0.000209 748.135553707143
0.00021 749.355197843887
0.000211 750.556173978148
0.000212 751.738663794062
0.000213 752.902842510208
0.000214 754.048753372243
0.000215 755.176434161252
0.000216 756.28592154927
0.000217 757.377250121625
0.000218 758.450454009179
0.000219 759.505566907101
0.00022 760.54262081277
0.000221 761.561649532973
0.000222 762.56268957021
0.000223 763.545775842333
0.000224 764.510940651948
0.000225 765.458216431826
0.000226 766.387635747318
0.000227 767.29923129871
0.000228 768.193037284144
0.000229 769.069088556436
0.00023 769.927418745112
0.000231 770.768061022569
0.000232 771.591048707601
0.000233 772.396415267431
0.000234 773.184194319682
0.000235 773.954419634298
0.000236 774.707125135425
0.000237 775.442344903237
0.000238 776.160113175716
0.000239 776.860464350385
0.00024 777.543432986006
0.000241 778.209056274008
0.000242 778.857371882493
0.000243 779.488415172466
0.000244 780.102221194601
0.000245 780.69882516955
0.000246 781.278262489376
0.000247 781.840568718945
0.000248 782.385779597275
0.000249 782.913931038852
0.00025 783.425059134908
0.000251 783.919200154661
0.000252 784.396390546522
0.000253 784.856666939262
0.000254 785.300066143155
0.000255 785.726625151071
0.000256 786.136381139554
0.000257 786.529371469853
0.000258 786.905633688931
0.000259 787.265205530434
0.00026 787.608124915636
0.000261 787.934429954354
0.000262 788.244158945824
0.000263 788.537350379561
0.000264 788.814042936182
0.000265 789.074275488197
0.000266 789.318087100787
0.000267 789.545517032539
0.000268 789.756604736163
0.000269 789.951389859181
0.00027 790.129912244587
0.000271 790.292211931488
0.000272 790.438329155716
0.000273 790.568304350411
0.000274 790.682178146591
0.000275 790.779991373687
0.000276 790.861785418052
0.000277 790.927601804492
0.000278 790.97748190022
0.000279 791.01146716724
0.00028 791.029599269209
0.000281 791.031920071844
0.000282 791.018471643298
0.000283 790.989296254519
0.000284 790.944436379591
0.000285 790.883934696055
0.000286 790.807834085205
0.000287 790.716178028626
0.000288 790.609010408368
0.000289 790.48637491352
0.00029 790.348315399534
0.000291 790.194875925987
0.000292 790.026100756773
0.000293 789.842034360274
0.000294 789.64272140952
0.000295 789.428206782326
0.000296 789.198535561411
0.000297 788.953753034508
0.000298 788.693904694449
0.000299 788.419036239241
0.0003 788.129193572118
0.000301 787.824422801589
0.000302 787.504770241458
0.000303 787.170282410837
0.000304 786.821006034144
0.000305 786.45698804108
0.000306 786.078275566596
0.000307 785.684915950849
0.000308 785.276956739135
0.000309 784.85444568182
0.00031 784.417430734243
0.000311 783.965960056619
0.000312 783.50008201392
0.000313 783.019845175747
0.000314 782.525298316188
0.000315 782.016490413663
0.000316 781.493470650756
0.000317 780.956288414036
0.000318 780.404993293868
0.000319 779.839635084203
0.00032 779.260263782369
0.000321 778.666929588839
0.000322 778.059682906994
0.000323 777.438575250101
0.000324 776.8036602304
0.000325 776.154990746054
0.000326 775.492618117547
0.000327 774.816593866248
0.000328 774.126969714098
0.000329 773.423797583302
0.00033 772.707129595998
0.000331 771.977018073922
0.000332 771.233515538064
0.000333 770.476674708305
0.000334 769.706548503058
0.000335 768.923190038887
0.000336 768.126652630123
0.000337 767.316990337037
0.000338 766.494258669216
0.000339 765.658512778521
0.00034 764.80980681554
0.000341 763.948195126223
0.000342 763.073732251435
0.000343 762.18647292651
0.000344 761.286472080789
0.000345 760.373788281219
0.000346 759.448480846721
0.000347 758.510605844477
0.000348 757.5602190676
0.000349 756.597377635116
0.00035 755.622139812768
0.000351 754.634562917419
0.000352 753.634707567271
0.000353 752.622635611595
0.000354 751.598408119451
0.000355 750.562092293956
0.000356 749.513867376674
0.000357 748.453921021028
0.000358 747.382326486004
0.000359 746.299144256784
0.00036 745.204434977972
0.000361 744.098259452882
0.000362 742.980678642814
0.000363 741.851753666315
0.000364 740.711545798438
0.000365 739.560116469987
0.000366 738.397527266752
0.000367 737.223839928737
0.000368 736.039116349376
0.000369 734.843418574739
0.00037 733.636808802733
0.000371 732.419349382289
0.000372 731.191102812541
0.000373 729.952131741996
0.000374 728.702498967692
0.000375 727.442267434354
0.000376 726.171500233532
0.000377 724.890260602734
0.000378 723.59861192455
0.000379 722.296617725767
0.00038 720.984342350919
0.000381 719.661852245842
0.000382 718.329213312278
0.000383 716.986490392014
0.000384 715.633750456672
0.000385 714.271059883989
0.000386 712.898483235156
0.000387 711.51608520722
0.000388 710.123930632113
0.000389 708.722084475668
0.00039 707.310611836626
0.000391 705.88957794564
0.000392 704.459048164256
0.000393 703.019087983906
0.000394 701.569763024867
0.000395 700.111139035233
0.000396 698.643281889865
0.000397 697.166257589333
0.000398 695.680132258852
0.000399 694.184972147211
0.0004 692.680843625683
0.000401 691.167813186938
0.000402 689.645947443935
0.000403 688.115313128816
0.000404 686.57597709178
0.000405 685.028006299957
0.000406 683.471467836263
0.000407 681.906428898255
0.000408 680.33295679697
0.000409 678.75111895576
0.00041 677.160982909107
0.000411 675.562616301444
0.000412 673.956086885951
0.000413 672.341462523353
0.000414 670.7188111807
0.000415 669.088200930144
0.000416 667.4496999477
0.000417 665.803376512001
0.000418 664.149299003044
0.000419 662.487535900921
0.00042 660.818155784545
0.000421 659.141227330364
0.000422 657.456819311065
0.000423 655.765000594267
0.000424 654.065840141204
0.000425 652.359407005403
0.000426 650.645770331337
0.000427 648.924999353089
0.000428 647.197163392984
0.000429 645.462333658929
0.00043 643.720584452906
0.000431 641.971988344365
0.000432 640.216615005506
0.000433 638.454534189024
0.000434 636.685815726676
0.000435 634.910529527846
0.000436 633.128745578095
0.000437 631.340533937697
0.000438 629.545964740169
0.000439 627.745108190786
0.00044 625.938037478293
0.000441 624.124827592022
0.000442 622.305550665296
0.000443 620.480277178658
0.000444 618.649077676019
0.000445 616.812022763103
0.000446 614.969183105872
0.000447 613.120629428946
0.000448 611.266432514006
0.000449 609.406663198189
0.00045 607.541392372469
0.000451 605.670690980018
0.000452 603.794630014571
0.000453 601.913280518761
0.000454 600.026713582453
0.000455 598.135000341057
0.000456 596.238211973836
0.000457 594.336419702194
0.000458 592.429694787952
0.000459 590.518108531618
0.00046 588.601732270629
0.000461 586.680637377597
0.000462 584.754895258527
0.000463 582.824577351025
0.000464 580.889755122502
0.000465 578.950500068345
0.000466 577.006883710095
0.000467 575.058977593592
0.000468 573.106853287119
0.000469 571.150582379526
0.00047 569.190236478338
0.000471 567.225887207853
0.000472 565.25760620722
0.000473 563.285471685733
0.000474 561.309565318396
0.000475 559.32996220736
0.000476 557.346733997792
0.000477 555.35995600726
0.000478 553.369711102834
0.000479 551.376078425607
0.00048 549.379129578672
0.000481 547.378936158006
0.000482 545.375569750379
0.000483 543.369101931243
0.000484 541.35960426261
0.000485 539.34714829091
0.000486 537.331805544832
0.000487 535.313647533153
0.000488 533.292745742548
0.000489 531.269171635382
0.00049 529.242996647487
0.000491 527.214292185923
0.000492 525.183129626721
0.000493 523.149580312609
0.000494 521.113715550721
0.000495 519.075606610289
0.000496 517.035324720317
0.000497 514.992941067243
0.000498 512.948526792571
0.000499 510.9021529905
0.0005 508.853890705528
0.000501 506.803810930036
0.000502 504.75198460186
0.000503 502.698482601842
0.000504 500.643375751364
0.000505 498.586735420222
0.000506 496.528638123465
0.000507 494.469159654459
0.000508 492.408370529274
0.000509 490.346341190741
0.00051 488.283142005871
0.000511 486.218843263262
0.000512 484.153515170481
0.000513 482.08722785143
0.000514 480.020051343696
0.000515 477.952055595877
0.000516 475.883310464895
0.000517 473.813885713288
0.000518 471.743851006484
0.000519 469.673275910057
0.00052 467.602229886959
0.000521 465.530782294746
0.000522 463.459002382769
0.000523 461.3869634278
0.000524 459.314746064858
0.000525 457.242426601169
0.000526 455.170073711131
0.000527 453.097755944221
0.000528 451.025541722073
0.000529 448.953499335551
0.00053 446.881696941792
0.000531 444.810202561244
0.000532 442.739084074672
0.000533 440.668409220158
0.000534 438.598245590079
0.000535 436.528666138476
0.000536 434.459743854229
0.000537 432.391546053276
0.000538 430.324139647953
0.000539 428.257591388281
0.00054 426.191967858845
0.000541 424.127335475645
0.000542 422.063767290814
0.000543 420.001340628406
0.000544 417.94012577155
0.000545 415.880188286611
0.000546 413.821593552979
0.000547 411.764411536446
0.000548 409.70871275375
0.000549 407.654562733361
0.00055 405.602044919927
0.000551 403.551245409493
0.000552 401.502235756726
0.000553 399.454707171718
0.000554 397.407361063272
0.000555 395.359708007014
0.000556 393.312244636008
0.000557 391.265041794368
0.000558 389.218170309375
0.000559 387.171700989089
0.00056 385.125704619926
0.000561 383.080251964197
0.000562 381.035413757616
0.000563 378.991260706765
0.000564 376.947863486527
0.000565 374.905292737477
0.000566 372.863619063238
0.000567 370.822913027796
0.000568 368.783245152773
0.000569 366.744685914664
0.00057 364.70730574203
0.000571 362.671175012651
0.000572 360.636364050632
0.000573 358.602943123474
0.000574 356.570982439098
0.000575 354.540552142816
0.000576 352.511722314275
0.000577 350.484562964337
0.000578 348.459144031929
0.000579 346.43553538083
0.00058 344.413806796427
0.000581 342.394027982409
0.000582 340.376268557423
0.000583 338.360602837381
0.000584 336.347112008755
0.000585 334.3358723354
0.000586 332.326952972352
0.000587 330.3204229688
0.000588 328.316351264409
0.000589 326.314806685582
0.00059 324.315862512452
0.000591 322.319601230072
0.000592 320.32610055663
0.000593 318.335428690883
0.000594 316.347653694763
0.000595 314.362844040472
0.000596 312.381081217553
0.000597 310.402445891412
0.000598 308.42700536431
0.000599 306.454826776704
};

\nextgroupplot[
    x label style={at={(axis description cs:0.5,-0.05)},anchor=north},
    y label style={at={(axis description cs:-0.005,.5)},rotate=0,anchor=south},
tick align=outside,
tick pos=left,
x grid style={white!69.0196078431373!black},
xlabel={Время},
xmin=-2.995e-05, xmax=0.00062895,
xtick style={color=black},
y grid style={white!69.0196078431373!black},
ylabel={$U$},
ymin=51.8468447060338, ymax=1464.19776929971,
ytick style={color=black}
,ytick={250, 500, ..., 1250}
]
\addplot [semithick, color0]
table {%
0 1400
1e-06 1399.98704108981
2e-06 1399.94912917103
3e-06 1399.88621351832
4e-06 1399.79851785438
5e-06 1399.68623537965
6e-06 1399.54954023774
7e-06 1399.38859063703
8e-06 1399.20353218967
9e-06 1398.99456634499
1e-05 1398.76196087896
1.1e-05 1398.50596135265
1.2e-05 1398.22679386515
1.3e-05 1397.92466753899
1.4e-05 1397.59977927143
1.5e-05 1397.25231487641
1.6e-05 1396.88244894038
1.7e-05 1396.4903487175
1.8e-05 1396.07617361086
1.9e-05 1395.64007518392
2e-05 1395.18219712249
2.1e-05 1394.70267751115
2.2e-05 1394.20165065424
2.3e-05 1393.67924520629
2.4e-05 1393.1355861441
2.5e-05 1392.57079492745
2.6e-05 1391.98499051391
2.7e-05 1391.37828726763
2.8e-05 1390.7507973471
2.9e-05 1390.10263027017
3e-05 1389.43389044966
3.1e-05 1388.7446623471
3.2e-05 1388.03502739883
3.3e-05 1387.30506408253
3.4e-05 1386.55484967208
3.5e-05 1385.78450896814
3.6e-05 1384.99418570693
3.7e-05 1384.18402093004
3.8e-05 1383.35415312705
3.9e-05 1382.50471846459
4e-05 1381.63585089175
4.1e-05 1380.74768179316
4.2e-05 1379.84034004545
4.3e-05 1378.9139525997
4.4e-05 1377.96864456458
4.5e-05 1377.00453928542
4.6e-05 1376.02175830365
4.7e-05 1375.02042145714
4.8e-05 1374.00064703563
4.9e-05 1372.96255184414
5e-05 1371.90625126316
5.1e-05 1370.83185930579
5.2e-05 1369.73948861336
5.3e-05 1368.62924992309
5.4e-05 1367.50125257888
5.5e-05 1366.3556047702
5.6e-05 1365.19241357621
5.7e-05 1364.01178500808
5.8e-05 1362.81382382743
5.9e-05 1361.59863379077
6e-05 1360.36631754853
6.1e-05 1359.11697675589
6.2e-05 1357.85071218964
6.3e-05 1356.56762367368
6.4e-05 1355.26781012758
6.5e-05 1353.95136968687
6.6e-05 1352.61839973072
6.7e-05 1351.26899690855
6.8e-05 1349.90325716546
6.9e-05 1348.5212757667
7e-05 1347.12314732096
7.1e-05 1345.70896580281
7.2e-05 1344.27882457413
7.3e-05 1342.83281640472
7.4e-05 1341.37103349207
7.5e-05 1339.89356748028
7.6e-05 1338.40050872016
7.7e-05 1336.89195205864
7.8e-05 1335.36800166802
7.9e-05 1333.8287607806
8e-05 1332.27433188852
8.1e-05 1330.70481679981
8.2e-05 1329.12031671577
8.3e-05 1327.52093210875
8.4e-05 1325.90676284837
8.5e-05 1324.27790824246
8.6e-05 1322.63446705056
8.7e-05 1320.97653741654
8.8e-05 1319.30421696233
8.9e-05 1317.61760280041
9e-05 1315.9167915457
9.1e-05 1314.20187932706
9.2e-05 1312.47296179842
9.3e-05 1310.73013401972
9.4e-05 1308.97349057188
9.5e-05 1307.20312559385
9.6e-05 1305.41913279242
9.7e-05 1303.62160545171
9.8e-05 1301.81063644232
9.9e-05 1299.98631823027
0.0001 1298.14874288563
0.000101 1296.29800209089
0.000102 1294.43418714906
0.000103 1292.55738893996
0.000104 1290.66769795813
0.000105 1288.76520434232
0.000106 1286.84999788268
0.000107 1284.92216802763
0.000108 1282.98180389068
0.000109 1281.02899425689
0.00011 1279.06382758926
0.000111 1277.08639203483
0.000112 1275.09677543064
0.000113 1273.09506530956
0.000114 1271.08134890582
0.000115 1269.05571316051
0.000116 1267.01824472681
0.000117 1264.96902997513
0.000118 1262.90815499806
0.000119 1260.83570561519
0.00012 1258.75176733781
0.000121 1256.65642532455
0.000122 1254.5497644762
0.000123 1252.43186938235
0.000124 1250.30282432727
0.000125 1248.16271335108
0.000126 1246.01162025364
0.000127 1243.84962859825
0.000128 1241.67682171527
0.000129 1239.49328270573
0.00013 1237.29909444468
0.000131 1235.09433958452
0.000132 1232.87910055826
0.000133 1230.65345958257
0.000134 1228.41749866086
0.000135 1226.17129958618
0.000136 1223.91494394405
0.000137 1221.64851311524
0.000138 1219.37208827837
0.000139 1217.08575041257
0.00014 1214.78958029991
0.000141 1212.48365852786
0.000142 1210.16806549161
0.000143 1207.84288139632
0.000144 1205.50818625936
0.000145 1203.16405991238
0.000146 1200.81058195732
0.000147 1198.44783181301
0.000148 1196.07588871918
0.000149 1193.69483173827
0.00015 1191.3047397572
0.000151 1188.90569148915
0.000152 1186.4977654752
0.000153 1184.08104007786
0.000154 1181.65559345055
0.000155 1179.22150357988
0.000156 1176.77884828712
0.000157 1174.32770522958
0.000158 1171.86815190191
0.000159 1169.40026563744
0.00016 1166.92412360939
0.000161 1164.43980283204
0.000162 1161.94738016193
0.000163 1159.44693229892
0.000164 1156.93853578726
0.000165 1154.42226701661
0.000166 1151.898202223
0.000167 1149.36641748977
0.000168 1146.82698874846
0.000169 1144.27999177966
0.00017 1141.72550221384
0.000171 1139.16359553208
0.000172 1136.5943470669
0.000173 1134.01783200287
0.000174 1131.43412537735
0.000175 1128.84330208109
0.000176 1126.24543685887
0.000177 1123.64060431002
0.000178 1121.02887888904
0.000179 1118.41033490602
0.00018 1115.78504652721
0.000181 1113.15308777541
0.000182 1110.51453253041
0.000183 1107.8694545294
0.000184 1105.21792736733
0.000185 1102.56002449725
0.000186 1099.89581923063
0.000187 1097.22538473764
0.000188 1094.54879404742
0.000189 1091.86612003651
0.00019 1089.17743542861
0.000191 1086.48281280686
0.000192 1083.78232459175
0.000193 1081.07604306452
0.000194 1078.36404036722
0.000195 1075.64638850284
0.000196 1072.92315933538
0.000197 1070.19442458994
0.000198 1067.46025585271
0.000199 1064.72072457106
0.0002 1061.97590205348
0.000201 1059.22585946963
0.000202 1056.47066785023
0.000203 1053.71039808708
0.000204 1050.94512093298
0.000205 1048.17490700162
0.000206 1045.39982676748
0.000207 1042.61995056576
0.000208 1039.83534859219
0.000209 1037.04609090293
0.00021 1034.25224741439
0.000211 1031.45388790303
0.000212 1028.65108190984
0.000213 1025.84389784861
0.000214 1023.03240395453
0.000215 1020.21666831958
0.000216 1017.39675889717
0.000217 1014.57274350541
0.000218 1011.74468983979
0.000219 1008.91266546359
0.00022 1006.07673782091
0.000221 1003.23697423624
0.000222 1000.39344190157
0.000223 997.546207868672
0.000224 994.695339069068
0.000225 991.840902313501
0.000226 988.982964291449
0.000227 986.12159157063
0.000228 983.256850596492
0.000229 980.388807681621
0.00023 977.517529011213
0.000231 974.643080646962
0.000232 971.765528526512
0.000233 968.884938462908
0.000234 966.001376144039
0.000235 963.114907132074
0.000236 960.225596862886
0.000237 957.333510645478
0.000238 954.438713661395
0.000239 951.541270964132
0.00024 948.641247478536
0.000241 945.738708000201
0.000242 942.83371717658
0.000243 939.926339521855
0.000244 937.01663941974
0.000245 934.10468112285
0.000246 931.190528752057
0.000247 928.274246295855
0.000248 925.355897609708
0.000249 922.4355464154
0.00025 919.513256300377
0.000251 916.589090717086
0.000252 913.663112982306
0.000253 910.73538627648
0.000254 907.805973643041
0.000255 904.874937987727
0.000256 901.942342077904
0.000257 899.008248541875
0.000258 896.072719868189
0.000259 893.135818404948
0.00026 890.197606359107
0.000261 887.258145795773
0.000262 884.317498637501
0.000263 881.375726663581
0.000264 878.432891509333
0.000265 875.489054665388
0.000266 872.54427747697
0.000267 869.598621143178
0.000268 866.652146716264
0.000269 863.704915100904
0.00027 860.756987053472
0.000271 857.808423181307
0.000272 854.859283941985
0.000273 851.909629642578
0.000274 848.95952043892
0.000275 846.00901633487
0.000276 843.058177181563
0.000277 840.107062674159
0.000278 837.155732353596
0.000279 834.204245606479
0.00028 831.252661664329
0.000281 828.301039602827
0.000282 825.349438341063
0.000283 822.397916640779
0.000284 819.446533105608
0.000285 816.495346180322
0.000286 813.544414150065
0.000287 810.593795139598
0.000288 807.643547109429
0.000289 804.693727857139
0.00029 801.744395017047
0.000291 798.795606059454
0.000292 795.847418289878
0.000293 792.899888848293
0.000294 789.953074708362
0.000295 787.007032676677
0.000296 784.061819391995
0.000297 781.117491324468
0.000298 778.174104774885
0.000299 775.231715873899
0.0003 772.29038058127
0.000301 769.350154685089
0.000302 766.411093801022
0.000303 763.473253371536
0.000304 760.536688665139
0.000305 757.601454775607
0.000306 754.667606621228
0.000307 751.735198944025
0.000308 748.804286308999
0.000309 745.874923103359
0.00031 742.94716353576
0.000311 740.021061635535
0.000312 737.096671251934
0.000313 734.174046053359
0.000314 731.253239526598
0.000315 728.334304976069
0.000316 725.417295523049
0.000317 722.50226410492
0.000318 719.589263474406
0.000319 716.67834619881
0.00032 713.769564659258
0.000321 710.86297104994
0.000322 707.958617377352
0.000323 705.056555459539
0.000324 702.15683691846
0.000325 699.259513172296
0.000326 696.364635448118
0.000327 693.472254781125
0.000328 690.582422013901
0.000329 687.695187795662
0.00033 684.810602581511
0.000331 681.928716631692
0.000332 679.049580010849
0.000333 676.173242587279
0.000334 673.299754032193
0.000335 670.429163818973
0.000336 667.561521222439
0.000337 664.696875318107
0.000338 661.835274977308
0.000339 658.976768861417
0.00034 656.121405430121
0.000341 653.26923294069
0.000342 650.420299447247
0.000343 647.57465280004
0.000344 644.732340644718
0.000345 641.893410421609
0.000346 639.057909339212
0.000347 636.225884395348
0.000348 633.397382380013
0.000349 630.572449874665
0.00035 627.751133242995
0.000351 624.933478631735
0.000352 622.119531976783
0.000353 619.309338972426
0.000354 616.502945093111
0.000355 613.700395577316
0.000356 610.901735397822
0.000357 608.107008454051
0.000358 605.316258321983
0.000359 602.529528352278
0.00036 599.746861669606
0.000361 596.968301172057
0.000362 594.193889530545
0.000363 591.423669188224
0.000364 588.6576823599
0.000365 585.895971031451
0.000366 583.138576959243
0.000367 580.385541669559
0.000368 577.63690645802
0.000369 574.892712389016
0.00037 572.153000295138
0.000371 569.417810776613
0.000372 566.687184200741
0.000373 563.961160701338
0.000374 561.239780178177
0.000375 558.523082296441
0.000376 555.811106486165
0.000377 553.1038919417
0.000378 550.401477621159
0.000379 547.703902245888
0.00038 545.011204299921
0.000381 542.323422024386
0.000382 539.640593407533
0.000383 536.962756198627
0.000384 534.289947902004
0.000385 531.62220576722
0.000386 528.959566803196
0.000387 526.3020677777
0.000388 523.64974521684
0.000389 521.002635404563
0.00039 518.360774382158
0.000391 515.72419794776
0.000392 513.092941655862
0.000393 510.467040816826
0.000394 507.846530496404
0.000395 505.231445515254
0.000396 502.62182044847
0.000397 500.017689625108
0.000398 497.419087127719
0.000399 494.826046791887
0.0004 492.23860220577
0.000401 489.656786709645
0.000402 487.080633395454
0.000403 484.510175106363
0.000404 481.945444436313
0.000405 479.386473729589
0.000406 476.833295080376
0.000407 474.28594033234
0.000408 471.744441078194
0.000409 469.20882865928
0.00041 466.679134165152
0.000411 464.155388433165
0.000412 461.637622048064
0.000413 459.125865341581
0.000414 456.620148392039
0.000415 454.120501023954
0.000416 451.626952807646
0.000417 449.139533058857
0.000418 446.658270838363
0.000419 444.183194951607
0.00042 441.714333948323
0.000421 439.251716122169
0.000422 436.795369510366
0.000423 434.345321893346
0.000424 431.901600794393
0.000425 429.4642334793
0.000426 427.033246956028
0.000427 424.608667974367
0.000428 422.190523025605
0.000429 419.778838342201
0.00043 417.373639884016
0.000431 414.974953329086
0.000432 412.582804095645
0.000433 410.197217341807
0.000434 407.818217965263
0.000435 405.445830602995
0.000436 403.080079630983
0.000437 400.720989163927
0.000438 398.368583054969
0.000439 396.022884895425
0.00044 393.683918014518
0.000441 391.351705457368
0.000442 389.026269993609
0.000443 386.707634130019
0.000444 384.39582011027
0.000445 382.090849914692
0.000446 379.792745260049
0.000447 377.501527599312
0.000448 375.21721812145
0.000449 372.939837751214
0.00045 370.669407148936
0.000451 368.40594671033
0.000452 366.149476566301
0.000453 363.90001658276
0.000454 361.657586360443
0.000455 359.422205234739
0.000456 357.193892275524
0.000457 354.972666286996
0.000458 352.75854580753
0.000459 350.55154910952
0.00046 348.351694199245
0.000461 346.158998816733
0.000462 343.97348043563
0.000463 341.795156263083
0.000464 339.624043239622
0.000465 337.460158039056
0.000466 335.303517068367
0.000467 333.154136467619
0.000468 331.012032109872
0.000469 328.8772196011
0.00047 326.749714280114
0.000471 324.629531218505
0.000472 322.516685220576
0.000473 320.411190823295
0.000474 318.313062247325
0.000475 316.222313420124
0.000476 314.138958001852
0.000477 312.063009385325
0.000478 309.994480668621
0.000479 307.933384626223
0.00048 305.879733765482
0.000481 303.833540326585
0.000482 301.794816282584
0.000483 299.763573339437
0.000484 297.739822936051
0.000485 295.723576244337
0.000486 293.714844169275
0.000487 291.713637348981
0.000488 289.719966154786
0.000489 287.733840691323
0.00049 285.755270796622
0.000491 283.784266042214
0.000492 281.820835733239
0.000493 279.864988908571
0.000494 277.916734340944
0.000495 275.976080537087
0.000496 274.043035737873
0.000497 272.117607918474
0.000498 270.19980478852
0.000499 268.289633792275
0.0005 266.387102108817
0.000501 264.492216652226
0.000502 262.604984071786
0.000503 260.725410752192
0.000504 258.853502813765
0.000505 256.989266112682
0.000506 255.13270623665
0.000507 253.283828470887
0.000508 251.442637837394
0.000509 249.609139095187
0.00051 247.783336740576
0.000511 245.965235007451
0.000512 244.154837867579
0.000513 242.352149030913
0.000514 240.557171945908
0.000515 238.769909799847
0.000516 236.990365519182
0.000517 235.218541769875
0.000518 233.454440957762
0.000519 231.698065228914
0.00052 229.949416470021
0.000521 228.208496308774
0.000522 226.475306114269
0.000523 224.74984699741
0.000524 223.032119780483
0.000525 221.322124972639
0.000526 219.619862826613
0.000527 217.925333339124
0.000528 216.238536251343
0.000529 214.559471049379
0.00053 212.888136964769
0.000531 211.224532974988
0.000532 209.568657803958
0.000533 207.920509922581
0.000534 206.28008754927
0.000535 204.647388650505
0.000536 203.022410900319
0.000537 201.4051517204
0.000538 199.795608282468
0.000539 198.193777508869
0.00054 196.599656073187
0.000541 195.013240400867
0.000542 193.434526669854
0.000543 191.863510760497
0.000544 190.300188273594
0.000545 188.744554564986
0.000546 187.196604746208
0.000547 185.656333685195
0.000548 184.123735971398
0.000549 182.598805946418
0.00055 181.081537710722
0.000551 179.571924983397
0.000552 178.069961222029
0.000553 176.575639611911
0.000554 175.088955910096
0.000555 173.609912995807
0.000556 172.138510290332
0.000557 170.674746953706
0.000558 169.218621881648
0.000559 167.770133705631
0.00056 166.329280792957
0.000561 164.896061246841
0.000562 163.470472906508
0.000563 162.052513347295
0.000564 160.642179880766
0.000565 159.239469554834
0.000566 157.844379153894
0.000567 156.456905198967
0.000568 155.077043947849
0.000569 153.704791395282
0.00057 152.340143273118
0.000571 150.983095050512
0.000572 149.63364193411
0.000573 148.291778868259
0.000574 146.957500535221
0.000575 145.630801355403
0.000576 144.311675487595
0.000577 143.000116829222
0.000578 141.696119016605
0.000579 140.399675425235
0.00058 139.110779170062
0.000581 137.829423105793
0.000582 136.555599827199
0.000583 135.289301669445
0.000584 134.030520672769
0.000585 132.779248566399
0.000586 131.535476821387
0.000587 130.299196650938
0.000588 129.070399010809
0.000589 127.849074599729
0.00059 126.635213859828
0.000591 125.428806943041
0.000592 124.229843675872
0.000593 123.038313630181
0.000594 121.854206123607
0.000595 120.677510220085
0.000596 119.50821472631
0.000597 118.346308098117
0.000598 117.191778540031
0.000599 116.044614005746
};

\nextgroupplot[
    x label style={at={(axis description cs:0.5,-0.05)},anchor=north},
    y label style={at={(axis description cs:-0.005,.5)},rotate=0,anchor=south},
tick align=outside,
tick pos=left,
x grid style={white!69.0196078431373!black},
xlabel={Время},
xmin=-3e-05, xmax=0.00063,
xtick style={color=black},
y grid style={white!69.0196078431373!black},
ylabel={$R_p$},
ymin=-27.959405762658, ymax=604.712856859887,
ytick={0,100,...,500},
ytick style={color=black}
]
\addplot [semithick, color0]
table {%
0 575.95502674068
1e-06 23.0827554578555
1e-06 21.5418264555666
2e-06 10.6234782932276
2e-06 10.6206403750673
3e-06 7.56920474967348
3e-06 7.57851666227927
4e-06 5.9994722442285
4e-06 6.00421986361188
5e-06 5.02885305628313
5e-06 5.03168038963841
6e-06 4.36166851384855
6e-06 4.36350769857728
7e-06 3.87132258067227
7e-06 3.87261805388201
8e-06 3.55674138647501
8e-06 3.55757638225747
9e-06 3.36976074171787
9e-06 3.37074487237578
1e-05 3.20588616800197
1e-05 3.20670446317028
1.1e-05 3.05977915833017
1.1e-05 3.06044767032009
1.2e-05 2.9281996032452
1.2e-05 2.92876962175676
1.3e-05 2.81037058723001
1.3e-05 2.81085546325094
1.4e-05 2.7038104367761
1.4e-05 2.70423605218895
1.5e-05 2.60606604951521
1.5e-05 2.60642766368552
1.6e-05 2.51726267644521
1.6e-05 2.51758403197157
1.7e-05 2.43552757089088
1.7e-05 2.4358153620215
1.8e-05 2.35962281645352
1.8e-05 2.35987519686576
1.9e-05 2.28854392786006
1.9e-05 2.28877000611181
2e-05 2.22242177245108
2e-05 2.22262503807913
2.1e-05 2.16121354484771
2.1e-05 2.16139837220693
2.2e-05 2.10355067073013
2.2e-05 2.10371834432639
2.3e-05 2.04962973296673
2.3e-05 2.04978392168387
2.4e-05 1.99902605189081
2.4e-05 1.99916671847969
2.5e-05 1.95165717019602
2.5e-05 1.95178767397657
2.6e-05 1.9064880918662
2.6e-05 1.90660810284281
2.7e-05 1.86397671190075
2.7e-05 1.86408805079168
2.8e-05 1.82370727775549
2.8e-05 1.8238120407647
2.9e-05 1.78473470648658
2.9e-05 1.78484143455765
3e-05 1.74282235737563
3e-05 1.74290705640239
3.1e-05 1.70266414721836
3.1e-05 1.70274253038295
3.2e-05 1.66412536233039
3.2e-05 1.6641977765066
3.3e-05 1.62749665850255
3.3e-05 1.62756461866695
3.4e-05 1.60463334571167
3.4e-05 1.60468655671028
3.5e-05 1.58771633509466
3.5e-05 1.58777638920819
3.6e-05 1.57126521190328
3.6e-05 1.57132305490478
3.7e-05 1.55524837722282
3.7e-05 1.55530397826559
3.8e-05 1.539673408813
3.8e-05 1.53972694722198
3.9e-05 1.52452148340933
3.9e-05 1.5245730804373
4e-05 1.50967936088897
4e-05 1.50972963198791
4.1e-05 1.4951281740969
4.1e-05 1.4951765036202
4.2e-05 1.48095788371131
4.2e-05 1.48100455970739
4.3e-05 1.46715331582969
4.3e-05 1.46719842969002
4.4e-05 1.45369995976573
4.4e-05 1.45374359637196
4.5e-05 1.44056184605999
4.5e-05 1.44060416977333
4.6e-05 1.42773269434805
4.6e-05 1.42777365805653
4.7e-05 1.41521726846373
4.7e-05 1.41525697193273
4.8e-05 1.40300387004636
4.8e-05 1.40304237749793
4.9e-05 1.39108137460227
4.9e-05 1.39111874601789
5e-05 1.37943922302891
5e-05 1.37947551448817
5.1e-05 1.36805712266617
5.1e-05 1.36809263206676
5.2e-05 1.35682682027335
5.2e-05 1.35686135220443
5.3e-05 1.34582057392676
5.3e-05 1.34585410098365
5.4e-05 1.33506256139506
5.4e-05 1.33509518809282
5.5e-05 1.32454437683385
5.5e-05 1.32457614465384
5.6e-05 1.31425795299471
5.6e-05 1.31428890092316
5.7e-05 1.30416062612899
5.7e-05 1.30419083953802
5.8e-05 1.29427825301339
5.8e-05 1.29430770971412
5.9e-05 1.28458281754714
5.9e-05 1.2846116203291
6e-05 1.27507986313123
6e-05 1.27510795234221
6.1e-05 1.26577573604567
6.1e-05 1.26580316452751
6.2e-05 1.25664858717496
6.2e-05 1.25667543880699
6.3e-05 1.24769564315522
6.3e-05 1.24772186146373
6.4e-05 1.23892459577628
6.4e-05 1.23895023016912
6.5e-05 1.23032998413663
6.5e-05 1.23035505807593
6.6e-05 1.22190655211687
6.6e-05 1.22193108782017
6.7e-05 1.21364926085822
6.7e-05 1.213673279379
6.8e-05 1.20555327740147
6.8e-05 1.20557679870361
6.9e-05 1.19761396404275
6.9e-05 1.19763700706933
7e-05 1.18982686835313
7e-05 1.18984945109025
7.1e-05 1.182187713814
7.1e-05 1.18220985334983
7.2e-05 1.17469239102361
7.2e-05 1.17471410360324
7.3e-05 1.16733694943426
7.3e-05 1.1673582505106
7.4e-05 1.16011758958277
7.4e-05 1.16013849386379
7.5e-05 1.15293383510117
7.5e-05 1.15295673687256
7.6e-05 1.14651058843997
7.6e-05 1.14652627555363
7.7e-05 1.14141685426902
7.7e-05 1.14143397234692
7.8e-05 1.13637057155248
7.8e-05 1.13638752274707
7.9e-05 1.13138987473712
7.9e-05 1.13140659061869
8e-05 1.12647819696973
8e-05 1.12649465733749
8.1e-05 1.12164142499459
8.1e-05 1.12165764700893
8.2e-05 1.11686184546819
8.2e-05 1.11687787854358
8.3e-05 1.11215131607251
8.3e-05 1.1121671187249
8.4e-05 1.10751152338202
8.4e-05 1.10752710627895
8.5e-05 1.10294102839465
8.5e-05 1.10295639760429
8.6e-05 1.09842928599954
8.6e-05 1.09844446563027
8.7e-05 1.09398422941152
8.7e-05 1.09399920632211
8.8e-05 1.08960456741051
8.8e-05 1.08961934711546
8.9e-05 1.08528901181514
8.9e-05 1.08530359955927
9e-05 1.08103631096072
9e-05 1.08105071180116
9.1e-05 1.07684524845563
9.1e-05 1.07685946727003
9.2e-05 1.07270061127773
9.2e-05 1.07271469083777
9.3e-05 1.06861269103779
9.3e-05 1.06862659304384
9.4e-05 1.06458327203818
9.4e-05 1.06459700495516
9.5e-05 1.06061126271503
9.5e-05 1.06062483080224
9.6e-05 1.05669559722106
9.6e-05 1.05670900459255
9.7e-05 1.05283523845584
9.7e-05 1.05284848908648
9.8e-05 1.04902917708491
9.8e-05 1.04904227481618
9.9e-05 1.04527643059949
9.9e-05 1.045289379145
0.0001 1.04157604241503
0.0001 1.04158884536573
0.000101 1.03792708100649
0.000101 1.0379397418357
0.000102 1.03432346438028
0.000102 1.03433600492653
0.000103 1.03076740233046
0.000103 1.03077980401906
0.000104 1.02726024050781
0.000104 1.02727250957363
0.000105 1.02380113962975
0.000105 1.0238132791294
0.000106 1.02038927866204
0.000106 1.02040129155571
0.000107 1.01702385721555
0.000107 1.01703574637077
0.000108 1.01370409489159
0.000108 1.01371586308681
0.000109 1.01042923065254
0.000109 1.01044088058056
0.00011 1.00719852221659
0.00011 1.00721005648779
0.000111 1.0040112454755
0.000111 1.00402266662098
0.000112 1.00086669393445
0.000112 1.000878004409
0.000113 0.997764178172811
0.000113 0.997775380357731
0.000114 0.994703025325134
0.000114 0.994714121530981
0.000115 0.991682578581307
0.000115 0.991693571050478
0.000116 0.988702196705122
0.000116 0.988713087614343
0.000117 0.985761253570444
0.000117 0.985772045033151
0.000118 0.982859137714226
0.000118 0.982869831782834
0.000119 0.97999163081815
0.000119 0.980002267871781
0.00012 0.977153840267181
0.00012 0.977164372225946
0.000121 0.974353300601483
0.000121 0.974363740641028
0.000122 0.971584339219052
0.000122 0.971594720095835
0.000123 0.968846641763684
0.000123 0.968856927011945
0.000124 0.966144712124074
0.000124 0.966154910537735
0.000125 0.963478030407515
0.000125 0.9634881436844
0.000126 0.960846081474098
0.000126 0.96085611126554
0.000127 0.958248361654699
0.000127 0.958258309567121
0.000128 0.95568437843229
0.000128 0.955694246028716
0.000129 0.953153650134055
0.000129 0.953163438935568
0.00013 0.950655705633909
0.00013 0.950665417121043
0.000131 0.948190084064982
0.000131 0.948199719679063
0.000132 0.945756334541699
0.000132 0.945765895686137
0.000133 0.943354015891064
0.000133 0.943363503932583
0.000134 0.940982696392793
0.000134 0.940992112662629
0.000135 0.938641953527967
0.000135 0.938651299323006
0.000136 0.936331373735852
0.000136 0.936340650319731
0.000137 0.934050552178603
0.000137 0.934059760782769
0.000138 0.931799092513527
0.000138 0.931808234338255
0.000139 0.929576606672643
0.000139 0.929585682888011
0.00014 0.927382714649235
0.00014 0.927391726396071
0.000141 0.925217044291173
0.000141 0.925225992681961
0.000142 0.92307923110072
0.000142 0.923088117220475
0.000143 0.920968918040605
0.000143 0.920977742947713
0.000144 0.918885755346122
0.000144 0.918894520073151
0.000145 0.916825720694781
0.000145 0.9168344374012
0.000146 0.914792051088934
0.000146 0.914800709278672
0.000147 0.9127845787835
0.000147 0.912793179657686
0.000148 0.910802980618407
0.000148 0.910811525115894
0.000149 0.908846939633361
0.000149 0.908855428670764
0.00015 0.90691614515051
0.00015 0.906924579622859
0.000151 0.905010292622454
0.000151 0.905018673403823
0.000152 0.903128450795226
0.000152 0.90313679463434
0.000153 0.901267859715374
0.000153 0.90127614710569
0.000154 0.899431381163139
0.000154 0.899439617283905
0.000155 0.897618737153179
0.000155 0.897626922802343
0.000156 0.895829650793581
0.000156 0.895837786750927
0.000157 0.894063850520802
0.000157 0.894071937548453
0.000158 0.892321069975691
0.000158 0.892329108818613
0.000159 0.890601047883075
0.000159 0.890609039269547
0.00016 0.888903527934778
0.00016 0.888911472576861
0.000161 0.887228258675975
0.000161 0.887236157269955
0.000162 0.88557499339475
0.000162 0.885582846621567
0.000163 0.883943490014766
0.000163 0.883951298540434
0.000164 0.882333510990937
0.000164 0.882341275466948
0.000165 0.880744823208008
0.000165 0.880752544271722
0.000166 0.879177197881944
0.000166 0.879184876156966
0.000167 0.877630410464043
0.000167 0.877638046560587
0.000168 0.876104240547669
0.000168 0.876111835062916
0.000169 0.87459847177755
0.000169 0.874606025295984
0.00017 0.873112891761522
0.00017 0.873120404855267
0.000171 0.871647291984674
0.000171 0.871654765213811
0.000172 0.870201467725796
0.000172 0.870208901638676
0.000173 0.868775217976067
0.000173 0.868782613109611
0.000174 0.867368345359916
0.000174 0.867375702239906
0.000175 0.865980656057972
0.000175 0.865987975199333
0.000176 0.864611959732057
0.000176 0.864619241639132
0.000177 0.863262069452152
0.000177 0.863269314618965
0.000178 0.861930801625264
0.000178 0.86193801053578
0.000179 0.860617975926155
0.000179 0.860625149054527
0.00018 0.859323415229865
0.00018 0.859330553040676
0.000181 0.85804694554597
0.000181 0.858054048494468
0.000182 0.85678839595454
0.000182 0.856795464486862
0.000183 0.855547598543718
0.000183 0.855554633097115
0.000184 0.854324388348904
0.000184 0.854331389351953
0.000185 0.853118603293474
0.000185 0.853125571166283
0.000186 0.85193008413099
0.000186 0.851937019285402
0.000187 0.850758674388869
0.000187 0.850765577228655
0.000188 0.849603388279666
0.000188 0.849610268238603
0.000189 0.848464054961031
0.000189 0.848470902157881
0.00019 0.847341355839527
0.00019 0.847348181816482
0.000191 0.846233611835489
0.000191 0.846240404788377
0.000192 0.845142279941423
0.000192 0.845149042427507
0.000193 0.844067220846792
0.000193 0.844073953226357
0.000194 0.843008295479932
0.000194 0.84301499810639
0.000195 0.84196536715159
0.000195 0.841972040371632
0.000196 0.840938301509645
0.000196 0.84094494566339
0.000197 0.839926966494909
0.000197 0.839933581916053
0.000198 0.838931232298
0.000198 0.838937819313958
0.000199 0.837950971317237
0.000199 0.83795753024928
0.0002 0.836986058117537
0.0002 0.836992589280929
0.000201 0.836036369390292
0.000201 0.836042873094413
0.000202 0.835101783914173
0.000202 0.835108260462657
0.000203 0.83418218251688
0.000203 0.834188632207729
0.000204 0.833277448037766
0.000204 0.833283871163477
0.000205 0.832387465291341
0.000205 0.832393862139019
0.000206 0.831512121031623
0.000206 0.831518491883091
0.000207 0.830651303917302
0.000207 0.830657649049222
0.000208 0.829804904477722
0.000208 0.829811224161692
0.000209 0.828972815079631
0.000209 0.828979109582295
0.00021 0.828154929894689
0.00021 0.828161199477837
0.000211 0.82734450881035
0.000211 0.827351268041861
0.000212 0.826482451739875
0.000212 0.826489150809064
0.000213 0.825631442521699
0.000213 0.825638129054933
0.000214 0.824793866462801
0.000214 0.824800534938203
0.000215 0.82396993737747
0.000215 0.823976598103923
0.000216 0.823159806474294
0.000216 0.823166440930056
0.000217 0.822364219011392
0.000217 0.822370838311741
0.000218 0.82158247239298
0.000218 0.821589066271906
0.000219 0.820815347809539
0.000219 0.820821917841067
0.00022 0.820062745779514
0.00022 0.820069292207181
0.000221 0.819323736991369
0.000221 0.819330277896793
0.000222 0.818597766820429
0.000222 0.818604282572165
0.000223 0.817886064295994
0.000223 0.817892557122291
0.000224 0.817188537145508
0.000224 0.817195007274108
0.000225 0.816505092753642
0.000225 0.816511540408343
0.000226 0.815835640131776
0.000226 0.815842065532507
0.000227 0.815180089891592
0.000227 0.815186493254487
0.000228 0.814537691799331
0.000228 0.814544081760387
0.000229 0.813908757083129
0.000229 0.813915124982636
0.00023 0.813293484290827
0.00023 0.813299830767601
0.000231 0.81269179047873
0.000231 0.812698115734732
0.000232 0.81210359375017
0.000232 0.812109897983895
0.000233 0.811528813665661
0.000233 0.811535097072196
0.000234 0.810967371220228
0.000234 0.810973633991317
0.000235 0.810419188821239
0.000235 0.810425431145334
0.000236 0.8098841902667
0.000236 0.809890412329024
0.000237 0.80936230072403
0.000237 0.809368502706631
0.000238 0.808853446709284
0.000238 0.808859628791087
0.000239 0.808357556066828
0.000239 0.808363718423688
0.00024 0.807874557949438
0.00024 0.807880700754193
0.000241 0.807403195852768
0.000241 0.807409335735879
0.000242 0.806944410735741
0.000242 0.806950531048517
0.000243 0.806498348194879
0.000243 0.806504449441966
0.000244 0.806064942577186
0.000244 0.806071024920421
0.000245 0.805644129110858
0.000245 0.805650192709295
0.000246 0.805235844227686
0.000246 0.805241889237634
0.000247 0.80484002554584
0.000247 0.804846052120911
0.000248 0.804456611853022
0.000248 0.80446262014417
0.000249 0.80408554308997
0.000249 0.804091533245534
0.00025 0.803726760334303
0.00025 0.803732732500042
0.000251 0.8033802057847
0.000251 0.803386160103837
0.000252 0.80304582274541
0.000252 0.803051759358667
0.000253 0.802723555611083
0.000253 0.802729474656719
0.000254 0.802413349851911
0.000254 0.802419251465756
0.000255 0.802115151999073
0.000255 0.802121036314565
0.000256 0.801828909630486
0.000256 0.801834776778705
0.000257 0.801554571356839
0.000257 0.801560421466538
0.000258 0.80129208680792
0.000258 0.801297920005558
0.000259 0.801041406619213
0.000259 0.80104722302899
0.00026 0.800802482418779
0.00026 0.800808282162663
0.000261 0.800575266814397
0.000261 0.800581050012152
0.000262 0.800359713380957
0.000262 0.800365480150177
0.000263 0.800155776648127
0.000263 0.800161527104263
0.000264 0.799963412088254
0.000264 0.799969146344639
0.000265 0.799782576104511
0.000265 0.79978829427239
0.000266 0.799613226019287
0.000266 0.799618928207841
0.000267 0.799455320062806
0.000267 0.799461006379179
0.000268 0.799308817361972
0.000268 0.799314487911296
0.000269 0.799173677929444
0.000269 0.799179332814862
0.00027 0.79904986265292
0.00027 0.799055501975609
0.000271 0.79893733328464
0.000271 0.798942957143837
0.000272 0.798836052431096
0.000272 0.798841660924115
0.000273 0.798745983542944
0.000273 0.798751576765205
0.000274 0.798667090905119
0.000274 0.798672668950163
0.000275 0.798599339627142
0.000275 0.798604902586654
0.000276 0.798542526341764
0.000276 0.798548093643519
0.000277 0.798496794802926
0.000277 0.798502347126343
0.000278 0.798462143310196
0.000278 0.798467680816207
0.000279 0.79843854024601
0.000279 0.798444063019581
0.00028 0.798425954702807
0.00028 0.798431462827137
0.000281 0.798424356548575
0.000281 0.798429850105114
0.000282 0.798433716418395
0.000282 0.798439195486861
0.000283 0.798454005706141
0.000283 0.798459470364537
0.000284 0.79848519655635
0.000284 0.798490646880979
0.000285 0.798527261856243
0.000285 0.798532697921725
0.000286 0.798580175227911
0.000286 0.798585597107195
0.000287 0.79864372359691
0.000287 0.798649112057372
0.000288 0.798717993703469
0.000288 0.798723368083644
0.000289 0.79880299826189
0.000289 0.798808358655419
0.00029 0.798898713734355
0.00029 0.798904060207183
0.000291 0.799005117249897
0.000291 0.79901044986637
0.000292 0.799122186623544
0.000292 0.799127505446425
0.000293 0.799249900349515
0.000293 0.79925520543999
0.000294 0.799388237594537
0.000294 0.799393529012231
0.000295 0.799537178191298
0.000295 0.799542455994287
0.000296 0.799696702632024
0.000296 0.799701966876843
0.000297 0.799866792062171
0.000297 0.799872042803824
0.000298 0.800047428274243
0.000298 0.800052665566215
0.000299 0.800238593701727
0.000299 0.800243817595992
0.0003 0.80044027141314
0.0003 0.800445481960173
0.000301 0.80065244510619
0.000301 0.800657642354971
0.000302 0.800875099102045
0.000302 0.800880283100071
0.000303 0.801108218339714
0.000303 0.801113389133008
0.000304 0.801351788370529
0.000304 0.801356946003643
0.000305 0.80160579535273
0.000305 0.801610939868758
0.000306 0.801870226046158
0.000306 0.801875357486741
0.000307 0.802145067807036
0.000307 0.802150186212368
0.000308 0.802430308582857
0.000308 0.802435413991692
0.000309 0.802725936907362
0.000309 0.802731029357021
0.00031 0.80303194189561
0.00031 0.803037021421984
0.000311 0.803348313239142
0.000311 0.803353379876702
0.000312 0.80367504120123
0.000312 0.803680094983029
0.000313 0.804012116612219
0.000313 0.804017157569897
0.000314 0.804359530864947
0.000314 0.804364559028738
0.000315 0.804717275910256
0.000315 0.80472229130899
0.000316 0.805085344252576
0.000316 0.805090346913685
0.000317 0.805463728945599
0.000317 0.805468718895118
0.000318 0.805852423588021
0.000318 0.805857400850593
0.000319 0.806251422319367
0.000319 0.806256386918247
0.00032 0.80666071981589
0.00032 0.806665671772947
0.000321 0.807080311286539
0.000321 0.807085250622261
0.000322 0.807510192469006
0.000322 0.807515119202497
0.000323 0.807949923192583
0.000323 0.807954822110493
0.000324 0.808399033054582
0.000324 0.808403918123654
0.000325 0.808858385664748
0.000325 0.808863258160606
0.000326 0.809327979910152
0.000326 0.809332839846122
0.000327 0.809807813922588
0.000327 0.809812661310625
0.000328 0.810297886335921
0.000328 0.810302721186613
0.000329 0.810798196282575
0.000329 0.810803018605142
0.00033 0.811308743390073
0.00033 0.811313553192369
0.000331 0.81182952777764
0.000331 0.811834325066151
0.000332 0.812360550052859
0.000332 0.812365334832704
0.000333 0.81290181130838
0.000333 0.81290658358331
0.000334 0.813453313118684
0.000334 0.813458072891083
0.000335 0.814015057536907
0.000335 0.814019804807786
0.000336 0.814587047091694
0.000336 0.814591781860699
0.000337 0.815169017405787
0.000337 0.815173732319715
0.000338 0.815760631961362
0.000338 0.815765333494626
0.000339 0.81636248218068
0.000339 0.816367171191188
0.00034 0.816974573205772
0.00034 0.816979249687562
0.000341 0.8175969097745
0.000341 0.817601573720234
0.000342 0.81822949708018
0.000342 0.818234148481141
0.000343 0.818872340768763
0.000343 0.818876979614851
0.000344 0.819525446936046
0.000344 0.819530073215775
0.000345 0.820187128128946
0.000345 0.820191725854792
0.000346 0.82085880243767
0.000346 0.820863387207404
0.000347 0.821540720633139
0.000347 0.821545292769449
0.000348 0.82223289065153
0.000348 0.822237450137187
0.000349 0.822934759468657
0.000349 0.82293929887666
0.00035 0.823646422734667
0.00035 0.823650948777571
0.000351 0.824368342753525
0.000351 0.824372856073483
0.000352 0.825098534349996
0.000352 0.825103019081967
0.000353 0.825838471439718
0.000353 0.825842936330246
0.000354 0.826587138438691
0.000354 0.826591575799575
0.000355 0.827342599338917
0.000355 0.827346978715933
0.000356 0.828050962530844
0.000356 0.828054911905714
0.000357 0.82876152537645
0.000357 0.828765452618751
0.000358 0.829481126202232
0.000358 0.829485039969022
0.000359 0.830209776866417
0.000359 0.83021367712412
0.00036 0.830947480809085
0.00036 0.830951367522909
0.000361 0.831694241738668
0.000361 0.831698114872604
0.000362 0.832450063629034
0.000362 0.832453923145857
0.000363 0.833214950716582
0.000363 0.833218796577845
0.000364 0.833988907497346
0.000364 0.833992739663377
0.000365 0.834771938724101
0.000365 0.834775757154001
0.000366 0.835564049403486
0.000366 0.835567854055125
0.000367 0.836365244793121
0.000367 0.836369035623137
0.000368 0.837175530398741
0.000368 0.837179307362536
0.000369 0.837994911971327
0.000369 0.837998675023059
0.00037 0.83882339550424
0.00037 0.838827144596826
0.000371 0.839660987230364
0.000371 0.839664722315473
0.000372 0.840507693619243
0.000372 0.840511414647292
0.000373 0.841363521374227
0.000373 0.841367228294379
0.000374 0.842228477429611
0.000374 0.84223217018977
0.000375 0.84310256894778
0.000375 0.843106247494585
0.000376 0.843985803316345
0.000376 0.84398946759517
0.000377 0.844878188145286
0.000377 0.844881838100231
0.000378 0.84577973126408
0.000378 0.845783366837972
0.000379 0.846690440718843
0.000379 0.846694061853225
0.00038 0.847609974911941
0.00038 0.847613575776818
0.000381 0.848537674502781
0.000381 0.848541259453954
0.000382 0.849474544376263
0.000382 0.849478114701737
0.000383 0.850420218622597
0.000383 0.850423768490182
0.000384 0.851374041734419
0.000384 0.851377575464803
0.000385 0.85233704136083
0.000385 0.85234056026815
0.000386 0.853309228069519
0.000386 0.853312732086491
0.000387 0.854290611238211
0.000387 0.854294100296241
0.000388 0.85528120043184
0.000388 0.855284674461019
0.000389 0.856281005399586
0.000389 0.856284464328687
0.00039 0.857290036071905
0.00039 0.857293479828377
0.000391 0.858308302557546
0.000391 0.858311731067506
0.000392 0.859335815140547
0.000392 0.85933922832878
0.000393 0.860372584277228
0.000393 0.860375982067176
0.000394 0.861418620593153
0.000394 0.861422002906914
0.000395 0.862473934880092
0.000395 0.862477301638411
0.000396 0.863538538092952
0.000396 0.863541889215217
0.000397 0.864612441346701
0.000397 0.864615776750938
0.000398 0.865695655913262
0.000398 0.865698975516129
0.000399 0.866788193218401
0.000399 0.86679149693518
0.0004 0.867890064838583
0.0004 0.867893352583177
0.000401 0.869001282497813
0.000401 0.869004554182737
0.000402 0.870121858064453
0.000402 0.870125113600833
0.000403 0.871251803548018
0.000403 0.87125504284558
0.000404 0.872391131095949
0.000404 0.872394354063016
0.000405 0.873539852990358
0.000405 0.873543059533843
0.000406 0.874697981644756
0.000406 0.874701171670154
0.000407 0.875865529600744
0.000407 0.875868703012131
0.000408 0.877042509524691
0.000408 0.877045666224711
0.000409 0.878228934204372
0.000409 0.878232074094236
0.00041 0.879424816545588
0.00041 0.879427939525067
0.000411 0.880630169568753
0.000411 0.880633275536169
0.000412 0.88184500640545
0.000412 0.881848095257673
0.000413 0.88306934029496
0.000413 0.8830724119274
0.000414 0.884303184580761
0.000414 0.884306238887363
0.000415 0.885546552706991
0.000415 0.885549589580226
0.000416 0.88679945821488
0.000416 0.88680247754574
0.000417 0.888061914739153
0.000417 0.888064916417147
0.000418 0.88933393600439
0.000418 0.889336919917535
0.000419 0.890615535821366
0.000419 0.890618501856179
0.00042 0.891906728083335
0.00042 0.891909676124833
0.000421 0.893207526762301
0.000421 0.893210456693988
0.000422 0.89451794590523
0.000422 0.894520857609093
0.000423 0.89583799963024
0.000423 0.895840892986746
0.000424 0.89716770212275
0.000424 0.897170577010833
0.000425 0.898507067631579
0.000425 0.898509923928639
0.000426 0.899856110465017
0.000426 0.899858948046914
0.000427 0.901214844986854
0.000427 0.901217663727897
0.000428 0.902583285612361
0.000428 0.902586085385307
0.000429 0.90396040458937
0.000429 0.903963178241023
0.00043 0.905345499124765
0.00043 0.905348251180899
0.000431 0.906740304983008
0.000431 0.906743037685752
0.000432 0.908144838828262
0.000432 0.908147552044633
0.000433 0.909559114979548
0.000433 0.909561808574993
0.000434 0.910983147780025
0.000434 0.910985821618414
0.000435 0.912416951592661
0.000435 0.912419605536279
0.000436 0.913860540795866
0.000436 0.913863174705411
0.000437 0.915313929779076
0.000437 0.915316543513652
0.000438 0.916777132938292
0.000438 0.916779726355402
0.000439 0.918250164671561
0.000439 0.918252737627104
0.00044 0.919731298733609
0.00044 0.919733843918711
0.000441 0.921221227549539
0.000441 0.92122375057542
0.000442 0.922720991529058
0.000442 0.922723493661442
0.000443 0.92423060616386
0.000443 0.924233087252789
0.000444 0.925750085508227
0.000444 0.925752545402123
0.000445 0.927279443588277
0.000445 0.927281882133942
0.000446 0.928818694397107
0.000446 0.928821111439716
0.000447 0.930367851889868
0.000447 0.930370247272969
0.000448 0.931926929978801
0.000448 0.931929303544305
0.000449 0.933495942528197
0.000449 0.933498294116378
0.00045 0.935074903349325
0.00045 0.935077232798813
0.000451 0.936663826195285
0.000451 0.936666133343066
0.000452 0.938262724755816
0.000452 0.938265009437225
0.000453 0.939871612652043
0.000453 0.939873874700762
0.000454 0.941490503431163
0.000454 0.941492742679219
0.000455 0.943119410561081
0.000455 0.943121626838839
0.000456 0.944758347424979
0.000456 0.944760540561144
0.000457 0.946407327315828
0.000457 0.946409497137438
0.000458 0.948066363430844
0.000458 0.94806850976327
0.000459 0.949735468865879
0.000459 0.949737591532822
0.00046 0.951414656609752
0.00046 0.95141675543324
0.000461 0.953103939538522
0.000461 0.953106014338909
0.000462 0.954803330409693
0.000462 0.954805381005658
0.000463 0.956512841856363
0.000463 0.956514868064908
0.000464 0.958232486381308
0.000464 0.958234488017756
0.000465 0.959962276351
0.000465 0.959964253228994
0.000466 0.961702223989567
0.000466 0.961704175921068
0.000467 0.963452341372682
0.000467 0.963454268167974
0.000468 0.965212640421397
0.000468 0.965214541889081
0.000469 0.9669831328959
0.000469 0.966985008842898
0.00047 0.968763830389225
0.00047 0.968765680620775
0.000471 0.970554744320874
0.000471 0.970556568640536
0.000472 0.97235588593039
0.000472 0.972357684140047
0.000473 0.974162912538909
0.000473 0.974164671909296
0.000474 0.975977811461493
0.000474 0.975979541188493
0.000475 0.977802886497725
0.000475 0.977804589566229
0.000476 0.979638150658251
0.000476 0.979639826865451
0.000477 0.981481142561248
0.000477 0.981482781558123
0.000478 0.983329200631232
0.000478 0.983330805413569
0.000479 0.985187377459646
0.000479 0.985188954830307
0.00048 0.987055688267349
0.00048 0.987057238019381
0.000481 0.988934141020392
0.000481 0.988935662945267
0.000482 0.990822743431287
0.000482 0.99082423731891
0.000483 0.992721502951984
0.000483 0.992722968590698
0.000484 0.99463042676678
0.000484 0.994631863943375
0.000485 0.996549521785168
0.000485 0.996550930284883
0.000486 0.99847879463462
0.000486 0.998480174241153
0.000487 1.00041825165331
0.000487 1.00041960214882
0.000488 1.00236789888276
0.000488 1.0023692200479
0.000489 1.00432774206048
0.000489 1.00432903367435
0.00049 1.00629778661242
0.00049 1.00629904845264
0.000491 1.00827803764554
0.000491 1.00827926948822
0.000492 1.01026849994015
0.000492 1.01026970155991
0.000493 1.01226917794228
0.000493 1.01227034911227
0.000494 1.01428007575598
0.000494 1.01428121624787
0.000495 1.01630119713554
0.000495 1.01630230671954
0.000496 1.01833254547764
0.000496 1.01833362392253
0.000497 1.02037412381349
0.000497 1.02037517088662
0.000498 1.02242593480085
0.000498 1.02242695026814
0.000499 1.02448798071605
0.000499 1.02448896434201
0.0005 1.0265602634459
0.0005 1.02656121499366
0.000501 1.02864278447956
0.000501 1.02864370371087
0.000502 1.03073554490039
0.000502 1.03073643157564
0.000503 1.03283854537767
0.000503 1.0328393992559
0.000504 1.03495178615835
0.000504 1.03495260699729
0.000505 1.03707480921606
0.000505 1.03707559452566
0.000506 1.0392041397913
0.000506 1.0392048862339
0.000507 1.04134365216838
0.000507 1.04134436489734
0.000508 1.0434933498574
0.000508 1.04349402862699
0.000509 1.0456532305057
0.000509 1.04565387506896
0.00051 1.0478232912872
0.00051 1.04782390139599
0.000511 1.05000352889381
0.000511 1.05000410429882
0.000512 1.05219393952679
0.000512 1.05219447997757
0.000513 1.05439451888809
0.000513 1.05439502413307
0.000514 1.05660526217163
0.000514 1.05660573195814
0.000515 1.05882616405454
0.000515 1.05882659812884
0.000516 1.06105721868834
0.000516 1.06105761679565
0.000517 1.06329841969015
0.000517 1.06329878157466
0.000518 1.06554976013374
0.000518 1.06555008553866
0.000519 1.06781123254068
0.000519 1.06781152120824
0.00052 1.07008282887138
0.00052 1.07008308054288
0.000521 1.07236454051606
0.000521 1.07236475493189
0.000522 1.07465635828582
0.000522 1.07465653518548
0.000523 1.07695491778923
0.000523 1.07695505175266
0.000524 1.07925740609422
0.000524 1.07925749381404
0.000525 1.08156985288611
0.000525 1.08156990246711
0.000526 1.08389225424364
0.000526 1.08389226542613
0.000527 1.08622459725897
0.000527 1.08622456978261
0.000528 1.08856686839246
0.000528 1.08856680199628
0.000529 1.09091905346366
0.000529 1.09091894788612
0.00053 1.09328113764233
0.00053 1.09328099262131
0.000531 1.09565310543939
0.000531 1.09565292071228
0.000532 1.09803494069792
0.000532 1.09803471600164
0.000533 1.10042662658411
0.000533 1.10042636165513
0.000534 1.10282814557826
0.000534 1.10282784015266
0.000535 1.10523475836159
0.000535 1.1052344064632
0.000536 1.1076508984173
0.000536 1.10765050530336
0.000537 1.11007675915229
0.000537 1.11007632484622
0.000538 1.1125123201795
0.000538 1.11251184441846
0.000539 1.11495756009666
0.000539 1.11495704261765
0.00054 1.11741245676343
0.00054 1.11741189730337
0.000541 1.11987698729257
0.000541 1.11987638558832
0.000542 1.12234509542721
0.000542 1.12234444568529
0.000543 1.12481868684974
0.000543 1.12481798909749
0.000544 1.12730172250051
0.000544 1.12730098192933
0.000545 1.12979418127168
0.000545 1.1297933976233
0.000546 1.13229603581956
0.000546 1.13229520883601
0.000547 1.13480291945835
0.000547 1.13480204480926
0.000548 1.13731835820036
0.000548 1.13731743882679
0.000549 1.13984304963098
0.000549 1.13984208626426
0.00055 1.14235951695141
0.00055 1.14235849419894
0.000551 1.14488210526861
0.000551 1.1448810346533
0.000552 1.14740929920395
0.000552 1.14740818078644
0.000553 1.15029829592003
0.000553 1.1502972340352
0.000554 1.15413187649012
0.000554 1.15413254315053
0.000555 1.1575910267802
0.000555 1.15759091325284
0.000556 1.16107052657387
0.000556 1.16107035183322
0.000557 1.16457117048714
0.000557 1.16457093378646
0.000558 1.16809303289457
0.000558 1.16809273347878
0.000559 1.17163618761525
0.000559 1.17163582472089
0.00056 1.1752007078855
0.00056 1.1752002807407
0.000561 1.17878666633088
0.000561 1.17878617415533
0.000562 1.18239413493776
0.000562 1.18239357694266
0.000563 1.18602318502425
0.000563 1.18602256041228
0.000564 1.18967388721064
0.000564 1.18967319517596
0.000565 1.19334631138925
0.000565 1.19334555111745
0.000566 1.19704052669371
0.000566 1.19703969736178
0.000567 1.2007566014677
0.000567 1.20075570224402
0.000568 1.20449460323305
0.000568 1.20449363327737
0.000569 1.20825459865733
0.000569 1.20825355712076
0.00057 1.21203665352082
0.00057 1.21203553954579
0.000571 1.21584083268285
0.000571 1.21583964540314
0.000572 1.21966720004767
0.000572 1.21966593858836
0.000573 1.22351581852956
0.000573 1.22351448200705
0.000574 1.22738675001747
0.000574 1.22738533753948
0.000575 1.23128005533899
0.000575 1.23127856600453
0.000576 1.23519579422373
0.000576 1.23519422712316
0.000577 1.23913402526611
0.000577 1.2391323794811
0.000578 1.24309480588748
0.000578 1.24309308049104
0.000579 1.24707819229771
0.000579 1.2470763863542
0.00058 1.25108423945609
0.00058 1.25108235202125
0.000581 1.25511300103162
0.000581 1.25511103115259
0.000582 1.25916452936276
0.000582 1.25916247607811
0.000583 1.26323358561356
0.000583 1.2632314451145
0.000584 1.26731762058146
0.000584 1.26731538075087
0.000585 1.2714243392026
0.000585 1.27142201329129
0.000586 1.27555379975228
0.000586 1.27555138677582
0.000587 1.27970604472321
0.000587 1.27970354368906
0.000588 1.28388111500903
0.000588 1.28387852491661
0.000589 1.28807904986017
0.000589 1.28807636970093
0.00059 1.29229461580768
0.00059 1.29229184547192
0.000591 1.29652214484933
0.000591 1.29651926286747
0.000592 1.30077236662407
0.000592 1.30076939187989
0.000593 1.30504533042862
0.000593 1.30504226189661
0.000594 1.3093410637751
0.000594 1.3093379004225
0.000595 1.31365893649393
0.000595 1.31365568838795
0.000596 1.31798383865537
0.000596 1.31798046520385
0.000597 1.32233130494336
0.000597 1.32232783396301
0.000598 1.32670138125204
0.000598 1.32669781169666
0.000599 1.33109408249291
0.000599 1.33109041330995
0.0006 1.33550942140247
};

\nextgroupplot[
    x label style={at={(axis description cs:0.5,-0.05)},anchor=north},
    y label style={at={(axis description cs:-0.005,.5)},rotate=0,anchor=south},
tick align=outside,
tick pos=left,
x grid style={white!69.0196078431373!black},
xlabel={Время},
xmin=-3e-05, xmax=0.00063,
xtick style={color=black},
y grid style={white!69.0196078431373!black},
ylabel={$I \cdot R_p$},
ymin=119.082547295159, ymax=655.990364623057,
ytick style={color=black}
]
\addplot [semithick, color0]
table {%
0 287.97751337034
1e-06 148.790884563562
1e-06 146.789302544322
2e-06 143.487448082791
2e-06 143.494647188219
3e-06 152.987763657292
3e-06 152.948644293139
4e-06 160.923874326811
4e-06 160.895321636402
5e-06 167.895129494809
5e-06 167.87184272148
6e-06 174.055077314755
6e-06 174.035794609653
7e-06 179.569776596813
7e-06 179.554196677621
8e-06 187.88934697265
8e-06 187.837160004124
9e-06 199.511740420672
9e-06 199.448575644181
1e-05 210.004954555098
1e-05 209.95257318125
1.1e-05 219.516947954216
1.1e-05 219.471817943179
1.2e-05 228.169554040884
1.2e-05 228.13160806702
1.3e-05 236.201486774399
1.3e-05 236.167842947556
1.4e-05 243.667921988841
1.4e-05 243.638754658493
1.5e-05 250.56202130436
1.5e-05 250.535655958177
1.6e-05 257.077729831861
1.6e-05 257.053992739239
1.7e-05 263.188658510476
1.7e-05 263.167619976647
1.8e-05 268.895498168595
1.8e-05 268.876240625048
1.9e-05 274.195302301168
1.9e-05 274.178247350665
2e-05 279.202416289343
2e-05 279.186766199497
2.1e-05 284.007743373718
2.1e-05 283.993082291367
2.2e-05 288.517077380564
2.2e-05 288.503764730882
2.3e-05 292.8293710772
2.3e-05 292.817037586647
2.4e-05 296.952562900528
2.4e-05 296.940891802156
2.5e-05 300.937263746026
2.5e-05 300.926249835781
2.6e-05 304.678449333415
2.6e-05 304.668390133722
2.7e-05 308.296003591107
2.7e-05 308.286387145879
2.8e-05 311.767714079765
2.8e-05 311.758846077257
2.9e-05 314.96912154666
2.9e-05 314.962981135851
3e-05 317.157102196529
3e-05 317.152798666652
3.1e-05 319.178964196465
3.1e-05 319.175092150855
3.2e-05 321.037537654954
3.2e-05 321.034023988176
3.3e-05 322.820740167704
3.3e-05 322.817385442041
3.4e-05 326.977906725648
3.4e-05 326.961901421478
3.5e-05 332.068085732502
3.5e-05 332.050015742215
3.6e-05 337.011740816588
3.6e-05 336.994397563136
3.7e-05 341.817059946699
3.7e-05 341.800368520194
3.8e-05 346.495481977185
3.8e-05 346.479390401447
3.9e-05 351.052268080889
3.9e-05 351.036741622489
4e-05 355.469907275356
4e-05 355.45512474146
4.1e-05 359.751365328036
4.1e-05 359.737136543675
4.2e-05 363.925755419849
4.2e-05 363.911997077572
4.3e-05 367.997219667232
4.3e-05 367.983906164435
4.4e-05 371.969726464927
4.4e-05 371.956834002043
4.5e-05 375.841225221723
4.5e-05 375.828776660972
4.6e-05 379.616786171254
4.6e-05 379.604723868525
4.7e-05 383.30419523438
4.7e-05 383.292490862356
4.8e-05 386.906641796118
4.8e-05 386.89527739879
4.9e-05 390.427162081724
4.9e-05 390.416120875019
5e-05 393.868640526046
5e-05 393.85790680508
5.1e-05 397.230839553228
5.1e-05 397.220480918308
5.2e-05 400.48711389227
5.2e-05 400.477154587547
5.3e-05 403.663117504516
5.3e-05 403.653437622068
5.4e-05 406.770804677984
5.4e-05 406.761374752558
5.5e-05 409.812414979935
5.5e-05 409.803223675879
5.6e-05 412.790100844483
5.6e-05 412.781137500063
5.7e-05 415.694769285771
5.7e-05 415.686078749914
5.8e-05 418.538745860101
5.8e-05 418.530264488009
5.9e-05 421.316959866366
5.9e-05 421.308725641676
6e-05 424.035036348197
6e-05 424.026998188457
6.1e-05 426.69885614787
6.1e-05 426.690999407321
6.2e-05 429.304701168525
6.2e-05 429.297055787583
6.3e-05 431.855079743641
6.3e-05 431.847607464102
6.4e-05 434.356045340414
6.4e-05 434.348732494287
6.5e-05 436.809030928399
6.5e-05 436.801871209545
6.6e-05 439.215417374009
6.6e-05 439.208404806949
6.7e-05 441.576529523318
6.7e-05 441.569658442037
6.8e-05 443.893639097356
6.8e-05 443.886904125705
6.9e-05 446.167967406166
6.9e-05 446.16136343972
7e-05 448.400687894878
7e-05 448.39421008412
7.1e-05 450.592928533968
7.1e-05 450.586572268699
7.2e-05 452.745774064863
7.2e-05 452.739534959732
7.3e-05 454.860268111136
7.3e-05 454.854141992195
7.4e-05 456.937415164708
7.4e-05 456.931398056909
7.5e-05 458.939641781362
7.5e-05 458.934797361792
7.6e-05 461.160009296014
7.6e-05 461.151741708313
7.7e-05 463.83746314954
7.7e-05 463.828473250039
7.8e-05 466.459063485846
7.8e-05 466.450287754058
7.9e-05 469.035198331276
7.9e-05 469.026575939176
8e-05 471.56871528612
8e-05 471.56022487031
8.1e-05 474.06354277059
8.1e-05 474.055175473674
8.2e-05 476.513593399517
8.2e-05 476.505384724031
8.3e-05 478.925283299969
8.3e-05 478.917192764987
8.4e-05 481.300708387584
8.4e-05 481.292730526256
8.5e-05 483.640607038819
8.5e-05 483.632738739308
8.6e-05 485.941652866441
8.6e-05 485.933911213667
8.7e-05 488.208620497398
8.7e-05 488.20098238898
8.8e-05 490.442191360599
8.8e-05 490.434653979445
8.9e-05 492.643023929515
8.9e-05 492.635584596241
9e-05 494.811758009212
9e-05 494.804414140336
9.1e-05 496.949015372292
9.1e-05 496.941764476023
9.2e-05 499.048872973696
9.2e-05 499.041749255819
9.3e-05 501.117183011228
9.3e-05 501.110149271201
9.4e-05 503.155852790259
9.4e-05 503.148904741136
9.5e-05 505.165435928855
9.5e-05 505.158571411719
9.6e-05 507.146472995407
9.6e-05 507.139689924979
9.7e-05 509.099489976773
9.7e-05 509.092786338325
9.8e-05 511.024998775865
9.8e-05 511.018372622284
9.9e-05 512.923497688585
9.9e-05 512.916947137583
0.0001 514.795471861137
0.0001 514.788995092588
0.000101 516.641393728635
0.000101 516.634988982056
0.000102 518.459129595834
0.000102 518.452811666619
0.000103 520.250656675007
0.000103 520.244408824492
0.000104 522.017513573949
0.000104 522.011332657851
0.000105 523.760124215615
0.000105 523.754008690974
0.000106 525.478903300188
0.000106 525.472851672714
0.000107 527.174255098517
0.000107 527.168265920696
0.000108 528.846573782855
0.000108 528.84064565214
0.000109 530.496243744814
0.000109 530.490375301906
0.00011 532.123639901147
0.00011 532.117829828341
0.000111 533.729127987873
0.000111 533.723375007494
0.000112 535.31306484329
0.000112 535.307367716186
0.000113 536.875798680353
0.000113 536.870156204464
0.000114 538.417669348903
0.000114 538.412080357883
0.000115 539.939008588165
0.000115 539.933471950077
0.000116 541.440140269968
0.000116 541.434654886022
0.000117 542.921380633062
0.000117 542.915945436422
0.000118 544.383038508939
0.000118 544.377652463565
0.000119 545.823398711775
0.000119 545.818087516667
0.00012 547.240324938505
0.00012 547.235066315172
0.000121 548.638625903341
0.000121 548.633413270843
0.000122 550.015676712694
0.000122 550.010534245162
0.000123 551.371861188884
0.000123 551.366766192142
0.000124 552.710301426189
0.000124 552.705249541717
0.000125 554.03125577825
0.000125 554.026246162849
0.000126 555.334980224848
0.000126 555.330012058403
0.000127 556.621725044493
0.000127 556.616797529196
0.000128 557.891734972838
0.000128 557.886847332428
0.000129 559.145249355695
0.000129 559.140400834744
0.00013 560.38250229691
0.00013 560.377692160122
0.000131 561.603722801254
0.000131 561.598950332806
0.000132 562.809134912574
0.000132 562.804399415471
0.000133 563.998957847342
0.000133 563.994258642805
0.000134 565.173406123824
0.000134 565.1687425507
0.000135 566.332689687009
0.000135 566.328061101201
0.000136 567.477014029474
0.000136 567.472419803394
0.000137 568.606580308332
0.000137 568.60201983038
0.000138 569.721585458429
0.000138 569.717058132483
0.000139 570.82222230191
0.000139 570.817727546842
0.00014 571.908679654298
0.00014 571.904216903505
0.000141 572.981142427232
0.000141 572.976711128185
0.000142 574.039791727958
0.000142 574.035391341769
0.000143 575.084804955721
0.000143 575.080434956725
0.000144 576.116355895157
0.000144 576.112015770508
0.000145 577.132298506315
0.000145 577.12800049305
0.000146 578.135046925113
0.000146 578.130777838494
0.000147 579.124861197026
0.000147 579.120620443329
0.000148 580.101901217144
0.000148 580.097688331997
0.000149 581.066323811474
0.000149 581.0621383415
0.00015 582.018282696466
0.00015 582.014124198964
0.000151 582.957928554256
0.000151 582.953796596887
0.000152 583.885000063142
0.000152 583.880905494203
0.000153 584.798039608894
0.000153 584.793972808471
0.000154 585.699231637286
0.000154 585.695190062435
0.000155 586.588713058946
0.000155 586.584696316786
0.000156 587.466620237264
0.000156 587.462627943853
0.000157 588.33308691042
0.000157 588.329118690507
0.000158 589.188244252468
0.000158 589.18429973925
0.000159 590.032220932663
0.000159 590.028299767545
0.00016 590.865143173086
0.00016 590.86124500546
0.000161 591.687134804644
0.000161 591.683259291663
0.000162 592.498317321475
0.000162 592.494464127844
0.000163 593.298809933826
0.000163 593.294978731599
0.000164 594.088729619456
0.000164 594.084920087831
0.000165 594.868191173597
0.000165 594.864402998729
0.000166 595.637307257539
0.000166 595.633540132352
0.000167 596.396188445872
0.000167 596.39244206988
0.000168 597.144943272428
0.000168 597.141217351562
0.000169 597.883678274977
0.000169 597.879972521413
0.00017 598.612498038696
0.00017 598.608812170696
0.000171 599.331505238475
0.000171 599.327838980226
0.000172 600.040800680076
0.000172 600.037153761538
0.000173 600.740483340188
0.000173 600.736855496947
0.000174 601.430650405425
0.000174 601.42704137855
0.000175 602.111397310277
0.000175 602.107806846179
0.000176 602.782817774064
0.000176 602.779245624362
0.000177 603.445003836921
0.000177 603.441449758311
0.000178 604.098045894835
0.000178 604.094509648965
0.000179 604.742032733775
0.000179 604.738514087122
0.00018 605.377051562936
0.00018 605.373550286686
0.000181 606.003188047123
0.000181 605.999703917057
0.000182 606.620526338307
0.000182 606.617059134688
0.000183 607.22914910637
0.000183 607.225698613836
0.000184 607.829137569067
0.000184 607.825703576525
0.000185 608.420571521237
0.000185 608.417153821759
0.000186 609.003529363263
0.000186 609.000127753991
0.000187 609.578088128833
0.000187 609.57470241088
0.000188 610.143725985897
0.000188 610.140363820258
0.000189 610.70050308097
0.000189 610.697156967155
0.00019 611.249089950664
0.00019 611.245766372994
0.000191 611.788448640333
0.000191 611.785141183669
0.000192 612.319809665015
0.000192 612.316517083382
0.000193 612.8432411272
0.000193 612.839963244345
0.000194 613.358811007257
0.000194 613.355547650287
0.000195 613.866586119852
0.000195 613.863337119159
0.000196 614.366632136097
0.000196 614.363397325282
0.000197 614.859013605172
0.000197 614.855792820976
0.000198 615.343793975422
0.000198 615.340587057655
0.000199 615.821035614956
0.000199 615.817842406431
0.0002 616.290799831754
0.0002 616.287620178221
0.000201 616.753146893298
0.000201 616.749980643376
0.000202 617.208136045736
0.000202 617.20498305086
0.000203 617.65582553261
0.000203 617.652685646961
0.000204 618.096272613125
0.000204 618.093145693577
0.000205 618.529533580011
0.000205 618.526419486074
0.000206 618.955663776962
0.000206 618.952562370722
0.000207 619.374717615671
0.000207 619.371628761742
0.000208 619.786748592479
0.000208 619.783672157945
0.000209 620.191809304636
0.000209 620.188745159005
0.00021 620.589951466196
0.00021 620.586899481346
0.000211 620.976245127822
0.000211 620.973602277469
0.000212 621.306396549967
0.000212 621.303849869494
0.000213 621.627814892705
0.000213 621.625294250269
0.000214 621.942318273822
0.000214 621.939815150912
0.000215 622.250187835019
0.000215 622.24770918844
0.000216 622.551659213801
0.000216 622.54919036722
0.000217 622.847416475602
0.000217 622.844964100762
0.000218 623.137043256426
0.000218 623.134600322905
0.000219 623.421250115548
0.000219 623.418816039653
0.00022 623.700074052351
0.00022 623.697648743323
0.000221 623.97291959801
0.000221 623.970517947391
0.000222 624.239475805629
0.000222 624.237083411923
0.000223 624.500791039071
0.000223 624.498407083609
0.000224 624.756899308548
0.000224 624.754523707203
0.000225 625.007834746874
0.000225 625.005467416972
0.000226 625.253630886373
0.000226 625.251271746666
0.000227 625.494320668618
0.000227 625.491969639259
0.000228 625.729427581704
0.000228 625.727092169336
0.000229 625.959290663861
0.000229 625.956963332693
0.00023 626.184158852654
0.00023 626.181839368735
0.000231 626.404062572509
0.000231 626.401750861555
0.000232 626.61903186222
0.000232 626.616727851216
0.000233 626.829096225086
0.000233 626.826799842271
0.000234 627.03428463724
0.000234 627.03199581208
0.000235 627.234625555795
0.000235 627.232344218962
0.000236 627.43014692681
0.000236 627.427873010163
0.000237 627.620876193091
0.000237 627.618609629652
0.000238 627.806840301817
0.000238 627.804581025757
0.000239 627.988065712015
0.000239 627.985813658631
0.00024 628.164578401861
0.00024 628.162333507555
0.000241 628.335480173366
0.000241 628.333257189842
0.000242 628.501585736666
0.000242 628.49936985163
0.000243 628.663084094735
0.000243 628.66087512506
0.000244 628.819999094353
0.000244 628.817796981029
0.000245 628.972354260736
0.000245 628.970158945768
0.000246 629.120172680853
0.000246 629.117984107241
0.000247 629.263477009683
0.000247 629.261295121411
0.000248 629.402289476346
0.000248 629.400114218363
0.000249 629.536631890112
0.000249 629.534463208319
0.00025 629.666525646273
0.00025 629.664363487506
0.000251 629.791991731903
0.000251 629.789836043924
0.000252 629.913050731494
0.000252 629.910901462972
0.000253 630.029722832476
0.000253 630.027579932977
0.000254 630.142027830626
0.000254 630.139891250599
0.000255 630.24998513536
0.000255 630.247854826123
0.000256 630.353613774923
0.000256 630.351489688653
0.000257 630.452932401467
0.000257 630.450814491187
0.000258 630.54795929603
0.000258 630.545847515596
0.000259 630.63871237341
0.000259 630.636606677501
0.00026 630.72520918694
0.00026 630.723109531046
0.000261 630.807466933171
0.000261 630.805373273583
0.000262 630.885502456448
0.000262 630.883414750247
0.000263 630.959332253413
0.000263 630.957250458459
0.000264 631.028972477397
0.000264 631.026896552321
0.000265 631.094438942735
0.000265 631.092368846927
0.000266 631.155747128995
0.000266 631.153682822595
0.000267 631.212912185114
0.000267 631.210853629003
0.000268 631.265948933459
0.000268 631.26389608925
0.000269 631.314871873802
0.000269 631.312824703833
0.00027 631.359695187221
0.00027 631.357653654542
0.000271 631.400432739918
0.000271 631.398396808287
0.000272 631.437098086961
0.000272 631.435067720835
0.000273 631.469704475962
0.000273 631.467679640486
0.000274 631.498264850663
0.000274 631.496245511666
0.000275 631.522791854472
0.000275 631.520777978459
0.000276 631.543163946099
0.000276 631.541171081095
0.000277 631.559533950041
0.000277 631.557546447897
0.000278 631.571937747908
0.000278 631.569955550712
0.000279 631.580386663053
0.000279 631.578409740091
0.00028 631.584891762885
0.00028 631.582920084078
0.000281 631.585463835426
0.000281 631.583497371323
0.000282 631.582113392346
0.000282 631.580152114121
0.000283 631.574850671959
0.000283 631.572894551401
0.000284 631.56368564214
0.000284 631.561734651651
0.000285 631.548628003203
0.000285 631.546682115791
0.000286 631.52968719071
0.000286 631.527746379981
0.000287 631.506724177794
0.000287 631.504773471961
0.000288 631.47983724109
0.000288 631.477891630648
0.000289 631.449064233356
0.000289 631.447123684142
0.00029 631.414413662714
0.00029 631.412478150595
0.000291 631.375893794297
0.000291 631.373963295724
0.000292 631.333512643328
0.000292 631.331587135326
0.000293 631.287277977593
0.000293 631.28535743776
0.000294 631.237197319875
0.000294 631.235281726379
0.000295 631.183277950337
0.000295 631.181367281907
0.000296 631.125526908856
0.000296 631.123621144784
0.000297 631.063950997321
0.000297 631.062050117455
0.000298 630.998556781878
0.000298 630.99666076662
0.000299 630.929350595144
0.000299 630.927459425443
0.0003 630.856338538362
0.0003 630.854452195716
0.000301 630.779526483538
0.000301 630.777644949986
0.000302 630.698920075514
0.000302 630.697043333635
0.000303 630.614524734021
0.000303 630.612652766933
0.000304 630.526345655684
0.000304 630.524478447036
0.000305 630.434387815989
0.000305 630.432525349963
0.000306 630.338655971218
0.000306 630.336798232525
0.000307 630.239154660344
0.000307 630.237301634222
0.000308 630.135888206895
0.000308 630.134039879104
0.000309 630.028860720778
0.000309 630.027017077601
0.00031 629.918076100072
0.00031 629.916237128313
0.000311 629.803538032791
0.000311 629.801703719769
0.000312 629.685249998608
0.000312 629.683420332158
0.000313 629.563215270553
0.000313 629.561390239025
0.000314 629.437436916676
0.000314 629.435616508933
0.000315 629.307917801684
0.000315 629.306102007098
0.000316 629.174660588544
0.000316 629.172849396996
0.000317 629.037667740059
0.000317 629.03586114194
0.000318 628.896941520415
0.000318 628.895139506621
0.000319 628.752483996702
0.000319 628.750686558634
0.00032 628.604297040404
0.00032 628.602504169969
0.000321 628.452382328868
0.000321 628.450594018474
0.000322 628.296741346737
0.000322 628.294957589298
0.000323 628.137036084304
0.000323 628.135245768031
0.000324 627.972907808375
0.000324 627.971122543052
0.000325 627.805035333182
0.000325 627.803254652468
0.000326 627.633419001913
0.000326 627.631642900779
0.000327 627.45805943254
0.000327 627.456287906459
0.000328 627.278957058331
0.000328 627.27719010328
0.000329 627.096112129136
0.000329 627.094349741593
0.00033 626.909524712659
0.00033 626.907766889605
0.000331 626.719194695704
0.000331 626.717441434623
0.000332 626.525121785406
0.000332 626.523373084282
0.000333 626.327305510435
0.000333 626.325561367757
0.000334 626.125745222185
0.000334 626.124005636944
0.000335 625.920440095945
0.000335 625.918705067635
0.000336 625.711389132046
0.000336 625.709658660666
0.000337 625.498385991298
0.000337 625.496654885374
0.000338 625.281170750723
0.000338 625.279444545009
0.000339 625.060195796066
0.000339 625.058474175453
0.00034 624.835459159295
0.00034 624.83374212585
0.000341 624.606959019635
0.000341 624.605246575932
0.000342 624.374693387954
0.000342 624.372985537079
0.000343 624.13866010781
0.000343 624.136956853357
0.000344 623.898856856473
0.000344 623.897158202548
0.000345 623.653993065062
0.000345 623.652289705119
0.000346 623.405150971126
0.000346 623.403452397357
0.000347 623.15251165448
0.000347 623.150817747233
0.000348 622.896072085953
0.000348 622.894382851512
0.000349 622.635404705404
0.000349 622.633715482962
0.00035 622.370576637443
0.00035 622.368892374125
0.000351 622.101929467244
0.000351 622.100249923997
0.000352 621.827955923365
0.000352 621.82627249871
0.000353 621.549770489342
0.000353 621.548087342089
0.000354 621.266599420138
0.000354 621.264912535909
0.000355 620.976989596815
0.000355 620.975279398114
0.000356 620.640561945576
0.000356 620.638639422703
0.000357 620.294664384544
0.000357 620.292752619271
0.000358 619.9443641793
0.000358 619.942458957384
0.000359 619.589655450275
0.000359 619.587756788009
0.00036 619.230536414812
0.00036 619.228644329082
0.000361 618.86700515866
0.000361 618.865119666948
0.000362 618.499059637397
0.000362 618.497180757782
0.000363 618.126697677852
0.000363 618.124825429011
0.000364 617.749916979527
0.000364 617.748051380734
0.000365 617.368715116001
0.000365 617.366856187134
0.000366 616.983089536352
0.000366 616.98123729789
0.000367 616.593037566556
0.000367 616.591192039584
0.000368 616.198556410896
0.000368 616.196717617103
0.000369 615.799643153365
0.000369 615.797811115046
0.00037 615.396294759063
0.00037 615.394469499124
0.000371 614.988508075601
0.000371 614.986689617559
0.000372 614.576279834499
0.000372 614.574468202483
0.000373 614.159606652582
0.000373 614.157801871337
0.000374 613.738485033383
0.000374 613.736687128268
0.000375 613.312911368537
0.000375 613.31112036553
0.000376 612.882881939181
0.000376 612.881097864884
0.000377 612.448392917357
0.000377 612.446615798993
0.000378 612.009440367414
0.000378 612.007670232829
0.000379 611.566020247406
0.000379 611.564257125076
0.00038 611.117876166927
0.00038 611.11611649916
0.000381 610.664526394395
0.000381 610.662774485652
0.000382 610.206690638812
0.000382 610.204945859696
0.000383 609.744094255318
0.000383 609.742353115726
0.000384 609.276260626477
0.000384 609.274527384586
0.000385 608.803921353562
0.000385 608.802195364845
0.000386 608.32707113221
0.000386 608.325352429706
0.000387 607.845705241013
0.000387 607.843993858405
0.000388 607.359818866046
0.000388 607.358114837666
0.000389 606.869407102328
0.000389 606.867710463157
0.00039 606.374464955285
0.00039 606.37277574096
0.000391 605.874987342223
0.000391 605.873305589033
0.000392 605.370969093802
0.000392 605.369294838699
0.000393 604.862404955529
0.000393 604.860738236123
0.000394 604.349289589244
0.000394 604.34763044381
0.000395 603.831617574624
0.000395 603.829966042102
0.000396 603.309383410691
0.000396 603.307739530694
0.000397 602.782581517336
0.000397 602.780945330145
0.000398 602.251206236838
0.000398 602.249577783413
0.000399 601.715251835411
0.000399 601.713631157387
0.0004 601.174712504743
0.0004 601.173099644437
0.000401 600.62958236356
0.000401 600.627977363972
0.000402 600.079855459189
0.000402 600.078258364008
0.000403 599.525525769144
0.000403 599.523936622746
0.000404 598.966587202714
0.000404 598.965006050168
0.000405 598.403033602561
0.000405 598.401460489633
0.000406 597.834858746347
0.000406 597.833293719499
0.000407 597.262056348349
0.000407 597.260499454748
0.000408 596.684620061112
0.000408 596.683071348626
0.000409 596.102543477092
0.000409 596.101002994301
0.00041 595.515820130335
0.00041 595.514287926525
0.000411 594.924443498149
0.000411 594.922919623325
0.000412 594.328407002811
0.000412 594.32689150769
0.000413 593.72770401327
0.000413 593.726196949293
0.000414 593.122327846878
0.000414 593.120829266208
0.000415 592.512271771134
0.000415 592.510781726661
0.000416 591.89752900544
0.000416 591.896047550781
0.000417 591.278092722877
0.000417 591.276619912384
0.000418 590.653956051997
0.000418 590.652491940757
0.000419 590.02511207863
0.000419 590.02365672247
0.00042 589.391553847716
0.00042 589.390107303203
0.000421 588.753274365142
0.000421 588.75183668959
0.000422 588.110266599609
0.000422 588.10883785108
0.000423 587.462523484515
0.000423 587.461103721822
0.000424 586.810037919849
0.000424 586.80862720256
0.000425 586.152802774118
0.000425 586.151401162556
0.000426 585.490810886277
0.000426 585.489418441531
0.000427 584.8240550677
0.000427 584.822671851618
0.000428 584.152528104149
0.000428 584.151154179349
0.000429 583.475550043704
0.000429 583.474182569193
0.00043 582.792662227428
0.00043 582.791305383585
0.000431 582.104978138945
0.000431 582.103630820531
0.000432 581.412489422782
0.000432 581.411151695551
0.000433 580.715188879389
0.000433 580.713860809873
0.000434 580.013069297043
0.000434 580.011750952556
0.000435 579.306123453988
0.000435 579.304814902627
0.000436 578.5943441206
0.000436 578.593045431252
0.000437 577.877724061575
0.000437 577.876435303916
0.000438 577.156256038141
0.000438 577.154977282639
0.000439 576.42993281029
0.000439 576.428664128212
0.00044 575.697657600354
0.00044 575.696397264844
0.000441 574.959863879348
0.000441 574.95861450156
0.000442 574.217190881403
0.000442 574.21595183479
0.000443 573.469630797254
0.000443 573.468402156297
0.000444 572.717176529266
0.000444 572.715958369251
0.000445 571.959820993594
0.000445 571.958613390614
0.000446 571.197557122614
0.000446 571.196360153571
0.000447 570.430377867359
0.000447 570.429191609967
0.000448 569.658276200002
0.000448 569.65710073279
0.000449 568.881245116354
0.000449 568.880080518664
0.00045 568.099277638393
0.00045 568.098123990386
0.000451 567.312366816819
0.000451 567.311224199476
0.000452 566.520505733643
0.000452 566.519374228767
0.000453 565.723687504799
0.000453 565.722567195015
0.000454 564.921905282784
0.000454 564.920796251542
0.000455 564.115152259332
0.000455 564.114054590908
0.000456 563.303421668113
0.000456 563.302335447612
0.000457 562.486706787464
0.000457 562.485632100818
0.000458 561.665000943148
0.000458 561.663937877123
0.000459 560.838297511146
0.000459 560.837246153336
0.00046 560.006589920473
0.00046 560.005550359307
0.000461 559.169871656035
0.000461 559.168843980773
0.000462 558.328136261501
0.000462 558.327120562241
0.000463 557.481377342228
0.000463 557.480373709902
0.000464 556.629588568194
0.000464 556.62859709457
0.000465 555.77276367698
0.000465 555.771784454661
0.000466 554.910896476773
0.000466 554.9099295992
0.000467 554.043980849405
0.000467 554.043026410857
0.000468 553.172010753427
0.000468 553.171068849019
0.000469 552.294980227208
0.000469 552.294050952892
0.00047 551.412883392071
0.00047 551.411966844637
0.000471 550.525714455465
0.000471 550.524810732537
0.000472 549.633467714156
0.000472 549.632576914199
0.000473 548.733685155161
0.000473 548.732806716005
0.000474 547.827515672884
0.000474 547.826652024161
0.000475 546.916254476571
0.000475 546.915404128262
0.000476 545.999894943941
0.000476 545.999057997622
0.000477 545.077059858163
0.000477 545.076234388002
0.000478 544.146291957872
0.000478 544.145483710223
0.000479 543.210416873592
0.000479 543.209622422557
0.00048 542.269426817282
0.00048 542.268646267416
0.000481 541.323317655041
0.000481 541.322551111693
0.000482 540.372085380971
0.000482 540.371332950285
0.000483 539.415726120734
0.000483 539.414987909646
0.000484 538.454236135136
0.000484 538.453512251368
0.000485 537.487611823746
0.000485 537.486902375805
0.000486 536.515849728542
0.000486 536.515154825717
0.000487 535.538946537598
0.000487 535.538266289952
0.000488 534.556899088788
0.000488 534.556233607159
0.000489 533.569704373539
0.000489 533.569053769534
0.00049 532.577359540602
0.00049 532.576723926591
0.000491 531.579861899861
0.000491 531.579241388973
0.000492 530.577208926169
0.000492 530.576603632288
0.000493 529.569398263222
0.000493 529.568808300982
0.000494 528.556427727454
0.000494 528.555853212232
0.000495 527.538295311971
0.000495 527.537736359883
0.000496 526.514999190514
0.000496 526.514455918405
0.000497 525.486537721446
0.000497 525.486010246889
0.000498 524.452909451776
0.000498 524.452397893061
0.000499 523.414113121211
0.000499 523.413617597342
0.0005 522.370147666233
0.0005 522.369668296918
0.000501 521.321012224214
0.000501 521.320549129857
0.000502 520.266706137546
0.000502 520.266259439242
0.000503 519.207228957816
0.000503 519.206798777341
0.000504 518.142580449994
0.000504 518.142166909796
0.000505 517.072532322057
0.000505 517.072135058532
0.000506 515.995364499229
0.000506 515.994986892968
0.000507 514.913033577225
0.000507 514.912673021696
0.000508 513.825537673325
0.000508 513.825194293242
0.000509 512.732877871292
0.000509 512.732551791983
0.00051 511.635055495083
0.00051 511.634746842483
0.000511 510.532072113212
0.000511 510.531781013848
0.000512 509.423929543133
0.000512 509.423656124114
0.000513 508.310629855645
0.000513 508.310374244651
0.000514 507.192175379319
0.000514 507.19193770459
0.000515 506.06856870495
0.000515 506.068349095268
0.000516 504.939812690023
0.000516 504.939611274708
0.000517 503.805910463208
0.000517 503.805727372096
0.000518 502.666865428861
0.000518 502.666700792296
0.000519 501.522681271559
0.000519 501.522535220377
0.00052 500.37336196064
0.00052 500.373234626156
0.000521 499.218911754765
0.000521 499.218803268756
0.000522 498.059335206497
0.000522 498.059245701188
0.000523 496.893089391187
0.000523 496.893021076389
0.000524 495.718926443111
0.000524 495.718881709793
0.000525 494.539672027295
0.000525 494.539646742849
0.000526 493.355328051649
0.000526 493.355322348935
0.000527 492.165901007756
0.000527 492.165915019977
0.000528 490.971397710361
0.000528 490.971431571038
0.000529 489.77182530197
0.000529 489.771879144929
0.00053 488.567191257468
0.00053 488.567265216819
0.000531 487.357503388733
0.000531 487.357597598849
0.000532 486.142769849256
0.000532 486.142884444757
0.000533 484.922999138767
0.000533 484.923134254494
0.000534 483.698200107856
0.000534 483.698355878853
0.000535 482.466321066748
0.000535 482.466501223732
0.000536 481.229353554327
0.000536 481.229554814105
0.000537 479.987395986366
0.000537 479.987618337394
0.000538 478.74045872023
0.000538 478.740702297529
0.000539 477.488552634168
0.000539 477.488817572836
0.00054 476.231688985422
0.00054 476.231975420609
0.000541 474.969879414783
0.000541 474.970187481671
0.000542 473.70058980669
0.000542 473.700924943853
0.000543 472.424703479156
0.000543 472.425063383895
0.000544 471.143932174876
0.000544 471.144314169935
0.000545 469.858286709504
0.000545 469.858690928549
0.000546 468.56778111635
0.000546 468.56820769292
0.000547 467.270643383314
0.000547 467.27109619124
0.000548 465.96838788722
0.000548 465.968863854115
0.000549 464.66132851342
0.000549 464.661827261138
0.00055 463.342404157919
0.00055 463.34294127874
0.000551 462.017605117387
0.000551 462.018167380047
0.000552 460.68636046908
0.000552 460.686949911312
0.000553 459.491424899115
0.000553 459.491644781969
0.000554 458.660960319054
0.000554 458.660768290694
0.000555 457.664772767597
0.000555 457.664805455214
0.000556 456.663135967921
0.000556 456.663186259842
0.000557 455.655826983955
0.000557 455.655895080467
0.000558 454.642830372813
0.000558 454.642916476285
0.000559 453.624130882049
0.000559 453.624235196941
0.00056 452.599713456928
0.00056 452.599836189797
0.000561 451.569563247834
0.000561 451.569704607336
0.000562 450.533665617793
0.000562 450.533825814693
0.000563 449.492006150142
0.000563 449.492185397317
0.000564 448.444570656326
0.000564 448.444769168769
0.000565 447.39134518383
0.000565 447.391563178652
0.000566 446.332316024243
0.000566 446.332553720677
0.000567 445.267469721462
0.000567 445.267727340865
0.000568 444.19679308003
0.000568 444.197070845884
0.000569 443.120273173616
0.000569 443.12057131153
0.00057 442.037897353628
0.00057 442.038216091334
0.000571 440.949653257966
0.000571 440.949992825322
0.000572 439.855528819919
0.000572 439.855889448908
0.000573 438.755512277201
0.000573 438.755894201922
0.000574 437.649592181123
0.000574 437.649995637794
0.000575 436.537757405911
0.000575 436.538182632863
0.000576 435.419997158166
0.000576 435.420444395836
0.000577 434.296300986468
0.000577 434.296770477393
0.000578 433.166658791111
0.000578 433.167150779922
0.000579 432.031060833996
0.000579 432.031575567409
0.00058 430.889497748658
0.00058 430.890035475465
0.000581 429.741960550434
0.000581 429.74252152149
0.000582 428.588440646777
0.000582 428.589025114992
0.000583 427.427139992413
0.000583 427.427753292076
0.000584 426.25722683551
0.000584 426.257868319831
0.000585 425.081322069418
0.000585 425.081987920176
0.000586 423.899415249585
0.000586 423.900105726865
0.000587 422.711500460225
0.000587 422.712215826041
0.000588 421.517572249087
0.000588 421.518312767353
0.000589 420.317625638706
0.000589 420.318391575224
0.00059 419.109946665635
0.00059 419.110744482035
0.000591 417.892742182948
0.000591 417.893571794548
0.000592 416.669531079146
0.000592 416.670387024303
0.000593 415.440305350322
0.000593 415.441187900578
0.000594 414.20506326371
0.000594 414.205972692287
0.000595 412.963597495057
0.000595 412.96453829158
0.000596 411.7111860676
0.000596 411.712162743993
0.000597 410.452789458094
0.000597 410.453793932412
0.000598 409.188400535113
0.000598 409.189433084983
0.000599 407.918021130549
0.000599 407.919082035033
0.0006 406.641653693917
};
\end{groupplot}

\end{tikzpicture}

\end{figure}

\begin{figure}[H]
    \caption{Метод Рунге-Кутта четвёртого порядка}\label{img:plot02}
    % This file was created by tikzplotlib v0.9.2.
\begin{tikzpicture}[scale=1.75]

\definecolor{color0}{rgb}{0.12156862745098,0.466666666666667,0.705882352941177}
\definecolor{color1}{rgb}{1,0.498039215686275,0.0549019607843137}
\definecolor{color2}{rgb}{0.172549019607843,0.627450980392157,0.172549019607843}
\definecolor{color3}{rgb}{0.83921568627451,0.152941176470588,0.156862745098039}
\definecolor{color4}{rgb}{0.580392156862745,0.403921568627451,0.741176470588235}
\definecolor{color5}{rgb}{0.549019607843137,0.337254901960784,0.294117647058824}
\definecolor{color6}{rgb}{0.890196078431372,0.466666666666667,0.76078431372549}
\definecolor{color7}{rgb}{0.737254901960784,0.741176470588235,0.133333333333333}
\definecolor{color8}{rgb}{0.0901960784313725,0.745098039215686,0.811764705882353}

\begin{axis}[
    x label style={at={(axis description cs:0.5,-0.05)},anchor=north},
    y label style={at={(axis description cs:-0.005,.5)},rotate=0,anchor=south},
    tick label style={font=\scriptsize},
tick align=outside,
tick pos=left,
x grid style={white!69.0196078431373!black},
xlabel={\footnotesize Время, c},
xmin=-0.9, xmax=18.9,
xtick style={color=black},
y grid style={white!69.0196078431373!black},
ylabel={\footnotesize Температура, K},
ymin=263.698013348176, ymax=1062.34171968815,
ytick style={color=black}
]
\addplot [semithick, color0]
table {%
0 300
1 599.930543206081
2 691.9724624112
3 755.219922274379
4 802.585678547953
5 839.990634877678
6 870.491502282562
7 895.903060044576
8 917.403896652301
9 935.808788017152
10 951.707409780728
11 965.541691793125
12 977.652040966931
13 988.30650388706
14 997.719990901908
15 1006.06741519652
16 1013.49294918595
17 1020.11671568309
18 1026.03973303634
};
\addplot [semithick, color1]
table {%
0 300
1 333.394658561331
2 373.621175240765
3 411.607515582236
4 445.372628097261
5 474.865027797009
6 500.532928665687
7 522.903946782287
8 542.463832064905
9 559.629009203146
10 574.74807143863
11 588.111018799887
12 599.959371789716
13 610.495128531151
14 619.888160968252
15 628.282161800075
16 635.799383793026
17 642.544414154532
18 648.607190570679
};
\addplot [semithick, color2]
table {%
0 300
1 303.748155297563
2 311.837753721511
3 322.88949514572
4 335.428999399817
5 348.423411651856
6 361.244372935233
7 373.53798589785
8 385.121248827281
9 395.914212973978
10 405.897875124974
11 415.088441535676
12 423.521657315748
13 431.24328137508
14 438.303322283714
15 444.752587470037
16 450.640662431259
17 456.014776662865
18 460.91921931379
};
\addplot [semithick, color3]
table {%
0 300
1 300.423637009955
2 301.754098259009
3 304.195142744808
4 307.669333593863
5 311.954261235521
6 316.795947421975
7 321.96631863137
8 327.282093776014
9 332.605048411659
10 337.835465279259
11 342.904212149289
12 347.765525586863
13 352.391084430106
14 356.765372344197
15 360.88213939271
16 364.741740056593
17 368.349147764878
18 371.712482827168
};
\addplot [semithick, color4]
table {%
0 300
1 300.048176909111
2 300.247178965463
3 300.708916019098
4 301.509810345145
5 302.674558766991
6 304.183404153429
7 305.988412528113
8 308.028860705083
9 310.242114190161
10 312.569973073417
11 314.961684614216
12 317.374869883059
13 319.775294668952
14 322.136082089848
15 324.436716251647
16 326.66202564864
17 328.80123893999
18 330.847151480283
};
\addplot [semithick, color5]
table {%
0 300
1 300.005508692328
2 300.033735156366
3 300.113107857459
4 300.276253870814
5 300.551552015656
6 300.957456062218
7 301.500724772732
8 302.177610051874
9 302.976499554865
10 303.880829309814
11 304.871611467812
12 305.929340539841
13 307.035282316754
14 308.172253209834
15 309.325021458159
16 310.480448475988
17 311.62746339775
18 312.756938484995
};
\addplot [semithick, color6]
table {%
0 300
1 300.000632947548
2 300.004506352159
3 300.017314923332
4 300.047811587871
5 300.1065686021
6 300.204149303033
7 300.3494204995
8 300.548452558754
9 300.804106182089
10 301.116176561166
11 301.481883409194
12 301.896510343444
13 302.354052686887
14 302.84779183658
15 303.370760463483
16 303.916092684948
17 304.477269671099
18 305.048278128466
};
\addplot [semithick, white!49.8039215686275!black]
table {%
0 300
1 300.000073043363
2 300.000592912477
3 300.002570284441
4 300.007927287313
5 300.019548710994
6 300.041063794374
7 300.076433174691
8 300.129472428116
9 300.203432294587
10 300.300707679604
11 300.422695786813
12 300.569786682182
13 300.74145131431
14 300.936388585274
15 301.15269831369
16 301.388055619189
17 301.63987105218
18 301.905428073458
};
\addplot [semithick, color7]
table {%
0 300
1 300.000009424732
2 300.000085451141
3 300.000410611901
4 300.001393451331
5 300.003754304339
6 300.008558428194
7 300.017178503708
8 300.031193763164
9 300.052250814681
10 300.081916721543
11 300.121550197918
12 300.172206987134
13 300.234585243203
14 300.309008790416
15 300.39544137121
16 300.493523088033
17 300.602620368568
18 300.721882061609
};
\addplot [semithick, color8]
table {%
0 300
1 300.000001095637
2 300.000011030567
3 300.000058483705
4 300.000217637626
5 300.000639169485
6 300.001579190726
7 300.003416682157
8 300.006652933153
9 300.011891582925
10 300.019802732067
11 300.031077655875
12 300.046381473912
13 300.066310153056
14 300.091356249845
15 300.121885597125
16 300.158125230584
17 300.200161480095
18 300.247946354757
};
\addplot [semithick, color0]
table {%
0 300
1 300.000000127781
2 300.000001414624
3 300.000008205117
4 300.000033236718
5 300.000105735827
6 300.000281656647
7 300.000654040095
8 300.001360963928
9 300.002588897624
10 300.004570269784
11 300.007575289226
12 300.01189913321
13 300.017846282818
14 300.025713987382
15 300.035776655747
16 300.048272546157
17 300.063393605829
18 300.08127881056
};
\addplot [semithick, color1]
table {%
0 300
1 300.000000014943
2 300.000000180487
3 300.000001137038
4 300.000004981881
5 300.00001707383
6 300.000048804499
7 300.000121151036
8 300.00026851188
9 300.000542126315
10 300.001012364649
11 300.001769345747
12 300.002921627709
13 300.004593037093
14 300.006917974041
15 300.010035707484
16 300.01408424304
17 300.019194319523
18 300.025483995864
};
\addplot [semithick, color2]
table {%
0 300
1 300.000000001569
2 300.000000020678
3 300.000000141175
4 300.000000667762
5 300.000002461801
6 300.000007543417
7 300.000020005856
8 300.000047216726
9 300.000101196155
10 300.000199991421
11 300.000368833643
12 300.000640874176
13 300.001057351238
14 300.001667115872
15 300.002525531442
16 300.003692835969
17 300.005232110343
18 300.007207023387
};
\addplot [semithick, color3]
table {%
0 300
1 300.000000000178
2 300.000000002611
3 300.000000019182
4 300.000000097151
5 300.000000382279
6 300.000001246696
7 300.000003509274
8 300.000008767194
9 300.000019838339
10 300.00004128929
11 300.00007999953
12 300.000145695377
13 300.000251381613
14 300.000413605889
15 300.000652508645
16 300.000991635493
17 300.001457515163
18 300.002079029548
};
\addplot [semithick, color4]
table {%
0 300
1 300.000000000015
2 300.000000000323
3 300.000000002582
4 300.000000013968
5 300.000000058458
6 300.000000202227
7 300.000000602344
8 300.000001588624
9 300.000003786269
10 300.000008281914
11 300.000016828244
12 300.000032074254
13 300.000057801122
14 300.00009913927
15 300.000162741324
16 300.000256888487
17 300.000391513759
18 300.00057813321
};
\addplot [semithick, color5]
table {%
0 300
1 299.999999999995
2 300.000000000035
3 300.000000000341
4 300.000000001983
5 300.000000008816
6 300.000000032263
7 300.000000101415
8 300.000000281669
9 300.000000705526
10 300.000001618692
11 300.000003443339
12 300.000006858116
13 300.000012891889
14 300.000023025445
15 300.00003929324
16 300.000064375988
17 300.00010167474
18 300.000155358162
};
\addplot [semithick, color6]
table {%
0 300
1 299.999999999993
2 299.999999999997
3 300.000000000039
4 300.000000000273
5 300.000000001308
6 300.000000005067
7 300.000000016779
8 300.000000048974
9 300.000000128661
10 300.000000309052
11 300.000000687129
12 300.000001428023
13 300.000002796544
14 300.000005195316
15 300.000009208003
16 300.000015645212
17 300.000025589865
18 300.000040438506
};
\end{axis}

\end{tikzpicture}

\end{figure}

\noindent$\blacksquare$

График зависимости $I(t)$ при $R_k + R_p = 0$. Обратить внимание на то, что в этом случае колебания тока будут не затухающими.

Изменим параметры:
\begin{itemize}
    \item $L_{\text{э}}  =$ 0 см;
    \item $R_k  =$ 0 Ом;
\end{itemize}
\begin{figure}[H]
    \caption{Графики при $R_k + R_p = 0$}\label{img:plot03}
    % This file was created by tikzplotlib v0.9.2.
\begin{tikzpicture}[scale=1.75]

\definecolor{color0}{rgb}{0.12156862745098,0.466666666666667,0.705882352941177}
\definecolor{color1}{rgb}{1,0.498039215686275,0.0549019607843137}
\definecolor{color2}{rgb}{0.172549019607843,0.627450980392157,0.172549019607843}
\definecolor{color3}{rgb}{0.83921568627451,0.152941176470588,0.156862745098039}
\definecolor{color4}{rgb}{0.580392156862745,0.403921568627451,0.741176470588235}
\definecolor{color5}{rgb}{0.549019607843137,0.337254901960784,0.294117647058824}
\definecolor{color6}{rgb}{0.890196078431372,0.466666666666667,0.76078431372549}
\definecolor{color7}{rgb}{0.737254901960784,0.741176470588235,0.133333333333333}
\definecolor{color8}{rgb}{0.0901960784313725,0.745098039215686,0.811764705882353}

\begin{axis}[
    x label style={at={(axis description cs:0.5,-0.05)},anchor=north},
    y label style={at={(axis description cs:-0.005,.5)},rotate=0,anchor=south},
    tick label style={font=\scriptsize},
tick align=outside,
tick pos=left,
x grid style={white!69.0196078431373!black},
xlabel={\footnotesize Время, c},
xmin=-2.85, xmax=59.85,
xtick style={color=black},
y grid style={white!69.0196078431373!black},
ylabel={\footnotesize Температура, K},
ymin=272.068074094823, ymax=1034.6634250431,
ytick style={color=black}
]
\addplot [semithick, color0]
table {%
0 1000
1 950.745424678849
2 905.054434466168
3 862.517492755391
4 822.859416700999
5 785.861186722951
6 751.335376988762
7 719.115496550553
8 689.050474158589
9 661.001458084649
10 634.839796788339
11 610.445675555632
12 587.707140330681
13 566.519360555963
14 546.784044766341
15 528.408956722855
16 511.30749961795
17 495.39834783531
18 480.60511322053
19 466.856037601416
20 454.083706399255
21 442.224780190496
22 431.219742376412
23 421.012661927667
24 411.550970642905
25 402.785254598873
26 394.669059546832
27 387.158709979676
28 380.213141496539
29 373.793745958253
30 367.864228782377
31 362.390477589834
32 357.340441300699
33 352.684018694001
34 348.392955401186
35 344.440748296638
36 340.802556279634
37 337.455116505999
38 334.376665217834
39 331.546862428571
40 328.946719840283
41 326.558531492909
42 324.36580676414
43 322.35320544881
44 320.50647474348
45 318.812388043407
46 317.258685523703
47 315.834016524842
48 314.527883795328
49 313.330589663433
50 312.233184217272
51 311.227415570495
52 310.305682281897
53 309.460987983195
54 308.686898252151
55 307.977499749625
56 307.32736162019
57 306.731499137927
};
\addplot [semithick, color1]
table {%
0 1000
1 952.833582165388
2 908.488389583496
3 866.870820407462
4 827.85400040178
5 791.301289134962
6 757.07628189113
7 725.046695271013
8 695.085778586217
9 667.072743033225
10 640.892798629214
11 616.437033858054
12 593.602232793842
13 572.290668023506
14 552.409884637302
15 533.872481197325
16 516.595889963595
17 500.502157425116
18 485.51772592091
19 471.573217221729
20 458.603219108034
21 446.546076112842
22 435.343685658904
23 424.941300800564
24 415.287340687753
25 406.333209715474
26 398.033126121493
27 390.343960563381
28 383.225084958945
29 376.638231626012
30 370.547362521906
31 364.918548171439
32 359.719855694305
33 354.92124520494
34 350.494473764225
35 346.413006013432
36 342.651930614473
37 339.187881652293
38 335.998964218501
39 333.064683482579
40 330.365876659704
41 327.884647394783
42 325.604302193016
43 323.509288632486
44 321.58513518902
45 319.818392584817
46 318.196576638319
47 316.70811264324
48 315.342281339811
49 314.08916656292
50 312.939604661398
51 311.88513578262
52 310.917957108814
53 310.030878118341
54 309.217277928266
55 308.471064755823
56 307.78663751686
57 307.158849560351
};
\addplot [semithick, color2]
table {%
0 1000
1 955.768306247379
2 913.853053440273
3 874.19958415281
4 836.743524820061
5 801.411032543054
6 768.121129763779
7 736.788330177289
8 707.324867764815
9 679.642398075708
10 653.653236927346
11 629.271241531195
12 606.412427438077
13 584.995392576594
14 564.941599880167
15 546.175555147424
16 528.624906342046
17 512.220483381572
18 496.896292580385
19 482.589476551165
20 469.240248014556
21 456.79180427363
22 445.190227850246
23 434.384377805862
24 424.325775482985
25 414.968487742763
26 406.269010200093
27 398.186152445988
28 390.680926784727
29 383.716441594366
30 377.25780004214
31 371.27200455235
32 365.727867136512
33 360.595925456076
34 355.848364299204
35 351.458942014979
36 347.402921359507
37 343.657004165153
38 340.19926924123
39 337.009112945121
40 334.067191919317
41 331.355367564396
42 328.856651902635
43 326.555154574871
44 324.436030798548
45 322.485430192783
46 320.690446443843
47 319.039067839446
48 317.520128742224
49 316.12326210174
50 314.838853121547
51 313.657994204413
52 312.572441296868
53 311.574571745374
54 310.657343762546
55 309.814257584717
56 309.039318383123
57 308.327000971233
};
\addplot [semithick, color3]
table {%
0 1000
1 958.448449386755
2 918.914193081055
3 881.341694636993
4 845.674432330267
5 811.854069554447
6 779.820069696909
7 749.509769537579
8 720.858766915139
9 693.801466251189
10 668.271667066092
11 644.203126527211
12 621.530061250465
13 600.187574979146
14 580.112010729055
15 561.241232007613
16 543.514840321064
17 526.874336966856
18 511.263236985208
19 496.627142613946
20 482.913782911481
21 470.073025508536
22 458.056865774252
23 446.819398054878
24 436.316773065612
25 426.507144983422
26 417.350611294507
27 408.809147988662
28 400.846542260485
29 393.428324472285
30 386.521700756031
31 380.095487283751
32 374.120046920298
33 368.567228692932
34 363.410310272022
35 358.623943458715
36 354.184102520162
37 350.068035100263
38 346.2542153622
39 342.72229898439
40 339.45307962932
41 336.428446528672
42 333.631342872148
43 331.045724744695
44 328.656520421201
45 326.449589893967
46 324.411684571506
47 322.530407144313
48 320.794171661675
49 319.192163902072
50 317.71430214789
51 316.35119849325
52 315.094120822696
53 313.934955599414
54 312.866171596032
55 311.88078469034
56 310.972323833838
57 310.134798284301
};
\addplot [semithick, color4]
table {%
0 1000
1 960.830362981604
2 923.44971933487
3 887.808236820798
4 853.855440651257
5 821.540292717141
6 790.811197253448
7 761.616009319611
8 733.902091317871
9 707.616430354312
10 682.705808233232
11 659.117006716147
12 636.797029444949
13 615.693324658899
14 595.753996870796
15 576.927999609369
16 559.165304614537
17 542.417045368155
18 526.635634631002
19 511.7748568759
20 497.789937296468
21 484.637589547924
22 472.276044630908
23 460.665063427062
24 449.765935383092
25 439.541465750706
26 429.955953646254
27 420.975163011677
28 412.566288349746
29 404.697916880295
30 397.339988527826
31 390.463754911622
32 384.041738273201
33 378.047691049371
34 372.456556587356
35 367.244431307219
36 362.388528450023
37 357.867143411287
38 353.659620550107
39 349.746321285212
40 346.108593239402
41 342.728740170614
42 339.58999242812
43 336.676477691796
44 333.973191786194
45 331.465969404864
46 329.141454629157
47 326.98707117591
48 324.990992356403
49 323.14211077229
50 321.430007811089
51 319.844923033199
52 318.377723563928
53 317.019873617839
54 315.763404289477
55 314.600883744998
56 313.525387944505
57 312.530472016056
};
\addplot [semithick, color5]
table {%
0 1000
1 962.954535377687
2 927.508826765944
3 893.620028008912
4 861.244549350034
5 830.338253336773
6 800.856617591108
7 772.754868271476
8 745.988094440257
9 720.511354066328
10 696.279779434285
11 673.248685792385
12 651.373683690152
13 630.610793216918
14 610.916557238802
15 592.248150473377
16 574.563481525598
17 557.821285575585
18 541.981206077292
19 527.00386448519
20 512.850917614433
21 499.485102732895
22 486.870270875393
23 474.971409166766
24 463.754653151281
25 453.18729026274
26 443.237755643814
27 433.875621544837
28 425.071581510584
29 416.797430506232
30 409.026042047844
31 401.731343294741
32 394.888288937151
33 388.472834578202
34 382.461910170481
35 376.833393929204
36 371.566087011819
37 366.639689131854
38 362.034775167095
39 357.732772731505
40 353.715940608667
41 349.967347892697
42 346.470853650273
43 343.211086903524
44 340.173426735568
45 337.343982336243
46 334.709572831525
47 332.257706773279
48 329.976561203225
49 327.854960243323
50 325.882353201961
51 324.048792219381
52 322.34490950519
53 320.761894244978
54 319.291469271364
55 317.925867607302
56 316.657808996593
57 315.480476538686
};
\addplot [semithick, color6]
table {%
0 1000
1 964.860056730858
2 931.159477011073
3 898.861539906332
4 867.928853213596
5 838.323504389502
6 810.007206073334
7 782.94143238012
8 757.087544172181
9 732.406903375834
10 708.860977564103
11 686.411436411698
12 665.020241450032
13 644.649730085227
14 625.262694318407
15 606.822454166715
16 589.292925486183
17 572.638681750371
18 556.825009318555
19 541.817955799837
20 527.584371250738
21 514.091942104673
22 501.309217900094
23 489.205631034691
24 477.751509915873
25 466.918085997248
26 456.677495284036
27 447.00277495669
28 437.867855802086
29 429.247551156754
30 421.117543058908
31 413.454366277864
32 406.23539084343
33 399.438803637143
34 393.043589534968
35 387.029512510631
36 381.377097023864
37 376.067609931923
38 371.083043079175
39 366.40609664183
40 362.020163235349
41 357.909312733215
42 354.058277698958
43 350.452439299425
44 347.077813546267
45 343.921037704005
46 340.969356705462
47 338.210609427169
48 335.633214696766
49 333.226156928974
50 330.978971314739
51 328.88172851726
52 326.925018857407
53 325.099935997885
54 323.398060159454
55 321.811440922757
56 320.332579685502
57 318.95441185664
};
\addplot [semithick, white!49.8039215686275!black]
table {%
0 1000
1 966.578999110166
2 934.4599983583
3 903.611348900307
4 874.000841586667
5 845.595818447454
6 818.363283384351
7 792.270010378944
8 767.282647620682
9 743.367816292004
10 720.492203190292
11 698.622646788233
12 677.726216656022
13 657.770286365988
14 638.722600086336
15 620.551333080024
16 603.225146292742
17 586.713235170462
18 570.985372811458
19 556.011947540291
20 541.763994994986
21 528.2132248416
22 515.332042268425
23 503.093564459774
24 491.471632301481
25 480.440817622286
26 469.976426323344
27 460.054497789295
28 450.65180100653
29 441.745827836098
30 433.3147838993
31 425.337577533152
32 417.793807260802
33 410.663748199049
34 403.928337792636
35 397.56916122396
36 391.568436799011
37 385.909001557676
38 380.574297300906
39 375.548357170977
40 370.815792866102
41 366.36178251935
42 362.172059225606
43 358.232900160953
44 354.531116207318
45 351.054041972132
46 347.78952607827
47 344.725921593476
48 341.852076469956
49 339.157323873058
50 336.631472291575
51 334.264795339916
52 332.048021182765
53 329.972321534554
54 328.029300207934
55 326.210981206357
56 324.509796375023
57 322.918572641311
};
\addplot [semithick, color7]
table {%
0 1000
1 968.063036709698
2 937.315036492268
3 907.728346258345
4 879.274846565342
5 851.926038352056
6 825.65312912538
7 800.427117854014
8 776.218877777866
9 752.999236353826
10 730.739051633273
11 709.409284489045
12 688.981066254528
13 669.425761482021
14 650.715025654165
15 632.820857783644
16 615.715647911369
17 599.372219566171
18 583.763867285387
19 568.86438932194
20 554.648115684293
21 541.089931674981
22 528.165297112882
23 515.850261445341
24 504.121474978097
25 492.956196473439
26 482.332297388806
27 472.228263048251
28 462.623191056468
29 453.496787278569
30 444.829359717553
31 436.601810624727
32 428.79562717582
33 421.392871036817
34 414.376167128736
35 407.728691879772
36 401.434161227005
37 395.476818598846
38 389.841423074558
39 384.513237879565
40 379.478019336201
41 374.722006350264
42 370.231910475748
43 365.994906564506
44 361.998623975731
45 358.231138292827
46 354.680963473316
47 351.337044341233
48 348.18874932127
49 345.225863309392
50 342.438580575646
51 339.817497600465
52 337.353605755451
53 335.038283752255
54 332.863289798064
55 330.820753412224
56 328.903166874952
57 327.103376295191
};
\addplot [semithick, color8]
table {%
0 1000
1 969.489066954018
2 940.063321405899
3 911.69862182012
4 884.370440669756
5 858.053932801399
6 832.724003253936
7 808.355374082737
8 784.922649740925
9 762.400380573737
10 740.76312399767
11 719.985502966472
12 700.042261371215
13 680.908316078337
14 662.558805372621
15 644.969133636669
16 628.115012160083
17 611.972496027863
18 596.518017086735
19 581.728413030502
20 567.580952681226
21 554.053357573434
22 541.123819974365
23 528.771017495634
24 516.974124471377
25 505.712820295167
26 494.967294923429
27 484.718251766381
28 474.946908198871
29 465.634993932505
30 456.764747496996
31 448.318911082322
32 440.280723993847
33 432.633914969755
34 425.362693603798
35 418.451741106348
36 411.886200623173
37 405.651667314225
38 399.73417837446
39 394.120203155706
40 388.796633523246
41 383.750774554029
42 378.970335655799
43 374.443422158886
44 370.158527405733
45 366.104525338261
46 362.270663560677
47 358.646556835862
48 355.222180957626
49 351.987866929109
50 348.934295369679
51 346.052491068642
52 343.333817603872
53 340.76997194663
54 338.352978979917
55 336.075185866157
56 333.929256210152
57 331.908163974688
};
\addplot [semithick, color0]
table {%
0 1000
1 970.793117816756
2 942.580676877353
3 915.341476232141
4 889.053993062704
5 863.696437400085
6 839.24680637696
7 815.682937688075
8 792.982561946772
9 771.123353639742
10 750.082980397747
11 729.839150318764
12 710.369657102017
13 691.652422777924
14 673.665537849148
15 656.387298691428
16 639.796242097835
17 623.871176885824
18 608.591212521173
19 593.935784746058
20 579.884678229204
21 566.418046284028
22 553.516427725766
23 541.160760960885
24 529.332395421596
25 518.013100475431
26 507.185071954556
27 496.830936462175
28 486.933753623869
29 477.477016460381
30 468.444650064934
31 459.821008772779
32 451.590872013234
33 443.739439034837
34 436.252322692426
35 429.115542480787
36 422.315516992969
37 415.839055972583
38 409.673352118073
39 403.805972783778
40 398.224851707107
41 392.918280874187
42 387.874902617987
43 383.083702023593
44 378.533999695658
45 374.215444923428
46 370.11800925981
47 366.231980513168
48 362.547957134366
49 359.056842967537
50 355.74984232136
51 352.618455308551
52 349.654473395028
53 346.849975096522
54 344.197321759581
55 341.689153365297
56 339.318384297748
57 337.078199024571
};
\addplot [semithick, color1]
table {%
0 1000
1 971.990206224676
2 944.895041634894
3 918.695786398323
4 893.373450240385
5 868.908816708247
6 845.282487062972
7 822.474923554714
8 800.466491847856
9 779.237502376246
10 758.768250423387
11 739.039054737852
12 720.030294510914
13 701.7224445612
14 684.09610859017
15 667.13205039246
16 650.811222926246
17 635.114795170684
18 620.024176719502
19 605.521040081804
20 591.58734068252
21 578.205334575339
22 565.357593900269
23 553.027020135832
24 541.19685521216
25 529.850690566098
26 518.972474232457
27 508.546516077164
28 498.557491287924
29 488.990442246482
30 479.830778913422
31 471.064277861791
32 462.677080099729
33 454.655687824623
34 446.986960252068
35 439.658108662182
36 432.65669080342
37 425.970604790051
38 419.588082623873
39 413.497683463523
40 407.688286756042
41 402.149085335202
42 396.869578579692
43 391.839565711732
44 387.0491393034
45 382.488679043984
46 378.148845807654
47 374.020576046594
48 370.095076521312
49 366.36381936693
50 362.818537482711
51 359.451220221725
52 356.254109348932
53 353.219695229086
54 350.34071320082
55 347.610140090184
56 345.021190815601
57 342.567315036677
};
\addplot [semithick, color2]
table {%
0 1000
1 973.145842736533
2 947.132435269742
3 921.94326275002
4 897.56158333712
5 873.970463969357
6 851.152815843674
7 829.091429423572
8 807.769008800481
9 787.168205243923
10 767.271649787001
11 748.061984705455
12 729.521893761343
13 711.634131095531
14 694.3815486669
15 677.747122150259
16 661.71397521932
17 646.265402155731
18 631.38488873966
19 617.056131392213
20 603.263054554188
21 589.989826299734
22 577.220872196996
23 564.940887440696
24 553.134847293619
25 541.788015885319
26 530.885953426655
27 520.414521908035
28 510.35988935781
29 500.708532744377
30 491.447239612058
31 482.563108545938
32 474.043548565278
33 465.876277548096
34 458.049319791858
35 450.551002816156
36 443.369953513297
37 436.495093751616
38 429.915635534132
39 423.621075811823
40 417.601191046438
41 411.846031612257
42 406.345916119765
43 401.091425736753
44 396.073398574148
45 391.282924194867
46 386.711338294574
47 382.350217593241
48 378.191374966539
49 374.226854836037
50 370.448928827584
51 366.850091698101
52 363.423057522568
53 360.160756125479
54 357.056329734608
55 354.10312982958
56 351.294714153773
57 348.624843855327
};
\addplot [semithick, color3]
table {%
0 1000
1 974.161103732006
2 949.100620222547
3 924.803823549667
4 901.255794379432
5 878.441449404809
6 856.345570556094
7 834.952833843396
8 814.247837697885
9 794.215130685535
10 774.839238475163
11 756.104689951395
12 737.996042372479
13 720.497905482726
14 703.594964499707
15 687.272001906524
16 671.513917990329
17 656.305750078673
18 641.63269043606
19 627.480102793472
20 613.833537494016
21 600.678745247903
22 588.001689499739
23 575.788557420477
24 564.025769545276
25 552.69998808693
26 541.798123962408
27 531.307342577264
28 521.215068419362
29 511.508988519273
30 502.177054840084
31 493.207485663939
32 484.58876604656
33 476.309647414252
34 468.359146380336
35 460.726542859792
36 453.401377561873
37 446.373448940756
38 439.632809683831
39 433.169762815952
40 426.974857495984
41 421.038884579207
42 415.352872015568
43 409.908080149552
44 404.695996982488
45 399.708333452508
46 394.937018781296
47 390.374195930122
48 386.012217200753
49 381.84364000963
50 377.861222856474
51 374.057921501246
52 370.426885356423
53 366.961454094875
54 363.655154467508
55 360.501697319253
56 357.49497478727
57 354.6290576611
};
\addplot [semithick, color4]
table {%
0 1000
1 975.102361131782
2 950.927487200369
3 927.462191597475
4 904.693121912156
5 882.60678433766
6 861.189567896328
7 840.427768375326
8 820.307611870497
9 800.815277840661
10 781.936921580755
11 763.658696028434
12 745.966772825773
13 728.84736256492
14 712.286734154069
15 696.271233248038
16 680.787299695428
17 665.821483962435
18 651.360462501182
19 637.391052038372
20 623.900222767638
21 610.875110436575
22 598.303027326634
23 586.171472131088
24 574.468138742965
25 563.180923971162
26 552.29793420899
27 541.807491084881
28 531.698136130278
29 521.95863450434
30 512.577977819513
31 503.545386115681
32 494.850309034161
33 486.482426245518
34 478.431647187684
35 470.688110172706
36 463.242180921873
37 456.084450589826
38 449.205733338665
39 442.59706352286
40 436.249692545083
41 430.155085441871
42 424.304917256224
43 418.69106925196
44 413.305625021804
45 408.140866537888
46 403.189270189551
47 398.443502849111
48 393.896418001684
49 389.541051970288
50 385.370620262294
51 381.37851405808
52 377.558296857408
53 373.903701293779
54 370.40862612194
55 367.067133378814
56 363.873445713641
57 360.821943879002
};
\addplot [semithick, color5]
table {%
0 1000
1 975.977424894961
2 952.627731588384
3 929.939065277992
4 907.899428225382
5 886.496700128646
6 865.7186583489
7 845.552997907542
8 825.987351174309
9 807.009307170129
10 788.606430413035
11 770.766279240132
12 753.476423543744
13 736.7244618652
14 720.498037795327
15 704.784855636571
16 689.572695287471
17 674.849426316233
18 660.603021196043
19 646.821567680717
20 633.49328030509
21 620.60651100018
22 608.149758818783
23 596.111678772383
24 584.481089785476
25 573.246981778184
26 562.398521892711
27 551.925059883491
28 541.816132694951
29 532.061468254498
30 522.650988511866
31 513.574811758948
32 504.823254267146
33 496.38683128163
34 488.256257414084
35 480.422446477268
36 472.876510806126
37 465.609760111328
38 458.613699911735
39 451.880029592765
40 445.400640137494
41 439.167611577013
42 433.17321020577
43 427.409885606481
44 421.870267527647
45 416.547162654841
46 411.433551314678
47 406.522584147735
48 401.807578783855
49 397.282016549954
50 392.939539237125
51 388.773945950048
52 384.779190058028
53 380.949376263022
54 377.278757796094
55 373.76173374992
56 370.392846551138
57 367.166779572765
};
\addplot [semithick, color6]
table {%
0 1000
1 976.79304297324
2 954.214079228051
3 932.25241211512
4 910.897221115628
5 890.137578954087
6 869.962468592863
7 850.360800043412
8 831.321426931443
9 812.833162756364
10 794.884796788432
11 777.465109550636
12 760.562887836069
13 744.166939215653
14 728.2661059952
15 712.849278585154
16 697.905408250814
17 683.423519215288
18 669.392720091945
19 655.802214627607
20 642.641311742223
21 629.899434854966
22 617.566130491116
23 605.631076168049
24 594.084087562736
25 582.9151249668
26 572.11429903894
27 561.671875867774
28 551.578281361469
29 541.824104983435
30 532.400102856152
31 523.297200257674
32 514.506493537713
33 506.019251482232
34 497.826916157252
35 489.921103264254
36 482.293602040774
37 474.936374740938
38 467.841555731519
39 461.001450239603
40 454.408532788339
41 448.055445357234
42 441.934995303303
43 436.040153078812
44 430.364049780668
45 424.899974565458
46 419.64137196281
47 414.58183911826
48 409.715122994936
49 405.035117561379
50 400.535860990529
51 396.21153289244
52 392.05645160071
53 388.065071529794
54 384.231980617545
55 380.551897864415
56 377.019670977843
57 373.630274127412
};
\end{axis}

\end{tikzpicture}

\end{figure}
Ввиду отсутствия сопротивления колебания тока стали незатухающими.

\noindent$\blacksquare$

График зависимости $I(t)$ при $R_k = 200$ Ом в интервале значений t 0--20 мкс.
\begin{figure}[H]
    \caption{Графики при $R_k = 200$ Ом}\label{img:plot04}
    % This file was created by tikzplotlib v0.9.2.
\begin{tikzpicture}[scale=0.875]

\definecolor{color0}{rgb}{0.12156862745098,0.466666666666667,0.705882352941177}

\begin{groupplot}[group style={group size=2 by 3, vertical sep=2.5cm, horizontal sep=2.5cm}]
\nextgroupplot[
    x label style={at={(axis description cs:0.5,-0.05)},anchor=north},
    y label style={at={(axis description cs:-0.005,.5)},rotate=0,anchor=south},
tick align=outside,
tick pos=left,
x grid style={white!69.0196078431373!black},
xlabel={Время},
xmin=-1e-06, xmax=2.1e-05,
xtick style={color=black},
y grid style={white!69.0196078431373!black},
ylabel={$I$},
ymin=0.2125492172434, ymax=6.5364664378886,
ytick style={color=black}
]
\addplot [semithick, color0]
table {%
0 0.5
1e-07 0.965861186973023
2e-07 1.35377173832092
3e-07 1.72530497982757
4e-07 2.08431511082624
5e-07 2.42794843244285
6e-07 2.75313961829968
7e-07 3.05799027495689
8e-07 3.34174334058725
9e-07 3.60439706282707
1e-06 3.84650547421535
1.1e-06 4.06896751961812
1.2e-06 4.27287145364503
1.3e-06 4.45940937751335
1.4e-06 4.62978382314568
1.5e-06 4.78521639133276
1.6e-06 4.92686515601568
1.7e-06 5.05578589665468
1.8e-06 5.17237553355271
1.9e-06 5.27766645525348
2e-06 5.37275376267293
2.1e-06 5.45861016022932
2.2e-06 5.53611066769451
2.3e-06 5.60605642193778
2.4e-06 5.66916264048638
2.5e-06 5.72610652191684
2.6e-06 5.77749838567998
2.7e-06 5.82386785837005
2.8e-06 5.8657003257452
2.9e-06 5.90343661533992
3e-06 5.9374720119718
3.1e-06 5.96816756930687
3.2e-06 5.9958499442012
3.3e-06 6.02081178642614
3.4e-06 6.04332431871141
3.5e-06 6.063629108117
3.6e-06 6.0819411864304
3.7e-06 6.09845458477748
3.8e-06 6.11334468351726
3.9e-06 6.12676999686395
4e-06 6.13887374699726
4.1e-06 6.14978530144218
4.2e-06 6.15962148329368
4.3e-06 6.16848776376995
4.4e-06 6.17647934630036
4.5e-06 6.18368215095769
4.6e-06 6.19017370757277
4.7e-06 6.19602454319028
4.8e-06 6.20129927035836
4.9e-06 6.20605449641773
5e-06 6.21034117640121
5.1e-06 6.21420528526662
5.2e-06 6.21768835771026
5.3e-06 6.2208283482854
5.4e-06 6.22365896500125
5.5e-06 6.22621053925046
5.6e-06 6.22851043507359
5.7e-06 6.23058333824784
5.8e-06 6.23245151744343
5.9e-06 6.23413506010141
6e-06 6.23565208544407
6.1e-06 6.23701893680557
6.2e-06 6.23825035526719
6.3e-06 6.2393596363956
6.4e-06 6.24035877171312
6.5e-06 6.24125857637492
6.6e-06 6.24206880438811
6.7e-06 6.24279825258005
6.8e-06 6.24345485440817
6.9e-06 6.24404576532189
7e-06 6.24457745617093
7.1e-06 6.24505575381229
7.2e-06 6.24548590015612
7.3e-06 6.24587262266939
7.4e-06 6.24622018482882
7.5e-06 6.24653243163227
7.6e-06 6.24681283065139
7.7e-06 6.24706450906129
7.8e-06 6.24729028704071
7.9e-06 6.24749270789777
8e-06 6.24767406524172
8.1e-06 6.24783642749001
8.2e-06 6.2479816599716
8.3e-06 6.24811144486193
8.4e-06 6.24822729916219
8.5e-06 6.24833059091437
8.6e-06 6.24842255382526
8.7e-06 6.24850430045515
8.8e-06 6.24857683411228
8.9e-06 6.24864105957967
9e-06 6.24869779278916
9.1e-06 6.24874776954575
9.2e-06 6.24879165339542
9.3e-06 6.24883004272068
9.4e-06 6.24886347713937
9.5e-06 6.2488924432753
9.6e-06 6.24891737996225
9.7e-06 6.24893868293701
9.8e-06 6.24895670907168
9.9e-06 6.24897178019027
1e-05 6.24898418651065
1.01e-05 6.24899418974839
1.02e-05 6.24900202591596
1.03e-05 6.24900790784688
1.04e-05 6.24901202747213
1.05e-05 6.24901455787294
1.06e-05 6.249015655132
1.07e-05 6.2490154600029
1.08e-05 6.24901409941559
1.09e-05 6.24901168783397
1.1e-05 6.24900832848026
1.11e-05 6.24900411443898
1.12e-05 6.24899912965265
1.13e-05 6.2489934498196
1.14e-05 6.24898714320363
1.15e-05 6.24898027136423
1.16e-05 6.24897288981497
1.17e-05 6.24896504861728
1.18e-05 6.24895679291597
1.19e-05 6.24894816342203
1.2e-05 6.24893919684815
1.21e-05 6.24892992630137
1.22e-05 6.24892038163719
1.23e-05 6.24891058977894
1.24e-05 6.24890057500575
1.25e-05 6.24889035921222
1.26e-05 6.24887996214263
1.27e-05 6.24886940160214
1.28e-05 6.24885869364721
1.29e-05 6.24884785275736
1.3e-05 6.24883689199005
1.31e-05 6.24882582312031
1.32e-05 6.24881465676676
1.33e-05 6.2488034025051
1.34e-05 6.24879206897061
1.35e-05 6.24878066395046
1.36e-05 6.24876919446703
1.37e-05 6.24875766685302
1.38e-05 6.24874608681919
1.39e-05 6.24873445951541
1.4e-05 6.24872278958581
1.41e-05 6.24871108121837
1.42e-05 6.24869933818978
1.43e-05 6.24868756390583
1.44e-05 6.24867576143778
1.45e-05 6.2486639335553
1.46e-05 6.24865208275602
1.47e-05 6.24864021129228
1.48e-05 6.24862832119521
1.49e-05 6.24861641429647
1.5e-05 6.24860449224781
1.51e-05 6.24859255653879
1.52e-05 6.24858060851265
1.53e-05 6.24856864938072
1.54e-05 6.24855668023536
1.55e-05 6.24854470206166
1.56e-05 6.24853271574795
1.57e-05 6.24852072209531
1.58e-05 6.24850872182617
1.59e-05 6.24849671559198
1.6e-05 6.24848470398023
1.61e-05 6.24847268752071
1.62e-05 6.24846066669117
1.63e-05 6.24844864192244
1.64e-05 6.24843661360306
1.65e-05 6.24842458208338
1.66e-05 6.24841254767936
1.67e-05 6.24840051067592
1.68e-05 6.24838847133001
1.69e-05 6.24837642987333
1.7e-05 6.24836438651484
1.71e-05 6.24835234144297
1.72e-05 6.24834029482763
1.73e-05 6.24832824682208
1.74e-05 6.24831619756449
1.75e-05 6.24830414717948
1.76e-05 6.24829209577943
1.77e-05 6.24828004346567
1.78e-05 6.2482679903296
1.79e-05 6.24825593645363
1.8e-05 6.24824388191207
1.81e-05 6.24823182677196
1.82e-05 6.24821977109371
1.83e-05 6.24820771493184
1.84e-05 6.2481956583355
1.85e-05 6.24818360134901
1.86e-05 6.24817154401233
1.87e-05 6.24815948636152
1.88e-05 6.24814742842908
1.89e-05 6.24813537024431
1.9e-05 6.24812331183366
1.91e-05 6.24811125322095
1.92e-05 6.2480991944277
1.93e-05 6.24808713547327
1.94e-05 6.24807507637516
1.95e-05 6.24806301714913
1.96e-05 6.24805095780939
1.97e-05 6.24803889836876
1.98e-05 6.24802683883881
1.99e-05 6.24801477922996
2e-05 6.24800271955161
};

\nextgroupplot[
    x label style={at={(axis description cs:0.5,-0.05)},anchor=north},
    y label style={at={(axis description cs:-0.005,.45)},rotate=0,anchor=south},
tick align=outside,
tick pos=left,
x grid style={white!69.0196078431373!black},
xlabel={Время},
xmin=-1e-06, xmax=2.1e-05,
xtick style={color=black},
y grid style={white!69.0196078431373!black},
ylabel={$U$},
ymin=1399.53489555248, ymax=1400.02214783083,
ytick style={color=black}
]
\addplot [semithick, color0]
table {%
0 1400
1e-07 1399.9997222845
2e-07 1399.99928865134
3e-07 1399.99871382027
4e-07 1399.99800264272
5e-07 1399.99716027267
6e-07 1399.99619304122
7e-07 1399.99510822648
8e-07 1399.99391358523
9e-07 1399.99261701408
1e-06 1399.9912262936
1.1e-06 1399.98974893117
1.2e-06 1399.98819205785
1.3e-06 1399.9865623797
1.4e-06 1399.98486615128
1.5e-06 1399.9831091747
1.6e-06 1399.98129680758
1.7e-06 1399.97943398275
1.8e-06 1399.97752537474
1.9e-06 1399.97557540535
2e-06 1399.97358808526
2.1e-06 1399.97156703481
2.2e-06 1399.96951553399
2.3e-06 1399.96743654772
2.4e-06 1399.96533276002
2.5e-06 1399.96320659508
2.6e-06 1399.96106023586
2.7e-06 1399.95889565326
2.8e-06 1399.9567146296
2.9e-06 1399.95451877397
3e-06 1399.95230953978
3.1e-06 1399.95008823972
3.2e-06 1399.94785605781
3.3e-06 1399.94561406294
3.4e-06 1399.94336321899
3.5e-06 1399.94110439377
3.6e-06 1399.93883837026
3.7e-06 1399.9365658552
3.8e-06 1399.93428748646
3.9e-06 1399.93200383969
4e-06 1399.92971543427
4.1e-06 1399.92742273877
4.2e-06 1399.92512617588
4.3e-06 1399.92282612684
4.4e-06 1399.92052293549
4.5e-06 1399.91821691191
4.6e-06 1399.9159083357
4.7e-06 1399.91359745893
4.8e-06 1399.91128450836
4.9e-06 1399.9089696882
5e-06 1399.90665318263
5.1e-06 1399.90433515775
5.2e-06 1399.90201576335
5.3e-06 1399.89969513443
5.4e-06 1399.89737339258
5.5e-06 1399.89505064749
5.6e-06 1399.89272699808
5.7e-06 1399.89040253358
5.8e-06 1399.88807733446
5.9e-06 1399.8857514733
6e-06 1399.88342501556
6.1e-06 1399.88109802026
6.2e-06 1399.87877054064
6.3e-06 1399.87644262472
6.4e-06 1399.87411431579
6.5e-06 1399.87178565291
6.6e-06 1399.86945667127
6.7e-06 1399.86712740264
6.8e-06 1399.86479787566
6.9e-06 1399.86246811614
7e-06 1399.86013814737
7.1e-06 1399.85780799034
7.2e-06 1399.85547766398
7.3e-06 1399.85314718537
7.4e-06 1399.85081656988
7.5e-06 1399.84848583141
7.6e-06 1399.84615498247
7.7e-06 1399.84382403436
7.8e-06 1399.84149299725
7.9e-06 1399.83916188033
8e-06 1399.83683069187
8.1e-06 1399.83449943936
8.2e-06 1399.8321681295
8.3e-06 1399.82983676839
8.4e-06 1399.8275053615
8.5e-06 1399.82517391376
8.6e-06 1399.82284242963
8.7e-06 1399.82051091313
8.8e-06 1399.81817936787
8.9e-06 1399.81584779712
9e-06 1399.81351620383
9.1e-06 1399.81118459065
9.2e-06 1399.80885295998
9.3e-06 1399.80652131398
9.4e-06 1399.80418965459
9.5e-06 1399.80185798358
9.6e-06 1399.79952630252
9.7e-06 1399.79719461285
9.8e-06 1399.79486291585
9.9e-06 1399.79253121269
1e-05 1399.79019950441
1.01e-05 1399.78786779196
1.02e-05 1399.78553607618
1.03e-05 1399.78320435785
1.04e-05 1399.78087263766
1.05e-05 1399.77854091624
1.06e-05 1399.77620919414
1.07e-05 1399.77387747188
1.08e-05 1399.77154574991
1.09e-05 1399.76921402865
1.1e-05 1399.76688230847
1.11e-05 1399.76455058971
1.12e-05 1399.76221887266
1.13e-05 1399.75988715761
1.14e-05 1399.75755544479
1.15e-05 1399.75522373443
1.16e-05 1399.75289202674
1.17e-05 1399.75056032189
1.18e-05 1399.74822862004
1.19e-05 1399.74589692134
1.2e-05 1399.74356522593
1.21e-05 1399.74123353392
1.22e-05 1399.73890184542
1.23e-05 1399.73657016053
1.24e-05 1399.73423847934
1.25e-05 1399.73190680192
1.26e-05 1399.72957512835
1.27e-05 1399.72724345868
1.28e-05 1399.72491179299
1.29e-05 1399.72258013132
1.3e-05 1399.72024847371
1.31e-05 1399.71791682022
1.32e-05 1399.71558517087
1.33e-05 1399.71325352571
1.34e-05 1399.71092188476
1.35e-05 1399.70859024805
1.36e-05 1399.70625861561
1.37e-05 1399.70392698747
1.38e-05 1399.70159536363
1.39e-05 1399.69926374412
1.4e-05 1399.69693212896
1.41e-05 1399.69460051817
1.42e-05 1399.69226891174
1.43e-05 1399.68993730971
1.44e-05 1399.68760571207
1.45e-05 1399.68527411885
1.46e-05 1399.68294253004
1.47e-05 1399.68061094565
1.48e-05 1399.6782793657
1.49e-05 1399.67594779019
1.5e-05 1399.67361621913
1.51e-05 1399.67128465251
1.52e-05 1399.66895309035
1.53e-05 1399.66662153266
1.54e-05 1399.66428997942
1.55e-05 1399.66195843066
1.56e-05 1399.65962688636
1.57e-05 1399.65729534654
1.58e-05 1399.6549638112
1.59e-05 1399.65263228033
1.6e-05 1399.65030075395
1.61e-05 1399.64796923205
1.62e-05 1399.64563771463
1.63e-05 1399.6433062017
1.64e-05 1399.64097469325
1.65e-05 1399.6386431893
1.66e-05 1399.63631168984
1.67e-05 1399.63398019486
1.68e-05 1399.63164870438
1.69e-05 1399.62931721839
1.7e-05 1399.6269857369
1.71e-05 1399.62465425989
1.72e-05 1399.62232278739
1.73e-05 1399.61999131938
1.74e-05 1399.61765985586
1.75e-05 1399.61532839684
1.76e-05 1399.61299694232
1.77e-05 1399.61066549229
1.78e-05 1399.60833404676
1.79e-05 1399.60600260573
1.8e-05 1399.6036711692
1.81e-05 1399.60133973716
1.82e-05 1399.59900830963
1.83e-05 1399.59667688659
1.84e-05 1399.59434546805
1.85e-05 1399.59201405401
1.86e-05 1399.58968264446
1.87e-05 1399.58735123942
1.88e-05 1399.58501983888
1.89e-05 1399.58268844283
1.9e-05 1399.58035705129
1.91e-05 1399.57802566424
1.92e-05 1399.5756942817
1.93e-05 1399.57336290365
1.94e-05 1399.5710315301
1.95e-05 1399.56870016106
1.96e-05 1399.56636879651
1.97e-05 1399.56403743646
1.98e-05 1399.56170608091
1.99e-05 1399.55937472986
2e-05 1399.55704338332
};

\nextgroupplot[
    x label style={at={(axis description cs:0.5,-0.05)},anchor=north},
    y label style={at={(axis description cs:-0.005,.5)},rotate=0,anchor=south},
tick align=outside,
tick pos=left,
x grid style={white!69.0196078431373!black},
xlabel={Время},
xmin=-1.005e-06, xmax=2.1105e-05,
xtick style={color=black},
y grid style={white!69.0196078431373!black},
ylabel={$R_p$},
ymin=-3.5984412578909, ymax=603.552810931088,
ytick style={color=black}
]
\addplot [semithick, color0]
table {%
0 575.95502674068
5e-08 510.013806086058
5e-08 519.872606773442
1e-07 463.573119716408
1e-07 464.142801754927
1.5e-07 380.677703076672
1.5e-07 384.85094750889
2e-07 312.926295036677
2e-07 312.393317134684
2.5e-07 257.299519505433
2.5e-07 258.12029249782
3e-07 215.179203469344
3e-07 215.175227633148
3.5e-07 181.802171205874
3.5e-07 182.324081297536
4e-07 155.964579794915
4e-07 155.991872755474
4.5e-07 135.062530335875
4.5e-07 135.50642739313
5e-07 118.699451182768
5e-07 118.721003333552
5.5e-07 105.159422467247
5.5e-07 105.525964362364
6e-07 94.3870410550528
6e-07 94.401233126752
6.5e-07 85.2245427452933
6.5e-07 85.5024086033803
7e-07 77.7902354406008
7e-07 77.8001582176379
7.5e-07 71.3581346406564
7.5e-07 71.5787994365471
8e-07 66.0923179839008
8e-07 66.0998325151751
8.5e-07 61.409620034034
8.5e-07 61.582468080143
9e-07 57.5304468136627
9e-07 57.536194183098
9.5e-07 54.0469144038434
9.5e-07 54.1823592750582
1e-06 51.1126283036468
1e-06 51.1169880090014
1.05e-06 48.4434494303995
1.05e-06 48.5516519308253
1.1e-06 46.1724505450014
1.1e-06 46.1759272621872
1.15e-06 44.0823666999625
1.15e-06 44.1706697533321
1.2e-06 42.3016369317253
1.2e-06 42.3044798157452
1.25e-06 40.620858197118
1.25e-06 40.6932427146543
1.3e-06 39.1900449173615
1.3e-06 39.1924940145331
1.35e-06 37.8415318932824
1.35e-06 37.900495670199
1.4e-06 36.6624397824178
1.4e-06 36.6643355110226
1.45e-06 35.5571295513016
1.45e-06 35.6076009778784
1.5e-06 34.5845761006431
1.5e-06 34.5861236186994
1.55e-06 33.6690549167412
1.55e-06 33.710819533754
1.6e-06 32.8601995042705
1.6e-06 32.8615332679694
1.65e-06 32.0835201228785
1.65e-06 32.1192664545005
1.7e-06 31.5534408726768
1.7e-06 31.5549015976204
1.75e-06 31.0564009903371
1.75e-06 31.0814787069411
1.8e-06 30.6196913652866
1.8e-06 30.6205593310237
1.85e-06 30.2010456720817
1.85e-06 30.2220751793191
1.9e-06 29.828700196439
1.9e-06 29.8294657559007
1.95e-06 29.4570993136557
1.95e-06 29.4757229107176
2e-06 29.131151463256
2e-06 29.1318122182182
2.05e-06 28.8159297658366
2.05e-06 28.8318452972158
2.1e-06 28.5383285554285
2.1e-06 28.5389020641153
2.15e-06 28.2625715027062
2.15e-06 28.2765772278833
2.2e-06 28.0185560608738
2.2e-06 28.01905207461
2.25e-06 27.7810735121226
2.25e-06 27.7931488722012
2.3e-06 27.5712318867467
2.3e-06 27.5716645378697
2.35e-06 27.3655046851962
2.35e-06 27.3762505375737
2.4e-06 27.1791123764365
2.4e-06 27.1794839402418
2.45e-06 26.9952110526121
2.45e-06 27.0048195959485
2.5e-06 26.8280209226692
2.5e-06 26.8283502498158
2.55e-06 26.6641071945626
2.55e-06 26.6724648626732
2.6e-06 26.518322495283
2.6e-06 26.5186220698031
2.65e-06 26.375155768148
2.65e-06 26.3824831365808
2.7e-06 26.2474594436759
2.7e-06 26.247725885132
2.75e-06 26.1203140179634
2.75e-06 26.1268268022021
2.8e-06 26.0068442143377
2.8e-06 26.0070791094944
2.85e-06 25.8947027015359
2.85e-06 25.9004598899623
2.9e-06 25.7945185985105
2.9e-06 25.7947259569921
2.95e-06 25.6956881528716
2.95e-06 25.7007747764142
3e-06 25.6072984066336
3e-06 25.6074819853722
3.05e-06 25.5188243962798
3.05e-06 25.5233779288209
3.1e-06 25.4396034174263
3.1e-06 25.4397676750812
3.15e-06 25.3612785768839
3.15e-06 25.3653175665313
3.2e-06 25.2910879140555
3.2e-06 25.2912338666418
3.25e-06 25.2216249052719
3.25e-06 25.2252132602273
3.3e-06 25.1586387018069
3.3e-06 25.1587700654593
3.35e-06 25.0951858966017
3.35e-06 25.0984557149774
3.4e-06 25.0381841427273
3.4e-06 25.0383019551763
3.45e-06 24.9816878724699
3.45e-06 24.9846012960278
3.5e-06 24.930848079725
3.5e-06 24.9309532216529
3.55e-06 24.8804231180622
3.55e-06 24.8830261219715
3.6e-06 24.8351002087766
3.6e-06 24.8351942823822
3.65e-06 24.7901217477971
3.65e-06 24.792446378386
3.7e-06 24.7496743220589
3.7e-06 24.749758417997
3.75e-06 24.7095123793056
3.75e-06 24.7115903119951
3.8e-06 24.6733805131616
3.8e-06 24.6734557510367
3.85e-06 24.6374861640841
3.85e-06 24.6393451071934
3.9e-06 24.605181103941
3.9e-06 24.6052484660486
3.95e-06 24.5730743961468
3.95e-06 24.5747386553312
4e-06 24.5441682213532
4e-06 24.5442285716986
4.05e-06 24.5154283035339
4.05e-06 24.5169192537672
4.1e-06 24.4895453255415
4.1e-06 24.4895994258311
4.15e-06 24.4638022082145
4.15e-06 24.4651386905677
4.2e-06 24.4406117729377
4.2e-06 24.4406602960929
4.25e-06 24.417539412568
4.25e-06 24.4187380694531
4.3e-06 24.3967498440374
4.3e-06 24.3967933856702
4.35e-06 24.3760603072767
4.35e-06 24.3771358670503
4.4e-06 24.3574137853863
4.4e-06 24.3574528735879
4.45e-06 24.3388523002576
4.45e-06 24.3398178199207
4.5e-06 24.3221205075909
4.5e-06 24.322155611316
4.55e-06 24.3054612472269
4.55e-06 24.3063283202669
4.6e-06 24.2904416527055
4.6e-06 24.2904731889837
4.65e-06 24.2753329650681
4.65e-06 24.2761314962822
4.7e-06 24.2614828339876
4.7e-06 24.2615110739966
4.75e-06 24.2476848310491
4.75e-06 24.2484007465214
4.8e-06 24.2352372088593
4.8e-06 24.2352631520692
4.85e-06 24.2228367402417
4.85e-06 24.2234805224863
4.9e-06 24.2116481512996
4.9e-06 24.2116714871267
4.95e-06 24.2005002870763
4.95e-06 24.2010793508601
5e-06 24.1904409430021
5e-06 24.1904619382316
5.05e-06 24.1804168724481
5.05e-06 24.180937842762
5.1e-06 24.1713708796627
5.1e-06 24.1713897728935
5.15e-06 24.1623555232715
5.15e-06 24.1628243245927
5.2e-06 24.1541173249537
5.2e-06 24.1541342726572
5.25e-06 24.1458995921099
5.25e-06 24.14632625003
5.3e-06 24.1384815597659
5.3e-06 24.1384970050955
5.35e-06 24.1310871143232
5.35e-06 24.1314712612389
5.4e-06 24.1244118322547
5.4e-06 24.1244257410129
5.45e-06 24.1177571911846
5.45e-06 24.1181031148248
5.5e-06 24.1117496113485
5.5e-06 24.1117621380742
5.55e-06 24.105760135641
5.55e-06 24.1060716815219
5.6e-06 24.1003529752323
5.6e-06 24.1003642586085
5.65e-06 24.0949617279952
5.65e-06 24.0952423468423
5.7e-06 24.0900946319705
5.7e-06 24.0901047965096
5.75e-06 24.0852415515986
5.75e-06 24.0854943413503
5.8e-06 24.0808603517859
5.8e-06 24.0808695093249
5.85e-06 24.076491519796
5.85e-06 24.0767192628659
5.9e-06 24.0725475783244
5.9e-06 24.0725558293537
5.95e-06 24.0686145699555
5.95e-06 24.0688197663392
6e-06 24.0650641964924
6e-06 24.0650716313384
6.05e-06 24.0615235038491
6.05e-06 24.061708400578
6.1e-06 24.0583274376006
6.1e-06 24.0583341374789
6.15e-06 24.0551399565308
6.15e-06 24.055306573908
6.2e-06 24.0522629052007
6.2e-06 24.0522689431557
6.25e-06 24.0493934782924
6.25e-06 24.0495436332896
6.3e-06 24.0468037072406
6.3e-06 24.0468091489845
6.35e-06 24.0442207163317
6.35e-06 24.0443560434756
6.4e-06 24.0418896820644
6.4e-06 24.0418945867265
6.45e-06 24.0395646845216
6.45e-06 24.0396866545558
6.5e-06 24.0374667073998
6.5e-06 24.0374711281973
6.55e-06 24.0353741111543
6.55e-06 24.0354840477296
6.6e-06 24.0334860828578
6.6e-06 24.0334900676958
6.65e-06 24.0316028558161
6.65e-06 24.0317019504281
6.7e-06 24.0299039776548
6.7e-06 24.0299075696637
6.75e-06 24.028209387643
6.75e-06 24.0282987130101
6.8e-06 24.0266809362845
6.8e-06 24.0266841743019
6.85e-06 24.0251563181499
6.85e-06 24.0252368402038
6.9e-06 24.0237801356873
6.9e-06 24.0237830619834
6.95e-06 24.0224035683297
6.95e-06 24.0224763854148
7e-06 24.021162392043
7e-06 24.0211650301876
7.05e-06 24.0199242944526
7.05e-06 24.0199899430752
7.1e-06 24.0188082399264
7.1e-06 24.0188106184278
7.15e-06 24.0176949436336
7.15e-06 24.0177541310189
7.2e-06 24.0166916628585
7.2e-06 24.0166938073198
7.25e-06 24.0156908547213
7.25e-06 24.0157442180362
7.3e-06 24.0147892236084
7.3e-06 24.0147911570986
7.35e-06 24.0138898103436
7.35e-06 24.0139379236897
7.4e-06 24.0130798064178
7.4e-06 24.0130815497247
7.45e-06 24.0122717928138
7.45e-06 24.0123151735125
7.5e-06 24.0115443853148
7.5e-06 24.0115459571717
7.55e-06 24.0108187647771
7.55e-06 24.0108578790211
7.6e-06 24.0101658158743
7.6e-06 24.0101672331645
7.65e-06 24.0095144720236
7.65e-06 24.0095497399581
7.7e-06 24.0089286479932
7.7e-06 24.0089299259333
7.75e-06 24.008344266173
7.75e-06 24.0083760664677
7.8e-06 24.0078189575097
7.8e-06 24.0078201098151
7.85e-06 24.0072949451843
7.85e-06 24.0073236191462
7.9e-06 24.0068241947311
7.9e-06 24.0068252337645
7.95e-06 24.0063546098637
7.95e-06 24.0063804651393
8e-06 24.0059330481393
8e-06 24.0059339850448
8.05e-06 24.0055125347351
8.05e-06 24.0055358486425
8.1e-06 24.0051353217267
8.1e-06 24.0051361665502
8.15e-06 24.0047590518141
8.15e-06 24.0047800743429
8.2e-06 24.0044218245783
8.2e-06 24.0044225863763
8.25e-06 24.0040854459745
8.25e-06 24.0041044024874
8.3e-06 24.00378427146
8.3e-06 24.0037849583969
8.35e-06 24.0034838607375
8.35e-06 24.0035009544027
8.4e-06 24.0032151933007
8.4e-06 24.0032158127371
8.45e-06 24.0029472134299
8.45e-06 24.0029626274127
8.5e-06 24.002707856574
8.5e-06 24.002708415146
8.55e-06 24.0024691187731
8.55e-06 24.0024830182095
8.6e-06 24.0022561906831
8.6e-06 24.002256694374
8.65e-06 24.002043820052
8.65e-06 24.0020563538272
8.7e-06 24.0018547225489
8.7e-06 24.0018551767531
8.75e-06 24.0016661271094
8.75e-06 24.0016774294596
8.8e-06 24.0014985176789
8.8e-06 24.0014989272601
8.85e-06 24.0013313604799
8.85e-06 24.0013415524365
8.9e-06 24.00118312707
8.9e-06 24.0011834964137
8.95e-06 24.0010353010537
8.95e-06 24.0010444917437
9e-06 24.0009045393649
9e-06 24.0009048724253
9.05e-06 24.0007741447167
9.05e-06 24.0007824325358
9.1e-06 24.0006591377373
9.1e-06 24.0006594380796
9.15e-06 24.0005444614745
9.15e-06 24.0005519351429
9.2e-06 24.0004436610362
9.2e-06 24.0004439318754
9.25e-06 24.000343158612
9.25e-06 24.0003498981273
9.3e-06 24.0002551687675
9.3e-06 24.0002554130024
9.35e-06 24.0001674474899
9.35e-06 24.0001735249858
9.4e-06 24.0000910095303
9.4e-06 24.0000912297747
9.45e-06 24.0000148136178
9.45e-06 24.0000202941373
9.5e-06 23.999948792567
9.5e-06 23.9999489911778
9.55e-06 23.999882989677
9.55e-06 23.9998879318702
9.6e-06 23.9998263621184
9.6e-06 23.9998265412211
9.65e-06 23.9997699312038
9.65e-06 23.9997743879567
9.7e-06 23.9997217743063
9.7e-06 23.9997219358171
9.75e-06 23.9996737946681
9.75e-06 23.9996778136695
9.8e-06 23.999633276293
9.8e-06 23.9996334219403
9.85e-06 23.999592917712
9.85e-06 23.9995965419645
9.9e-06 23.9995592874948
9.9e-06 23.9995594188367
9.95e-06 23.9995258013341
9.95e-06 23.9995290696148
1e-05 23.9994983826435
1e-05 23.9994985010854
1.005e-05 23.9994710938278
1.005e-05 23.9994740411038
1.01e-05 23.999449276517
1.01e-05 23.9994493833259
1.015e-05 23.9994275763006
1.015e-05 23.9994302341034
1.02e-05 23.9994108101714
1.02e-05 23.99941090649
1.025e-05 23.9993941496181
1.025e-05 23.9993965463814
1.03e-05 23.9993819385286
1.03e-05 23.9993820253873
1.035e-05 23.9993698226331
1.035e-05 23.9993719839971
1.04e-05 23.9993617191843
1.04e-05 23.9993617975122
1.045e-05 23.9993537015717
1.045e-05 23.999355650658
1.05e-05 23.9993493023175
1.05e-05 23.9993493729527
1.055e-05 23.9993449804644
1.055e-05 23.9993467381229
1.06e-05 23.9993439215921
1.06e-05 23.9993439852901
1.065e-05 23.9993429325173
1.065e-05 23.9993445175499
1.07e-05 23.9993448859531
1.07e-05 23.9993449433951
1.075e-05 23.9993469023318
1.075e-05 23.9993483316935
1.08e-05 23.9993515722291
1.08e-05 23.9993516240297
1.085e-05 23.9993562988899
1.085e-05 23.9993575878702
1.09e-05 23.999363418463
1.09e-05 23.9993634651762
1.095e-05 23.9993705892285
1.095e-05 23.9993717516152
1.1e-05 23.999379917897
1.1e-05 23.9993799600225
1.105e-05 23.9993892927354
1.105e-05 23.9993903409618
1.11e-05 23.9994006135501
1.11e-05 23.9994006515385
1.115e-05 23.9994119760065
1.115e-05 23.9994129212846
1.12e-05 23.9994250933274
1.12e-05 23.999425127585
1.125e-05 23.9994382482073
1.125e-05 23.9994391006483
1.13e-05 23.9994529856107
1.13e-05 23.999453016504
1.135e-05 23.9994677568922
1.135e-05 23.9994685256136
1.14e-05 23.9994839552825
1.14e-05 23.9994839831417
1.145e-05 23.9995001842319
1.145e-05 23.9995008774562
1.15e-05 23.9995177001399
1.15e-05 23.9995177252632
1.155e-05 23.9995352436144
1.155e-05 23.9995358687562
1.16e-05 23.9995539476619
1.16e-05 23.9995539703179
1.165e-05 23.9995726765777
1.165e-05 23.9995732403233
1.17e-05 23.9995924520935
1.17e-05 23.9995924725245
1.175e-05 23.9996122500446
1.175e-05 23.999612758424
1.18e-05 23.9996329918166
1.18e-05 23.9996330102412
1.185e-05 23.9996537538302
1.185e-05 23.9996542122807
1.19e-05 23.9996753669795
1.19e-05 23.9996753835946
1.195e-05 23.9996969983922
1.195e-05 23.9996974118174
1.2e-05 23.9997193973583
1.2e-05 23.9997194123418
1.205e-05 23.9997418128041
1.205e-05 23.999742185626
1.21e-05 23.9997649204299
1.21e-05 23.9997649339419
1.215e-05 23.9997880429268
1.215e-05 23.9997883791329
1.22e-05 23.9998117896325
1.22e-05 23.9998118018175
1.225e-05 23.9998355497591
1.225e-05 23.9998358529453
1.23e-05 23.9998598727987
1.23e-05 23.9998598837872
1.235e-05 23.9998842079515
1.235e-05 23.9998844813607
1.24e-05 23.9999090507414
1.24e-05 23.9999090606507
1.245e-05 23.9999339044651
1.245e-05 23.9999341510218
1.25e-05 23.9999592159788
1.25e-05 23.999959224915
1.255e-05 23.9999845373628
1.255e-05 23.999984759704
1.26e-05 24.000010271585
1.26e-05 24.0000102796437
1.265e-05 24.0000360147186
1.265e-05 24.0000362152224
1.27e-05 24.0000621301532
1.27e-05 24.0000621374205
1.275e-05 24.0000882536344
1.275e-05 24.0000884344455
1.28e-05 24.0001147128606
1.28e-05 24.0001147194142
1.285e-05 24.0001411793536
1.285e-05 24.0001413424059
1.29e-05 24.0001679486256
1.29e-05 24.0001679545357
1.295e-05 24.0001947244614
1.295e-05 24.000194871499
1.3e-05 24.0002217733476
1.3e-05 24.0002217786773
1.305e-05 24.0002488281635
1.305e-05 24.0002489607591
1.31e-05 24.0002761292209
1.31e-05 24.0002761340273
1.315e-05 24.0003034356362
1.315e-05 24.0003035552081
1.32e-05 24.000330964117
1.32e-05 24.0003309684514
1.325e-05 24.00035849744
1.325e-05 24.0003586052673
1.33e-05 24.0003862310268
1.33e-05 24.0003862349356
1.335e-05 24.0004139689907
1.335e-05 24.0004140662267
1.34e-05 24.0004418875577
1.34e-05 24.0004418910827
1.345e-05 24.0004698100826
1.345e-05 24.0004698977675
1.35e-05 24.0004978954806
1.35e-05 24.0004978986594
1.355e-05 24.0005259844583
1.355e-05 24.0005260635301
1.36e-05 24.0005542203205
1.36e-05 24.0005542231872
1.365e-05 24.0005824594215
1.365e-05 24.0005825307259
1.37e-05 24.0006108309884
1.37e-05 24.0006108335736
1.375e-05 24.0006392054866
1.375e-05 24.0006392697865
1.38e-05 24.0006676994482
1.38e-05 24.0006677017795
1.385e-05 24.0006961960637
1.385e-05 24.0006962540471
1.39e-05 24.0007248004172
1.39e-05 24.0007248025196
1.395e-05 24.0007534071745
1.395e-05 24.0007534594616
1.4e-05 24.0007821110956
1.4e-05 24.0007821129916
1.405e-05 24.0008108171951
1.405e-05 24.0008108643453
1.41e-05 24.0008396109228
1.41e-05 24.0008396126327
1.415e-05 24.0008684066256
1.415e-05 24.0008684491434
1.42e-05 24.0008972813573
1.42e-05 24.0008972828993
1.425e-05 24.0009261578807
1.425e-05 24.0009261962211
1.43e-05 24.0009551056785
1.43e-05 24.000955107069
1.435e-05 24.0009840551025
1.435e-05 24.0009840896756
1.44e-05 24.0010130688078
1.44e-05 24.0010130700618
1.445e-05 24.0010420839902
1.445e-05 24.001042115166
1.45e-05 24.0010711571474
1.45e-05 24.0010711582784
1.455e-05 24.0011002316474
1.455e-05 24.0011002597596
1.46e-05 24.0011293584351
1.46e-05 24.001129359455
1.465e-05 24.0011584864443
1.465e-05 24.0011585117938
1.47e-05 24.0011876616127
1.47e-05 24.0011876625325
1.475e-05 24.0012168378932
1.475e-05 24.0012168607512
1.48e-05 24.001246056708
1.48e-05 24.0012460575375
1.485e-05 24.0012752765363
1.485e-05 24.0012752971475
1.49e-05 24.0013045347282
1.49e-05 24.0013045354763
1.495e-05 24.0013337938446
1.495e-05 24.0013338124297
1.5e-05 24.0013630875635
1.5e-05 24.0013630882382
1.505e-05 24.0013923821268
1.505e-05 24.0013923988848
1.51e-05 24.0014217079008
1.51e-05 24.0014217085093
1.515e-05 24.0014510344469
1.515e-05 24.0014510495571
1.52e-05 24.001480389145
1.52e-05 24.0014803896937
1.525e-05 24.0015097445499
1.525e-05 24.0015097581742
1.53e-05 24.0015391253485
1.53e-05 24.0015391258434
1.535e-05 24.0015685067952
1.535e-05 24.0015685190795
1.54e-05 24.0015979111482
1.54e-05 24.0015979115945
1.545e-05 24.0016273160961
1.545e-05 24.0016273271721
1.55e-05 24.0016567417072
1.55e-05 24.0016567421098
1.555e-05 24.0016861678655
1.555e-05 24.0016861778518
1.56e-05 24.0017156126642
1.56e-05 24.0017156130272
1.565e-05 24.0017450579668
1.565e-05 24.0017450669703
1.57e-05 24.0017745200855
1.57e-05 24.001774520413
1.575e-05 24.0018039826694
1.575e-05 24.0018039907867
1.58e-05 24.0018334604244
1.58e-05 24.0018334607197
1.585e-05 24.0018629386093
1.585e-05 24.0018629459275
1.59e-05 24.0018924304819
1.59e-05 24.0018924307483
1.595e-05 24.0019219227529
1.595e-05 24.0019219293504
1.6e-05 24.0019514273736
1.6e-05 24.0019514276139
1.605e-05 24.0019809323642
1.605e-05 24.0019809383118
1.61e-05 24.0020104484982
1.61e-05 24.0020104487149
1.615e-05 24.0020399649763
1.615e-05 24.0020399703378
1.62e-05 24.0020694915098
1.62e-05 24.0020694917052
1.625e-05 24.0020990183642
1.625e-05 24.0020990231972
1.63e-05 24.0021285542929
1.63e-05 24.0021285544692
1.635e-05 24.0021580905216
1.635e-05 24.002158094878
1.64e-05 24.0021876349398
1.64e-05 24.0021876350989
1.645e-05 24.0022171796392
1.645e-05 24.0022171835657
1.65e-05 24.0022467317302
1.65e-05 24.0022467318737
1.655e-05 24.0022762840854
1.655e-05 24.0022762876243
1.66e-05 24.0023058431127
1.66e-05 24.0023058432421
1.665e-05 24.0023354023887
1.665e-05 24.0023354055781
1.67e-05 24.002364967688
1.67e-05 24.0023649678048
1.675e-05 24.0023945332223
1.675e-05 24.0023945360965
1.68e-05 24.0024241041947
1.68e-05 24.0024241043
1.685e-05 24.0024536753896
1.685e-05 24.0024536779795
1.69e-05 24.0024832514949
1.69e-05 24.0024832515899
1.695e-05 24.0025128278115
1.695e-05 24.002512830145
1.7e-05 24.0025424085626
1.7e-05 24.0025424086483
1.705e-05 24.0025719895148
1.705e-05 24.0025719916171
1.71e-05 24.0026015744724
1.71e-05 24.0026015745498
1.715e-05 24.002631159622
1.715e-05 24.0026311615159
1.72e-05 24.0026607483901
1.72e-05 24.0026607484599
1.725e-05 24.0026903373418
1.725e-05 24.0026903390477
1.73e-05 24.002719929563
1.73e-05 24.002719929626
1.735e-05 24.0027495219605
1.735e-05 24.0027495234969
1.74e-05 24.0027791173128
1.74e-05 24.0027791173696
1.745e-05 24.0028087128346
1.745e-05 24.002808714218
1.75e-05 24.0028383110273
1.75e-05 24.0028383110786
1.755e-05 24.0028679093836
1.755e-05 24.0028679106292
1.76e-05 24.0028975101549
1.76e-05 24.0028975102012
1.765e-05 24.0029271110842
1.765e-05 24.0029271122054
1.77e-05 24.0029567141977
1.77e-05 24.0029567142396
1.775e-05 24.0029863174644
1.775e-05 24.0029863184735
1.78e-05 24.0030159227072
1.78e-05 24.0030159227449
1.785e-05 24.0030455280986
1.785e-05 24.0030455290066
1.79e-05 24.0030751352785
1.79e-05 24.0030751353126
1.795e-05 24.003104742603
1.795e-05 24.0031047434198
1.8e-05 24.0031343515467
1.8e-05 24.0031343515774
1.805e-05 24.0031639606314
1.805e-05 24.003163961366
1.81e-05 24.0031935711826
1.81e-05 24.0031935712104
1.815e-05 24.0032231818717
1.815e-05 24.0032231825322
1.82e-05 24.0032527938896
1.82e-05 24.0032527939147
1.825e-05 24.0032824060424
1.825e-05 24.003282406636
1.83e-05 24.0033120193999
1.83e-05 24.0033120194225
1.835e-05 24.0033416328896
1.835e-05 24.0033416334229
1.84e-05 24.0033712474721
1.84e-05 24.0033712474926
1.845e-05 24.0034008621845
1.845e-05 24.0034008626634
1.85e-05 24.0034304778887
1.85e-05 24.0034304779072
1.855e-05 24.0034600937205
1.855e-05 24.0034600941504
1.86e-05 24.0034897104532
1.86e-05 24.00348971047
1.865e-05 24.0035193273117
1.865e-05 24.0035193276973
1.87e-05 24.0035489449888
1.87e-05 24.003548945004
1.875e-05 24.0035785627899
1.875e-05 24.0035785631357
1.88e-05 24.0036081813357
1.88e-05 24.0036081813494
1.885e-05 24.003637800004
1.885e-05 24.0036378003138
1.89e-05 24.0036674193501
1.89e-05 24.0036674193625
1.895e-05 24.0036970388173
1.895e-05 24.0036970390946
1.9e-05 24.0037266589021
1.9e-05 24.0037266589133
1.905e-05 24.0037562791066
1.905e-05 24.0037562793548
1.91e-05 24.0037858998746
1.91e-05 24.0037858998848
1.915e-05 24.0038155207611
1.915e-05 24.0038155209829
1.92e-05 24.0038451421621
1.92e-05 24.0038451421713
1.925e-05 24.0038747636806
1.925e-05 24.0038747638785
1.93e-05 24.0039043856694
1.93e-05 24.0039043856777
1.935e-05 24.0039340077747
1.935e-05 24.0039340079512
1.94e-05 24.0039636303106
1.94e-05 24.0039636303181
1.945e-05 24.0039932529621
1.945e-05 24.0039932531192
1.95e-05 24.0040228760082
1.95e-05 24.0040228760151
1.955e-05 24.0040524991693
1.955e-05 24.004052499309
1.96e-05 24.0040821226926
1.96e-05 24.0040821226988
1.965e-05 24.0041117463301
1.965e-05 24.0041117464541
1.97e-05 24.0041413703006
1.97e-05 24.0041413703063
1.975e-05 24.0041709943848
1.975e-05 24.0041709944946
1.98e-05 24.0042006187756
1.98e-05 24.0042006187808
1.985e-05 24.0042302432796
1.985e-05 24.0042302433766
1.99e-05 24.0042598680664
1.99e-05 24.0042598680711
1.995e-05 24.0042894929658
1.995e-05 24.0042894930513
2e-05 24.0043191181267
2e-05 24.004319118131
2.005e-05 24.0043487433997
2.005e-05 24.0043487434748
2.01e-05 24.0043783689149
};

\nextgroupplot[
    x label style={at={(axis description cs:0.5,-0.05)},anchor=north},
    y label style={at={(axis description cs:-0.005,.5)},rotate=0,anchor=south},
tick align=outside,
tick pos=left,
x grid style={white!69.0196078431373!black},
xlabel={Время},
xmin=-1.005e-06, xmax=2.1105e-05,
xtick style={color=black},
y grid style={white!69.0196078431373!black},
ylabel={$I \cdot R_p$},
ymin=135.024724780376, ymax=463.870826177106,
ytick style={color=black}
]
\addplot [semithick, color0]
table {%
0 287.97751337034
5e-08 393.01370524909
5e-08 378.487740143935
1e-07 448.923276113618
1e-07 448.297517427998
1.5e-07 444.889156577803
1.5e-07 445.943039714143
2e-07 423.114578265303
2e-07 422.909243977261
2.5e-07 396.918356680431
2.5e-07 397.371405457633
3e-07 371.245625064134
3e-07 371.242891771
3.5e-07 346.89873780203
3.5e-07 347.299005540979
4e-07 325.111151114113
4e-07 325.136217550318
4.5e-07 305.275166435978
4.5e-07 305.723182684228
5e-07 288.224682581505
5e-07 288.24847394174
5.5e-07 272.92767600442
5.5e-07 273.366479118126
6e-07 259.881896177462
6e-07 259.899774937605
6.5e-07 248.067466928921
6.5e-07 248.426106988374
7e-07 237.898484839159
7e-07 237.912127219644
7.5e-07 228.715577349941
7.5e-07 229.035446980291
8e-07 220.877259926317
8e-07 220.888675121519
8.5e-07 213.601650538665
8.5e-07 213.873110458161
9e-07 207.373857821426
9e-07 207.383289319807
9.5e-07 201.623530533337
9.5e-07 201.846786271682
1e-06 196.614238088791
1e-06 196.621774202024
1.05e-06 191.960391422223
1.05e-06 192.148319952395
1.1e-06 187.882026120182
1.1e-06 187.888348218089
1.15e-06 184.064465553616
1.15e-06 184.226765518875
1.2e-06 180.756312074926
1.2e-06 180.761604166
1.25e-06 177.528264037174
1.25e-06 177.667611156506
1.3e-06 174.770517963741
1.3e-06 174.775375336545
1.35e-06 172.123527043632
1.35e-06 172.237954401474
1.4e-06 169.744111656349
1.4e-06 169.747947435318
1.45e-06 167.514691872923
1.45e-06 167.618805533482
1.5e-06 165.498908489277
1.5e-06 165.502085652862
1.55e-06 163.611324880299
1.55e-06 163.697582467995
1.6e-06 161.901536441782
1.6e-06 161.904343231208
1.65e-06 160.238993252367
1.65e-06 160.315764965839
1.7e-06 159.533007067881
1.7e-06 159.534826467575
1.75e-06 158.918482660348
1.75e-06 158.950023266713
1.8e-06 158.379982257208
1.8e-06 158.381031907486
1.85e-06 157.883685139391
1.85e-06 157.908130734055
1.9e-06 157.428945754119
1.9e-06 157.42997079805
1.95e-06 156.937696866835
1.95e-06 156.961971912293
2e-06 156.517212949706
2e-06 156.518053708913
2.05e-06 156.122271633319
2.05e-06 156.14191324682
2.1e-06 155.782021597916
2.1e-06 155.78274076897
2.15e-06 155.426526716481
2.15e-06 155.444591232408
2.2e-06 155.115949851608
2.2e-06 155.116573088936
2.25e-06 154.821432088156
2.25e-06 154.836218534372
2.3e-06 154.567790611113
2.3e-06 154.568307046039
2.35e-06 154.321243241862
2.35e-06 154.335031138205
2.4e-06 154.084441557229
2.4e-06 154.084914941718
2.45e-06 153.849000375529
2.45e-06 153.862191923102
2.5e-06 153.621547213873
2.5e-06 153.62199133774
2.55e-06 153.402446488369
2.55e-06 153.413523055927
2.6e-06 153.210905867821
2.6e-06 153.211296198745
2.65e-06 153.025931973034
2.65e-06 153.035322618093
2.7e-06 152.862935096482
2.7e-06 152.863287137728
2.75e-06 152.696197755177
2.75e-06 152.704677217123
2.8e-06 152.54942727622
2.8e-06 152.549732404243
2.85e-06 152.404769210044
2.85e-06 152.412146196348
2.9e-06 152.277267506098
2.9e-06 152.277529697166
2.95e-06 152.15311912348
2.95e-06 152.159468834421
3e-06 152.043481256028
3e-06 152.04370758522
3.05e-06 151.929524708773
3.05e-06 151.93535867569
3.1e-06 151.828588229634
3.1e-06 151.828796409121
3.15e-06 151.729846873826
3.15e-06 151.734912839467
3.2e-06 151.642261938696
3.2e-06 151.642443168084
3.25e-06 151.556435295627
3.25e-06 151.560848082431
3.3e-06 151.47604275662
3.3e-06 151.476219342103
3.35e-06 151.390688713567
3.35e-06 151.395080510302
3.4e-06 151.314422051337
3.4e-06 151.314579104956
3.45e-06 151.239384672339
3.45e-06 151.243240679794
3.5e-06 151.171914565119
3.5e-06 151.172053647918
3.55e-06 151.105436127341
3.55e-06 151.108856847727
3.6e-06 151.046068131778
3.6e-06 151.046190979021
3.65e-06 150.987513887344
3.65e-06 150.990531259321
3.7e-06 150.935169171288
3.7e-06 150.935277696369
3.75e-06 150.883487447591
3.75e-06 150.88615419076
3.8e-06 150.837243494077
3.8e-06 150.837339539598
3.85e-06 150.791541249309
3.85e-06 150.793902293751
3.9e-06 150.750612925743
3.9e-06 150.750698067169
3.95e-06 150.710128463162
3.95e-06 150.7122222496
4e-06 150.673844829734
4e-06 150.673920419101
4.05e-06 150.637925902455
4.05e-06 150.639785441367
4.1e-06 150.605711385506
4.1e-06 150.605778587183
4.15e-06 150.573797373689
4.15e-06 150.575451112043
4.2e-06 150.545156405616
4.2e-06 150.545216225697
4.25e-06 150.516763737368
4.25e-06 150.518236271071
4.3e-06 150.491268164173
4.3e-06 150.49132147473
4.35e-06 150.465978333376
4.35e-06 150.467290995471
4.4e-06 150.443257043011
4.4e-06 150.4433046022
4.45e-06 150.420706716456
4.45e-06 150.421878064775
4.5e-06 150.400437057725
4.5e-06 150.40047952651
4.55e-06 150.380309871017
4.55e-06 150.381356094156
4.6e-06 150.362210522817
4.6e-06 150.362248478948
4.65e-06 150.343294060306
4.65e-06 150.344356778549
4.7e-06 150.324880561035
4.7e-06 150.324918069366
4.75e-06 150.306571826193
4.75e-06 150.307520910948
4.8e-06 150.290085371043
4.8e-06 150.29011970187
4.85e-06 150.273690151293
4.85e-06 150.274540616825
4.9e-06 150.258921921083
4.9e-06 150.258952698471
4.95e-06 150.244230806786
4.95e-06 150.244993345909
5e-06 150.230994244486
5e-06 150.231021851166
5.05e-06 150.217823059584
5.05e-06 150.218507124605
5.1e-06 150.205953304198
5.1e-06 150.205978078954
5.15e-06 150.194139140826
5.15e-06 150.194753101813
5.2e-06 150.182856204096
5.2e-06 150.182879457671
5.25e-06 150.17158707643
5.25e-06 150.172171853347
5.3e-06 150.161425306717
5.3e-06 150.161446454301
5.35e-06 150.15130598018
5.35e-06 150.151831437348
5.4e-06 150.14217953109
5.4e-06 150.142198538562
5.45e-06 150.133089529716
5.45e-06 150.133561846448
5.5e-06 150.124890454628
5.5e-06 150.124907543978
5.55e-06 150.116722768173
5.55e-06 150.117147450312
5.6e-06 150.10935490462
5.6e-06 150.109370273818
5.65e-06 150.102014148418
5.65e-06 150.102396107107
5.7e-06 150.095391736008
5.7e-06 150.095405561777
5.75e-06 150.088792797362
5.75e-06 150.089136417663
5.8e-06 150.082839274499
5.8e-06 150.082851714749
5.85e-06 150.076906132554
5.85e-06 150.077215333385
5.9e-06 150.071553086078
5.9e-06 150.071564282022
5.95e-06 150.066217780296
5.95e-06 150.066496066906
6e-06 150.061404026292
6e-06 150.061414104316
6.05e-06 150.056605752642
6.05e-06 150.05685626266
6.1e-06 150.052276530157
6.1e-06 150.052285603452
6.15e-06 150.047960847384
6.15e-06 150.048186391322
6.2e-06 150.044067109617
6.2e-06 150.044075279623
6.25e-06 150.040185240712
6.25e-06 150.040388337655
6.3e-06 150.036683030603
6.3e-06 150.036690388282
6.35e-06 150.033191246817
6.35e-06 150.033374155934
6.4e-06 150.030041145919
6.4e-06 150.030047772881
6.45e-06 150.026900222562
6.45e-06 150.02706497098
6.5e-06 150.024066863725
6.5e-06 150.024072833226
6.55e-06 150.021241598549
6.55e-06 150.021390006044
6.6e-06 150.018693234313
6.6e-06 150.018698612135
6.65e-06 150.016152020593
6.65e-06 150.016285721503
6.7e-06 150.013860140314
6.7e-06 150.013864985557
6.75e-06 150.011574587866
6.75e-06 150.011695050553
6.8e-06 150.009513577603
6.8e-06 150.009517943377
6.85e-06 150.007458175876
6.85e-06 150.007566720052
6.9e-06 150.005596898216
6.9e-06 150.005600895189
6.95e-06 150.003716849162
6.95e-06 150.00381629028
7e-06 150.002022016378
7e-06 150.002025618471
7.05e-06 150.00033167645
7.05e-06 150.000421297429
7.1e-06 149.998808205815
7.1e-06 149.99881145234
7.15e-06 149.997288733987
7.15e-06 149.997369509451
7.2e-06 149.995919615765
7.2e-06 149.995922541982
7.25e-06 149.994554060722
7.25e-06 149.994626867662
7.3e-06 149.9933239896
7.3e-06 149.993326627245
7.35e-06 149.992097096859
7.35e-06 149.992162724581
7.4e-06 149.990992298193
7.4e-06 149.990994675831
7.45e-06 149.989890337437
7.45e-06 149.989949496467
7.5e-06 149.988898411754
7.5e-06 149.988900555102
7.55e-06 149.987909022258
7.55e-06 149.987962352315
7.6e-06 149.987018806002
7.6e-06 149.987020738218
7.65e-06 149.986130858156
7.65e-06 149.9861789353
7.7e-06 149.985332298899
7.7e-06 149.985334040838
7.75e-06 149.984535770079
7.75e-06 149.984579113115
7.8e-06 149.983819814622
7.8e-06 149.983821385068
7.85e-06 149.983105677851
7.85e-06 149.983144754085
7.9e-06 149.982464171843
7.9e-06 149.98246558772
7.95e-06 149.981824295884
7.95e-06 149.98185952628
8e-06 149.981249893719
8e-06 149.981251170269
8.05e-06 149.980676953371
8.05e-06 149.980708717187
8.1e-06 149.98016303727
8.1e-06 149.980164188231
8.15e-06 149.979650432809
8.15e-06 149.979679071763
8.2e-06 149.979191040141
8.2e-06 149.979192077887
8.25e-06 149.978732824926
8.25e-06 149.978758646935
8.3e-06 149.978322582879
8.3e-06 149.978323518564
8.35e-06 149.977913398288
8.35e-06 149.977936680836
8.4e-06 149.97754746515
8.4e-06 149.977548308826
8.45e-06 149.977182482078
8.45e-06 149.977203475238
8.5e-06 149.976856494431
8.5e-06 149.976857255155
8.55e-06 149.976531360673
8.55e-06 149.976550289831
8.6e-06 149.976241385891
8.6e-06 149.97624207183
8.65e-06 149.975952178812
8.65e-06 149.975969247115
8.7e-06 149.975694672332
8.7e-06 149.975695290843
8.75e-06 149.975437856272
8.75e-06 149.975453246832
8.8e-06 149.975209623129
8.8e-06 149.975210180848
8.85e-06 149.974982011074
8.85e-06 149.97499588895
8.9e-06 149.97478017129
8.9e-06 149.974780674196
8.95e-06 149.974578890362
8.95e-06 149.974591404349
9e-06 149.974400847782
9e-06 149.974401301266
9.05e-06 149.974223308174
9.05e-06 149.974234592408
9.1e-06 149.974066722405
9.1e-06 149.974067131327
9.15e-06 149.973910589405
9.15e-06 149.973920764811
9.2e-06 149.973773350545
9.2e-06 149.973773719288
9.25e-06 149.973636519341
9.25e-06 149.97364569494
9.3e-06 149.973516725226
9.3e-06 149.973517057739
9.35e-06 149.97339729821
9.35e-06 149.973405572292
9.4e-06 149.973293233908
9.4e-06 149.973293533752
9.45e-06 149.973189500234
9.45e-06 149.973196961416
9.5e-06 149.973099619578
9.5e-06 149.973099889964
9.55e-06 149.973010036744
9.55e-06 149.973016764926
9.6e-06 149.972932945691
9.6e-06 149.972933189515
9.65e-06 149.972856122947
9.65e-06 149.972862190167
9.7e-06 149.972790564588
9.7e-06 149.97279078446
9.75e-06 149.972725247976
9.75e-06 149.972730719188
9.8e-06 149.972670089021
9.8e-06 149.972670287295
9.85e-06 149.972615147907
9.85e-06 149.972620081677
9.9e-06 149.972569366513
9.9e-06 149.97256954531
9.95e-06 149.972523781437
9.95e-06 149.972528230574
1e-05 149.972486456233
1e-05 149.972486617468
1.005e-05 149.972449307969
1.005e-05 149.972453320089
1.01e-05 149.972419608166
1.01e-05 149.972419753564
1.015e-05 149.972390067852
1.015e-05 149.972393685891
1.02e-05 149.972367244329
1.02e-05 149.972367375446
1.025e-05 149.972344564577
1.025e-05 149.97234782725
1.03e-05 149.972327941844
1.03e-05 149.972328060083
1.035e-05 149.972311448726
1.035e-05 149.972314390944
1.04e-05 149.972300417683
1.04e-05 149.972300524309
1.045e-05 149.9722895035
1.045e-05 149.972292156743
1.05e-05 149.972283514906
1.05e-05 149.97228361106
1.055e-05 149.97227763168
1.055e-05 149.972280024334
1.06e-05 149.972276190265
1.06e-05 149.972276276976
1.065e-05 149.972274843864
1.065e-05 149.972277001527
1.07e-05 149.972277503024
1.07e-05 149.972277581219
1.075e-05 149.972280247868
1.075e-05 149.972282193621
1.08e-05 149.972286604879
1.08e-05 149.972286675394
1.085e-05 149.972293039165
1.085e-05 149.972294793823
1.09e-05 149.972302730872
1.09e-05 149.972302794462
1.095e-05 149.972312492275
1.095e-05 149.972314074607
1.1e-05 149.972325191198
1.1e-05 149.972325248543
1.105e-05 149.972337952988
1.105e-05 149.97233937992
1.11e-05 149.972353363821
1.11e-05 149.972353415534
1.115e-05 149.972368831364
1.115e-05 149.972370118158
1.12e-05 149.972386687808
1.12e-05 149.972386734443
1.125e-05 149.972404595414
1.125e-05 149.972405755834
1.13e-05 149.972424657332
1.13e-05 149.972424699386
1.135e-05 149.972444765409
1.135e-05 149.972445811866
1.14e-05 149.972466816249
1.14e-05 149.972466854174
1.145e-05 149.97248890874
1.145e-05 149.972489852429
1.15e-05 149.972512753225
1.15e-05 149.972512787426
1.155e-05 149.972536635295
1.155e-05 149.972537486306
1.16e-05 149.972562097326
1.16e-05 149.972562128168
1.165e-05 149.972587593276
1.165e-05 149.972588360713
1.17e-05 149.972614514051
1.17e-05 149.972614541864
1.175e-05 149.972641465441
1.175e-05 149.97264215751
1.18e-05 149.972669701755
1.18e-05 149.972669726837
1.185e-05 149.972697965706
1.185e-05 149.972698589809
1.19e-05 149.97272738842
1.19e-05 149.972727411039
1.195e-05 149.972756836085
1.195e-05 149.972757398897
1.2e-05 149.972787328743
1.2e-05 149.97278734914
1.205e-05 149.97281784393
1.205e-05 149.97281835147
1.21e-05 149.972849301513
1.21e-05 149.972849319907
1.215e-05 149.972880779442
1.215e-05 149.972881237139
1.22e-05 149.972913107246
1.22e-05 149.972913123834
1.225e-05 149.972945453428
1.225e-05 149.972945866175
1.23e-05 149.972978566049
1.23e-05 149.972978581008
1.235e-05 149.973011695273
1.235e-05 149.973012067486
1.24e-05 149.973045515696
1.24e-05 149.973045529186
1.245e-05 149.97307935112
1.245e-05 149.973079686779
1.25e-05 149.973113809892
1.25e-05 149.973113822058
1.255e-05 149.973148282222
1.255e-05 149.973148584916
1.26e-05 149.973183316711
1.26e-05 149.973183327682
1.265e-05 149.973218363458
1.265e-05 149.973218636424
1.27e-05 149.973253917183
1.27e-05 149.973253927077
1.275e-05 149.973289481992
1.275e-05 149.973289728151
1.28e-05 149.97332550402
1.28e-05 149.973325512942
1.285e-05 149.973361536073
1.285e-05 149.973361758056
1.29e-05 149.97339798047
1.29e-05 149.973397988516
1.295e-05 149.973434433939
1.295e-05 149.973434634121
1.3e-05 149.973471259286
1.3e-05 149.973471266542
1.305e-05 149.973508092843
1.305e-05 149.973508273365
1.31e-05 149.973545261784
1.31e-05 149.973545268328
1.315e-05 149.973582438161
1.315e-05 149.973582600952
1.32e-05 149.973619917011
1.32e-05 149.973619922912
1.325e-05 149.973657402597
1.325e-05 149.973657549401
1.33e-05 149.97369516098
1.33e-05 149.973695166302
1.335e-05 149.973732925468
1.335e-05 149.973733057852
1.34e-05 149.973770935988
1.34e-05 149.973770940787
1.345e-05 149.973808952044
1.345e-05 149.973809071426
1.35e-05 149.973847189998
1.35e-05 149.973847194326
1.355e-05 149.973885432975
1.355e-05 149.973885540631
1.36e-05 149.973923876085
1.36e-05 149.973923879988
1.365e-05 149.973962323754
1.365e-05 149.973962420836
1.37e-05 149.974000951929
1.37e-05 149.974000955449
1.375e-05 149.974039584246
1.375e-05 149.974039671792
1.38e-05 149.974078379368
1.38e-05 149.974078382542
1.385e-05 149.974117178257
1.385e-05 149.974117257203
1.39e-05 149.974156123988
1.39e-05 149.97415612685
1.395e-05 149.974195073147
1.395e-05 149.974195144337
1.4e-05 149.974234154753
1.4e-05 149.974234157334
1.405e-05 149.97427323948
1.405e-05 149.974273303678
1.41e-05 149.974312443674
1.41e-05 149.974312446003
1.415e-05 149.974351650714
1.415e-05 149.974351708605
1.42e-05 149.974390965514
1.42e-05 149.974390967614
1.425e-05 149.974430282911
1.425e-05 149.974430335114
1.43e-05 149.974469697511
1.43e-05 149.974469699404
1.435e-05 149.974509114484
1.435e-05 149.974509161558
1.44e-05 149.974548619139
1.44e-05 149.974548620847
1.445e-05 149.974588125965
1.445e-05 149.974588168414
1.45e-05 149.974627711889
1.45e-05 149.974627713428
1.455e-05 149.9746672998
1.455e-05 149.974667338078
1.46e-05 149.974706959067
1.46e-05 149.974706960455
1.465e-05 149.974746620157
1.465e-05 149.974746654673
1.47e-05 149.97478634562
1.47e-05 149.974786346873
1.475e-05 149.974826072759
1.475e-05 149.974826103883
1.48e-05 149.974865857974
1.48e-05 149.974865859104
1.485e-05 149.974905644731
1.485e-05 149.974905672796
1.49e-05 149.974945483887
1.49e-05 149.974945484905
1.495e-05 149.974985324463
1.495e-05 149.974985349769
1.5e-05 149.975025212317
1.5e-05 149.975025213236
1.505e-05 149.975065101484
1.505e-05 149.975065124303
1.51e-05 149.975105033311
1.51e-05 149.97510503414
1.515e-05 149.975144966352
1.515e-05 149.975144986927
1.52e-05 149.97518493789
1.52e-05 149.975184938637
1.525e-05 149.975224910552
1.525e-05 149.975224929104
1.53e-05 149.975264917956
1.53e-05 149.97526491863
1.535e-05 149.975304926405
1.535e-05 149.975304943133
1.54e-05 149.975344966209
1.54e-05 149.975344966817
1.545e-05 149.975385006987
1.545e-05 149.975385022069
1.55e-05 149.975425076065
1.55e-05 149.975425076613
1.555e-05 149.975465146051
1.555e-05 149.97546515965
1.56e-05 149.975505241585
1.56e-05 149.975505242079
1.565e-05 149.975545337968
1.565e-05 149.975545350228
1.57e-05 149.975585457414
1.57e-05 149.97558545786
1.575e-05 149.975625577657
1.575e-05 149.975625588711
1.58e-05 149.975665718724
1.58e-05 149.975665719126
1.585e-05 149.97570586054
1.585e-05 149.975705870506
1.59e-05 149.97574602116
1.59e-05 149.975746021523
1.595e-05 149.975786182486
1.595e-05 149.97578619147
1.6e-05 149.975826360795
1.6e-05 149.975826361122
1.605e-05 149.975866539771
1.605e-05 149.97586654787
1.61e-05 149.975906734087
1.61e-05 149.975906734382
1.615e-05 149.975946929035
1.615e-05 149.975946936336
1.62e-05 149.975987137842
1.62e-05 149.975987138108
1.625e-05 149.97602734725
1.625e-05 149.976027353832
1.63e-05 149.976067569181
1.63e-05 149.976067569421
1.635e-05 149.976107791685
1.635e-05 149.976107797617
1.64e-05 149.976148025506
1.64e-05 149.976148025722
1.645e-05 149.976188259875
1.645e-05 149.976188265222
1.65e-05 149.976228504475
1.65e-05 149.97622850467
1.655e-05 149.976268749599
1.655e-05 149.976268754419
1.66e-05 149.976309003975
1.66e-05 149.976309004151
1.665e-05 149.976349258855
1.665e-05 149.976349263198
1.67e-05 149.976389522102
1.67e-05 149.976389522261
1.675e-05 149.976429785835
1.675e-05 149.976429789749
1.68e-05 149.976470057138
1.68e-05 149.976470057282
1.685e-05 149.97651032891
1.685e-05 149.976510332437
1.69e-05 149.976550607534
1.69e-05 149.976550607664
1.695e-05 149.976590886612
1.695e-05 149.97659088979
1.7e-05 149.976631171893
1.7e-05 149.97663117201
1.705e-05 149.976671457614
1.705e-05 149.976671460477
1.71e-05 149.976711748955
1.71e-05 149.976711749061
1.715e-05 149.976752040723
1.715e-05 149.976752043302
1.72e-05 149.976792337584
1.72e-05 149.97679233768
1.725e-05 149.976832634861
1.725e-05 149.976832637184
1.73e-05 149.976872936756
1.73e-05 149.976872936842
1.735e-05 149.976913239056
1.735e-05 149.976913241148
1.74e-05 149.976953545546
1.74e-05 149.976953545623
1.745e-05 149.976993852432
1.745e-05 149.976993854316
1.75e-05 149.977034163121
1.75e-05 149.977034163191
1.755e-05 149.977074474199
1.755e-05 149.977074475895
1.76e-05 149.977114788731
1.76e-05 149.977114788794
1.765e-05 149.977155103643
1.765e-05 149.97715510517
1.77e-05 149.977195421696
1.77e-05 149.977195421753
1.775e-05 149.977235740123
1.775e-05 149.977235741498
1.78e-05 149.977276061407
1.78e-05 149.977276061459
1.785e-05 149.977316383059
1.785e-05 149.977316384296
1.79e-05 149.977356707313
1.79e-05 149.977356707359
1.795e-05 149.977397031929
1.795e-05 149.977397033041
1.8e-05 149.977437358915
1.8e-05 149.977437358957
1.805e-05 149.97747768626
1.805e-05 149.97747768726
1.81e-05 149.977518015767
1.81e-05 149.977518015805
1.815e-05 149.977558345628
1.815e-05 149.977558346527
1.82e-05 149.977598677464
1.82e-05 149.977598677498
1.825e-05 149.977639009649
1.825e-05 149.977639010458
1.83e-05 149.977679343641
1.83e-05 149.977679343672
1.835e-05 149.977719677979
1.835e-05 149.977719678705
1.84e-05 149.977760013971
1.84e-05 149.977760013999
1.845e-05 149.977800350305
1.845e-05 149.977800350957
1.85e-05 149.977840688155
1.85e-05 149.977840688181
1.855e-05 149.977881026346
1.855e-05 149.977881026931
1.86e-05 149.977921365929
1.86e-05 149.977921365951
1.865e-05 149.977961705848
1.865e-05 149.977961706374
1.87e-05 149.978002047049
1.87e-05 149.97800204707
1.875e-05 149.978042388584
1.875e-05 149.978042389055
1.88e-05 149.978082731299
1.88e-05 149.978082731318
1.885e-05 149.978123074346
1.885e-05 149.978123074768
1.89e-05 149.978163418483
1.89e-05 149.9781634185
1.895e-05 149.97820376295
1.895e-05 149.978203763328
1.9e-05 149.978244108424
1.9e-05 149.978244108439
1.905e-05 149.978284454227
1.905e-05 149.978284454565
1.91e-05 149.978324800963
1.91e-05 149.978324800977
1.915e-05 149.978365148026
1.915e-05 149.978365148328
1.92e-05 149.978405495955
1.92e-05 149.978405495968
1.925e-05 149.978445844211
1.925e-05 149.97844584448
1.93e-05 149.978486193272
1.93e-05 149.978486193283
1.935e-05 149.978526542658
1.935e-05 149.978526542898
1.94e-05 149.978566892796
1.94e-05 149.978566892806
1.945e-05 149.978607243257
1.945e-05 149.978607243472
1.95e-05 149.978647594422
1.95e-05 149.978647594431
1.955e-05 149.978687945909
1.955e-05 149.978687946099
1.96e-05 149.978728298055
1.96e-05 149.978728298063
1.965e-05 149.978768650522
1.965e-05 149.978768650691
1.97e-05 149.978809003609
1.97e-05 149.978809003617
1.975e-05 149.978849357016
1.975e-05 149.978849357166
1.98e-05 149.978889711006
1.98e-05 149.978889711014
1.985e-05 149.978930065317
1.985e-05 149.978930065449
1.99e-05 149.978970420178
1.99e-05 149.978970420185
1.995e-05 149.979010775359
1.995e-05 149.979010775475
2e-05 149.979051131061
2e-05 149.979051131067
2.005e-05 149.979091487082
2.005e-05 149.979091487184
2.01e-05 149.979131843598
};
\end{groupplot}

\end{tikzpicture}

\end{figure}

