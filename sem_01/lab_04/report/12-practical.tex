\chapter{Практическая часть}

\section{Реализация}
\begin{lstlisting}[caption={Теплоёмкость стержня}]
def c(T):
    return a2 + b2 * T ** m2 - c2 / T ** 2
\end{lstlisting}

\begin{lstlisting}[caption={Коэффициенты теплопроводности материала стержня и теплоотдачи при обдуве}]
def k(T):
    return a1 * (b1 + c1 * T ** m1)


def alpha(x):
    d = (alphaN * l) / (alphaN - alpha0)
    c = -alpha0 * d
    return c / (x - d)
\end{lstlisting}

\begin{lstlisting}[caption={Замены p и f}]
def p(x):
    return 2 * alpha(x) / R


def f(x):
    return 2 * alpha(x) * T0 / R
\end{lstlisting}

\begin{lstlisting}[caption={Метод средних}]
def FAddHalf(x, h, F):
    return (F(x) + F(x + h)) / 2


def FSubHalf(x, h, F):
    return (F(x) + F(x - h)) / 2
\end{lstlisting}

\begin{lstlisting}[caption={Параметры разностной схемы}]
def A(T):
    return FSubHalf(T, t, k) * t / h


def D(T):
    return FAddHalf(T, t, k) * t / h


def B(x, T):
    return A(T) + D(T) + c(T) * h + p(x) * h * t


def F(x, T):
    return f(x) * h * t + c(T) * T * h
\end{lstlisting}

\begin{lstlisting}[caption={Краевые условия}]
def bounds_left(prevT):
    K0 = h / 8 * FAddHalf(
            prevT[0], t, c) + h / 4 * c(prevT[0]) + \
            FAddHalf(prevT[0], t, k) * t / h + \
            t * h / 8 * p(h / 2) + t * h / 4 * p(0)

    M0 = h / 8 * FAddHalf(
            prevT[0], t, c) - FAddHalf(prevT[0], t, k) * \
            t / h + t * h * p(h / 2) / 8

    P0 = h / 8 * FAddHalf(
            prevT[0], t, c) * (prevT[0] + prevT[1]) + \
            h / 4 * c(prevT[0]) * prevT[0] + F0 * t + \
            t * h / 8 * (3 * f(0) + f(h))

    return K0, M0, P0


def bounds_right(prevT):
    KN = h / 8 * FSubHalf(
            prevT[-1], t, c) + h / 4 * c(prevT[-1]) + \
            FSubHalf(prevT[-1], t, k) * t / h + t * \
            alphaN + t * h / 8 * p(l - h / 2) + t * h / 4 * p(l)

    MN = h / 8 * FSubHalf(
            prevT[-1], t, c) - FSubHalf(prevT[-1], t, k) * \
            t / h + t * h * p(l - h / 2) / 8

    PN = h / 8 * FSubHalf(
            prevT[-1], t, c) * (prevT[-1] + prevT[-2]) + h / \
            4 * c(prevT[-1]) * prevT[-1] + t * alphaN * T0 + \
            t * h / 4 * (f(l) + f(l - h / 2))

    return KN, MN, PN
\end{lstlisting}

\begin{lstlisting}[caption={Метод прогонки}]
def thomas(prevT):
    K0, M0, P0 = bounds_left(prevT)
    KN, MN, PN = bounds_right(prevT)

    # Прямой ход
    eps = [0, -M0 / K0]
    eta = [0, P0 / K0]

    x = h
    n = 1
    while (x + h < l):
        eps.append(D(prevT[n]) / (B(x, prevT[n]) - A(prevT[n]) * eps[n]))
        eta.append((
            F(x, prevT[n]) + A(prevT[n]) * eta[n]) /
            (B(x, prevT[n]) - A(prevT[n]) * eps[n]))

        n += 1
        x += h

    # Обратный ход
    y = [0] * (n + 1)
    y[n] = (PN - MN * eta[n]) / (KN + MN * eps[n])

    for i in range(n - 1, -1, -1):
        y[i] = eps[i + 1] * y[i + 1] + eta[i + 1]

    return y
\end{lstlisting}

\begin{lstlisting}[caption={Метод простых итераций}]
def termtest1(T, newT):
    max = fabs((T[0] - newT[0]) / newT[0])

    for i, j in zip(T, newT):
        d = fabs(i - j) / j

        if d > max:
            max = d

    return max < 1


def termtest2(T, newT):
    for i, j in zip(T, newT):
        if fabs((i - j) / j) > 1e-2:
            return True

    return False


def fixed_point_iteration():
    n = int(l / h)
    ti = 0

    result = []
    newT = [0] * (n + 1)
    T = [T0] * (n + 1)

    result.append(T)

    while (True):
        buf = T

        while True:
            newT = thomas(buf)
            if termtest1(buf, newT):
                break
            buf = newT

        result.append(newT)
        ti += t

        if (termtest2(T, newT) == False):
            break

        T = newT

    return result, ti
\end{lstlisting}

\section{Результаты работы}

\subsection{Представить разностный аналог краевого условия при $x = l$ и его краткий вывод интегро-интерполяционным методом}
См. раздел~\ref{theory}.

\subsection{График зависимости температуры от координаты при нескольких фиксированных значениях времени}
На рисунке представлены графики зависимости температуры от координаты при нескольких фиксированных $t$. Последняя~--- синяя кривая соответствует установившемуся режиму, когда поле перестает меняться с точностью $1\text{e--}3$.
\begin{figure}[H]
    \caption{Зависимость температуры от координаты стержня}\label{img:plot01}
    % This file was created by tikzplotlib v0.9.2.
\begin{tikzpicture}[scale=1.25]

\definecolor{color0}{rgb}{0.12156862745098,0.466666666666667,0.705882352941177}

\begin{axis}[
    x label style={at={(axis description cs:0.5,-0.05)},anchor=north},
    y label style={at={(axis description cs:-0.005,.5)},rotate=0,anchor=south},
tick align=outside,
tick pos=left,
title={\(\displaystyle T(x)\)},
x grid style={white!69.0196078431373!black},
xlabel={\(\displaystyle x\), см},
xmin=-0.49995, xmax=10.49895,
xtick style={color=black},
ytick={300,325,...,450},
y grid style={white!69.0196078431373!black},
ylabel={\(\displaystyle T\), K},
ymin=293.406278639121, ymax=440.034664472604,
ytick style={color=black}
]
\addplot [semithick, color0]
table {%
0 433.36973784381
0.001 433.244763542199
0.002 433.119915523658
0.003 432.995193647028
0.004 432.870597771326
0.005 432.74612775575
0.006 432.621783459678
0.007 432.497564742664
0.008 432.373471464445
0.009 432.24950348493
0.01 432.125660664211
0.011 432.001942862556
0.012 431.878349940408
0.013 431.754881758391
0.014 431.631538177304
0.015 431.508319058122
0.016 431.385224261997
0.017 431.262253650257
0.018 431.139407084405
0.019 431.016684426122
0.02 430.894085537261
0.021 430.771610279852
0.022 430.649258516099
0.023 430.527030108382
0.024 430.404924919253
0.025 430.28294281144
0.026 430.161083647843
0.027 430.039347291537
0.028 429.917733605769
0.029 429.796242453959
0.03 429.674873699701
0.031 429.55362720676
0.032 429.432502839075
0.033 429.311500460755
0.034 429.190619936082
0.035 429.069861129508
0.036 428.949223905657
0.037 428.828708129326
0.038 428.708313665479
0.039 428.588040379253
0.04 428.467888135954
0.041 428.347856801058
0.042 428.227946240212
0.043 428.108156319231
0.044 427.9884869041
0.045 427.868937860973
0.046 427.749509056171
0.047 427.630200356185
0.048 427.511011627676
0.049 427.391942737469
0.05 427.272993552559
0.051 427.154163940108
0.052 427.035453767447
0.053 426.916862902071
0.054 426.798391211643
0.055 426.680038563992
0.056 426.561804827116
0.057 426.443689869175
0.058 426.325693558498
0.059 426.207815763577
0.06 426.090056353071
0.061 425.972415195803
0.062 425.854892160762
0.063 425.7374871171
0.064 425.620199934135
0.065 425.503030481348
0.066 425.385978628383
0.067 425.26904424505
0.068 425.15222720132
0.069 425.035527367329
0.07 424.918944613375
0.071 424.802478809919
0.072 424.686129827583
0.073 424.569897537153
0.074 424.453781809576
0.075 424.337782515961
0.076 424.221899527579
0.077 424.106132715861
0.078 423.990481952399
0.079 423.874947108948
0.08 423.759528057421
0.081 423.644224669892
0.082 423.529036818596
0.083 423.413964375927
0.084 423.299007214439
0.085 423.184165206844
0.086 423.069438226016
0.087 422.954826144985
0.088 422.840328836942
0.089 422.725946175234
0.09 422.611678033368
0.091 422.49752428501
0.092 422.38348480398
0.093 422.26955946426
0.094 422.155748139986
0.095 422.042050705452
0.096 421.928467035111
0.097 421.814997003568
0.098 421.701640485589
0.099 421.588397356094
0.1 421.475267490159
0.101 421.362250763017
0.102 421.249347050053
0.103 421.136556226813
0.104 421.023878168993
0.105 420.911312752445
0.106 420.798859853178
0.107 420.686519347353
0.108 420.574291111286
0.109 420.462175021446
0.11 420.350170954457
0.111 420.238278787096
0.112 420.126498396294
0.113 420.014829659132
0.114 419.903272452849
0.115 419.791826654831
0.116 419.680492142621
0.117 419.569268793912
0.118 419.458156486548
0.119 419.347155098527
0.12 419.236264507998
0.121 419.12548459326
0.122 419.014815232764
0.123 418.904256305112
0.124 418.793807689056
0.125 418.683469263499
0.126 418.573240907494
0.127 418.463122500245
0.128 418.353113921103
0.129 418.243215049572
0.13 418.133425765303
0.131 418.023745948097
0.132 417.914175477905
0.133 417.804714234823
0.134 417.6953620991
0.135 417.586118951132
0.136 417.47698467146
0.137 417.367959140777
0.138 417.259042239922
0.139 417.15023384988
0.14 417.041533851786
0.141 416.93294212692
0.142 416.82445855671
0.143 416.716083022729
0.144 416.607815406699
0.145 416.499655590484
0.146 416.3916034561
0.147 416.283658885702
0.148 416.175821761597
0.149 416.068091966232
0.15 415.960469382202
0.151 415.852953892247
0.152 415.745545379251
0.153 415.638243726242
0.154 415.531048816393
0.155 415.423960533021
0.156 415.316978759587
0.157 415.210103379696
0.158 415.103334277096
0.159 414.996671335678
0.16 414.890114439477
0.161 414.78366347267
0.162 414.677318319577
0.163 414.571078864662
0.164 414.464944992527
0.165 414.358916587922
0.166 414.252993535733
0.167 414.147175720992
0.168 414.041463028871
0.169 413.935855344682
0.17 413.830352553881
0.171 413.724954542061
0.172 413.619661194959
0.173 413.514472398451
0.174 413.409388038554
0.175 413.304408001423
0.176 413.199532173356
0.177 413.094760440789
0.178 412.990092690297
0.179 412.885528808595
0.18 412.781068682537
0.181 412.676712199115
0.182 412.572459245462
0.183 412.468309708847
0.184 412.364263476678
0.185 412.260320436503
0.186 412.156480476005
0.187 412.052743483006
0.188 411.949109345465
0.189 411.845577951481
0.19 411.742149189285
0.191 411.63882294725
0.192 411.535599113883
0.193 411.432477577828
0.194 411.329458227865
0.195 411.226540952912
0.196 411.12372564202
0.197 411.021012184379
0.198 410.918400469312
0.199 410.815890386279
0.2 410.713481824874
0.201 410.611174674826
0.202 410.508968826001
0.203 410.406864168396
0.204 410.304860592147
0.205 410.202957987519
0.206 410.101156244916
0.207 409.999455254873
0.208 409.897854908059
0.209 409.796355095277
0.21 409.694955707463
0.211 409.593656635686
0.212 409.49245777115
0.213 409.391359005188
0.214 409.290360229269
0.215 409.189461334991
0.216 409.088662214088
0.217 408.987962758423
0.218 408.887362859992
0.219 408.786862410923
0.22 408.686461303475
0.221 408.586159430037
0.222 408.485956683133
0.223 408.385852955412
0.224 408.28584813966
0.225 408.185942128788
0.226 408.086134815841
0.227 407.986426093993
0.228 407.886815856547
0.229 407.787303996938
0.23 407.68789040873
0.231 407.588574985613
0.232 407.489357621412
0.233 407.390238210077
0.234 407.291216645687
0.235 407.192292822453
0.236 407.09346663471
0.237 406.994737976926
0.238 406.896106743692
0.239 406.797572829733
0.24 406.699136129896
0.241 406.60079653916
0.242 406.502553952628
0.243 406.404408265534
0.244 406.306359373236
0.245 406.208407171221
0.246 406.1105515551
0.247 406.012792420614
0.248 405.915129663629
0.249 405.817563180135
0.25 405.720092866252
0.251 405.622718618223
0.252 405.525440332418
0.253 405.428257905332
0.254 405.331171233585
0.255 405.234180213924
0.256 405.137284743218
0.257 405.040484718464
0.258 404.943780036781
0.259 404.847170595415
0.26 404.750656291733
0.261 404.654237023229
0.262 404.557912687521
0.263 404.461683182348
0.264 404.365548405575
0.265 404.26950825519
0.266 404.173562629304
0.267 404.077711426152
0.268 403.98195454409
0.269 403.886291881599
0.27 403.79072333728
0.271 403.695248809859
0.272 403.599868198183
0.273 403.504581401221
0.274 403.409388318065
0.275 403.314288847927
0.276 403.219282890141
0.277 403.124370344164
0.278 403.029551109573
0.279 402.934825086065
0.28 402.84019217346
0.281 402.745652271698
0.282 402.651205280839
0.283 402.556851101064
0.284 402.462589632673
0.285 402.368420776088
0.286 402.27434443185
0.287 402.18036050062
0.288 402.086468883177
0.289 401.992669480422
0.29 401.898962193373
0.291 401.805346923168
0.292 401.711823571065
0.293 401.618392038439
0.294 401.525052226784
0.295 401.431804037713
0.296 401.338647372957
0.297 401.245582134364
0.298 401.152608223902
0.299 401.059725543656
0.3 400.966933995827
0.301 400.874233482736
0.302 400.781623906819
0.303 400.689105170631
0.304 400.596677176843
0.305 400.504339828243
0.306 400.412093027735
0.307 400.31993667834
0.308 400.227870683197
0.309 400.135894945557
0.31 400.044009368791
0.311 399.952213856385
0.312 399.860508311939
0.313 399.768892639169
0.314 399.677366741908
0.315 399.585930524103
0.316 399.494583889817
0.317 399.403326743225
0.318 399.31215898862
0.319 399.221080530408
0.32 399.13009127311
0.321 399.039191121361
0.322 398.94837997991
0.323 398.85765775362
0.324 398.767024347469
0.325 398.676479666546
0.326 398.586023616056
0.327 398.495656101316
0.328 398.405377027756
0.329 398.31518630092
0.33 398.225083826465
0.331 398.135069510158
0.332 398.045143257883
0.333 397.955304975633
0.334 397.865554569514
0.335 397.775891945745
0.336 397.686317010656
0.337 397.596829670689
0.338 397.507429832397
0.339 397.418117402446
0.34 397.328892287612
0.341 397.239754394783
0.342 397.150703630958
0.343 397.061739903246
0.344 396.972863118867
0.345 396.884073185153
0.346 396.795370009545
0.347 396.706753499593
0.348 396.618223562961
0.349 396.529780107419
0.35 396.441423040848
0.351 396.35315227124
0.352 396.264967706697
0.353 396.176869255426
0.354 396.088856825749
0.355 396.000930326092
0.356 395.913089664994
0.357 395.8253347511
0.358 395.737665493165
0.359 395.650081800052
0.36 395.562583580732
0.361 395.475170744286
0.362 395.3878431999
0.363 395.30060085687
0.364 395.213443624599
0.365 395.126371412599
0.366 395.039384130488
0.367 394.952481687991
0.368 394.865663994941
0.369 394.778930961279
0.37 394.69228249705
0.371 394.605718512408
0.372 394.519238917613
0.373 394.432843623032
0.374 394.346532539137
0.375 394.260305576507
0.376 394.174162645827
0.377 394.088103657888
0.378 394.002128523587
0.379 393.916237153924
0.38 393.830429460008
0.381 393.744705353052
0.382 393.659064744372
0.383 393.573507545394
0.384 393.488033667643
0.385 393.402643022753
0.386 393.31733552246
0.387 393.232111078606
0.388 393.146969603138
0.389 393.061911008104
0.39 392.976935205659
0.391 392.892042108061
0.392 392.807231627671
0.393 392.722503676955
0.394 392.637858168481
0.395 392.55329501492
0.396 392.468814129049
0.397 392.384415423746
0.398 392.30009881199
0.399 392.215864206867
0.4 392.131711521563
0.401 392.047640669366
0.402 391.963651563668
0.403 391.879744117963
0.404 391.795918245845
0.405 391.712173861013
0.406 391.628510877266
0.407 391.544929208504
0.408 391.46142876873
0.409 391.378009472048
0.41 391.294671232663
0.411 391.211413964881
0.412 391.12823758311
0.413 391.045142001857
0.414 390.96212713573
0.415 390.879192899441
0.416 390.796339207797
0.417 390.713565975709
0.418 390.630873118188
0.419 390.548260550344
0.42 390.465728187386
0.421 390.383275944624
0.422 390.300903737469
0.423 390.218611481428
0.424 390.136399092112
0.425 390.054266485225
0.426 389.972213576577
0.427 389.890240282072
0.428 389.808346517715
0.429 389.726532199608
0.43 389.644797243955
0.431 389.563141567054
0.432 389.481565085305
0.433 389.400067715204
0.434 389.318649373345
0.435 389.237309976422
0.436 389.156049441225
0.437 389.074867684641
0.438 388.993764623657
0.439 388.912740175355
0.44 388.831794256915
0.441 388.750926785615
0.442 388.670137678829
0.443 388.589426854029
0.444 388.508794228781
0.445 388.428239720751
0.446 388.3477632477
0.447 388.267364727485
0.448 388.187044078059
0.449 388.106801217472
0.45 388.02663606387
0.451 387.946548535494
0.452 387.866538550682
0.453 387.786606027866
0.454 387.706750885574
0.455 387.62697304243
0.456 387.547272417153
0.457 387.467648928557
0.458 387.388102495549
0.459 387.308633037135
0.46 387.229240472412
0.461 387.149924720573
0.462 387.070685700905
0.463 386.99152333279
0.464 386.912437535703
0.465 386.833428229215
0.466 386.754495332989
0.467 386.675638766783
0.468 386.596858450449
0.469 386.518154303931
0.47 386.439526247267
0.471 386.36097420059
0.472 386.282498084123
0.473 386.204097818186
0.474 386.125773323188
0.475 386.047524519634
0.476 385.96935132812
0.477 385.891253669334
0.478 385.813231464059
0.479 385.735284633168
0.48 385.657413097627
0.481 385.579616778494
0.482 385.50189559692
0.483 385.424249474145
0.484 385.346678331505
0.485 385.269182090423
0.486 385.191760672417
0.487 385.114413999095
0.488 385.037141992156
0.489 384.95994457339
0.49 384.882821664679
0.491 384.805773187995
0.492 384.728799065402
0.493 384.651899219052
0.494 384.57507357119
0.495 384.498322044151
0.496 384.42164456036
0.497 384.345041042331
0.498 384.26851141267
0.499 384.192055594071
0.5 384.115673509321
0.501 384.039365081292
0.502 383.963130232949
0.503 383.886968887346
0.504 383.810880967626
0.505 383.734866397021
0.506 383.658925098851
0.507 383.583056996527
0.508 383.507262013549
0.509 383.431540073503
0.51 383.355891100066
0.511 383.280315017003
0.512 383.204811748167
0.513 383.1293812175
0.514 383.054023349032
0.515 382.97873806688
0.516 382.90352529525
0.517 382.828384958435
0.518 382.753316980818
0.519 382.678321286866
0.52 382.603397801136
0.521 382.528546448271
0.522 382.453767153002
0.523 382.379059840148
0.524 382.304424434613
0.525 382.229860861389
0.526 382.155369045555
0.527 382.080948912276
0.528 382.006600386804
0.529 381.932323394478
0.53 381.858117860721
0.531 381.783983711046
0.532 381.709920871048
0.533 381.635929266412
0.534 381.562008822904
0.535 381.488159466382
0.536 381.414381122783
0.537 381.340673718135
0.538 381.267037178549
0.539 381.19347143022
0.54 381.119976399432
0.541 381.04655201255
0.542 380.973198196026
0.543 380.899914876398
0.544 380.826701980286
0.545 380.753559434396
0.546 380.68048716552
0.547 380.607485100532
0.548 380.534553166391
0.549 380.461691290141
0.55 380.388899398911
0.551 380.31617741991
0.552 380.243525280436
0.553 380.170942907867
0.554 380.098430229666
0.555 380.025987173379
0.556 379.953613666637
0.557 379.881309637153
0.558 379.809075012723
0.559 379.736909721227
0.56 379.664813690627
0.561 379.592786848969
0.562 379.520829124381
0.563 379.448940445075
0.564 379.377120739343
0.565 379.305369935562
0.566 379.23368796219
0.567 379.162074747767
0.568 379.090530220916
0.569 379.019054310343
0.57 378.947646944833
0.571 378.876308053256
0.572 378.805037564561
0.573 378.733835407781
0.574 378.662701512028
0.575 378.591635806498
0.576 378.520638220467
0.577 378.449708683291
0.578 378.378847124409
0.579 378.308053473341
0.58 378.237327659686
0.581 378.166669613126
0.582 378.096079263421
0.583 378.025556540415
0.584 377.95510137403
0.585 377.884713694268
0.586 377.814393431213
0.587 377.744140515029
0.588 377.673954875958
0.589 377.603836444324
0.59 377.53378515053
0.591 377.463800925058
0.592 377.393883698472
0.593 377.324033401413
0.594 377.254249964603
0.595 377.184533318843
0.596 377.114883395011
0.597 377.045300124068
0.598 376.975783437051
0.599 376.906333265077
0.6 376.836949539343
0.601 376.767632191121
0.602 376.698381151767
0.603 376.62919635271
0.604 376.560077725461
0.605 376.491025201608
0.606 376.422038712818
0.607 376.353118190834
0.608 376.28426356748
0.609 376.215474774656
0.61 376.14675174434
0.611 376.078094408587
0.612 376.009502699531
0.613 375.940976549384
0.614 375.872515890433
0.615 375.804120655043
0.616 375.735790775659
0.617 375.667526184799
0.618 375.599326815061
0.619 375.531192599118
0.62 375.46312346972
0.621 375.395119359696
0.622 375.327180201948
0.623 375.259305929457
0.624 375.191496475279
0.625 375.123751772549
0.626 375.056071754474
0.627 374.98845635434
0.628 374.920905505509
0.629 374.853419141418
0.63 374.785997195579
0.631 374.718639601582
0.632 374.65134629309
0.633 374.584117203844
0.634 374.516952267658
0.635 374.449851418425
0.636 374.382814590108
0.637 374.31584171675
0.638 374.248932732466
0.639 374.182087571446
0.64 374.115306167958
0.641 374.048588456341
0.642 373.98193437101
0.643 373.915343846456
0.644 373.848816817241
0.645 373.782353218006
0.646 373.715952983462
0.647 373.649616048396
0.648 373.58334234767
0.649 373.517131816218
0.65 373.450984389049
0.651 373.384900001246
0.652 373.318878587965
0.653 373.252920084437
0.654 373.187024425964
0.655 373.121191547924
0.656 373.055421385767
0.657 372.989713875015
0.658 372.924068951267
0.659 372.858486550191
0.66 372.792966607529
0.661 372.727509059098
0.662 372.662113840785
0.663 372.596780888551
0.664 372.53151013843
0.665 372.466301526527
0.666 372.401154989021
0.667 372.336070462162
0.668 372.271047882273
0.669 372.206087185749
0.67 372.141188309056
0.671 372.076351188734
0.672 372.011575761393
0.673 371.946861963716
0.674 371.882209732456
0.675 371.817619004438
0.676 371.753089716561
0.677 371.688621805793
0.678 371.624215209172
0.679 371.559869863811
0.68 371.495585706891
0.681 371.431362675665
0.682 371.367200707457
0.683 371.303099739662
0.684 371.239059709746
0.685 371.175080555244
0.686 371.111162213763
0.687 371.047304622981
0.688 370.983507720645
0.689 370.919771444573
0.69 370.856095732653
0.691 370.792480522844
0.692 370.728925753173
0.693 370.665431361738
0.694 370.601997286709
0.695 370.538623466322
0.696 370.475309838884
0.697 370.412056342774
0.698 370.348862916437
0.699 370.28572949839
0.7 370.222656027217
0.701 370.159642441574
0.702 370.096688680183
0.703 370.033794681838
0.704 369.9709603854
0.705 369.9081857298
0.706 369.845470654037
0.707 369.782815097179
0.708 369.720218998362
0.709 369.657682296792
0.71 369.595204931743
0.711 369.532786842556
0.712 369.470427968642
0.713 369.408128249478
0.714 369.345887624613
0.715 369.28370603366
0.716 369.221583416301
0.717 369.159519712287
0.718 369.097514861436
0.719 369.035568803634
0.72 368.973681478834
0.721 368.911852827057
0.722 368.85008278839
0.723 368.78837130299
0.724 368.726718311079
0.725 368.665123752948
0.726 368.603587568952
0.727 368.542109699517
0.728 368.480690085133
0.729 368.419328666358
0.73 368.358025383816
0.731 368.296780178198
0.732 368.235592990262
0.733 368.174463760833
0.734 368.1133924308
0.735 368.052378941121
0.736 367.991423232818
0.737 367.930525246981
0.738 367.869684924765
0.739 367.808902207392
0.74 367.748177036148
0.741 367.687509352387
0.742 367.626899097526
0.743 367.566346213052
0.744 367.505850640513
0.745 367.445412321524
0.746 367.385031197768
0.747 367.324707210989
0.748 367.264440302999
0.749 367.204230415675
0.75 367.144077490958
0.751 367.083981470855
0.752 367.023942297437
0.753 366.96395991284
0.754 366.904034259266
0.755 366.844165278981
0.756 366.784352914314
0.757 366.72459710766
0.758 366.664897801479
0.759 366.605254938294
0.76 366.545668460693
0.761 366.486138311328
0.762 366.426664432914
0.763 366.367246768233
0.764 366.307885260129
0.765 366.248579851508
0.766 366.189330485343
0.767 366.130137104669
0.768 366.070999652585
0.769 366.011918072253
0.77 365.9528923069
0.771 365.893922299815
0.772 365.835007994351
0.773 365.776149333922
0.774 365.71734626201
0.775 365.658598722154
0.776 365.599906657961
0.777 365.541270013099
0.778 365.482688731298
0.779 365.424162756351
0.78 365.365692032115
0.781 365.307276502509
0.782 365.248916111514
0.783 365.190610803174
0.784 365.132360521594
0.785 365.074165210943
0.786 365.016024815452
0.787 364.957939279413
0.788 364.899908547181
0.789 364.841932563173
0.79 364.784011271866
0.791 364.726144617803
0.792 364.668332545584
0.793 364.610574999873
0.794 364.552871925395
0.795 364.495223266939
0.796 364.437628969351
0.797 364.380088977541
0.798 364.322603236481
0.799 364.265171691202
0.8 364.207794286797
0.801 364.150470968421
0.802 364.093201681289
0.803 364.035986370677
0.804 363.978824981922
0.805 363.921717460422
0.806 363.864663751634
0.807 363.807663801079
0.808 363.750717554334
0.809 363.693824957041
0.81 363.636985954899
0.811 363.580200493668
0.812 363.523468519171
0.813 363.466789977286
0.814 363.410164813956
0.815 363.35359297518
0.816 363.297074407021
0.817 363.240609055598
0.818 363.184196867093
0.819 363.127837787744
0.82 363.071531763852
0.821 363.015278741777
0.822 362.959078667936
0.823 362.902931488808
0.824 362.846837150931
0.825 362.790795600901
0.826 362.734806785374
0.827 362.678870651065
0.828 362.622987144748
0.829 362.567156213256
0.83 362.511377803481
0.831 362.455651862373
0.832 362.399978336943
0.833 362.344357174257
0.834 362.288788321442
0.835 362.233271725684
0.836 362.177807334226
0.837 362.122395094371
0.838 362.067034953477
0.839 362.011726858965
0.84 361.95647075831
0.841 361.901266599047
0.842 361.846114328769
0.843 361.791013895127
0.844 361.735965245829
0.845 361.680968328642
0.846 361.626023091389
0.847 361.571129481952
0.848 361.516287448271
0.849 361.461496938342
0.85 361.406757900219
0.851 361.352070282015
0.852 361.297434031897
0.853 361.242849098093
0.854 361.188315428885
0.855 361.133832972614
0.856 361.079401677677
0.857 361.025021492528
0.858 360.970692365679
0.859 360.916414245697
0.86 360.862187081208
0.861 360.808010820893
0.862 360.75388541349
0.863 360.699810807793
0.864 360.645786952653
0.865 360.591813796978
0.866 360.537891289731
0.867 360.484019379933
0.868 360.430198016659
0.869 360.376427149042
0.87 360.322706726269
0.871 360.269036697586
0.872 360.215417012292
0.873 360.161847619742
0.874 360.10832846935
0.875 360.054859510582
0.876 360.001440692961
0.877 359.948071966065
0.878 359.89475327953
0.879 359.841484583043
0.88 359.788265826351
0.881 359.735096959254
0.882 359.681977931606
0.883 359.628908693318
0.884 359.575889194356
0.885 359.522919384741
0.886 359.469999214547
0.887 359.417128633905
0.888 359.364307593001
0.889 359.311536042074
0.89 359.25881393142
0.891 359.206141211387
0.892 359.153517832378
0.893 359.100943744854
0.894 359.048418899325
0.895 358.995943246359
0.896 358.943516736578
0.897 358.891139320657
0.898 358.838810949325
0.899 358.786531573366
0.9 358.734301143619
0.901 358.682119610974
0.902 358.629986926377
0.903 358.577903040827
0.904 358.525867905379
0.905 358.473881471137
0.906 358.421943689263
0.907 358.370054510971
0.908 358.318213887529
0.909 358.266421770256
0.91 358.214678110527
0.911 358.162982859771
0.912 358.111335969466
0.913 358.059737391148
0.914 358.008187076403
0.915 357.956684976871
0.916 357.905231044244
0.917 357.85382523027
0.918 357.802467486747
0.919 357.751157765525
0.92 357.69989601851
0.921 357.648682197657
0.922 357.597516254977
0.923 357.54639814253
0.924 357.495327812432
0.925 357.44430521685
0.926 357.393330308001
0.927 357.342403038157
0.928 357.291523359642
0.929 357.240691224832
0.93 357.189906586153
0.931 357.139169396086
0.932 357.088479607162
0.933 357.037837171965
0.934 356.987242043129
0.935 356.936694173342
0.936 356.886193515343
0.937 356.835740021921
0.938 356.785333645919
0.939 356.73497434023
0.94 356.684662057799
0.941 356.634396751622
0.942 356.584178374746
0.943 356.53400688027
0.944 356.483882221344
0.945 356.433804351169
0.946 356.383773222997
0.947 356.333788790131
0.948 356.283851005925
0.949 356.233959823784
0.95 356.184115197163
0.951 356.13431707957
0.952 356.08456542456
0.953 356.034860185743
0.954 355.985201316775
0.955 355.935588771366
0.956 355.886022503276
0.957 355.836502466313
0.958 355.787028614338
0.959 355.737600901261
0.96 355.688219281042
0.961 355.638883707693
0.962 355.589594135272
0.963 355.540350517892
0.964 355.491152809712
0.965 355.442000964943
0.966 355.392894937846
0.967 355.34383468273
0.968 355.294820153955
0.969 355.245851305931
0.97 355.196928093117
0.971 355.14805047002
0.972 355.0992183912
0.973 355.050431811264
0.974 355.001690684868
0.975 354.952994966718
0.976 354.90434461157
0.977 354.855739574229
0.978 354.807179809547
0.979 354.758665272428
0.98 354.710195917822
0.981 354.661771700732
0.982 354.613392576206
0.983 354.565058499342
0.984 354.516769425287
0.985 354.468525309238
0.986 354.420326106438
0.987 354.372171772181
0.988 354.324062261807
0.989 354.275997530707
0.99 354.22797753432
0.991 354.180002228131
0.992 354.132071567675
0.993 354.084185508537
0.994 354.036344006347
0.995 353.988547016784
0.996 353.940794495577
0.997 353.893086398501
0.998 353.845422681379
0.999 353.797803300083
1 353.750228210531
1.001 353.702697368691
1.002 353.655210730577
1.003 353.607768252252
1.004 353.560369889826
1.005 353.513015599455
1.006 353.465705337345
1.007 353.418439059749
1.008 353.371216722965
1.009 353.324038283342
1.01 353.276903697272
1.011 353.229812921198
1.012 353.182765911608
1.013 353.135762625038
1.014 353.088803018071
1.015 353.041887047335
1.016 352.995014669508
1.017 352.948185841314
1.018 352.901400519521
1.019 352.854658660948
1.02 352.807960222458
1.021 352.761305160961
1.022 352.714693433414
1.023 352.66812499682
1.024 352.62159980823
1.025 352.575117824739
1.026 352.52867900349
1.027 352.482283301672
1.028 352.43593067652
1.029 352.389621085315
1.03 352.343354485385
1.031 352.297130834103
1.032 352.250950088889
1.033 352.204812207208
1.034 352.158717146572
1.035 352.112664864537
1.036 352.066655318707
1.037 352.020688466731
1.038 351.974764266303
1.039 351.928882675162
1.04 351.883043651096
1.041 351.837247151935
1.042 351.791493135555
1.043 351.74578155988
1.044 351.700112382875
1.045 351.654485562554
1.046 351.608901056975
1.047 351.563358824241
1.048 351.517858822501
1.049 351.472401009947
1.05 351.426985344819
1.051 351.381611785399
1.052 351.336280290016
1.053 351.290990817044
1.054 351.245743324899
1.055 351.200537772046
1.056 351.155374116992
1.057 351.110252318289
1.058 351.065172334533
1.059 351.020134124367
1.06 350.975137646475
1.061 350.930182859589
1.062 350.885269722483
1.063 350.840398193975
1.064 350.79556823293
1.065 350.750779798255
1.066 350.706032848902
1.067 350.661327343865
1.068 350.616663242186
1.069 350.572040502949
1.07 350.527459085281
1.071 350.482918948354
1.072 350.438420051384
1.073 350.39396235363
1.074 350.349545814396
1.075 350.305170393028
1.076 350.260836048918
1.077 350.2165427415
1.078 350.172290430251
1.079 350.128079074693
1.08 350.08390863439
1.081 350.039779068951
1.082 349.995690338028
1.083 349.951642401314
1.084 349.907635218549
1.085 349.863668749513
1.086 349.819742954031
1.087 349.77585779197
1.088 349.732013223241
1.089 349.688209207798
1.09 349.644445705636
1.091 349.600722676795
1.092 349.557040081358
1.093 349.513397879449
1.094 349.469796031235
1.095 349.426234496928
1.096 349.38271323678
1.097 349.339232211087
1.098 349.295791380187
1.099 349.25239070446
1.1 349.20903014433
1.101 349.165709660261
1.102 349.122429212762
1.103 349.079188762383
1.104 349.035988269716
1.105 348.992827695395
1.106 348.949707000096
1.107 348.906626144539
1.108 348.863585089484
1.109 348.820583795733
1.11 348.777622224131
1.111 348.734700335565
1.112 348.691818090963
1.113 348.648975451295
1.114 348.606172377572
1.115 348.563408830848
1.116 348.520684772219
1.117 348.47800016282
1.118 348.43535496383
1.119 348.392749136469
1.12 348.350182641997
1.121 348.307655441717
1.122 348.265167496973
1.123 348.22271876915
1.124 348.180309219675
1.125 348.137938810014
1.126 348.095607501677
1.127 348.053315256213
1.128 348.011062035214
1.129 347.96884780031
1.13 347.926672513175
1.131 347.884536135522
1.132 347.842438629106
1.133 347.800379955722
1.134 347.758360077207
1.135 347.716378955436
1.136 347.674436552327
1.137 347.632532829839
1.138 347.59066774997
1.139 347.548841274758
1.14 347.507053366285
1.141 347.465303986668
1.142 347.42359309807
1.143 347.38192066269
1.144 347.34028664277
1.145 347.29869100059
1.146 347.257133698473
1.147 347.215614698779
1.148 347.17413396391
1.149 347.132691456307
1.15 347.091287138453
1.151 347.049920972869
1.152 347.008592922116
1.153 346.967302948796
1.154 346.926051015549
1.155 346.884837085057
1.156 346.84366112004
1.157 346.802523083258
1.158 346.761422937511
1.159 346.720360645639
1.16 346.679336170521
1.161 346.638349475074
1.162 346.597400522257
1.163 346.556489275068
1.164 346.515615696541
1.165 346.474779749754
1.166 346.433981397822
1.167 346.393220603898
1.168 346.352497331177
1.169 346.31181154289
1.17 346.27116320231
1.171 346.230552272747
1.172 346.18997871755
1.173 346.149442500109
1.174 346.10894358385
1.175 346.068481932239
1.176 346.028057508783
1.177 345.987670277024
1.178 345.947320200544
1.179 345.907007242966
1.18 345.866731367949
1.181 345.82649253919
1.182 345.786290720427
1.183 345.746125875436
1.184 345.705997968029
1.185 345.665906962058
1.186 345.625852821415
1.187 345.585835510028
1.188 345.545854991863
1.189 345.505911230926
1.19 345.46600419126
1.191 345.426133836947
1.192 345.386300132105
1.193 345.346503040893
1.194 345.306742527506
1.195 345.267018556176
1.196 345.227331091176
1.197 345.187680096815
1.198 345.148065537439
1.199 345.108487377433
1.2 345.06894558122
1.201 345.029440113259
1.202 344.989970938048
1.203 344.950538020123
1.204 344.911141324055
1.205 344.871780814456
1.206 344.832456455973
1.207 344.793168213291
1.208 344.753916051132
1.209 344.714699934257
1.21 344.675519827462
1.211 344.636375695582
1.212 344.597267503488
1.213 344.558195216089
1.214 344.519158798329
1.215 344.480158215194
1.216 344.441193431701
1.217 344.402264412907
1.218 344.363371123907
1.219 344.324513529831
1.22 344.285691595846
1.221 344.246905287157
1.222 344.208154569003
1.223 344.169439406664
1.224 344.130759765453
1.225 344.092115610721
1.226 344.053506907856
1.227 344.014933622282
1.228 343.976395719459
1.229 343.937893164884
1.23 343.899425924091
1.231 343.860993962649
1.232 343.822597246164
1.233 343.78423574028
1.234 343.745909410674
1.235 343.707618223062
1.236 343.669362143194
1.237 343.631141136858
1.238 343.592955169876
1.239 343.554804208109
1.24 343.51668821745
1.241 343.478607163832
1.242 343.440561013222
1.243 343.402549731621
1.244 343.36457328507
1.245 343.326631639642
1.246 343.288724761448
1.247 343.250852616634
1.248 343.213015171381
1.249 343.175212391907
1.25 343.137444244465
1.251 343.099710695343
1.252 343.062011710864
1.253 343.024347257389
1.254 342.986717301311
1.255 342.949121809061
1.256 342.911560747104
1.257 342.874034081941
1.258 342.836541780108
1.259 342.799083808176
1.26 342.761660132751
1.261 342.724270720475
1.262 342.686915538024
1.263 342.64959455211
1.264 342.612307729479
1.265 342.575055036913
1.266 342.537836441228
1.267 342.500651909275
1.268 342.463501407941
1.269 342.426384904147
1.27 342.389302364848
1.271 342.352253757034
1.272 342.315239047732
1.273 342.278258203999
1.274 342.241311192932
1.275 342.204397981658
1.276 342.167518537342
1.277 342.130672827181
1.278 342.093860818407
1.279 342.057082478287
1.28 342.020337774122
1.281 341.983626673248
1.282 341.946949143035
1.283 341.910305150885
1.284 341.873694664238
1.285 341.837117650565
1.286 341.800574077373
1.287 341.764063912203
1.288 341.727587122629
1.289 341.691143676259
1.29 341.654733540736
1.291 341.618356683736
1.292 341.58201307297
1.293 341.545702676182
1.294 341.50942546115
1.295 341.473181395685
1.296 341.436970447634
1.297 341.400792584874
1.298 341.364647775319
1.299 341.328535986915
1.3 341.292457187642
1.301 341.256411345514
1.302 341.220398428578
1.303 341.184418404914
1.304 341.148471242636
1.305 341.112556909891
1.306 341.07667537486
1.307 341.040826605757
1.308 341.005010570828
1.309 340.969227238355
1.31 340.933476576651
1.311 340.897758554063
1.312 340.86207313897
1.313 340.826420299786
1.314 340.790800004955
1.315 340.755212222959
1.316 340.719656922307
1.317 340.684134071545
1.318 340.648643639252
1.319 340.613185594037
1.32 340.577759904543
1.321 340.542366539447
1.322 340.507005467458
1.323 340.471676657317
1.324 340.436380077799
1.325 340.401115697711
1.326 340.365883485891
1.327 340.330683411211
1.328 340.295515442578
1.329 340.260379548926
1.33 340.225275699226
1.331 340.19020386248
1.332 340.155164007721
1.333 340.120156104017
1.334 340.085180120465
1.335 340.050236026197
1.336 340.015323790376
1.337 339.980443382197
1.338 339.945594770889
1.339 339.91077792571
1.34 339.875992815952
1.341 339.841239410939
1.342 339.806517680027
1.343 339.771827592603
1.344 339.737169118087
1.345 339.70254222593
1.346 339.667946885616
1.347 339.63338306666
1.348 339.598850738609
1.349 339.56434987104
1.35 339.529880433565
1.351 339.495442395826
1.352 339.461035727496
1.353 339.42666039828
1.354 339.392316377916
1.355 339.35800363617
1.356 339.323722142843
1.357 339.289471867767
1.358 339.255252780804
1.359 339.221064851847
1.36 339.186908050822
1.361 339.152782347686
1.362 339.118687712426
1.363 339.084624115062
1.364 339.050591525643
1.365 339.016589914252
1.366 338.982619251
1.367 338.948679506032
1.368 338.914770649522
1.369 338.880892651676
1.37 338.84704548273
1.371 338.813229112953
1.372 338.779443512643
1.373 338.745688652128
1.374 338.71196450177
1.375 338.67827103196
1.376 338.644608213119
1.377 338.6109760157
1.378 338.577374410186
1.379 338.543803367091
1.38 338.51026285696
1.381 338.476752850367
1.382 338.443273317919
1.383 338.409824230252
1.384 338.376405558032
1.385 338.343017271957
1.386 338.309659342755
1.387 338.276331741182
1.388 338.243034438029
1.389 338.209767404112
1.39 338.176530610282
1.391 338.143324027417
1.392 338.110147626427
1.393 338.077001378251
1.394 338.043885253859
1.395 338.010799224252
1.396 337.977743260458
1.397 337.944717333538
1.398 337.911721414582
1.399 337.878755474709
1.4 337.84581948507
1.401 337.812913416844
1.402 337.780037241241
1.403 337.7471909295
1.404 337.714374452891
1.405 337.681587782712
1.406 337.648830890293
1.407 337.616103746992
1.408 337.583406324197
1.409 337.550738593325
1.41 337.518100525825
1.411 337.485492093173
1.412 337.452913266876
1.413 337.420364018469
1.414 337.387844319518
1.415 337.355354141618
1.416 337.322893456393
1.417 337.290462235497
1.418 337.258060450613
1.419 337.225688073453
1.42 337.193345075758
1.421 337.1610314293
1.422 337.128747105878
1.423 337.096492077321
1.424 337.064266315489
1.425 337.032069792267
1.426 336.999902479572
1.427 336.967764349351
1.428 336.935655373577
1.429 336.903575524254
1.43 336.871524773414
1.431 336.839503093118
1.432 336.807510455457
1.433 336.775546832549
1.434 336.743612196543
1.435 336.711706519615
1.436 336.67982977397
1.437 336.647981931842
1.438 336.616162965494
1.439 336.584372847217
1.44 336.552611549331
1.441 336.520879044185
1.442 336.489175304156
1.443 336.457500301649
1.444 336.425854009098
1.445 336.394236398966
1.446 336.362647443744
1.447 336.331087115951
1.448 336.299555388134
1.449 336.26805223287
1.45 336.236577622763
1.451 336.205131530444
1.452 336.173713928576
1.453 336.142324789847
1.454 336.110964086974
1.455 336.079631792702
1.456 336.048327879805
1.457 336.017052321083
1.458 335.985805089367
1.459 335.954586157513
1.46 335.923395498407
1.461 335.892233084962
1.462 335.86109889012
1.463 335.82999288685
1.464 335.798915048148
1.465 335.767865347039
1.466 335.736843756576
1.467 335.705850249839
1.468 335.674884799937
1.469 335.643947380004
1.47 335.613037963204
1.471 335.582156522728
1.472 335.551303031795
1.473 335.52047746365
1.474 335.489679791568
1.475 335.458909988849
1.476 335.428168028823
1.477 335.397453884844
1.478 335.366767530298
1.479 335.336108938593
1.48 335.30547808317
1.481 335.274874937492
1.482 335.244299475053
1.483 335.213751669374
1.484 335.183231494
1.485 335.152738922507
1.486 335.122273928496
1.487 335.091836485597
1.488 335.061426567464
1.489 335.031044147781
1.49 335.000689200257
1.491 334.970361698631
1.492 334.940061616665
1.493 334.90978892815
1.494 334.879543606905
1.495 334.849325626774
1.496 334.819134961629
1.497 334.788971585367
1.498 334.758835471915
1.499 334.728726595224
1.5 334.698644929273
1.501 334.668590448067
1.502 334.638563125638
1.503 334.608562936045
1.504 334.578589853373
1.505 334.548643851734
1.506 334.518724905268
1.507 334.488832988137
1.508 334.458968074535
1.509 334.429130138679
1.51 334.399319154814
1.511 334.36953509721
1.512 334.339777940165
1.513 334.310047658002
1.514 334.28034422507
1.515 334.250667615747
1.516 334.221017804435
1.517 334.191394765561
1.518 334.161798473582
1.519 334.132228902978
1.52 334.102686028256
1.521 334.07316982395
1.522 334.043680264619
1.523 334.014217324848
1.524 333.984780979249
1.525 333.95537120246
1.526 333.925987969143
1.527 333.896631253988
1.528 333.867301031711
1.529 333.837997277053
1.53 333.80871996478
1.531 333.779469069685
1.532 333.750244566588
1.533 333.721046430332
1.534 333.691874635787
1.535 333.66272915785
1.536 333.633609971442
1.537 333.60451705151
1.538 333.575450373027
1.539 333.546409910992
1.54 333.517395640427
1.541 333.488407536384
1.542 333.459445573936
1.543 333.430509728185
1.544 333.401599974256
1.545 333.3727162873
1.546 333.343858642494
1.547 333.315027015041
1.548 333.286221380168
1.549 333.257441713128
1.55 333.228687989197
1.551 333.199960183681
1.552 333.171258271907
1.553 333.14258222923
1.554 333.113932031027
1.555 333.085307652703
1.556 333.056709069688
1.557 333.028136257435
1.558 332.999589191424
1.559 332.971067847158
1.56 332.942572200168
1.561 332.914102226008
1.562 332.885657900257
1.563 332.857239198518
1.564 332.828846096421
1.565 332.800478569621
1.566 332.772136593795
1.567 332.743820144647
1.568 332.715529197905
1.569 332.687263729323
1.57 332.659023714679
1.571 332.630809129774
1.572 332.602619950436
1.573 332.574456152517
1.574 332.546317711892
1.575 332.518204604465
1.576 332.490116806158
1.577 332.462054292924
1.578 332.434017040736
1.579 332.406005025594
1.58 332.37801822352
1.581 332.350056610563
1.582 332.322120162796
1.583 332.294208856314
1.584 332.26632266724
1.585 332.238461571718
1.586 332.210625545918
1.587 332.182814566034
1.588 332.155028608283
1.589 332.12726764891
1.59 332.099531664179
1.591 332.071820630382
1.592 332.044134523833
1.593 332.016473320871
1.594 331.98883699786
1.595 331.961225531186
1.596 331.933638897261
1.597 331.906077072518
1.598 331.878540033419
1.599 331.851027756444
1.6 331.823540218102
1.601 331.796077394922
1.602 331.76863926346
1.603 331.741225800294
1.604 331.713836982027
1.605 331.686472785283
1.606 331.659133186713
1.607 331.631818162991
1.608 331.604527690814
1.609 331.577261746902
1.61 331.550020308
1.611 331.522803350876
1.612 331.495610852322
1.613 331.468442789154
1.614 331.44129913821
1.615 331.414179876352
1.616 331.387084980467
1.617 331.360014427464
1.618 331.332968194275
1.619 331.305946257857
1.62 331.278948595189
1.621 331.251975183274
1.622 331.225025999139
1.623 331.198101019833
1.624 331.171200222429
1.625 331.144323584023
1.626 331.117471081735
1.627 331.090642692707
1.628 331.063838394104
1.629 331.037058163116
1.63 331.010301976956
1.631 330.983569812857
1.632 330.956861648078
1.633 330.930177459901
1.634 330.903517225629
1.635 330.876880922591
1.636 330.850268528136
1.637 330.823680019637
1.638 330.797115374492
1.639 330.770574570118
1.64 330.744057583958
1.641 330.717564393476
1.642 330.691094976161
1.643 330.664649309523
1.644 330.638227371095
1.645 330.611829138433
1.646 330.585454589115
1.647 330.559103700743
1.648 330.532776450942
1.649 330.506472817357
1.65 330.480192777658
1.651 330.453936309537
1.652 330.427703390709
1.653 330.401493998911
1.654 330.375308111902
1.655 330.349145707464
1.656 330.323006763402
1.657 330.296891257543
1.658 330.270799167737
1.659 330.244730471854
1.66 330.21868514779
1.661 330.192663173462
1.662 330.166664526807
1.663 330.140689185788
1.664 330.114737128387
1.665 330.088808332611
1.666 330.062902776487
1.667 330.037020438065
1.668 330.011161295419
1.669 329.985325326643
1.67 329.959512509852
1.671 329.933722823187
1.672 329.907956244807
1.673 329.882212752897
1.674 329.856492325659
1.675 329.830794941323
1.676 329.805120578136
1.677 329.779469214369
1.678 329.753840828316
1.679 329.728235398291
1.68 329.702652902631
1.681 329.677093319693
1.682 329.65155662786
1.683 329.626042805532
1.684 329.600551831134
1.685 329.575083683111
1.686 329.549638339931
1.687 329.524215780084
1.688 329.49881598208
1.689 329.473438924451
1.69 329.448084585753
1.691 329.42275294456
1.692 329.397443979471
1.693 329.372157669104
1.694 329.3468939921
1.695 329.321652927121
1.696 329.296434452851
1.697 329.271238547994
1.698 329.246065191279
1.699 329.220914361452
1.7 329.195786037283
1.701 329.170680197563
1.702 329.145596821104
1.703 329.12053588674
1.704 329.095497373326
1.705 329.070481259737
1.706 329.045487524871
1.707 329.020516147648
1.708 328.995567107006
1.709 328.970640381908
1.71 328.945735951334
1.711 328.92085379429
1.712 328.895993889799
1.713 328.871156216907
1.714 328.846340754681
1.715 328.82154748221
1.716 328.7967763786
1.717 328.772027422984
1.718 328.747300594512
1.719 328.722595872355
1.72 328.697913235706
1.721 328.67325266378
1.722 328.648614135811
1.723 328.623997631054
1.724 328.599403128787
1.725 328.574830608305
1.726 328.550280048928
1.727 328.525751429995
1.728 328.501244730864
1.729 328.476759930917
1.73 328.452297009554
1.731 328.427855946197
1.732 328.40343672029
1.733 328.379039311294
1.734 328.354663698694
1.735 328.330309861994
1.736 328.305977780719
1.737 328.281667434415
1.738 328.257378802647
1.739 328.233111865003
1.74 328.20886660109
1.741 328.184642990535
1.742 328.160441012986
1.743 328.136260648112
1.744 328.112101875602
1.745 328.087964675164
1.746 328.063849026529
1.747 328.039754909447
1.748 328.015682303687
1.749 327.991631189041
1.75 327.967601545319
1.751 327.943593352352
1.752 327.919606589992
1.753 327.895641238111
1.754 327.8716972766
1.755 327.847774685371
1.756 327.823873444357
1.757 327.799993533508
1.758 327.776134932799
1.759 327.75229762222
1.76 327.728481581785
1.761 327.704686791525
1.762 327.680913231494
1.763 327.657160881764
1.764 327.633429722427
1.765 327.609719733595
1.766 327.586030895402
1.767 327.562363187999
1.768 327.538716591558
1.769 327.515091086272
1.77 327.491486652351
1.771 327.467903270028
1.772 327.444340919554
1.773 327.4207995812
1.774 327.397279235258
1.775 327.373779862037
1.776 327.350301441869
1.777 327.326843955103
1.778 327.303407382108
1.779 327.279991703275
1.78 327.256596899013
1.781 327.233222949749
1.782 327.209869835933
1.783 327.186537538032
1.784 327.163226036532
1.785 327.139935311942
1.786 327.116665344787
1.787 327.093416115613
1.788 327.070187604986
1.789 327.046979793489
1.79 327.023792661728
1.791 327.000626190325
1.792 326.977480359924
1.793 326.954355151187
1.794 326.931250544794
1.795 326.908166521448
1.796 326.885103061867
1.797 326.862060146792
1.798 326.839037756981
1.799 326.816035873212
1.8 326.793054476281
1.801 326.770093547006
1.802 326.747153066222
1.803 326.724233014783
1.804 326.701333373562
1.805 326.678454123453
1.806 326.655595245367
1.807 326.632756720236
1.808 326.609938529009
1.809 326.587140652655
1.81 326.564363072162
1.811 326.541605768537
1.812 326.518868722806
1.813 326.496151916013
1.814 326.473455329223
1.815 326.450778943519
1.816 326.42812274
1.817 326.405486699789
1.818 326.382870804024
1.819 326.360275033863
1.82 326.337699370482
1.821 326.315143795078
1.822 326.292608288865
1.823 326.270092833075
1.824 326.247597408961
1.825 326.225121997792
1.826 326.202666580859
1.827 326.180231139468
1.828 326.157815654947
1.829 326.135420108639
1.83 326.11304448191
1.831 326.09068875614
1.832 326.06835291273
1.833 326.046036933101
1.834 326.023740798688
1.835 326.00146449095
1.836 325.97920799136
1.837 325.956971281412
1.838 325.934754342616
1.839 325.912557156504
1.84 325.890379704624
1.841 325.868221968541
1.842 325.846083929842
1.843 325.82396557013
1.844 325.801866871026
1.845 325.779787814171
1.846 325.757728381223
1.847 325.735688553858
1.848 325.713668313772
1.849 325.691667642677
1.85 325.669686522304
1.851 325.647724934403
1.852 325.625782860742
1.853 325.603860283105
1.854 325.581957183298
1.855 325.56007354314
1.856 325.538209344474
1.857 325.516364569156
1.858 325.494539199062
1.859 325.472733216087
1.86 325.450946602143
1.861 325.429179339159
1.862 325.407431409084
1.863 325.385702793884
1.864 325.363993475542
1.865 325.34230343606
1.866 325.320632657458
1.867 325.298981121773
1.868 325.277348811061
1.869 325.255735707394
1.87 325.234141792864
1.871 325.21256704958
1.872 325.191011459667
1.873 325.169475005271
1.874 325.147957668554
1.875 325.126459431694
1.876 325.104980276889
1.877 325.083520186355
1.878 325.062079142324
1.879 325.040657127047
1.88 325.019254122791
1.881 324.997870111841
1.882 324.976505076502
1.883 324.955158999094
1.884 324.933831861955
1.885 324.91252364744
1.886 324.891234337924
1.887 324.869963915796
1.888 324.848712363465
1.889 324.827479663357
1.89 324.806265797914
1.891 324.785070749597
1.892 324.763894500885
1.893 324.742737034272
1.894 324.72159833227
1.895 324.70047837741
1.896 324.67937715224
1.897 324.658294639323
1.898 324.637230821241
1.899 324.616185680593
1.9 324.595159199996
1.901 324.574151362084
1.902 324.553162149506
1.903 324.532191544931
1.904 324.511239531044
1.905 324.490306090548
1.906 324.46939120616
1.907 324.448494860619
1.908 324.427617036677
1.909 324.406757717106
1.91 324.385916884692
1.911 324.365094522241
1.912 324.344290612575
1.913 324.323505138532
1.914 324.302738082967
1.915 324.281989428754
1.916 324.261259158783
1.917 324.240547255959
1.918 324.219853703207
1.919 324.199178483466
1.92 324.178521579695
1.921 324.157882974867
1.922 324.137262651972
1.923 324.11666059402
1.924 324.096076784034
1.925 324.075511205056
1.926 324.054963840144
1.927 324.034434672373
1.928 324.013923684834
1.929 323.993430860636
1.93 323.972956182903
1.931 323.952499634779
1.932 323.93206119942
1.933 323.911640860002
1.934 323.891238599717
1.935 323.870854401772
1.936 323.850488249393
1.937 323.830140125821
1.938 323.809810014313
1.939 323.789497898146
1.94 323.769203760608
1.941 323.748927585009
1.942 323.728669354671
1.943 323.708429052936
1.944 323.68820666316
1.945 323.668002168717
1.946 323.647815552996
1.947 323.627646799403
1.948 323.607495891361
1.949 323.587362812309
1.95 323.567247545702
1.951 323.547150075012
1.952 323.527070383726
1.953 323.507008455349
1.954 323.486964273401
1.955 323.466937821419
1.956 323.446929082955
1.957 323.426938041578
1.958 323.406964680875
1.959 323.387008984446
1.96 323.36707093591
1.961 323.347150518899
1.962 323.327247717065
1.963 323.307362514073
1.964 323.287494893605
1.965 323.26764483936
1.966 323.247812335051
1.967 323.227997364411
1.968 323.208199911184
1.969 323.188419959133
1.97 323.168657492037
1.971 323.14891249369
1.972 323.129184947902
1.973 323.109474838501
1.974 323.089782149327
1.975 323.070106864239
1.976 323.050448967112
1.977 323.030808441835
1.978 323.011185272314
1.979 322.991579442471
1.98 322.971990936243
1.981 322.952419737584
1.982 322.932865830462
1.983 322.913329198863
1.984 322.893809826786
1.985 322.874307698249
1.986 322.854822797284
1.987 322.835355107938
1.988 322.815904614275
1.989 322.796471300374
1.99 322.77705515033
1.991 322.757656148254
1.992 322.738274278271
1.993 322.718909524523
1.994 322.699561871168
1.995 322.680231302378
1.996 322.660917802343
1.997 322.641621355265
1.998 322.622341945365
1.999 322.603079556878
2 322.583834174053
2.001 322.564605781158
2.002 322.545394362474
2.003 322.526199902297
2.004 322.50702238494
2.005 322.487861794731
2.006 322.468718116013
2.007 322.449591333145
2.008 322.4304814305
2.009 322.411388392468
2.01 322.392312203453
2.011 322.373252847875
2.012 322.35421031017
2.013 322.335184574788
2.014 322.316175626195
2.015 322.297183448872
2.016 322.278208027315
2.017 322.259249346035
2.018 322.24030738956
2.019 322.221382142431
2.02 322.202473589206
2.021 322.183581714455
2.022 322.164706502768
2.023 322.145847938746
2.024 322.127006007007
2.025 322.108180692183
2.026 322.089371978922
2.027 322.070579851887
2.028 322.051804295757
2.029 322.033045295223
2.03 322.014302834993
2.031 321.995576899791
2.032 321.976867474355
2.033 321.958174543436
2.034 321.939498091804
2.035 321.92083810424
2.036 321.902194565542
2.037 321.883567460523
2.038 321.86495677401
2.039 321.846362490845
2.04 321.827784595886
2.041 321.809223074005
2.042 321.790677910087
2.043 321.772149089035
2.044 321.753636595766
2.045 321.73514041521
2.046 321.716660532313
2.047 321.698196932036
2.048 321.679749599354
2.049 321.661318519257
2.05 321.642903676751
2.051 321.624505056854
2.052 321.606122644601
2.053 321.587756425041
2.054 321.569406383237
2.055 321.551072504267
2.056 321.532754773223
2.057 321.514453175213
2.058 321.496167695359
2.059 321.477898318798
2.06 321.459645030679
2.061 321.441407816169
2.062 321.423186660447
2.063 321.404981548708
2.064 321.386792466161
2.065 321.368619398029
2.066 321.350462329551
2.067 321.332321245977
2.068 321.314196132576
2.069 321.296086974628
2.07 321.277993757428
2.071 321.259916466288
2.072 321.241855086529
2.073 321.223809603493
2.074 321.205780002531
2.075 321.18776626901
2.076 321.169768388312
2.077 321.151786345833
2.078 321.133820126983
2.079 321.115869717186
2.08 321.097935101881
2.081 321.08001626652
2.082 321.062113196572
2.083 321.044225877516
2.084 321.026354294848
2.085 321.008498434079
2.086 320.99065828073
2.087 320.972833820342
2.088 320.955025038464
2.089 320.937231920664
2.09 320.919454452521
2.091 320.90169261963
2.092 320.883946407599
2.093 320.866215802051
2.094 320.848500788621
2.095 320.830801352961
2.096 320.813117480734
2.097 320.795449157618
2.098 320.777796369308
2.099 320.760159101507
2.1 320.742537339938
2.101 320.724931070333
2.102 320.707340278442
2.103 320.689764950025
2.104 320.67220507086
2.105 320.654660626735
2.106 320.637131603454
2.107 320.619617986835
2.108 320.602119762709
2.109 320.584636916921
2.11 320.56716943533
2.111 320.549717303808
2.112 320.532280508243
2.113 320.514859034534
2.114 320.497452868594
2.115 320.480061996353
2.116 320.46268640375
2.117 320.445326076742
2.118 320.427981001297
2.119 320.410651163397
2.12 320.393336549038
2.121 320.37603714423
2.122 320.358752934997
2.123 320.341483907375
2.124 320.324230047415
2.125 320.306991341181
2.126 320.289767774751
2.127 320.272559334215
2.128 320.255366005679
2.129 320.238187775261
2.13 320.221024629093
2.131 320.20387655332
2.132 320.186743534101
2.133 320.169625557608
2.134 320.152522610026
2.135 320.135434677556
2.136 320.11836174641
2.137 320.101303802812
2.138 320.084260833004
2.139 320.067232823238
2.14 320.05021975978
2.141 320.033221628909
2.142 320.016238416918
2.143 319.999270110113
2.144 319.982316694815
2.145 319.965378157355
2.146 319.948454484081
2.147 319.93154566135
2.148 319.914651675536
2.149 319.897772513025
2.15 319.880908160215
2.151 319.86405860352
2.152 319.847223829364
2.153 319.830403824186
2.154 319.813598574439
2.155 319.796808066586
2.156 319.780032287107
2.157 319.763271222493
2.158 319.746524859247
2.159 319.729793183889
2.16 319.713076182947
2.161 319.696373842967
2.162 319.679686150504
2.163 319.66301309213
2.164 319.646354654425
2.165 319.629710823988
2.166 319.613081587425
2.167 319.59646693136
2.168 319.579866842427
2.169 319.563281307274
2.17 319.546710312563
2.171 319.530153844966
2.172 319.51361189117
2.173 319.497084437876
2.174 319.480571471795
2.175 319.464072979653
2.176 319.447588948189
2.177 319.431119364153
2.178 319.414664214309
2.179 319.398223485435
2.18 319.38179716432
2.181 319.365385237766
2.182 319.348987692589
2.183 319.332604515617
2.184 319.316235693691
2.185 319.299881213664
2.186 319.283541062402
2.187 319.267215226786
2.188 319.250903693705
2.189 319.234606450066
2.19 319.218323482784
2.191 319.202054778791
2.192 319.185800325028
2.193 319.16956010845
2.194 319.153334116027
2.195 319.137122334737
2.196 319.120924751574
2.197 319.104741353543
2.198 319.088572127663
2.199 319.072417060965
2.2 319.056276140492
2.201 319.0401493533
2.202 319.024036686457
2.203 319.007938127045
2.204 318.991853662156
2.205 318.975783278897
2.206 318.959726964386
2.207 318.943684705754
2.208 318.927656490144
2.209 318.911642304712
2.21 318.895642136627
2.211 318.879655973069
2.212 318.863683801232
2.213 318.847725608319
2.214 318.83178138155
2.215 318.815851108155
2.216 318.799934775376
2.217 318.784032370467
2.218 318.768143880697
2.219 318.752269293345
2.22 318.736408595702
2.221 318.720561775073
2.222 318.704728818774
2.223 318.688909714134
2.224 318.673104448493
2.225 318.657313009205
2.226 318.641535383635
2.227 318.62577155916
2.228 318.610021523171
2.229 318.594285263069
2.23 318.578562766269
2.231 318.562854020196
2.232 318.54715901229
2.233 318.53147773
2.234 318.51581016079
2.235 318.500156292134
2.236 318.484516111519
2.237 318.468889606444
2.238 318.453276764421
2.239 318.437677572972
2.24 318.422092019632
2.241 318.406520091949
2.242 318.390961777482
2.243 318.375417063802
2.244 318.359885938493
2.245 318.344368389149
2.246 318.328864403378
2.247 318.313373968799
2.248 318.297897073043
2.249 318.282433703753
2.25 318.266983848585
2.251 318.251547495204
2.252 318.23612463129
2.253 318.220715244534
2.254 318.205319322637
2.255 318.189936853315
2.256 318.174567824293
2.257 318.15921222331
2.258 318.143870038115
2.259 318.128541256471
2.26 318.113225866151
2.261 318.09792385494
2.262 318.082635210635
2.263 318.067359921045
2.264 318.052097973991
2.265 318.036849357305
2.266 318.021614058831
2.267 318.006392066426
2.268 317.991183367955
2.269 317.9759879513
2.27 317.96080580435
2.271 317.945636915008
2.272 317.930481271188
2.273 317.915338860816
2.274 317.90020967183
2.275 317.885093692178
2.276 317.869990909822
2.277 317.854901312733
2.278 317.839824888896
2.279 317.824761626306
2.28 317.80971151297
2.281 317.794674536906
2.282 317.779650686145
2.283 317.764639948728
2.284 317.749642312708
2.285 317.734657766151
2.286 317.719686297131
2.287 317.704727893738
2.288 317.689782544069
2.289 317.674850236235
2.29 317.659930958359
2.291 317.645024698573
2.292 317.630131445022
2.293 317.615251185863
2.294 317.600383909263
2.295 317.585529603401
2.296 317.570688256468
2.297 317.555859856665
2.298 317.541044392205
2.299 317.526241851313
2.3 317.511452222224
2.301 317.496675493186
2.302 317.481911652456
2.303 317.467160688304
2.304 317.452422589012
2.305 317.437697342871
2.306 317.422984938185
2.307 317.408285363268
2.308 317.393598606446
2.309 317.378924656057
2.31 317.364263500449
2.311 317.34961512798
2.312 317.334979527023
2.313 317.320356685958
2.314 317.305746593179
2.315 317.29114923709
2.316 317.276564606106
2.317 317.261992688653
2.318 317.24743347317
2.319 317.232886948105
2.32 317.218353101917
2.321 317.203831923078
2.322 317.189323400068
2.323 317.174827521383
2.324 317.160344275524
2.325 317.145873651008
2.326 317.13141563636
2.327 317.116970220117
2.328 317.102537390828
2.329 317.088117137052
2.33 317.073709447358
2.331 317.059314310328
2.332 317.044931714553
2.333 317.030561648637
2.334 317.016204101194
2.335 317.001859060847
2.336 316.987526516234
2.337 316.973206455999
2.338 316.958898868802
2.339 316.94460374331
2.34 316.930321068202
2.341 316.916050832169
2.342 316.901793023912
2.343 316.887547632141
2.344 316.873314645581
2.345 316.859094052964
2.346 316.844885843034
2.347 316.830690004546
2.348 316.816506526267
2.349 316.802335396972
2.35 316.788176605449
2.351 316.774030140497
2.352 316.759895990922
2.353 316.745774145546
2.354 316.731664593199
2.355 316.71756732272
2.356 316.703482322963
2.357 316.689409582789
2.358 316.67534909107
2.359 316.661300836692
2.36 316.647264808547
2.361 316.633240995541
2.362 316.619229386589
2.363 316.605229970618
2.364 316.591242736564
2.365 316.577267673373
2.366 316.563304770006
2.367 316.549354015428
2.368 316.535415398621
2.369 316.521488908572
2.37 316.507574534283
2.371 316.493672264764
2.372 316.479782089036
2.373 316.465903996131
2.374 316.45203797509
2.375 316.438184014967
2.376 316.424342104825
2.377 316.410512233737
2.378 316.396694390786
2.379 316.382888565069
2.38 316.369094745689
2.381 316.355312921761
2.382 316.341543082413
2.383 316.327785216778
2.384 316.314039314005
2.385 316.30030536325
2.386 316.28658335368
2.387 316.272873274473
2.388 316.259175114816
2.389 316.245488863909
2.39 316.231814510959
2.391 316.218152045185
2.392 316.204501455817
2.393 316.190862732095
2.394 316.177235863267
2.395 316.163620838594
2.396 316.150017647347
2.397 316.136426278806
2.398 316.122846722261
2.399 316.109278967014
2.4 316.095723002376
2.401 316.082178817669
2.402 316.068646402223
2.403 316.055125745382
2.404 316.041616836497
2.405 316.02811966493
2.406 316.014634220053
2.407 316.00116049125
2.408 315.987698467912
2.409 315.974248139442
2.41 315.960809495254
2.411 315.947382524771
2.412 315.933967217424
2.413 315.920563562659
2.414 315.907171549927
2.415 315.893791168692
2.416 315.880422408427
2.417 315.867065258616
2.418 315.853719708753
2.419 315.84038574834
2.42 315.827063366892
2.421 315.813752553931
2.422 315.800453298991
2.423 315.787165591616
2.424 315.773889421358
2.425 315.760624777782
2.426 315.747371650461
2.427 315.734130028977
2.428 315.720899902924
2.429 315.707681261906
2.43 315.694474095534
2.431 315.681278393433
2.432 315.668094145235
2.433 315.654921340582
2.434 315.641759969128
2.435 315.628610020534
2.436 315.615471484473
2.437 315.602344350628
2.438 315.589228608689
2.439 315.576124248359
2.44 315.563031259349
2.441 315.549949631381
2.442 315.536879354186
2.443 315.523820417504
2.444 315.510772811087
2.445 315.497736524696
2.446 315.484711548099
2.447 315.471697871078
2.448 315.458695483422
2.449 315.44570437493
2.45 315.432724535411
2.451 315.419755954685
2.452 315.40679862258
2.453 315.393852528934
2.454 315.380917663595
2.455 315.36799401642
2.456 315.355081577277
2.457 315.342180336043
2.458 315.329290282604
2.459 315.316411406855
2.46 315.303543698704
2.461 315.290687148065
2.462 315.277841744863
2.463 315.265007479032
2.464 315.252184340517
2.465 315.239372319272
2.466 315.226571405258
2.467 315.21378158845
2.468 315.201002858829
2.469 315.188235206388
2.47 315.175478621127
2.471 315.162733093057
2.472 315.149998612199
2.473 315.137275168582
2.474 315.124562752246
2.475 315.111861353239
2.476 315.09917096162
2.477 315.086491567456
2.478 315.073823160824
2.479 315.061165731812
2.48 315.048519270513
2.481 315.035883767035
2.482 315.023259211492
2.483 315.010645594007
2.484 314.998042904715
2.485 314.985451133759
2.486 314.97287027129
2.487 314.96030030747
2.488 314.94774123247
2.489 314.935193036471
2.49 314.922655709662
2.491 314.910129242242
2.492 314.897613624418
2.493 314.88510884641
2.494 314.872614898443
2.495 314.860131770754
2.496 314.847659453587
2.497 314.835197937198
2.498 314.82274721185
2.499 314.810307267817
2.5 314.797878095381
2.501 314.785459684833
2.502 314.773052026474
2.503 314.760655110614
2.504 314.748268927572
2.505 314.735893467677
2.506 314.723528721265
2.507 314.711174678685
2.508 314.698831330291
2.509 314.686498666449
2.51 314.674176677532
2.511 314.661865353924
2.512 314.649564686017
2.513 314.637274664213
2.514 314.624995278923
2.515 314.612726520565
2.516 314.600468379569
2.517 314.588220846373
2.518 314.575983911424
2.519 314.563757565176
2.52 314.551541798096
2.521 314.539336600658
2.522 314.527141963344
2.523 314.514957876646
2.524 314.502784331066
2.525 314.490621317113
2.526 314.478468825308
2.527 314.466326846177
2.528 314.454195370258
2.529 314.442074388097
2.53 314.429963890248
2.531 314.417863867277
2.532 314.405774309754
2.533 314.393695208263
2.534 314.381626553394
2.535 314.369568335746
2.536 314.357520545928
2.537 314.345483174557
2.538 314.33345621226
2.539 314.321439649671
2.54 314.309433477435
2.541 314.297437686205
2.542 314.285452266642
2.543 314.273477209416
2.544 314.261512505208
2.545 314.249558144705
2.546 314.237614118604
2.547 314.225680417612
2.548 314.213757032442
2.549 314.201843953818
2.55 314.189941172472
2.551 314.178048679145
2.552 314.166166464586
2.553 314.154294519555
2.554 314.142432834819
2.555 314.130581401152
2.556 314.11874020934
2.557 314.106909250176
2.558 314.095088514463
2.559 314.08327799301
2.56 314.071477676638
2.561 314.059687556173
2.562 314.047907622455
2.563 314.036137866326
2.564 314.024378278642
2.565 314.012628850266
2.566 314.000889572069
2.567 313.98916043493
2.568 313.977441429738
2.569 313.965732547391
2.57 313.954033778794
2.571 313.942345114862
2.572 313.930666546518
2.573 313.918998064693
2.574 313.907339660327
2.575 313.895691324369
2.576 313.884053047777
2.577 313.872424821516
2.578 313.86080663656
2.579 313.849198483892
2.58 313.837600354504
2.581 313.826012239395
2.582 313.814434129574
2.583 313.802866016057
2.584 313.79130788987
2.585 313.779759742047
2.586 313.768221563629
2.587 313.756693345668
2.588 313.745175079223
2.589 313.733666755361
2.59 313.722168365158
2.591 313.710679899699
2.592 313.699201350077
2.593 313.687732707393
2.594 313.676273962755
2.595 313.664825107284
2.596 313.653386132104
2.597 313.641957028351
2.598 313.630537787168
2.599 313.619128399706
2.6 313.607728857125
2.601 313.596339150593
2.602 313.584959271287
2.603 313.573589210392
2.604 313.5622289591
2.605 313.550878508614
2.606 313.539537850142
2.607 313.528206974903
2.608 313.516885874124
2.609 313.505574539038
2.61 313.494272960889
2.611 313.482981130927
2.612 313.471699040412
2.613 313.460426680612
2.614 313.449164042802
2.615 313.437911118267
2.616 313.426667898298
2.617 313.415434374195
2.618 313.404210537269
2.619 313.392996378834
2.62 313.381791890217
2.621 313.370597062751
2.622 313.359411887776
2.623 313.348236356642
2.624 313.337070460707
2.625 313.325914191337
2.626 313.314767539905
2.627 313.303630497794
2.628 313.292503056393
2.629 313.281385207101
2.63 313.270276941324
2.631 313.259178250477
2.632 313.248089125981
2.633 313.237009559268
2.634 313.225939541777
2.635 313.214879064953
2.636 313.203828120252
2.637 313.192786699136
2.638 313.181754793077
2.639 313.170732393552
2.64 313.159719492049
2.641 313.148716080063
2.642 313.137722149097
2.643 313.126737690661
2.644 313.115762696274
2.645 313.104797157463
2.646 313.093841065763
2.647 313.082894412716
2.648 313.071957189874
2.649 313.061029388795
2.65 313.050111001045
2.651 313.039202018199
2.652 313.028302431839
2.653 313.017412233556
2.654 313.006531414948
2.655 312.995659967621
2.656 312.984797883189
2.657 312.973945153274
2.658 312.963101769505
2.659 312.952267723521
2.66 312.941443006967
2.661 312.930627611496
2.662 312.91982152877
2.663 312.909024750457
2.664 312.898237268234
2.665 312.887459073786
2.666 312.876690158806
2.667 312.865930514994
2.668 312.855180134057
2.669 312.844439007712
2.67 312.833707127682
2.671 312.822984485698
2.672 312.812271073501
2.673 312.801566882837
2.674 312.79087190546
2.675 312.780186133133
2.676 312.769509557626
2.677 312.758842170716
2.678 312.748183964191
2.679 312.737534929842
2.68 312.726895059472
2.681 312.716264344888
2.682 312.705642777907
2.683 312.695030350353
2.684 312.684427054057
2.685 312.67383288086
2.686 312.663247822609
2.687 312.652671871157
2.688 312.642105018368
2.689 312.63154725611
2.69 312.620998576263
2.691 312.610458970711
2.692 312.599928431348
2.693 312.589406950072
2.694 312.578894518793
2.695 312.568391129427
2.696 312.557896773895
2.697 312.54741144413
2.698 312.536935132069
2.699 312.526467829659
2.7 312.516009528852
2.701 312.50556022161
2.702 312.495119899901
2.703 312.484688555702
2.704 312.474266180995
2.705 312.463852767773
2.706 312.453448308033
2.707 312.443052793782
2.708 312.432666217033
2.709 312.422288569808
2.71 312.411919844134
2.711 312.401560032048
2.712 312.391209125594
2.713 312.380867116822
2.714 312.370533997791
2.715 312.360209760566
2.716 312.349894397221
2.717 312.339587899836
2.718 312.329290260499
2.719 312.319001471307
2.72 312.308721524361
2.721 312.298450411772
2.722 312.288188125658
2.723 312.277934658144
2.724 312.267690001362
2.725 312.257454147452
2.726 312.24722708856
2.727 312.237008816843
2.728 312.226799324461
2.729 312.216598603583
2.73 312.206406646386
2.731 312.196223445054
2.732 312.186048991778
2.733 312.175883278757
2.734 312.165726298196
2.735 312.155578042307
2.736 312.145438503312
2.737 312.135307673438
2.738 312.12518554492
2.739 312.11507211
2.74 312.104967360928
2.741 312.094871289959
2.742 312.084783889358
2.743 312.074705151396
2.744 312.064635068351
2.745 312.054573632509
2.746 312.044520836163
2.747 312.034476671611
2.748 312.024441131163
2.749 312.014414207131
2.75 312.004395891838
2.751 311.994386177612
2.752 311.984385056789
2.753 311.974392521712
2.754 311.964408564731
2.755 311.954433178203
2.756 311.944466354494
2.757 311.934508085975
2.758 311.924558365024
2.759 311.914617184027
2.76 311.904684535377
2.761 311.894760411474
2.762 311.884844804726
2.763 311.874937707547
2.764 311.865039112358
2.765 311.855149011587
2.766 311.84526739767
2.767 311.83539426305
2.768 311.825529600176
2.769 311.815673401505
2.77 311.8058256595
2.771 311.795986366633
2.772 311.786155515381
2.773 311.776333098228
2.774 311.766519107667
2.775 311.756713536197
2.776 311.746916376323
2.777 311.737127620557
2.778 311.727347261421
2.779 311.717575291441
2.78 311.70781170315
2.781 311.698056489089
2.782 311.688309641806
2.783 311.678571153856
2.784 311.668841017799
2.785 311.659119226205
2.786 311.649405771649
2.787 311.639700646713
2.788 311.630003843987
2.789 311.620315356066
2.79 311.610635175554
2.791 311.600963295062
2.792 311.591299707205
2.793 311.581644404607
2.794 311.5719973799
2.795 311.562358625721
2.796 311.552728134713
2.797 311.543105899529
2.798 311.533491912827
2.799 311.52388616727
2.8 311.514288655532
2.801 311.504699370291
2.802 311.495118304232
2.803 311.485545450047
2.804 311.475980800435
2.805 311.466424348103
2.806 311.456876085763
2.807 311.447336006133
2.808 311.437804101942
2.809 311.428280365921
2.81 311.41876479081
2.811 311.409257369357
2.812 311.399758094313
2.813 311.39026695844
2.814 311.380783954503
2.815 311.371309075277
2.816 311.361842313541
2.817 311.352383662083
2.818 311.342933113697
2.819 311.333490661182
2.82 311.324056297345
2.821 311.314630015002
2.822 311.305211806971
2.823 311.29580166608
2.824 311.286399585163
2.825 311.27700555706
2.826 311.267619574619
2.827 311.258241630694
2.828 311.248871718144
2.829 311.239509829838
2.83 311.230155958648
2.831 311.220810097455
2.832 311.211472239146
2.833 311.202142376615
2.834 311.192820502762
2.835 311.183506610494
2.836 311.174200692723
2.837 311.164902742371
2.838 311.155612752364
2.839 311.146330715635
2.84 311.137056625123
2.841 311.127790473776
2.842 311.118532254545
2.843 311.109281960391
2.844 311.100039584279
2.845 311.090805119182
2.846 311.081578558078
2.847 311.072359893955
2.848 311.063149119802
2.849 311.05394622862
2.85 311.044751213413
2.851 311.035564067193
2.852 311.026384782978
2.853 311.017213353792
2.854 311.008049772667
2.855 310.99889403264
2.856 310.989746126755
2.857 310.980606048062
2.858 310.97147378962
2.859 310.96234934449
2.86 310.953232705743
2.861 310.944123866455
2.862 310.935022819708
2.863 310.925929558593
2.864 310.916844076204
2.865 310.907766365643
2.866 310.898696420019
2.867 310.889634232447
2.868 310.880579796047
2.869 310.871533103948
2.87 310.862494149282
2.871 310.853462925191
2.872 310.84443942482
2.873 310.835423641324
2.874 310.82641556786
2.875 310.817415197596
2.876 310.808422523703
2.877 310.79943753936
2.878 310.79046023775
2.879 310.781490612066
2.88 310.772528655504
2.881 310.763574361269
2.882 310.75462772257
2.883 310.745688732624
2.884 310.736757384653
2.885 310.727833671886
2.886 310.718917587558
2.887 310.71000912491
2.888 310.701108277191
2.889 310.692215037654
2.89 310.683329399559
2.891 310.674451356173
2.892 310.665580900769
2.893 310.656718026625
2.894 310.647862727027
2.895 310.639014995266
2.896 310.63017482464
2.897 310.621342208452
2.898 310.612517140012
2.899 310.603699612637
2.9 310.594889619649
2.901 310.586087154377
2.902 310.577292210155
2.903 310.568504780325
2.904 310.559724858234
2.905 310.550952437234
2.906 310.542187510685
2.907 310.533430071954
2.908 310.524680114411
2.909 310.515937631435
2.91 310.50720261641
2.911 310.498475062726
2.912 310.489754963779
2.913 310.481042312971
2.914 310.472337103712
2.915 310.463639329415
2.916 310.454948983501
2.917 310.446266059397
2.918 310.437590550537
2.919 310.428922450358
2.92 310.420261752307
2.921 310.411608449834
2.922 310.402962536396
2.923 310.394324005456
2.924 310.385692850485
2.925 310.377069064957
2.926 310.368452642353
2.927 310.359843576161
2.928 310.351241859874
2.929 310.342647486993
2.93 310.334060451021
2.931 310.325480745471
2.932 310.316908363861
2.933 310.308343299713
2.934 310.299785546557
2.935 310.291235097929
2.936 310.28269194737
2.937 310.274156088427
2.938 310.265627514654
2.939 310.25710621961
2.94 310.24859219686
2.941 310.240085439976
2.942 310.231585942535
2.943 310.22309369812
2.944 310.21460870032
2.945 310.20613094273
2.946 310.19766041895
2.947 310.189197122589
2.948 310.180741047257
2.949 310.172292186575
2.95 310.163850534167
2.951 310.155416083662
2.952 310.146988828697
2.953 310.138568762915
2.954 310.130155879964
2.955 310.121750173497
2.956 310.113351637175
2.957 310.104960264662
2.958 310.096576049631
2.959 310.088198985758
2.96 310.079829066728
2.961 310.071466286228
2.962 310.063110637955
2.963 310.054762115608
2.964 310.046420712894
2.965 310.038086423525
2.966 310.02975924122
2.967 310.021439159702
2.968 310.013126172702
2.969 310.004820273954
2.97 309.9965214572
2.971 309.988229716187
2.972 309.979945044667
2.973 309.971667436401
2.974 309.963396885151
2.975 309.955133384687
2.976 309.946876928787
2.977 309.938627511231
2.978 309.930385125807
2.979 309.922149766308
2.98 309.913921426533
2.981 309.905700100285
2.982 309.897485781377
2.983 309.889278463623
2.984 309.881078140845
2.985 309.87288480687
2.986 309.864698455532
2.987 309.85651908067
2.988 309.848346676127
2.989 309.840181235754
2.99 309.832022753407
2.991 309.823871222947
2.992 309.815726638242
2.993 309.807588993164
2.994 309.799458281591
2.995 309.791334497407
2.996 309.783217634503
2.997 309.775107686774
2.998 309.76700464812
2.999 309.758908512449
3 309.750819273672
3.001 309.742736925707
3.002 309.734661462478
3.003 309.726592877914
3.004 309.718531165949
3.005 309.710476320523
3.006 309.702428335583
3.007 309.69438720508
3.008 309.68635292297
3.009 309.678325483216
3.01 309.670304879787
3.011 309.662291106655
3.012 309.654284157801
3.013 309.646284027209
3.014 309.638290708868
3.015 309.630304196776
3.016 309.622324484933
3.017 309.614351567346
3.018 309.606385438028
3.019 309.598426090996
3.02 309.590473520274
3.021 309.58252771989
3.022 309.57458868388
3.023 309.566656406283
3.024 309.558730881145
3.025 309.550812102516
3.026 309.542900064452
3.027 309.534994761016
3.028 309.527096186275
3.029 309.519204334302
3.03 309.511319199174
3.031 309.503440774976
3.032 309.495569055796
3.033 309.487704035729
3.034 309.479845708876
3.035 309.47199406934
3.036 309.464149111235
3.037 309.456310828674
3.038 309.448479215781
3.039 309.440654266682
3.04 309.43283597551
3.041 309.425024336403
3.042 309.417219343503
3.043 309.40942099096
3.044 309.401629272927
3.045 309.393844183564
3.046 309.386065717036
3.047 309.378293867513
3.048 309.37052862917
3.049 309.362769996188
3.05 309.355017962754
3.051 309.347272523058
3.052 309.339533671298
3.053 309.331801401677
3.054 309.324075708401
3.055 309.316356585683
3.056 309.308644027742
3.057 309.300938028801
3.058 309.293238583089
3.059 309.28554568484
3.06 309.277859328294
3.061 309.270179507694
3.062 309.262506217293
3.063 309.254839451343
3.064 309.247179204107
3.065 309.23952546985
3.066 309.231878242843
3.067 309.224237517362
3.068 309.216603287689
3.069 309.208975548111
3.07 309.201354292921
3.071 309.193739516414
3.072 309.186131212895
3.073 309.178529376671
3.074 309.170934002054
3.075 309.163345083364
3.076 309.155762614924
3.077 309.148186591062
3.078 309.140617006112
3.079 309.133053854414
3.08 309.125497130311
3.081 309.117946828154
3.082 309.110402942297
3.083 309.102865467099
3.084 309.095334396927
3.085 309.08780972615
3.086 309.080291449143
3.087 309.072779560288
3.088 309.065274053969
3.089 309.057774924578
3.09 309.05028216651
3.091 309.042795774168
3.092 309.035315741957
3.093 309.027842064288
3.094 309.020374735578
3.095 309.012913750249
3.096 309.005459102728
3.097 308.998010787446
3.098 308.990568798841
3.099 308.983133131354
3.1 308.975703779433
3.101 308.968280737531
3.102 308.960864000103
3.103 308.953453561614
3.104 308.94604941653
3.105 308.938651559324
3.106 308.931259984474
3.107 308.923874686462
3.108 308.916495659777
3.109 308.909122898911
3.11 308.901756398362
3.111 308.894396152634
3.112 308.887042156233
3.113 308.879694403673
3.114 308.872352889472
3.115 308.865017608153
3.116 308.857688554244
3.117 308.850365722279
3.118 308.843049106796
3.119 308.835738702336
3.12 308.82843450345
3.121 308.821136504689
3.122 308.813844700612
3.123 308.806559085782
3.124 308.799279654767
3.125 308.79200640214
3.126 308.784739322479
3.127 308.777478410367
3.128 308.770223660392
3.129 308.762975067147
3.13 308.75573262523
3.131 308.748496329243
3.132 308.741266173794
3.133 308.734042153496
3.134 308.726824262967
3.135 308.719612496829
3.136 308.712406849709
3.137 308.705207316239
3.138 308.698013891058
3.139 308.690826568807
3.14 308.683645344134
3.141 308.67647021169
3.142 308.669301166132
3.143 308.662138202121
3.144 308.654981314326
3.145 308.647830497417
3.146 308.64068574607
3.147 308.633547054967
3.148 308.626414418794
3.149 308.619287832243
3.15 308.612167290008
3.151 308.605052786791
3.152 308.597944317297
3.153 308.590841876237
3.154 308.583745458325
3.155 308.576655058282
3.156 308.569570670834
3.157 308.562492290708
3.158 308.555419912641
3.159 308.548353531371
3.16 308.541293141643
3.161 308.534238738206
3.162 308.527190315812
3.163 308.520147869222
3.164 308.513111393198
3.165 308.506080882508
3.166 308.499056331926
3.167 308.492037736229
3.168 308.485025090199
3.169 308.478018388625
3.17 308.471017626297
3.171 308.464022798012
3.172 308.457033898573
3.173 308.450050922786
3.174 308.44307386546
3.175 308.436102721413
3.176 308.429137485465
3.177 308.42217815244
3.178 308.41522471717
3.179 308.408277174488
3.18 308.401335519234
3.181 308.394399746252
3.182 308.387469850392
3.183 308.380545826506
3.184 308.373627669452
3.185 308.366715374094
3.186 308.3598089353
3.187 308.352908347941
3.188 308.346013606894
3.189 308.339124707041
3.19 308.33224164327
3.191 308.325364410469
3.192 308.318493003536
3.193 308.31162741737
3.194 308.304767646876
3.195 308.297913686965
3.196 308.29106553255
3.197 308.28422317855
3.198 308.277386619889
3.199 308.270555851495
3.2 308.2637308683
3.201 308.256911665243
3.202 308.250098237265
3.203 308.243290579313
3.204 308.236488686338
3.205 308.229692553297
3.206 308.222902175148
3.207 308.216117546859
3.208 308.209338663398
3.209 308.202565519739
3.21 308.195798110862
3.211 308.18903643175
3.212 308.182280477391
3.213 308.175530242777
3.214 308.168785722906
3.215 308.162046912779
3.216 308.155313807403
3.217 308.148586401788
3.218 308.14186469095
3.219 308.135148669908
3.22 308.128438333688
3.221 308.121733677317
3.222 308.11503469583
3.223 308.108341384265
3.224 308.101653737663
3.225 308.094971751073
3.226 308.088295419546
3.227 308.081624738138
3.228 308.074959701909
3.229 308.068300305924
3.23 308.061646545254
3.231 308.054998414972
3.232 308.048355910157
3.233 308.041719025892
3.234 308.035087757264
3.235 308.028462099366
3.236 308.021842047294
3.237 308.015227596149
3.238 308.008618741036
3.239 308.002015477065
3.24 307.995417799351
3.241 307.988825703012
3.242 307.982239183172
3.243 307.975658234958
3.244 307.969082853502
3.245 307.962513033941
3.246 307.955948771415
3.247 307.949390061071
3.248 307.942836898058
3.249 307.93628927753
3.25 307.929747194646
3.251 307.923210644569
3.252 307.916679622466
3.253 307.910154123509
3.254 307.903634142874
3.255 307.897119675742
3.256 307.890610717299
3.257 307.884107262732
3.258 307.877609307237
3.259 307.871116846011
3.26 307.864629874256
3.261 307.858148387181
3.262 307.851672379995
3.263 307.845201847914
3.264 307.838736786159
3.265 307.832277189953
3.266 307.825823054526
3.267 307.81937437511
3.268 307.812931146942
3.269 307.806493365265
3.27 307.800061025324
3.271 307.793634122369
3.272 307.787212651655
3.273 307.780796608442
3.274 307.774385987991
3.275 307.767980785571
3.276 307.761580996454
3.277 307.755186615915
3.278 307.748797639235
3.279 307.7424140617
3.28 307.736035878597
3.281 307.72966308522
3.282 307.723295676867
3.283 307.716933648839
3.284 307.710576996444
3.285 307.70422571499
3.286 307.697879799793
3.287 307.691539246171
3.288 307.685204049448
3.289 307.678874204952
3.29 307.672549708013
3.291 307.666230553967
3.292 307.659916738156
3.293 307.653608255922
3.294 307.647305102615
3.295 307.641007273588
3.296 307.634714764197
3.297 307.628427569804
3.298 307.622145685774
3.299 307.615869107477
3.3 307.609597830287
3.301 307.603331849581
3.302 307.597071160743
3.303 307.590815759158
3.304 307.584565640218
3.305 307.578320799317
3.306 307.572081231853
3.307 307.565846933231
3.308 307.559617898858
3.309 307.553394124144
3.31 307.547175604507
3.311 307.540962335365
3.312 307.534754312143
3.313 307.528551530269
3.314 307.522353985175
3.315 307.516161672297
3.316 307.509974587077
3.317 307.503792724959
3.318 307.497616081392
3.319 307.491444651829
3.32 307.485278431727
3.321 307.479117416547
3.322 307.472961601754
3.323 307.466810982819
3.324 307.460665555215
3.325 307.454525314419
3.326 307.448390255913
3.327 307.442260375183
3.328 307.436135667719
3.329 307.430016129016
3.33 307.42390175457
3.331 307.417792539886
3.332 307.411688480468
3.333 307.405589571827
3.334 307.399495809478
3.335 307.39340718894
3.336 307.387323705734
3.337 307.381245355387
3.338 307.375172133431
3.339 307.3691040354
3.34 307.363041056833
3.341 307.356983193272
3.342 307.350930440265
3.343 307.344882793362
3.344 307.338840248119
3.345 307.332802800094
3.346 307.32677044485
3.347 307.320743177955
3.348 307.314720994979
3.349 307.308703891498
3.35 307.302691863091
3.351 307.296684905339
3.352 307.290683013832
3.353 307.28468618416
3.354 307.278694411917
3.355 307.272707692703
3.356 307.266726022122
3.357 307.260749395779
3.358 307.254777809286
3.359 307.248811258259
3.36 307.242849738315
3.361 307.236893245078
3.362 307.230941774175
3.363 307.224995321236
3.364 307.219053881897
3.365 307.213117451796
3.366 307.207186026575
3.367 307.201259601882
3.368 307.195338173367
3.369 307.189421736685
3.37 307.183510287493
3.371 307.177603821454
3.372 307.171702334236
3.373 307.165805821507
3.374 307.159914278942
3.375 307.154027702219
3.376 307.14814608702
3.377 307.14226942903
3.378 307.136397723941
3.379 307.130530967445
3.38 307.124669155239
3.381 307.118812283026
3.382 307.112960346511
3.383 307.107113341403
3.384 307.101271263415
3.385 307.095434108264
3.386 307.08960187167
3.387 307.08377454936
3.388 307.07795213706
3.389 307.072134630505
3.39 307.066322025429
3.391 307.060514317574
3.392 307.054711502682
3.393 307.048913576503
3.394 307.043120534788
3.395 307.037332373293
3.396 307.031549087776
3.397 307.025770674001
3.398 307.019997127735
3.399 307.01422844475
3.4 307.008464620819
3.401 307.002705651721
3.402 306.996951533239
3.403 306.991202261159
3.404 306.98545783127
3.405 306.979718239368
3.406 306.973983481248
3.407 306.968253552713
3.408 306.962528449567
3.409 306.95680816762
3.41 306.951092702685
3.411 306.945382050577
3.412 306.939676207118
3.413 306.93397516813
3.414 306.928278929443
3.415 306.922587486888
3.416 306.9169008363
3.417 306.911218973518
3.418 306.905541894385
3.419 306.899869594748
3.42 306.894202070457
3.421 306.888539317367
3.422 306.882881331335
3.423 306.877228108223
3.424 306.871579643896
3.425 306.865935934223
3.426 306.860296975078
3.427 306.854662762336
3.428 306.849033291879
3.429 306.843408559589
3.43 306.837788561356
3.431 306.832173293069
3.432 306.826562750625
3.433 306.820956929922
3.434 306.815355826863
3.435 306.809759437354
3.436 306.804167757304
3.437 306.798580782628
3.438 306.792998509243
3.439 306.78742093307
3.44 306.781848050034
3.441 306.776279856062
3.442 306.770716347088
3.443 306.765157519045
3.444 306.759603367875
3.445 306.75405388952
3.446 306.748509079927
3.447 306.742968935045
3.448 306.737433450829
3.449 306.731902623238
3.45 306.726376448231
3.451 306.720854921774
3.452 306.715338039835
3.453 306.709825798387
3.454 306.704318193406
3.455 306.69881522087
3.456 306.693316876763
3.457 306.687823157072
3.458 306.682334057787
3.459 306.676849574902
3.46 306.671369704415
3.461 306.665894442326
3.462 306.660423784641
3.463 306.654957727367
3.464 306.649496266518
3.465 306.644039398107
3.466 306.638587118155
3.467 306.633139422685
3.468 306.627696307722
3.469 306.622257769296
3.47 306.616823803441
3.471 306.611394406194
3.472 306.605969573596
3.473 306.60054930169
3.474 306.595133586525
3.475 306.589722424151
3.476 306.584315810624
3.477 306.578913742003
3.478 306.573516214348
3.479 306.568123223726
3.48 306.562734766206
3.481 306.557350837859
3.482 306.551971434763
3.483 306.546596552997
3.484 306.541226188645
3.485 306.535860337792
3.486 306.530498996529
3.487 306.52514216095
3.488 306.519789827153
3.489 306.514441991237
3.49 306.509098649308
3.491 306.503759797473
3.492 306.498425431843
3.493 306.493095548534
3.494 306.487770143663
3.495 306.482449213352
3.496 306.477132753727
3.497 306.471820760917
3.498 306.466513231053
3.499 306.461210160272
3.5 306.455911544712
3.501 306.450617380517
3.502 306.445327663832
3.503 306.440042390808
3.504 306.434761557597
3.505 306.429485160355
3.506 306.424213195244
3.507 306.418945658426
3.508 306.413682546068
3.509 306.408423854341
3.51 306.403169579418
3.511 306.397919717477
3.512 306.392674264698
3.513 306.387433217266
3.514 306.382196571368
3.515 306.376964323194
3.516 306.371736468941
3.517 306.366513004805
3.518 306.361293926987
3.519 306.356079231692
3.52 306.350868915129
3.521 306.345662973509
3.522 306.340461403046
3.523 306.33526419996
3.524 306.33007136047
3.525 306.324882880804
3.526 306.319698757189
3.527 306.314518985857
3.528 306.309343563043
3.529 306.304172484987
3.53 306.299005747929
3.531 306.293843348116
3.532 306.288685281797
3.533 306.283531545222
3.534 306.278382134648
3.535 306.273237046335
3.536 306.268096276543
3.537 306.262959821538
3.538 306.25782767759
3.539 306.252699840971
3.54 306.247576307956
3.541 306.242457074825
3.542 306.23734213786
3.543 306.232231493346
3.544 306.227125137572
3.545 306.222023066832
3.546 306.21692527742
3.547 306.211831765635
3.548 306.20674252778
3.549 306.20165756016
3.55 306.196576859085
3.551 306.191500420866
3.552 306.18642824182
3.553 306.181360318264
3.554 306.176296646522
3.555 306.171237222919
3.556 306.166182043783
3.557 306.161131105447
3.558 306.156084404246
3.559 306.151041936519
3.56 306.146003698607
3.561 306.140969686856
3.562 306.135939897614
3.563 306.130914327234
3.564 306.12589297207
3.565 306.120875828481
3.566 306.115862892827
3.567 306.110854161475
3.568 306.105849630791
3.569 306.100849297149
3.57 306.095853156921
3.571 306.090861206487
3.572 306.085873442226
3.573 306.080889860525
3.574 306.075910457769
3.575 306.070935230351
3.576 306.065964174664
3.577 306.060997287105
3.578 306.056034564076
3.579 306.05107600198
3.58 306.046121597223
3.581 306.041171346217
3.582 306.036225245374
3.583 306.031283291112
3.584 306.02634547985
3.585 306.021411808011
3.586 306.016482272021
3.587 306.011556868311
3.588 306.006635593313
3.589 306.001718443462
3.59 305.996805415198
3.591 305.991896504963
3.592 305.986991709203
3.593 305.982091024366
3.594 305.977194446904
3.595 305.972301973272
3.596 305.967413599929
3.597 305.962529323335
3.598 305.957649139955
3.599 305.952773046258
3.6 305.947901038713
3.601 305.943033113795
3.602 305.938169267981
3.603 305.933309497751
3.604 305.92845379959
3.605 305.923602169983
3.606 305.91875460542
3.607 305.913911102395
3.608 305.909071657403
3.609 305.904236266943
3.61 305.899404927519
3.611 305.894577635635
3.612 305.889754387801
3.613 305.884935180527
3.614 305.880120010329
3.615 305.875308873724
3.616 305.870501767235
3.617 305.865698687384
3.618 305.8608996307
3.619 305.856104593712
3.62 305.851313572955
3.621 305.846526564965
3.622 305.841743566282
3.623 305.836964573448
3.624 305.83218958301
3.625 305.827418591516
3.626 305.822651595519
3.627 305.817888591574
3.628 305.813129576239
3.629 305.808374546076
3.63 305.803623497649
3.631 305.798876427525
3.632 305.794133332275
3.633 305.789394208473
3.634 305.784659052695
3.635 305.779927861522
3.636 305.775200631536
3.637 305.770477359323
3.638 305.765758041471
3.639 305.761042674574
3.64 305.756331255225
3.641 305.751623780024
3.642 305.74692024557
3.643 305.742220648469
3.644 305.737524985328
3.645 305.732833252757
3.646 305.728145447368
3.647 305.72346156578
3.648 305.71878160461
3.649 305.714105560481
3.65 305.709433430018
3.651 305.704765209851
3.652 305.700100896609
3.653 305.695440486928
3.654 305.690783977446
3.655 305.686131364802
3.656 305.681482645639
3.657 305.676837816606
3.658 305.67219687435
3.659 305.667559815524
3.66 305.662926636784
3.661 305.658297334787
3.662 305.653671906197
3.663 305.649050347676
3.664 305.644432655892
3.665 305.639818827516
3.666 305.635208859221
3.667 305.630602747682
3.668 305.62600048958
3.669 305.621402081597
3.67 305.616807520417
3.671 305.612216802729
3.672 305.607629925224
3.673 305.603046884595
3.674 305.598467677541
3.675 305.593892300761
3.676 305.589320750957
3.677 305.584753024836
3.678 305.580189119107
3.679 305.57562903048
3.68 305.571072755671
3.681 305.566520291398
3.682 305.561971634381
3.683 305.557426781342
3.684 305.55288572901
3.685 305.548348474112
3.686 305.543815013382
3.687 305.539285343554
3.688 305.534759461366
3.689 305.53023736356
3.69 305.525719046878
3.691 305.521204508069
3.692 305.516693743881
3.693 305.512186751068
3.694 305.507683526384
3.695 305.503184066588
3.696 305.498688368442
3.697 305.494196428709
3.698 305.489708244156
3.699 305.485223811554
3.7 305.480743127674
3.701 305.476266189294
3.702 305.471792993191
3.703 305.467323536147
3.704 305.462857814946
3.705 305.458395826375
3.706 305.453937567225
3.707 305.449483034288
3.708 305.44503222436
3.709 305.440585134239
3.71 305.436141760728
3.711 305.43170210063
3.712 305.427266150752
3.713 305.422833907905
3.714 305.418405368901
3.715 305.413980530557
3.716 305.40955938969
3.717 305.405141943122
3.718 305.400728187677
3.719 305.396318120183
3.72 305.391911737469
3.721 305.387509036368
3.722 305.383110013715
3.723 305.378714666348
3.724 305.37432299111
3.725 305.369934984844
3.726 305.365550644396
3.727 305.361169966616
3.728 305.356792948357
3.729 305.352419586474
3.73 305.348049877824
3.731 305.343683819269
3.732 305.339321407672
3.733 305.3349626399
3.734 305.330607512822
3.735 305.326256023309
3.736 305.321908168237
3.737 305.317563944483
3.738 305.313223348927
3.739 305.308886378453
3.74 305.304553029946
3.741 305.300223300295
3.742 305.295897186391
3.743 305.291574685129
3.744 305.287255793406
3.745 305.282940508121
3.746 305.278628826177
3.747 305.274320744479
3.748 305.270016259935
3.749 305.265715369456
3.75 305.261418069955
3.751 305.257124358349
3.752 305.252834231556
3.753 305.248547686499
3.754 305.244264720101
3.755 305.23998532929
3.756 305.235709510997
3.757 305.231437262152
3.758 305.227168579693
3.759 305.222903460557
3.76 305.218641901685
3.761 305.21438390002
3.762 305.210129452509
3.763 305.205878556101
3.764 305.201631207747
3.765 305.197387404403
3.766 305.193147143025
3.767 305.188910420573
3.768 305.18467723401
3.769 305.180447580301
3.77 305.176221456413
3.771 305.171998859318
3.772 305.167779785989
3.773 305.163564233402
3.774 305.159352198536
3.775 305.155143678372
3.776 305.150938669894
3.777 305.146737170089
3.778 305.142539175947
3.779 305.138344684459
3.78 305.134153692621
3.781 305.12996619743
3.782 305.125782195886
3.783 305.121601684992
3.784 305.117424661754
3.785 305.113251123179
3.786 305.109081066279
3.787 305.104914488067
3.788 305.100751385558
3.789 305.096591755773
3.79 305.092435595732
3.791 305.08828290246
3.792 305.084133672982
3.793 305.079987904329
3.794 305.075845593533
3.795 305.071706737627
3.796 305.06757133365
3.797 305.063439378642
3.798 305.059310869644
3.799 305.055185803702
3.8 305.051064177864
3.801 305.046945989181
3.802 305.042831234705
3.803 305.038719911491
3.804 305.034612016599
3.805 305.030507547089
3.806 305.026406500025
3.807 305.022308872472
3.808 305.0182146615
3.809 305.014123864179
3.81 305.010036477585
3.811 305.005952498792
3.812 305.00187192488
3.813 304.997794752932
3.814 304.993720980031
3.815 304.989650603264
3.816 304.98558361972
3.817 304.981520026493
3.818 304.977459820676
3.819 304.973402999366
3.82 304.969349559663
3.821 304.96529949867
3.822 304.961252813492
3.823 304.957209501236
3.824 304.953169559012
3.825 304.949132983933
3.826 304.945099773114
3.827 304.941069923673
3.828 304.93704343273
3.829 304.933020297408
3.83 304.929000514833
3.831 304.924984082132
3.832 304.920970996436
3.833 304.916961254878
3.834 304.912954854593
3.835 304.908951792721
3.836 304.9049520664
3.837 304.900955672776
3.838 304.896962608993
3.839 304.8929728722
3.84 304.888986459547
3.841 304.885003368189
3.842 304.88102359528
3.843 304.87704713798
3.844 304.873073993449
3.845 304.869104158851
3.846 304.865137631351
3.847 304.861174408118
3.848 304.857214486324
3.849 304.85325786314
3.85 304.849304535745
3.851 304.845354501316
3.852 304.841407757034
3.853 304.837464300082
3.854 304.833524127647
3.855 304.829587236917
3.856 304.825653625083
3.857 304.821723289339
3.858 304.81779622688
3.859 304.813872434905
3.86 304.809951910615
3.861 304.806034651213
3.862 304.802120653906
3.863 304.798209915901
3.864 304.794302434409
3.865 304.790398206644
3.866 304.786497229822
3.867 304.782599501161
3.868 304.778705017881
3.869 304.774813777205
3.87 304.77092577636
3.871 304.767041012574
3.872 304.763159483076
3.873 304.759281185101
3.874 304.755406115882
3.875 304.751534272659
3.876 304.747665652672
3.877 304.743800253163
3.878 304.739938071379
3.879 304.736079104565
3.88 304.732223349973
3.881 304.728370804854
3.882 304.724521466465
3.883 304.720675332063
3.884 304.716832398906
3.885 304.712992664259
3.886 304.709156125384
3.887 304.70532277955
3.888 304.701492624026
3.889 304.697665656084
3.89 304.693841872998
3.891 304.690021272044
3.892 304.686203850503
3.893 304.682389605656
3.894 304.678578534787
3.895 304.674770635182
3.896 304.670965904129
3.897 304.667164338922
3.898 304.663365936852
3.899 304.659570695216
3.9 304.655778611312
3.901 304.651989682441
3.902 304.648203905906
3.903 304.644421279013
3.904 304.64064179907
3.905 304.636865463387
3.906 304.633092269277
3.907 304.629322214054
3.908 304.625555295037
3.909 304.621791509545
3.91 304.6180308549
3.911 304.614273328427
3.912 304.610518927453
3.913 304.606767649307
3.914 304.60301949132
3.915 304.599274450828
3.916 304.595532525166
3.917 304.591793711672
3.918 304.588058007689
3.919 304.58432541056
3.92 304.580595917629
3.921 304.576869526246
3.922 304.573146233761
3.923 304.569426037527
3.924 304.565708934898
3.925 304.561994923233
3.926 304.558283999891
3.927 304.554576162234
3.928 304.550871407627
3.929 304.547169733437
3.93 304.543471137033
3.931 304.539775615787
3.932 304.536083167071
3.933 304.532393788263
3.934 304.528707476741
3.935 304.525024229885
3.936 304.521344045079
3.937 304.517666919708
3.938 304.51399285116
3.939 304.510321836825
3.94 304.506653874094
3.941 304.502988960364
3.942 304.49932709303
3.943 304.495668269492
3.944 304.492012487152
3.945 304.488359743413
3.946 304.484710035681
3.947 304.481063361365
3.948 304.477419717875
3.949 304.473779102625
3.95 304.470141513029
3.951 304.466506946506
3.952 304.462875400475
3.953 304.459246872358
3.954 304.45562135958
3.955 304.451998859567
3.956 304.448379369748
3.957 304.444762887555
3.958 304.44114941042
3.959 304.43753893578
3.96 304.433931461073
3.961 304.430326983738
3.962 304.426725501219
3.963 304.42312701096
3.964 304.419531510409
3.965 304.415938997013
3.966 304.412349468226
3.967 304.408762921501
3.968 304.405179354294
3.969 304.401598764062
3.97 304.398021148268
3.971 304.394446504373
3.972 304.390874829842
3.973 304.387306122143
3.974 304.383740378745
3.975 304.38017759712
3.976 304.376617774742
3.977 304.373060909087
3.978 304.369506997633
3.979 304.365956037862
3.98 304.362408027255
3.981 304.358862963298
3.982 304.355320843478
3.983 304.351781665285
3.984 304.348245426211
3.985 304.344712123749
3.986 304.341181755395
3.987 304.337654318648
3.988 304.334129811009
3.989 304.33060822998
3.99 304.327089573066
3.991 304.323573837774
3.992 304.320061021614
3.993 304.316551122097
3.994 304.313044136737
3.995 304.30954006305
3.996 304.306038898554
3.997 304.30254064077
3.998 304.299045287219
3.999 304.295552835427
4 304.29206328292
4.001 304.288576627228
4.002 304.285092865881
4.003 304.281611996414
4.004 304.278134016361
4.005 304.274658923261
4.006 304.271186714653
4.007 304.26771738808
4.008 304.264250941085
4.009 304.260787371215
4.01 304.257326676019
4.011 304.253868853048
4.012 304.250413899854
4.013 304.246961813992
4.014 304.24351259302
4.015 304.240066234497
4.016 304.236622735984
4.017 304.233182095045
4.018 304.229744309247
4.019 304.226309376156
4.02 304.222877293343
4.021 304.219448058381
4.022 304.216021668843
4.023 304.212598122307
4.024 304.20917741635
4.025 304.205759548554
4.026 304.202344516502
4.027 304.198932317778
4.028 304.195522949969
4.029 304.192116410666
4.03 304.188712697459
4.031 304.185311807942
4.032 304.181913739711
4.033 304.178518490363
4.034 304.175126057498
4.035 304.171736438718
4.036 304.168349631628
4.037 304.164965633834
4.038 304.161584442943
4.039 304.158206056567
4.04 304.154830472317
4.041 304.151457687809
4.042 304.148087700659
4.043 304.144720508486
4.044 304.141356108912
4.045 304.137994499558
4.046 304.13463567805
4.047 304.131279642016
4.048 304.127926389084
4.049 304.124575916886
4.05 304.121228223056
4.051 304.117883305229
4.052 304.114541161042
4.053 304.111201788136
4.054 304.107865184152
4.055 304.104531346734
4.056 304.101200273528
4.057 304.097871962182
4.058 304.094546410346
4.059 304.091223615672
4.06 304.087903575814
4.061 304.084586288429
4.062 304.081271751174
4.063 304.077959961711
4.064 304.074650917702
4.065 304.071344616811
4.066 304.068041056704
4.067 304.064740235051
4.068 304.061442149522
4.069 304.05814679779
4.07 304.054854177529
4.071 304.051564286417
4.072 304.048277122132
4.073 304.044992682355
4.074 304.041710964769
4.075 304.038431967058
4.076 304.035155686911
4.077 304.031882122015
4.078 304.028611270062
4.079 304.025343128745
4.08 304.02207769576
4.081 304.018814968802
4.082 304.015554945572
4.083 304.01229762377
4.084 304.009043001101
4.085 304.005791075268
4.086 304.002541843979
4.087 303.999295304944
4.088 303.996051455874
4.089 303.992810294482
4.09 303.989571818483
4.091 303.986336025594
4.092 303.983102913536
4.093 303.979872480028
4.094 303.976644722796
4.095 303.973419639563
4.096 303.970197228057
4.097 303.966977486007
4.098 303.963760411146
4.099 303.960546001205
4.1 303.95733425392
4.101 303.954125167029
4.102 303.950918738271
4.103 303.947714965387
4.104 303.94451384612
4.105 303.941315378215
4.106 303.93811955942
4.107 303.934926387483
4.108 303.931735860157
4.109 303.928547975193
4.11 303.925362730348
4.111 303.922180123378
4.112 303.919000152042
4.113 303.915822814101
4.114 303.912648107318
4.115 303.909476029458
4.116 303.906306578288
4.117 303.903139751577
4.118 303.899975547095
4.119 303.896813962616
4.12 303.893654995914
4.121 303.890498644765
4.122 303.887344906948
4.123 303.884193780245
4.124 303.881045262437
4.125 303.877899351308
4.126 303.874756044646
4.127 303.871615340239
4.128 303.868477235876
4.129 303.865341729351
4.13 303.862208818457
4.131 303.85907850099
4.132 303.855950774748
4.133 303.852825637532
4.134 303.849703087143
4.135 303.846583121385
4.136 303.843465738063
4.137 303.840350934987
4.138 303.837238709964
4.139 303.834129060807
4.14 303.831021985329
4.141 303.827917481346
4.142 303.824815546674
4.143 303.821716179133
4.144 303.818619376544
4.145 303.81552513673
4.146 303.812433457515
4.147 303.809344336727
4.148 303.806257772195
4.149 303.803173761749
4.15 303.800092303221
4.151 303.797013394446
4.152 303.793937033261
4.153 303.790863217503
4.154 303.787791945012
4.155 303.784723213632
4.156 303.781657021205
4.157 303.778593365578
4.158 303.775532244597
4.159 303.772473656114
4.16 303.769417597978
4.161 303.766364068044
4.162 303.763313064166
4.163 303.760264584203
4.164 303.757218626011
4.165 303.754175187454
4.166 303.751134266393
4.167 303.748095860693
4.168 303.74505996822
4.169 303.742026586843
4.17 303.738995714432
4.171 303.735967348859
4.172 303.732941487998
4.173 303.729918129724
4.174 303.726897271917
4.175 303.723878912454
4.176 303.720863049218
4.177 303.717849680091
4.178 303.714838802959
4.179 303.711830415708
4.18 303.708824516228
4.181 303.70582110241
4.182 303.702820172144
4.183 303.699821723326
4.184 303.696825753853
4.185 303.693832261621
4.186 303.690841244532
4.187 303.687852700486
4.188 303.684866627387
4.189 303.681883023141
4.19 303.678901885654
4.191 303.675923212836
4.192 303.672947002598
4.193 303.669973252852
4.194 303.667001961513
4.195 303.664033126497
4.196 303.661066745722
4.197 303.658102817109
4.198 303.655141338579
4.199 303.652182308055
4.2 303.649225723464
4.201 303.646271582733
4.202 303.64331988379
4.203 303.640370624566
4.204 303.637423802995
4.205 303.63447941701
4.206 303.631537464548
4.207 303.628597943548
4.208 303.625660851948
4.209 303.622726187692
4.21 303.619793948722
4.211 303.616864132983
4.212 303.613936738423
4.213 303.611011762991
4.214 303.608089204638
4.215 303.605169061315
4.216 303.602251330977
4.217 303.599336011581
4.218 303.596423101085
4.219 303.593512597447
4.22 303.590604498629
4.221 303.587698802595
4.222 303.584795507309
4.223 303.581894610739
4.224 303.578996110853
4.225 303.576100005622
4.226 303.573206293016
4.227 303.570314971011
4.228 303.567426037582
4.229 303.564539490707
4.23 303.561655328364
4.231 303.558773548534
4.232 303.555894149201
4.233 303.553017128349
4.234 303.550142483964
4.235 303.547270214034
4.236 303.544400316548
4.237 303.5415327895
4.238 303.538667630881
4.239 303.535804838686
4.24 303.532944410914
4.241 303.530086345561
4.242 303.527230640628
4.243 303.524377294119
4.244 303.521526304035
4.245 303.518677668383
4.246 303.515831385171
4.247 303.512987452406
4.248 303.5101458681
4.249 303.507306630266
4.25 303.504469736917
4.251 303.50163518607
4.252 303.498802975742
4.253 303.495973103952
4.254 303.493145568723
4.255 303.490320368076
4.256 303.487497500036
4.257 303.48467696263
4.258 303.481858753886
4.259 303.479042871833
4.26 303.476229314504
4.261 303.47341807993
4.262 303.470609166148
4.263 303.467802571194
4.264 303.464998293106
4.265 303.462196329925
4.266 303.459396679691
4.267 303.45659934045
4.268 303.453804310245
4.269 303.451011587124
4.27 303.448221169136
4.271 303.44543305433
4.272 303.442647240759
4.273 303.439863726477
4.274 303.437082509538
4.275 303.434303588001
4.276 303.431526959923
4.277 303.428752623366
4.278 303.425980576392
4.279 303.423210817064
4.28 303.420443343449
4.281 303.417678153612
4.282 303.414915245625
4.283 303.412154617556
4.284 303.409396267479
4.285 303.406640193467
4.286 303.403886393596
4.287 303.401134865944
4.288 303.398385608589
4.289 303.395638619613
4.29 303.392893897097
4.291 303.390151439126
4.292 303.387411243786
4.293 303.384673309163
4.294 303.381937633348
4.295 303.37920421443
4.296 303.376473050503
4.297 303.37374413966
4.298 303.371017479997
4.299 303.368293069612
4.3 303.365570906603
4.301 303.362850989072
4.302 303.36013331512
4.303 303.357417882852
4.304 303.354704690373
4.305 303.351993735791
4.306 303.349285017215
4.307 303.346578532756
4.308 303.343874280524
4.309 303.341172258636
4.31 303.338472465206
4.311 303.335774898351
4.312 303.333079556191
4.313 303.330386436845
4.314 303.327695538436
4.315 303.325006859088
4.316 303.322320396927
4.317 303.319636150078
4.318 303.316954116671
4.319 303.314274294836
4.32 303.311596682706
4.321 303.308921278413
4.322 303.306248080093
4.323 303.303577085883
4.324 303.30090829392
4.325 303.298241702347
4.326 303.295577309303
4.327 303.292915112932
4.328 303.290255111379
4.329 303.287597302791
4.33 303.284941685316
4.331 303.282288257104
4.332 303.279637016306
4.333 303.276987961074
4.334 303.274341089564
4.335 303.271696399932
4.336 303.269053890336
4.337 303.266413558934
4.338 303.263775403889
4.339 303.261139423362
4.34 303.258505615518
4.341 303.255873978523
4.342 303.253244510545
4.343 303.250617209751
4.344 303.247992074313
4.345 303.245369102403
4.346 303.242748292195
4.347 303.240129641865
4.348 303.237513149588
4.349 303.234898813544
4.35 303.232286631913
4.351 303.229676602876
4.352 303.227068724617
4.353 303.224462995322
4.354 303.221859413175
4.355 303.219257976365
4.356 303.216658683083
4.357 303.214061531519
4.358 303.211466519866
4.359 303.208873646318
4.36 303.206282909072
4.361 303.203694306324
4.362 303.201107836274
4.363 303.198523497123
4.364 303.195941287072
4.365 303.193361204325
4.366 303.190783247088
4.367 303.188207413568
4.368 303.185633701973
4.369 303.183062110513
4.37 303.1804926374
4.371 303.177925280846
4.372 303.175360039067
4.373 303.172796910278
4.374 303.170235892698
4.375 303.167676984546
4.376 303.165120184042
4.377 303.16256548941
4.378 303.160012898872
4.379 303.157462410655
4.38 303.154914022986
4.381 303.152367734093
4.382 303.149823542206
4.383 303.147281445557
4.384 303.14474144238
4.385 303.142203530909
4.386 303.13966770938
4.387 303.137133976031
4.388 303.134602329102
4.389 303.132072766833
4.39 303.129545287467
4.391 303.127019889247
4.392 303.12449657042
4.393 303.121975329233
4.394 303.119456163933
4.395 303.116939072771
4.396 303.114424053998
4.397 303.111911105868
4.398 303.109400226634
4.399 303.106891414555
4.4 303.104384667886
4.401 303.101879984887
4.402 303.099377363818
4.403 303.096876802943
4.404 303.094378300524
4.405 303.091881854827
4.406 303.089387464118
4.407 303.086895126666
4.408 303.08440484074
4.409 303.081916604612
4.41 303.079430416555
4.411 303.076946274842
4.412 303.074464177749
4.413 303.071984123553
4.414 303.069506110534
4.415 303.067030136971
4.416 303.064556201147
4.417 303.062084301344
4.418 303.059614435847
4.419 303.057146602942
4.42 303.054680800917
4.421 303.052217028061
4.422 303.049755282664
4.423 303.04729556302
4.424 303.044837867421
4.425 303.042382194162
4.426 303.039928541541
4.427 303.037476907855
4.428 303.035027291403
4.429 303.032579690487
4.43 303.03013410341
4.431 303.027690528474
4.432 303.025248963986
4.433 303.022809408252
4.434 303.020371859581
4.435 303.017936316283
4.436 303.015502776669
4.437 303.013071239051
4.438 303.010641701744
4.439 303.008214163064
4.44 303.005788621328
4.441 303.003365074854
4.442 303.000943521963
4.443 302.998523960976
4.444 302.996106390215
4.445 302.993690808007
4.446 302.991277212676
4.447 302.988865602549
4.448 302.986455975956
4.449 302.984048331228
4.45 302.981642666694
4.451 302.97923898069
4.452 302.976837271548
4.453 302.974437537606
4.454 302.972039777201
4.455 302.969643988672
4.456 302.967250170359
4.457 302.964858320604
4.458 302.962468437751
4.459 302.960080520143
4.46 302.957694566127
4.461 302.955310574051
4.462 302.952928542264
4.463 302.950548469116
4.464 302.948170352959
4.465 302.945794192146
4.466 302.943419985032
4.467 302.941047729973
4.468 302.938677425327
4.469 302.936309069453
4.47 302.93394266071
4.471 302.931578197462
4.472 302.929215678071
4.473 302.926855100901
4.474 302.92449646432
4.475 302.922139766694
4.476 302.919785006392
4.477 302.917432181786
4.478 302.915081291245
4.479 302.912732333145
4.48 302.910385305858
4.481 302.908040207762
4.482 302.905697037233
4.483 302.903355792651
4.484 302.901016472396
4.485 302.898679074848
4.486 302.896343598393
4.487 302.894010041412
4.488 302.891678402294
4.489 302.889348679424
4.49 302.887020871191
4.491 302.884694975986
4.492 302.882370992199
4.493 302.880048918225
4.494 302.877728752455
4.495 302.875410493288
4.496 302.873094139118
4.497 302.870779688345
4.498 302.868467139369
4.499 302.86615649059
4.5 302.863847740411
4.501 302.861540887236
4.502 302.859235929471
4.503 302.856932865521
4.504 302.854631693796
4.505 302.852332412704
4.506 302.850035020656
4.507 302.847739516064
4.508 302.845445897343
4.509 302.843154162906
4.51 302.840864311171
4.511 302.838576340555
4.512 302.836290249476
4.513 302.834006036356
4.514 302.831723699616
4.515 302.829443237679
4.516 302.82716464897
4.517 302.824887931915
4.518 302.822613084941
4.519 302.820340106476
4.52 302.81806899495
4.521 302.815799748796
4.522 302.813532366445
4.523 302.811266846331
4.524 302.809003186891
4.525 302.80674138656
4.526 302.804481443777
4.527 302.802223356981
4.528 302.799967124613
4.529 302.797712745115
4.53 302.795460216931
4.531 302.793209538506
4.532 302.790960708285
4.533 302.788713724717
4.534 302.78646858625
4.535 302.784225291335
4.536 302.781983838422
4.537 302.779744225965
4.538 302.777506452419
4.539 302.775270516238
4.54 302.773036415879
4.541 302.770804149802
4.542 302.768573716465
4.543 302.76634511433
4.544 302.764118341858
4.545 302.761893397513
4.546 302.759670279761
4.547 302.757448987067
4.548 302.755229517898
4.549 302.753011870725
4.55 302.750796044017
4.551 302.748582036246
4.552 302.746369845884
4.553 302.744159471405
4.554 302.741950911286
4.555 302.739744164003
4.556 302.737539228034
4.557 302.735336101859
4.558 302.733134783958
4.559 302.730935272813
4.56 302.728737566908
4.561 302.726541664728
4.562 302.724347564758
4.563 302.722155265486
4.564 302.719964765401
4.565 302.717776062993
4.566 302.715589156752
4.567 302.713404045171
4.568 302.711220726745
4.569 302.709039199968
4.57 302.706859463337
4.571 302.704681515349
4.572 302.702505354504
4.573 302.700330979302
4.574 302.698158388245
4.575 302.695987579835
4.576 302.693818552576
4.577 302.691651304975
4.578 302.689485835539
4.579 302.687322142774
4.58 302.685160225192
4.581 302.683000081301
4.582 302.680841709615
4.583 302.678685108647
4.584 302.676530276911
4.585 302.674377212922
4.586 302.672225915199
4.587 302.670076382259
4.588 302.667928612623
4.589 302.66578260481
4.59 302.663638357344
4.591 302.661495868748
4.592 302.659355137547
4.593 302.657216162266
4.594 302.655078941433
4.595 302.652943473578
4.596 302.650809757228
4.597 302.648677790916
4.598 302.646547573175
4.599 302.644419102537
4.6 302.642292377537
4.601 302.640167396713
4.602 302.638044158601
4.603 302.63592266174
4.604 302.633802904671
4.605 302.631684885934
4.606 302.629568604071
4.607 302.627454057628
4.608 302.625341245148
4.609 302.623230165179
4.61 302.621120816267
4.611 302.619013196961
4.612 302.616907305812
4.613 302.614803141371
4.614 302.61270070219
4.615 302.610599986823
4.616 302.608500993825
4.617 302.606403721753
4.618 302.604308169164
4.619 302.602214334617
4.62 302.600122216671
4.621 302.598031813889
4.622 302.595943124833
4.623 302.593856148066
4.624 302.591770882153
4.625 302.589687325662
4.626 302.587605477158
4.627 302.585525335211
4.628 302.583446898391
4.629 302.581370165269
4.63 302.579295134418
4.631 302.57722180441
4.632 302.575150173822
4.633 302.573080241229
4.634 302.571012005208
4.635 302.568945464338
4.636 302.566880617199
4.637 302.564817462371
4.638 302.562755998438
4.639 302.560696223982
4.64 302.558638137588
4.641 302.556581737843
4.642 302.554527023332
4.643 302.552473992646
4.644 302.550422644372
4.645 302.548372977102
4.646 302.546324989429
4.647 302.544278679944
4.648 302.542234047243
4.649 302.540191089921
4.65 302.538149806575
4.651 302.536110195803
4.652 302.534072256204
4.653 302.53203598638
4.654 302.530001384931
4.655 302.52796845046
4.656 302.525937181571
4.657 302.523907576871
4.658 302.521879634965
4.659 302.51985335446
4.66 302.517828733967
4.661 302.515805772094
4.662 302.513784467454
4.663 302.511764818659
4.664 302.509746824322
4.665 302.507730483058
4.666 302.505715793483
4.667 302.503702754215
4.668 302.501691363873
4.669 302.499681621075
4.67 302.497673524442
4.671 302.495667072597
4.672 302.493662264163
4.673 302.491659097764
4.674 302.489657572026
4.675 302.487657685576
4.676 302.485659437041
4.677 302.483662825051
4.678 302.481667848236
4.679 302.479674505227
4.68 302.477682794658
4.681 302.475692715161
4.682 302.473704265373
4.683 302.471717443929
4.684 302.469732249466
4.685 302.467748680624
4.686 302.465766736042
4.687 302.46378641436
4.688 302.461807714222
4.689 302.459830634269
4.69 302.457855173147
4.691 302.455881329501
4.692 302.453909101978
4.693 302.451938489226
4.694 302.449969489893
4.695 302.448002102631
4.696 302.44603632609
4.697 302.444072158923
4.698 302.442109599784
4.699 302.440148647327
4.7 302.438189300209
4.701 302.436231557087
4.702 302.434275416619
4.703 302.432320877465
4.704 302.430367938285
4.705 302.428416597742
4.706 302.426466854497
4.707 302.424518707216
4.708 302.422572154564
4.709 302.420627195206
4.71 302.418683827811
4.711 302.416742051047
4.712 302.414801863584
4.713 302.412863264093
4.714 302.410926251246
4.715 302.408990823717
4.716 302.40705698018
4.717 302.40512471931
4.718 302.403194039784
4.719 302.40126494028
4.72 302.399337419478
4.721 302.397411476056
4.722 302.395487108697
4.723 302.393564316083
4.724 302.391643096897
4.725 302.389723449824
4.726 302.38780537355
4.727 302.385888866761
4.728 302.383973928147
4.729 302.382060556395
4.73 302.380148750196
4.731 302.378238508242
4.732 302.376329829225
4.733 302.37442271184
4.734 302.37251715478
4.735 302.370613156741
4.736 302.368710716421
4.737 302.366809832519
4.738 302.364910503732
4.739 302.363012728762
4.74 302.36111650631
4.741 302.359221835079
4.742 302.357328713773
4.743 302.355437141096
4.744 302.353547115754
4.745 302.351658636454
4.746 302.349771701906
4.747 302.347886310817
4.748 302.346002461899
4.749 302.344120153862
4.75 302.342239385421
4.751 302.340360155287
4.752 302.338482462178
4.753 302.336606304807
4.754 302.334731681893
4.755 302.332858592153
4.756 302.330987034307
4.757 302.329117007075
4.758 302.327248509178
4.759 302.32538153934
4.76 302.323516096283
4.761 302.321652178734
4.762 302.319789785416
4.763 302.317928915058
4.764 302.316069566388
4.765 302.314211738134
4.766 302.312355429027
4.767 302.310500637798
4.768 302.30864736318
4.769 302.306795603906
4.77 302.304945358711
4.771 302.30309662633
4.772 302.3012494055
4.773 302.299403694959
4.774 302.297559493446
4.775 302.295716799701
4.776 302.293875612465
4.777 302.292035930481
4.778 302.290197752491
4.779 302.28836107724
4.78 302.286525903473
4.781 302.284692229937
4.782 302.28286005538
4.783 302.281029378549
4.784 302.279200198196
4.785 302.27737251307
4.786 302.275546321923
4.787 302.273721623509
4.788 302.271898416581
4.789 302.270076699895
4.79 302.268256472207
4.791 302.266437732273
4.792 302.264620478853
4.793 302.262804710705
4.794 302.260990426591
4.795 302.25917762527
4.796 302.257366305507
4.797 302.255556466065
4.798 302.253748105708
4.799 302.251941223202
4.8 302.250135817315
4.801 302.248331886813
4.802 302.246529430465
4.803 302.244728447043
4.804 302.242928935315
4.805 302.241130894056
4.806 302.239334322038
4.807 302.237539218034
4.808 302.235745580822
4.809 302.233953409175
4.81 302.232162701873
4.811 302.230373457693
4.812 302.228585675415
4.813 302.226799353819
4.814 302.225014491687
4.815 302.223231087801
4.816 302.221449140945
4.817 302.219668649904
4.818 302.217889613463
4.819 302.216112030409
4.82 302.21433589953
4.821 302.212561219615
4.822 302.210787989454
4.823 302.209016207837
4.824 302.207245873557
4.825 302.205476985406
4.826 302.203709542179
4.827 302.20194354267
4.828 302.200178985676
4.829 302.198415869993
4.83 302.19665419442
4.831 302.194893957757
4.832 302.193135158803
4.833 302.191377796359
4.834 302.189621869228
4.835 302.187867376213
4.836 302.186114316119
4.837 302.18436268775
4.838 302.182612489914
4.839 302.180863721417
4.84 302.179116381068
4.841 302.177370467676
4.842 302.175625980053
4.843 302.173882917008
4.844 302.172141277356
4.845 302.170401059909
4.846 302.168662263482
4.847 302.16692488689
4.848 302.16518892895
4.849 302.16345438848
4.85 302.161721264297
4.851 302.159989555223
4.852 302.158259260077
4.853 302.156530377681
4.854 302.154802906858
4.855 302.153076846431
4.856 302.151352195225
4.857 302.149628952066
4.858 302.14790711578
4.859 302.146186685195
4.86 302.144467659139
4.861 302.142750036444
4.862 302.141033815938
4.863 302.139318996454
4.864 302.137605576824
4.865 302.135893555883
4.866 302.134182932464
4.867 302.132473705404
4.868 302.130765873539
4.869 302.129059435707
4.87 302.127354390746
4.871 302.125650737497
4.872 302.123948474799
4.873 302.122247601495
4.874 302.120548116428
4.875 302.11885001844
4.876 302.117153306377
4.877 302.115457979083
4.878 302.113764035407
4.879 302.112071474195
4.88 302.110380294296
4.881 302.108690494559
4.882 302.107002073836
4.883 302.105315030977
4.884 302.103629364835
4.885 302.101945074264
4.886 302.100262158118
4.887 302.098580615253
4.888 302.096900444525
4.889 302.095221644791
4.89 302.09354421491
4.891 302.091868153741
4.892 302.090193460146
4.893 302.088520132984
4.894 302.086848171118
4.895 302.085177573412
4.896 302.08350833873
4.897 302.081840465937
4.898 302.0801739539
4.899 302.078508801485
4.9 302.076845007561
4.901 302.075182570996
4.902 302.073521490661
4.903 302.071861765428
4.904 302.070203394167
4.905 302.068546375752
4.906 302.066890709057
4.907 302.065236392956
4.908 302.063583426327
4.909 302.061931808044
4.91 302.060281536987
4.911 302.058632612034
4.912 302.056985032065
4.913 302.05533879596
4.914 302.053693902601
4.915 302.052050350871
4.916 302.050408139653
4.917 302.048767267832
4.918 302.047127734293
4.919 302.045489537923
4.92 302.043852677608
4.921 302.042217152238
4.922 302.040582960701
4.923 302.038950101888
4.924 302.03731857469
4.925 302.035688377999
4.926 302.034059510707
4.927 302.03243197171
4.928 302.030805759902
4.929 302.029180874179
4.93 302.027557313438
4.931 302.025935076576
4.932 302.024314162493
4.933 302.022694570087
4.934 302.02107629826
4.935 302.019459345914
4.936 302.017843711949
4.937 302.016229395271
4.938 302.014616394783
4.939 302.013004709391
4.94 302.011394338001
4.941 302.00978527952
4.942 302.008177532856
4.943 302.006571096919
4.944 302.004965970618
4.945 302.003362152864
4.946 302.001759642569
4.947 302.000158438645
4.948 301.998558540008
4.949 301.99695994557
4.95 301.995362654248
4.951 301.993766664958
4.952 301.992171976618
4.953 301.990578588146
4.954 301.988986498461
4.955 301.987395706483
4.956 301.985806211133
4.957 301.984218011334
4.958 301.982631106008
4.959 301.98104549408
4.96 301.979461174473
4.961 301.977878146115
4.962 301.976296407931
4.963 301.974715958848
4.964 301.973136797796
4.965 301.971558923704
4.966 301.969982335502
4.967 301.968407032122
4.968 301.966833012494
4.969 301.965260275553
4.97 301.963688820233
4.971 301.962118645467
4.972 301.960549750193
4.973 301.958982133346
4.974 301.957415793865
4.975 301.955850730687
4.976 301.954286942752
4.977 301.952724429001
4.978 301.951163188374
4.979 301.949603219814
4.98 301.948044522264
4.981 301.946487094667
4.982 301.944930935969
4.983 301.943376045115
4.984 301.941822421051
4.985 301.940270062726
4.986 301.938718969088
4.987 301.937169139086
4.988 301.93562057167
4.989 301.934073265791
4.99 301.932527220402
4.991 301.930982434455
4.992 301.929438906904
4.993 301.927896636704
4.994 301.92635562281
4.995 301.924815864179
4.996 301.923277359767
4.997 301.921740108535
4.998 301.92020410944
4.999 301.918669361442
5 301.917135863503
5.001 301.915603614584
5.002 301.914072613649
5.003 301.912542859659
5.004 301.911014351581
5.005 301.909487088379
5.006 301.90796106902
5.007 301.90643629247
5.008 301.904912757698
5.009 301.903390463673
5.01 301.901869409364
5.011 301.900349593742
5.012 301.898831015778
5.013 301.897313674446
5.014 301.895797568717
5.015 301.894282697567
5.016 301.892769059969
5.017 301.891256654901
5.018 301.889745481339
5.019 301.888235538261
5.02 301.886726824644
5.021 301.885219339469
5.022 301.883713081716
5.023 301.882208050365
5.024 301.8807042444
5.025 301.879201662802
5.026 301.877700304555
5.027 301.876200168644
5.028 301.874701254055
5.029 301.873203559773
5.03 301.871707084787
5.031 301.870211828083
5.032 301.868717788651
5.033 301.867224965481
5.034 301.865733357564
5.035 301.86424296389
5.036 301.862753783452
5.037 301.861265815244
5.038 301.859779058259
5.039 301.858293511492
5.04 301.85680917394
5.041 301.855326044598
5.042 301.853844122465
5.043 301.852363406539
5.044 301.850883895818
5.045 301.849405589304
5.046 301.847928485996
5.047 301.846452584898
5.048 301.84497788501
5.049 301.843504385338
5.05 301.842032084884
5.051 301.840560982655
5.052 301.839091077656
5.053 301.837622368895
5.054 301.836154855378
5.055 301.834688536115
5.056 301.833223410114
5.057 301.831759476387
5.058 301.830296733944
5.059 301.828835181797
5.06 301.827374818959
5.061 301.825915644443
5.062 301.824457657264
5.063 301.823000856438
5.064 301.82154524098
5.065 301.820090809907
5.066 301.818637562238
5.067 301.817185496991
5.068 301.815734613185
5.069 301.81428490984
5.07 301.812836385979
5.071 301.811389040622
5.072 301.809942872793
5.073 301.808497881515
5.074 301.807054065813
5.075 301.805611424713
5.076 301.804169957239
5.077 301.80272966242
5.078 301.801290539282
5.079 301.799852586856
5.08 301.798415804169
5.081 301.796980190253
5.082 301.795545744138
5.083 301.794112464857
5.084 301.792680351442
5.085 301.791249402927
5.086 301.789819618346
5.087 301.788390996734
5.088 301.786963537128
5.089 301.785537238564
5.09 301.784112100081
5.091 301.782688120716
5.092 301.781265299509
5.093 301.779843635501
5.094 301.778423127731
5.095 301.777003775242
5.096 301.775585577077
5.097 301.774168532279
5.098 301.772752639892
5.099 301.77133789896
5.1 301.769924308532
5.101 301.768511867651
5.102 301.767100575367
5.103 301.765690430728
5.104 301.764281432782
5.105 301.76287358058
5.106 301.761466873172
5.107 301.76006130961
5.108 301.758656888946
5.109 301.757253610233
5.11 301.755851472525
5.111 301.754450474878
5.112 301.753050616346
5.113 301.751651895986
5.114 301.750254312854
5.115 301.74885786601
5.116 301.747462554512
5.117 301.746068377419
5.118 301.744675333792
5.119 301.743283422691
5.12 301.74189264318
5.121 301.74050299432
5.122 301.739114475176
5.123 301.737727084811
5.124 301.736340822291
5.125 301.734955686682
5.126 301.73357167705
5.127 301.732188792463
5.128 301.73080703199
5.129 301.729426394699
5.13 301.72804687966
5.131 301.726668485945
5.132 301.725291212624
5.133 301.723915058771
5.134 301.722540023457
5.135 301.721166105758
5.136 301.719793304746
5.137 301.718421619499
5.138 301.717051049092
5.139 301.715681592602
5.14 301.714313249107
5.141 301.712946017686
5.142 301.711579897418
5.143 301.710214887383
5.144 301.708850986662
5.145 301.707488194336
5.146 301.706126509489
5.147 301.704765931204
5.148 301.703406458564
5.149 301.702048090655
5.15 301.700690826561
5.151 301.69933466537
5.152 301.697979606169
5.153 301.696625648045
5.154 301.695272790088
5.155 301.693921031386
5.156 301.692570371031
5.157 301.691220808112
5.158 301.689872341723
5.159 301.688524970955
5.16 301.687178694901
5.161 301.685833512657
5.162 301.684489423317
5.163 301.683146425977
5.164 301.681804519732
5.165 301.68046370368
5.166 301.67912397692
5.167 301.677785338549
5.168 301.676447787668
5.169 301.675111323377
5.17 301.673775944776
5.171 301.672441650968
5.172 301.671108441055
5.173 301.66977631414
5.174 301.668445269328
5.175 301.667115305723
5.176 301.665786422431
5.177 301.664458618558
5.178 301.663131893212
5.179 301.6618062455
5.18 301.660481674531
5.181 301.659158179414
5.182 301.65783575926
5.183 301.656514413179
5.184 301.655194140284
5.185 301.653874939686
5.186 301.652556810498
5.187 301.651239751835
5.188 301.649923762812
5.189 301.648608842543
5.19 301.647294990145
5.191 301.645982204735
5.192 301.64467048543
5.193 301.64335983135
5.194 301.642050241613
5.195 301.640741715339
5.196 301.639434251648
5.197 301.638127849663
5.198 301.636822508506
5.199 301.635518227298
5.2 301.634215005165
5.201 301.63291284123
5.202 301.631611734619
5.203 301.630311684457
5.204 301.629012689871
5.205 301.627714749989
5.206 301.626417863938
5.207 301.625122030848
5.208 301.623827249848
5.209 301.622533520068
5.21 301.621240840639
5.211 301.619949210694
5.212 301.618658629364
5.213 301.617369095784
5.214 301.616080609086
5.215 301.614793168407
5.216 301.613506772881
5.217 301.612221421644
5.218 301.610937113834
5.219 301.609653848589
5.22 301.608371625046
5.221 301.607090442344
5.222 301.605810299625
5.223 301.604531196027
5.224 301.603253130694
5.225 301.601976102766
5.226 301.600700111387
5.227 301.599425155699
5.228 301.598151234848
5.229 301.596878347978
5.23 301.595606494236
5.231 301.594335672766
5.232 301.593065882718
5.233 301.591797123238
5.234 301.590529393475
5.235 301.589262692578
5.236 301.587997019697
5.237 301.586732373984
5.238 301.585468754589
5.239 301.584206160666
5.24 301.582944591366
5.241 301.581684045843
5.242 301.580424523253
5.243 301.579166022749
5.244 301.577908543488
5.245 301.576652084625
5.246 301.57539664532
5.247 301.574142224728
5.248 301.57288882201
5.249 301.571636436324
5.25 301.570385066831
5.251 301.569134712691
5.252 301.567885373066
5.253 301.566637047117
5.254 301.565389734009
5.255 301.564143432904
5.256 301.562898142967
5.257 301.561653863363
5.258 301.560410593257
5.259 301.559168331818
5.26 301.55792707821
5.261 301.556686831603
5.262 301.555447591165
5.263 301.554209356065
5.264 301.552972125473
5.265 301.55173589856
5.266 301.550500674497
5.267 301.549266452457
5.268 301.548033231612
5.269 301.546801011135
5.27 301.545569790202
5.271 301.544339567986
5.272 301.543110343664
5.273 301.541882116412
5.274 301.540654885407
5.275 301.539428649826
5.276 301.538203408848
5.277 301.536979161652
5.278 301.535755907417
5.279 301.534533645325
5.28 301.533312374557
5.281 301.532092094294
5.282 301.530872803719
5.283 301.529654502015
5.284 301.528437188366
5.285 301.527220861957
5.286 301.526005521973
5.287 301.524791167601
5.288 301.523577798027
5.289 301.522365412437
5.29 301.521154010021
5.291 301.519943589968
5.292 301.518734151466
5.293 301.517525693706
5.294 301.516318215878
5.295 301.515111717175
5.296 301.513906196788
5.297 301.51270165391
5.298 301.511498087736
5.299 301.510295497458
5.3 301.509093882273
5.301 301.507893241375
5.302 301.506693573961
5.303 301.505494879229
5.304 301.504297156375
5.305 301.503100404598
5.306 301.501904623097
5.307 301.500709811072
5.308 301.499515967723
5.309 301.498323092252
5.31 301.497131183859
5.311 301.495940241748
5.312 301.494750265122
5.313 301.493561253184
5.314 301.492373205139
5.315 301.491186120191
5.316 301.489999997547
5.317 301.488814836413
5.318 301.487630635997
5.319 301.486447395505
5.32 301.485265114147
5.321 301.484083791132
5.322 301.482903425669
5.323 301.481724016969
5.324 301.480545564243
5.325 301.479368066704
5.326 301.478191523562
5.327 301.477015934032
5.328 301.475841297327
5.329 301.474667612662
5.33 301.473494879253
5.331 301.472323096314
5.332 301.471152263062
5.333 301.469982378715
5.334 301.46881344249
5.335 301.467645453605
5.336 301.466478411281
5.337 301.465312314736
5.338 301.464147163191
5.339 301.462982955867
5.34 301.461819691987
5.341 301.460657370771
5.342 301.459495991444
5.343 301.45833555323
5.344 301.457176055352
5.345 301.456017497035
5.346 301.454859877506
5.347 301.453703195991
5.348 301.452547451716
5.349 301.451392643909
5.35 301.450238771799
5.351 301.449085834615
5.352 301.447933831586
5.353 301.446782761943
5.354 301.445632624916
5.355 301.444483419737
5.356 301.443335145639
5.357 301.442187801854
5.358 301.441041387616
5.359 301.439895902158
5.36 301.438751344717
5.361 301.437607714527
5.362 301.436465010824
5.363 301.435323232845
5.364 301.434182379828
5.365 301.43304245101
5.366 301.431903445631
5.367 301.430765362929
5.368 301.429628202145
5.369 301.428491962519
5.37 301.427356643293
5.371 301.426222243708
5.372 301.425088763008
5.373 301.423956200434
5.374 301.422824555231
5.375 301.421693826645
5.376 301.420564013918
5.377 301.419435116298
5.378 301.418307133031
5.379 301.417180063364
5.38 301.416053906544
5.381 301.41492866182
5.382 301.413804328441
5.383 301.412680905656
5.384 301.411558392716
5.385 301.410436788871
5.386 301.409316093373
5.387 301.408196305474
5.388 301.407077424426
5.389 301.405959449483
5.39 301.4048423799
5.391 301.40372621493
5.392 301.402610953829
5.393 301.401496595854
5.394 301.400383140259
5.395 301.399270586303
5.396 301.398158933244
5.397 301.397048180339
5.398 301.395938326847
5.399 301.394829372029
5.4 301.393721315144
5.401 301.392614155454
5.402 301.391507892221
5.403 301.390402524705
5.404 301.38929805217
5.405 301.38819447388
5.406 301.387091789098
5.407 301.385989997089
5.408 301.384889097118
5.409 301.383789088451
5.41 301.382689970356
5.411 301.381591742098
5.412 301.380494402945
5.413 301.379397952167
5.414 301.378302389032
5.415 301.377207712809
5.416 301.376113922769
5.417 301.375021018183
5.418 301.373928998321
5.419 301.372837862457
5.42 301.371747609863
5.421 301.370658239812
5.422 301.369569751577
5.423 301.368482144434
5.424 301.367395417658
5.425 301.366309570525
5.426 301.36522460231
5.427 301.364140512291
5.428 301.363057299745
5.429 301.361974963951
5.43 301.360893504186
5.431 301.359812919732
5.432 301.358733209867
5.433 301.357654373873
5.434 301.35657641103
5.435 301.35549932062
5.436 301.354423101927
5.437 301.353347754232
5.438 301.35227327682
5.439 301.351199668974
5.44 301.350126929981
5.441 301.349055059124
5.442 301.347984055691
5.443 301.346913918967
5.444 301.345844648241
5.445 301.3447762428
5.446 301.343708701932
5.447 301.342642024927
5.448 301.341576211074
5.449 301.340511259664
5.45 301.339447169987
5.451 301.338383941335
5.452 301.337321573001
5.453 301.336260064276
5.454 301.335199414453
5.455 301.334139622828
5.456 301.333080688694
5.457 301.332022611346
5.458 301.330965390081
5.459 301.329909024193
5.46 301.32885351298
5.461 301.32779885574
5.462 301.326745051771
5.463 301.325692100371
5.464 301.324640000839
5.465 301.323588752475
5.466 301.32253835458
5.467 301.321488806454
5.468 301.3204401074
5.469 301.319392256718
5.47 301.318345253713
5.471 301.317299097687
5.472 301.316253787945
5.473 301.31520932379
5.474 301.314165704528
5.475 301.313122929465
5.476 301.312080997907
5.477 301.31103990916
5.478 301.309999662532
5.479 301.308960257332
5.48 301.307921692867
5.481 301.306883968448
5.482 301.305847083383
5.483 301.304811036983
5.484 301.30377582856
5.485 301.302741457424
5.486 301.301707922887
5.487 301.300675224263
5.488 301.299643360865
5.489 301.298612332006
5.49 301.297582137001
5.491 301.296552775164
5.492 301.295524245813
5.493 301.294496548261
5.494 301.293469681828
5.495 301.292443645828
5.496 301.291418439582
5.497 301.290394062406
5.498 301.28937051362
5.499 301.288347792544
5.5 301.287325898497
5.501 301.286304830801
5.502 301.285284588777
5.503 301.284265171747
5.504 301.283246579033
5.505 301.282228809958
5.506 301.281211863845
5.507 301.28019574002
5.508 301.279180437807
5.509 301.278165956531
5.51 301.277152295518
5.511 301.276139454094
5.512 301.275127431587
5.513 301.274116227324
5.514 301.273105840633
5.515 301.272096270843
5.516 301.271087517283
5.517 301.270079579284
5.518 301.269072456175
5.519 301.268066147287
5.52 301.267060651952
5.521 301.266055969503
5.522 301.265052099271
5.523 301.26404904059
5.524 301.263046792794
5.525 301.262045355218
5.526 301.261044727195
5.527 301.260044908062
5.528 301.259045897155
5.529 301.258047693809
5.53 301.257050297363
5.531 301.256053707155
5.532 301.255057922521
5.533 301.254062942802
5.534 301.253068767336
5.535 301.252075395464
5.536 301.251082826525
5.537 301.250091059862
5.538 301.249100094815
5.539 301.248109930728
5.54 301.247120566942
5.541 301.2461320028
5.542 301.245144237648
5.543 301.244157270828
5.544 301.243171101686
5.545 301.242185729568
5.546 301.241201153819
5.547 301.240217373786
5.548 301.239234388817
5.549 301.238252198258
5.55 301.237270801459
5.551 301.236290197767
5.552 301.235310386532
5.553 301.234331367105
5.554 301.233353138834
5.555 301.232375701073
5.556 301.231399053171
5.557 301.230423194482
5.558 301.229448124357
5.559 301.228473842151
5.56 301.227500347215
5.561 301.226527638906
5.562 301.225555716577
5.563 301.224584579584
5.564 301.223614227283
5.565 301.22264465903
5.566 301.221675874182
5.567 301.220707872096
5.568 301.219740652132
5.569 301.218774213646
5.57 301.217808555999
5.571 301.21684367855
5.572 301.21587958066
5.573 301.214916261688
5.574 301.213953720997
5.575 301.212991957947
5.576 301.212030971903
5.577 301.211070762225
5.578 301.210111328278
5.579 301.209152669426
5.58 301.208194785033
5.581 301.207237674464
5.582 301.206281337085
5.583 301.205325772262
5.584 301.204370979361
5.585 301.20341695775
5.586 301.202463706796
5.587 301.201511225868
5.588 301.200559514335
5.589 301.199608571565
5.59 301.198658396928
5.591 301.197708989795
5.592 301.196760349537
5.593 301.195812475525
5.594 301.194865367131
5.595 301.193919023727
5.596 301.192973444687
5.597 301.192028629385
5.598 301.191084577193
5.599 301.190141287487
5.6 301.189198759641
5.601 301.188256993032
5.602 301.187315987036
5.603 301.186375741029
5.604 301.185436254388
5.605 301.184497526492
5.606 301.183559556717
5.607 301.182622344444
5.608 301.181685889052
5.609 301.18075018992
5.61 301.179815246428
5.611 301.178881057958
5.612 301.177947623891
5.613 301.177014943608
5.614 301.176083016493
5.615 301.175151841928
5.616 301.174221419297
5.617 301.173291747983
5.618 301.172362827372
5.619 301.171434656847
5.62 301.170507235795
5.621 301.169580563602
5.622 301.168654639654
5.623 301.167729463338
5.624 301.166805034043
5.625 301.165881351155
5.626 301.164958414064
5.627 301.164036222158
5.628 301.163114774828
5.629 301.162194071464
5.63 301.161274111456
5.631 301.160354894195
5.632 301.159436419074
5.633 301.158518685484
5.634 301.157601692818
5.635 301.156685440469
5.636 301.155769927831
5.637 301.154855154298
5.638 301.153941119266
5.639 301.153027822129
5.64 301.152115262283
5.641 301.151203439125
5.642 301.150292352051
5.643 301.149382000458
5.644 301.148472383745
5.645 301.147563501309
5.646 301.146655352549
5.647 301.145747936866
5.648 301.144841253658
5.649 301.143935302326
5.65 301.14303008227
5.651 301.142125592893
5.652 301.141221833595
5.653 301.14031880378
5.654 301.13941650285
5.655 301.138514930209
5.656 301.13761408526
5.657 301.136713967407
5.658 301.135814576056
5.659 301.134915910612
5.66 301.134017970481
5.661 301.133120755069
5.662 301.132224263783
5.663 301.13132849603
5.664 301.130433451218
5.665 301.129539128755
5.666 301.128645528051
5.667 301.127752648515
5.668 301.126860489556
5.669 301.125969050585
5.67 301.125078331013
5.671 301.12418833025
5.672 301.12329904771
5.673 301.122410482803
5.674 301.121522634944
5.675 301.120635503544
5.676 301.119749088019
5.677 301.118863387783
5.678 301.117978402249
5.679 301.117094130833
5.68 301.116210572952
5.681 301.115327728021
5.682 301.114445595456
5.683 301.113564174676
5.684 301.112683465098
5.685 301.11180346614
5.686 301.110924177221
5.687 301.11004559776
5.688 301.109167727177
5.689 301.108290564891
5.69 301.107414110324
5.691 301.106538362897
5.692 301.105663322031
5.693 301.104788987148
5.694 301.103915357672
5.695 301.103042433024
5.696 301.102170212628
5.697 301.10129869591
5.698 301.100427882292
5.699 301.099557771201
5.7 301.098688362061
5.701 301.097819654299
5.702 301.09695164734
5.703 301.096084340613
5.704 301.095217733544
5.705 301.094351825561
5.706 301.093486616093
5.707 301.092622104568
5.708 301.091758290416
5.709 301.090895173067
5.71 301.09003275195
5.711 301.089171026498
5.712 301.08830999614
5.713 301.087449660309
5.714 301.086590018437
5.715 301.085731069956
5.716 301.0848728143
5.717 301.084015250902
5.718 301.083158379196
5.719 301.082302198618
5.72 301.081446708601
5.721 301.080591908582
5.722 301.079737797997
5.723 301.078884376282
5.724 301.078031642874
5.725 301.07717959721
5.726 301.076328238728
5.727 301.075477566867
5.728 301.074627581066
5.729 301.073778280763
5.73 301.072929665399
5.731 301.072081734414
5.732 301.071234487248
5.733 301.070387923344
5.734 301.069542042141
5.735 301.068696843083
5.736 301.067852325612
5.737 301.067008489172
5.738 301.066165333205
5.739 301.065322857156
5.74 301.064481060468
5.741 301.063639942588
5.742 301.06279950296
5.743 301.061959741031
5.744 301.061120656246
5.745 301.060282248052
5.746 301.059444515897
5.747 301.058607459229
5.748 301.057771077495
5.749 301.056935370144
5.75 301.056100336625
5.751 301.055265976388
5.752 301.054432288883
5.753 301.05359927356
5.754 301.05276692987
5.755 301.051935257264
5.756 301.051104255196
5.757 301.050273923116
5.758 301.049444260478
5.759 301.048615266734
5.76 301.04778694134
5.761 301.046959283748
5.762 301.046132293413
5.763 301.04530596979
5.764 301.044480312336
5.765 301.043655320505
5.766 301.042830993755
5.767 301.042007331542
5.768 301.041184333324
5.769 301.040361998558
5.77 301.039540326703
5.771 301.038719317217
5.772 301.03789896956
5.773 301.037079283191
5.774 301.036260257571
5.775 301.035441892159
5.776 301.034624186417
5.777 301.033807139806
5.778 301.032990751788
5.779 301.032175021825
5.78 301.031359949381
5.781 301.030545533918
5.782 301.029731774901
5.783 301.028918671792
5.784 301.028106224058
5.785 301.027294431162
5.786 301.026483292571
5.787 301.02567280775
5.788 301.024862976165
5.789 301.024053797284
5.79 301.023245270574
5.791 301.022437395502
5.792 301.021630171537
5.793 301.020823598147
5.794 301.020017674802
5.795 301.01921240097
5.796 301.018407776122
5.797 301.017603799728
5.798 301.01680047126
5.799 301.015997790187
5.8 301.015195755983
5.801 301.014394368119
5.802 301.013593626067
5.803 301.012793529301
5.804 301.011994077294
5.805 301.01119526952
5.806 301.010397105454
5.807 301.00959958457
5.808 301.008802706344
5.809 301.008006470251
5.81 301.007210875767
5.811 301.006415922369
5.812 301.005621609535
5.813 301.00482793674
5.814 301.004034903464
5.815 301.003242509184
5.816 301.00245075338
5.817 301.00165963553
5.818 301.000869155114
5.819 301.000079311613
5.82 300.999290104506
5.821 300.998501533275
5.822 300.9977135974
5.823 300.996926296365
5.824 300.99613962965
5.825 300.99535359674
5.826 300.994568197116
5.827 300.993783430262
5.828 300.992999295663
5.829 300.992215792802
5.83 300.991432921165
5.831 300.990650680237
5.832 300.989869069503
5.833 300.98908808845
5.834 300.988307736563
5.835 300.987528013331
5.836 300.986748918241
5.837 300.98597045078
5.838 300.985192610437
5.839 300.9844153967
5.84 300.983638809058
5.841 300.982862847002
5.842 300.98208751002
5.843 300.981312797605
5.844 300.980538709245
5.845 300.979765244433
5.846 300.97899240266
5.847 300.978220183418
5.848 300.9774485862
5.849 300.976677610499
5.85 300.975907255809
5.851 300.975137521622
5.852 300.974368407433
5.853 300.973599912738
5.854 300.97283203703
5.855 300.972064779806
5.856 300.971298140561
5.857 300.970532118792
5.858 300.969766713996
5.859 300.969001925669
5.86 300.968237753309
5.861 300.967474196415
5.862 300.966711254484
5.863 300.965948927016
5.864 300.96518721351
5.865 300.964426113466
5.866 300.963665626383
5.867 300.962905751763
5.868 300.962146489106
5.869 300.961387837913
5.87 300.960629797687
5.871 300.95987236793
5.872 300.959115548144
5.873 300.958359337832
5.874 300.957603736498
5.875 300.956848743646
5.876 300.95609435878
5.877 300.955340581404
5.878 300.954587411024
5.879 300.953834847145
5.88 300.953082889274
5.881 300.952331536916
5.882 300.951580789579
5.883 300.950830646769
5.884 300.950081107994
5.885 300.949332172763
5.886 300.948583840583
5.887 300.947836110964
5.888 300.947088983414
5.889 300.946342457444
5.89 300.945596532563
5.891 300.944851208282
5.892 300.944106484111
5.893 300.943362359563
5.894 300.942618834149
5.895 300.94187590738
5.896 300.94113357877
5.897 300.940391847831
5.898 300.939650714077
5.899 300.938910177022
5.9 300.938170236179
5.901 300.937430891062
5.902 300.936692141188
5.903 300.935953986072
5.904 300.935216425228
5.905 300.934479458174
5.906 300.933743084425
5.907 300.9330073035
5.908 300.932272114914
5.909 300.931537518187
5.91 300.930803512837
5.911 300.930070098381
5.912 300.929337274339
5.913 300.92860504023
5.914 300.927873395575
5.915 300.927142339893
5.916 300.926411872704
5.917 300.925681993531
5.918 300.924952701894
5.919 300.924223997315
5.92 300.923495879316
5.921 300.92276834742
5.922 300.92204140115
5.923 300.92131504003
5.924 300.920589263582
5.925 300.919864071332
5.926 300.919139462804
5.927 300.918415437522
5.928 300.917691995013
5.929 300.916969134802
5.93 300.916246856415
5.931 300.915525159379
5.932 300.914804043221
5.933 300.914083507468
5.934 300.913363551648
5.935 300.912644175289
5.936 300.91192537792
5.937 300.911207159069
5.938 300.910489518267
5.939 300.909772455042
5.94 300.909055968925
5.941 300.908340059446
5.942 300.907624726137
5.943 300.906909968528
5.944 300.906195786151
5.945 300.905482178539
5.946 300.904769145223
5.947 300.904056685737
5.948 300.903344799615
5.949 300.902633486388
5.95 300.901922745593
5.951 300.901212576762
5.952 300.900502979431
5.953 300.899793953135
5.954 300.89908549741
5.955 300.898377611791
5.956 300.897670295815
5.957 300.896963549019
5.958 300.89625737094
5.959 300.895551761115
5.96 300.894846719083
5.961 300.894142244381
5.962 300.893438336548
5.963 300.892734995124
5.964 300.892032219648
5.965 300.891330009659
5.966 300.890628364699
5.967 300.889927284306
5.968 300.889226768024
5.969 300.888526815392
5.97 300.887827425953
5.971 300.887128599248
5.972 300.886430334821
5.973 300.885732632214
5.974 300.88503549097
5.975 300.884338910634
5.976 300.883642890749
5.977 300.88294743086
5.978 300.882252530511
5.979 300.881558189247
5.98 300.880864406615
5.981 300.880171182161
5.982 300.879478515429
5.983 300.878786405968
5.984 300.878094853324
5.985 300.877403857045
5.986 300.876713416678
5.987 300.876023531773
5.988 300.875334201877
5.989 300.874645426539
5.99 300.873957205309
5.991 300.873269537736
5.992 300.872582423371
5.993 300.871895861765
5.994 300.871209852467
5.995 300.870524395029
5.996 300.869839489003
5.997 300.86915513394
5.998 300.868471329394
5.999 300.867788074916
6 300.867105370061
6.001 300.86642321438
6.002 300.86574160743
6.003 300.865060548762
6.004 300.864380037933
6.005 300.863700074496
6.006 300.863020658008
6.007 300.862341788024
6.008 300.8616634641
6.009 300.860985685793
6.01 300.860308452659
6.011 300.859631764255
6.012 300.85895562014
6.013 300.85828001987
6.014 300.857604963005
6.015 300.856930449102
6.016 300.856256477722
6.017 300.855583048423
6.018 300.854910160765
6.019 300.854237814308
6.02 300.853566008613
6.021 300.852894743241
6.022 300.852224017753
6.023 300.85155383171
6.024 300.850884184674
6.025 300.850215076209
6.026 300.849546505875
6.027 300.848878473238
6.028 300.848210977859
6.029 300.847544019303
6.03 300.846877597133
6.031 300.846211710915
6.032 300.845546360214
6.033 300.844881544594
6.034 300.844217263621
6.035 300.843553516861
6.036 300.84289030388
6.037 300.842227624246
6.038 300.841565477525
6.039 300.840903863284
6.04 300.840242781091
6.041 300.839582230515
6.042 300.838922211124
6.043 300.838262722487
6.044 300.837603764173
6.045 300.836945335751
6.046 300.836287436792
6.047 300.835630066866
6.048 300.834973225543
6.049 300.834316912395
6.05 300.833661126993
6.051 300.833005868908
6.052 300.832351137713
6.053 300.831696932979
6.054 300.831043254281
6.055 300.830390101191
6.056 300.829737473282
6.057 300.829085370129
6.058 300.828433791305
6.059 300.827782736385
6.06 300.827132204944
6.061 300.826482196558
6.062 300.825832710802
6.063 300.825183747252
6.064 300.824535305484
6.065 300.823887385075
6.066 300.823239985602
6.067 300.822593106642
6.068 300.821946747774
6.069 300.821300908575
6.07 300.820655588624
6.071 300.820010787499
6.072 300.81936650478
6.073 300.818722740047
6.074 300.818079492878
6.075 300.817436762855
6.076 300.816794549557
6.077 300.816152852566
6.078 300.815511671464
6.079 300.814871005831
6.08 300.814230855249
6.081 300.813591219302
6.082 300.812952097571
6.083 300.81231348964
6.084 300.811675395091
6.085 300.811037813509
6.086 300.810400744478
6.087 300.809764187581
6.088 300.809128142405
6.089 300.808492608533
6.09 300.807857585551
6.091 300.807223073046
6.092 300.806589070603
6.093 300.805955577808
6.094 300.805322594248
6.095 300.804690119511
6.096 300.804058153185
6.097 300.803426694856
6.098 300.802795744113
6.099 300.802165300545
6.1 300.80153536374
6.101 300.800905933288
6.102 300.800277008778
6.103 300.7996485898
6.104 300.799020675945
6.105 300.798393266802
6.106 300.797766361964
6.107 300.797139961021
6.108 300.796514063564
6.109 300.795888669186
6.11 300.795263777479
6.111 300.794639388036
6.112 300.794015500448
6.113 300.793392114311
6.114 300.792769229217
6.115 300.792146844761
6.116 300.791524960536
6.117 300.790903576138
6.118 300.790282691162
6.119 300.789662305202
6.12 300.789042417854
6.121 300.788423028715
6.122 300.787804137381
6.123 300.787185743448
6.124 300.786567846513
6.125 300.785950446174
6.126 300.785333542029
6.127 300.784717133675
6.128 300.78410122071
6.129 300.783485802735
6.13 300.782870879347
6.131 300.782256450145
6.132 300.78164251473
6.133 300.781029072702
6.134 300.780416123661
6.135 300.779803667207
6.136 300.779191702943
6.137 300.778580230468
6.138 300.777969249384
6.139 300.777358759295
6.14 300.776748759801
6.141 300.776139250506
6.142 300.775530231013
6.143 300.774921700925
6.144 300.774313659845
6.145 300.773706107378
6.146 300.773099043128
6.147 300.7724924667
6.148 300.771886377698
6.149 300.771280775729
6.15 300.770675660397
6.151 300.770071031308
6.152 300.76946688807
6.153 300.768863230288
6.154 300.768260057569
6.155 300.767657369522
6.156 300.767055165752
6.157 300.76645344587
6.158 300.765852209481
6.159 300.765251456196
6.16 300.764651185623
6.161 300.764051397371
6.162 300.76345209105
6.163 300.76285326627
6.164 300.76225492264
6.165 300.761657059772
6.166 300.761059677277
6.167 300.760462774765
6.168 300.759866351848
6.169 300.759270408139
6.17 300.758674943248
6.171 300.758079956789
6.172 300.757485448374
6.173 300.756891417617
6.174 300.756297864131
6.175 300.755704787529
6.176 300.755112187426
6.177 300.754520063437
6.178 300.753928415175
6.179 300.753337242256
6.18 300.752746544296
6.181 300.75215632091
6.182 300.751566571713
6.183 300.750977296324
6.184 300.750388494357
6.185 300.74980016543
6.186 300.74921230916
6.187 300.748624925166
6.188 300.748038013064
6.189 300.747451572473
6.19 300.746865603012
6.191 300.7462801043
6.192 300.745695075955
6.193 300.745110517598
6.194 300.744526428847
6.195 300.743942809324
6.196 300.743359658649
6.197 300.742776976442
6.198 300.742194762325
6.199 300.741613015919
6.2 300.741031736845
6.201 300.740450924726
6.202 300.739870579185
6.203 300.739290699843
6.204 300.738711286324
6.205 300.73813233825
6.206 300.737553855247
6.207 300.736975836937
6.208 300.736398282945
6.209 300.735821192895
6.21 300.735244566412
6.211 300.734668403122
6.212 300.73409270265
6.213 300.733517464621
6.214 300.732942688662
6.215 300.732368374399
6.216 300.731794521459
6.217 300.731221129469
6.218 300.730648198056
6.219 300.730075726849
6.22 300.729503715474
6.221 300.72893216356
6.222 300.728361070736
6.223 300.727790436631
6.224 300.727220260873
6.225 300.726650543093
6.226 300.72608128292
6.227 300.725512479985
6.228 300.724944133917
6.229 300.724376244348
6.23 300.723808810908
6.231 300.72324183323
6.232 300.722675310944
6.233 300.722109243682
6.234 300.721543631077
6.235 300.720978472762
6.236 300.720413768369
6.237 300.719849517531
6.238 300.719285719882
6.239 300.718722375056
6.24 300.718159482687
6.241 300.717597042409
6.242 300.717035053857
6.243 300.716473516666
6.244 300.715912430471
6.245 300.715351794909
6.246 300.714791609614
6.247 300.714231874224
6.248 300.713672588374
6.249 300.713113751702
6.25 300.712555363844
6.251 300.711997424438
6.252 300.711439933122
6.253 300.710882889533
6.254 300.710326293311
6.255 300.709770144094
6.256 300.70921444152
6.257 300.70865918523
6.258 300.708104374862
6.259 300.707550010056
6.26 300.706996090453
6.261 300.706442615693
6.262 300.705889585417
6.263 300.705336999265
6.264 300.70478485688
6.265 300.704233157902
6.266 300.703681901974
6.267 300.703131088738
6.268 300.702580717836
6.269 300.702030788911
6.27 300.701481301607
6.271 300.700932255567
6.272 300.700383650434
6.273 300.699835485852
6.274 300.699287761467
6.275 300.698740476921
6.276 300.698193631861
6.277 300.697647225932
6.278 300.697101258778
6.279 300.696555730046
6.28 300.696010639382
6.281 300.695465986433
6.282 300.694921770844
6.283 300.694377992263
6.284 300.693834650337
6.285 300.693291744713
6.286 300.69274927504
6.287 300.692207240966
6.288 300.691665642138
6.289 300.691124478207
6.29 300.69058374882
6.291 300.690043453627
6.292 300.689503592278
6.293 300.688964164422
6.294 300.68842516971
6.295 300.687886607792
6.296 300.687348478319
6.297 300.686810780942
6.298 300.686273515312
6.299 300.685736681081
6.3 300.685200277901
6.301 300.684664305424
6.302 300.684128763302
6.303 300.683593651188
6.304 300.683058968736
6.305 300.682524715598
6.306 300.681990891429
6.307 300.681457495882
6.308 300.680924528611
6.309 300.680391989272
6.31 300.679859877518
6.311 300.679328193005
6.312 300.678796935388
6.313 300.678266104323
6.314 300.677735699466
6.315 300.677205720472
6.316 300.676676166999
6.317 300.676147038704
6.318 300.675618335243
6.319 300.675090056273
6.32 300.674562201453
6.321 300.67403477044
6.322 300.673507762893
6.323 300.67298117847
6.324 300.67245501683
6.325 300.671929277631
6.326 300.671403960534
6.327 300.670879065198
6.328 300.670354591282
6.329 300.669830538447
6.33 300.669306906354
6.331 300.668783694663
6.332 300.668260903036
6.333 300.667738531132
6.334 300.667216578616
6.335 300.666695045147
6.336 300.666173930388
6.337 300.665653234002
6.338 300.665132955651
6.339 300.664613094999
6.34 300.664093651709
6.341 300.663574625443
6.342 300.663056015867
6.343 300.662537822645
6.344 300.66202004544
6.345 300.661502683917
6.346 300.660985737742
6.347 300.660469206579
6.348 300.659953090094
6.349 300.659437387953
6.35 300.658922099822
6.351 300.658407225367
6.352 300.657892764256
6.353 300.657378716154
6.354 300.656865080729
6.355 300.656351857648
6.356 300.65583904658
6.357 300.655326647192
6.358 300.654814659152
6.359 300.654303082129
6.36 300.653791915793
6.361 300.653281159811
6.362 300.652770813854
6.363 300.65226087759
6.364 300.651751350691
6.365 300.651242232826
6.366 300.650733523666
6.367 300.650225222881
6.368 300.649717330143
6.369 300.649209845122
6.37 300.648702767491
6.371 300.648196096921
6.372 300.647689833084
6.373 300.647183975653
6.374 300.6466785243
6.375 300.646173478699
6.376 300.645668838522
6.377 300.645164603443
6.378 300.644660773136
6.379 300.644157347274
6.38 300.643654325533
6.381 300.643151707586
6.382 300.642649493109
6.383 300.642147681776
6.384 300.641646273263
6.385 300.641145267246
6.386 300.6406446634
6.387 300.640144461402
6.388 300.639644660927
6.389 300.639145261654
6.39 300.638646263258
6.391 300.638147665418
6.392 300.63764946781
6.393 300.637151670112
6.394 300.636654272002
6.395 300.63615727316
6.396 300.635660673262
6.397 300.635164471989
6.398 300.634668669019
6.399 300.634173264031
6.4 300.633678256706
6.401 300.633183646723
6.402 300.632689433762
6.403 300.632195617505
6.404 300.631702197631
6.405 300.631209173822
6.406 300.630716545758
6.407 300.630224313122
6.408 300.629732475595
6.409 300.629241032859
6.41 300.628749984596
6.411 300.628259330489
6.412 300.627769070221
6.413 300.627279203475
6.414 300.626789729934
6.415 300.626300649282
6.416 300.625811961203
6.417 300.625323665381
6.418 300.6248357615
6.419 300.624348249245
6.42 300.623861128301
6.421 300.623374398352
6.422 300.622888059086
6.423 300.622402110186
6.424 300.62191655134
6.425 300.621431382233
6.426 300.620946602552
6.427 300.620462211984
6.428 300.619978210216
6.429 300.619494596934
6.43 300.619011371827
6.431 300.618528534583
6.432 300.618046084888
6.433 300.617564022433
6.434 300.617082346904
6.435 300.616601057992
6.436 300.616120155384
6.437 300.615639638772
6.438 300.615159507843
6.439 300.614679762288
6.44 300.614200401797
6.441 300.61372142606
6.442 300.613242834769
6.443 300.612764627613
6.444 300.612286804284
6.445 300.611809364474
6.446 300.611332307874
6.447 300.610855634176
6.448 300.610379343072
6.449 300.609903434255
6.45 300.609427907417
6.451 300.608952762251
6.452 300.608477998451
6.453 300.608003615709
6.454 300.60752961372
6.455 300.607055992177
6.456 300.606582750775
6.457 300.606109889208
6.458 300.60563740717
6.459 300.605165304358
6.46 300.604693580465
6.461 300.604222235187
6.462 300.603751268221
6.463 300.603280679262
6.464 300.602810468006
6.465 300.60234063415
6.466 300.60187117739
6.467 300.601402097424
6.468 300.600933393948
6.469 300.600465066661
6.47 300.599997115259
6.471 300.599529539441
6.472 300.599062338905
6.473 300.59859551335
6.474 300.598129062474
6.475 300.597662985976
6.476 300.597197283556
6.477 300.596731954913
6.478 300.596266999746
6.479 300.595802417756
6.48 300.595338208643
6.481 300.594874372108
6.482 300.59441090785
6.483 300.593947815572
6.484 300.593485094974
6.485 300.593022745758
6.486 300.592560767626
6.487 300.592099160279
6.488 300.591637923419
6.489 300.59117705675
6.49 300.590716559974
6.491 300.590256432793
6.492 300.589796674911
6.493 300.589337286031
6.494 300.588878265858
6.495 300.588419614094
6.496 300.587961330445
6.497 300.587503414614
6.498 300.587045866306
6.499 300.586588685227
6.5 300.586131871081
6.501 300.585675423573
6.502 300.58521934241
6.503 300.584763627298
6.504 300.584308277941
6.505 300.583853294048
6.506 300.583398675323
6.507 300.582944421475
6.508 300.582490532211
6.509 300.582037007237
6.51 300.581583846261
6.511 300.581131048992
6.512 300.580678615137
6.513 300.580226544404
6.514 300.579774836502
6.515 300.57932349114
6.516 300.578872508027
6.517 300.578421886872
6.518 300.577971627385
6.519 300.577521729275
6.52 300.577072192253
6.521 300.576623016028
6.522 300.576174200311
6.523 300.575725744813
6.524 300.575277649245
6.525 300.574829913318
6.526 300.574382536744
6.527 300.573935519234
6.528 300.5734888605
6.529 300.573042560254
6.53 300.572596618209
6.531 300.572151034077
6.532 300.571705807571
6.533 300.571260938405
6.534 300.570816426291
6.535 300.570372270943
6.536 300.569928472076
6.537 300.569485029403
6.538 300.569041942638
6.539 300.568599211496
6.54 300.568156835691
6.541 300.56771481494
6.542 300.567273148956
6.543 300.566831837457
6.544 300.566390880156
6.545 300.56595027677
6.546 300.565510027015
6.547 300.565070130609
6.548 300.564630587266
6.549 300.564191396705
6.55 300.563752558643
6.551 300.563314072796
6.552 300.562875938882
6.553 300.56243815662
6.554 300.562000725727
6.555 300.561563645922
6.556 300.561126916922
6.557 300.560690538447
6.558 300.560254510217
6.559 300.559818831949
6.56 300.559383503364
6.561 300.558948524181
6.562 300.55851389412
6.563 300.558079612901
6.564 300.557645680245
6.565 300.557212095872
6.566 300.556778859503
6.567 300.55634597086
6.568 300.555913429663
6.569 300.555481235634
6.57 300.555049388496
6.571 300.554617887969
6.572 300.554186733776
6.573 300.553755925639
6.574 300.553325463282
6.575 300.552895346427
6.576 300.552465574797
6.577 300.552036148116
6.578 300.551607066107
6.579 300.551178328494
6.58 300.550749935002
6.581 300.550321885354
6.582 300.549894179275
6.583 300.54946681649
6.584 300.549039796724
6.585 300.548613119702
6.586 300.548186785149
6.587 300.547760792792
6.588 300.547335142355
6.589 300.546909833566
6.59 300.54648486615
6.591 300.546060239834
6.592 300.545635954345
6.593 300.54521200941
6.594 300.544788404756
6.595 300.544365140111
6.596 300.543942215202
6.597 300.543519629757
6.598 300.543097383505
6.599 300.542675476173
6.6 300.54225390749
6.601 300.541832677186
6.602 300.541411784989
6.603 300.540991230628
6.604 300.540571013834
6.605 300.540151134335
6.606 300.539731591863
6.607 300.539312386146
6.608 300.538893516915
6.609 300.538474983901
6.61 300.538056786835
6.611 300.537638925448
6.612 300.537221399471
6.613 300.536804208635
6.614 300.536387352673
6.615 300.535970831316
6.616 300.535554644297
6.617 300.535138791347
6.618 300.5347232722
6.619 300.534308086588
6.62 300.533893234245
6.621 300.533478714903
6.622 300.533064528297
6.623 300.532650674159
6.624 300.532237152224
6.625 300.531823962227
6.626 300.5314111039
6.627 300.53099857698
6.628 300.530586381201
6.629 300.530174516297
6.63 300.529762982004
6.631 300.529351778059
6.632 300.528940904195
6.633 300.52853036015
6.634 300.528120145659
6.635 300.527710260459
6.636 300.527300704287
6.637 300.526891476878
6.638 300.526482577971
6.639 300.526074007302
6.64 300.525665764609
6.641 300.52525784963
6.642 300.524850262101
6.643 300.524443001763
6.644 300.524036068352
6.645 300.523629461607
6.646 300.523223181267
6.647 300.522817227072
6.648 300.522411598759
6.649 300.522006296069
6.65 300.521601318742
6.651 300.521196666516
6.652 300.520792339133
6.653 300.520388336332
6.654 300.519984657853
6.655 300.519581303439
6.656 300.519178272828
6.657 300.518775565764
6.658 300.518373181986
6.659 300.517971121236
6.66 300.517569383257
6.661 300.51716796779
6.662 300.516766874577
6.663 300.516366103361
6.664 300.515965653885
6.665 300.515565525891
6.666 300.515165719121
6.667 300.514766233321
6.668 300.514367068232
6.669 300.513968223599
6.67 300.513569699165
6.671 300.513171494675
6.672 300.512773609873
6.673 300.512376044503
6.674 300.511978798311
6.675 300.511581871041
6.676 300.511185262437
6.677 300.510788972247
6.678 300.510393000215
6.679 300.509997346087
6.68 300.509602009609
6.681 300.509206990527
6.682 300.508812288589
6.683 300.508417903539
6.684 300.508023835125
6.685 300.507630083095
6.686 300.507236647195
6.687 300.506843527173
6.688 300.506450722777
6.689 300.506058233754
6.69 300.505666059853
6.691 300.505274200822
6.692 300.504882656409
6.693 300.504491426363
6.694 300.504100510433
6.695 300.503709908369
6.696 300.503319619919
6.697 300.502929644833
6.698 300.502539982861
6.699 300.502150633753
6.7 300.501761597258
6.701 300.501372873128
6.702 300.500984461113
6.703 300.500596360963
6.704 300.50020857243
6.705 300.499821095265
6.706 300.499433929219
6.707 300.499047074044
6.708 300.498660529492
6.709 300.498274295315
6.71 300.497888371264
6.711 300.497502757093
6.712 300.497117452555
6.713 300.496732457401
6.714 300.496347771385
6.715 300.49596339426
6.716 300.495579325781
6.717 300.495195565699
6.718 300.49481211377
6.719 300.494428969747
6.72 300.494046133385
6.721 300.493663604437
6.722 300.493281382659
6.723 300.492899467806
6.724 300.492517859633
6.725 300.492136557895
6.726 300.491755562347
6.727 300.491374872746
6.728 300.490994488846
6.729 300.490614410404
6.73 300.490234637177
6.731 300.489855168921
6.732 300.489476005393
6.733 300.489097146349
6.734 300.488718591547
6.735 300.488340340744
6.736 300.487962393698
6.737 300.487584750165
6.738 300.487207409905
6.739 300.486830372675
6.74 300.486453638233
6.741 300.486077206339
6.742 300.48570107675
6.743 300.485325249226
6.744 300.484949723525
6.745 300.484574499408
6.746 300.484199576632
6.747 300.48382495496
6.748 300.483450634149
6.749 300.48307661396
6.75 300.482702894154
6.751 300.482329474491
6.752 300.481956354732
6.753 300.481583534638
6.754 300.481211013969
6.755 300.480838792487
6.756 300.480466869953
6.757 300.48009524613
6.758 300.479723920779
6.759 300.479352893662
6.76 300.478982164542
6.761 300.47861173318
6.762 300.47824159934
6.763 300.477871762784
6.764 300.477502223276
6.765 300.477132980578
6.766 300.476764034455
6.767 300.47639538467
6.768 300.476027030986
6.769 300.475658973168
6.77 300.475291210979
6.771 300.474923744186
6.772 300.474556572551
6.773 300.47418969584
6.774 300.473823113818
6.775 300.473456826251
6.776 300.473090832902
6.777 300.472725133539
6.778 300.472359727927
6.779 300.471994615831
6.78 300.471629797019
6.781 300.471265271256
6.782 300.470901038309
6.783 300.470537097944
6.784 300.470173449929
6.785 300.469810094031
6.786 300.469447030017
6.787 300.469084257655
6.788 300.468721776712
6.789 300.468359586956
6.79 300.467997688155
6.791 300.467636080079
6.792 300.467274762494
6.793 300.46691373517
6.794 300.466552997876
6.795 300.466192550381
6.796 300.465832392453
6.797 300.465472523864
6.798 300.465112944381
6.799 300.464753653775
6.8 300.464394651817
6.801 300.464035938275
6.802 300.463677512921
6.803 300.463319375525
6.804 300.462961525857
6.805 300.46260396369
6.806 300.462246688794
6.807 300.46188970094
6.808 300.4615329999
6.809 300.461176585446
6.81 300.460820457349
6.811 300.460464615382
6.812 300.460109059316
6.813 300.459753788925
6.814 300.459398803981
6.815 300.459044104257
6.816 300.458689689526
6.817 300.458335559561
6.818 300.457981714136
6.819 300.457628153024
6.82 300.457274875999
6.821 300.456921882834
6.822 300.456569173305
6.823 300.456216747186
6.824 300.45586460425
6.825 300.455512744273
6.826 300.45516116703
6.827 300.454809872295
6.828 300.454458859844
6.829 300.454108129453
6.83 300.453757680897
6.831 300.453407513951
6.832 300.453057628393
6.833 300.452708023998
6.834 300.452358700542
6.835 300.452009657802
6.836 300.451660895555
6.837 300.451312413578
6.838 300.450964211648
6.839 300.450616289543
6.84 300.450268647039
6.841 300.449921283914
6.842 300.449574199947
6.843 300.449227394916
6.844 300.448880868598
6.845 300.448534620772
6.846 300.448188651216
6.847 300.44784295971
6.848 300.447497546032
6.849 300.447152409962
6.85 300.446807551279
6.851 300.446462969763
6.852 300.446118665192
6.853 300.445774637347
6.854 300.445430886008
6.855 300.445087410956
6.856 300.44474421197
6.857 300.444401288832
6.858 300.444058641321
6.859 300.44371626922
6.86 300.443374172309
6.861 300.443032350369
6.862 300.442690803183
6.863 300.442349530531
6.864 300.442008532196
6.865 300.441667807959
6.866 300.441327357604
6.867 300.440987180911
6.868 300.440647277665
6.869 300.440307647647
6.87 300.439968290641
6.871 300.43962920643
6.872 300.439290394796
6.873 300.438951855524
6.874 300.438613588398
6.875 300.4382755932
6.876 300.437937869716
6.877 300.437600417729
6.878 300.437263237024
6.879 300.436926327385
6.88 300.436589688596
6.881 300.436253320444
6.882 300.435917222713
6.883 300.435581395188
6.884 300.435245837655
6.885 300.434910549899
6.886 300.434575531707
6.887 300.434240782863
6.888 300.433906303155
6.889 300.433572092368
6.89 300.43323815029
6.891 300.432904476707
6.892 300.432571071405
6.893 300.432237934172
6.894 300.431905064796
6.895 300.431572463062
6.896 300.43124012876
6.897 300.430908061677
6.898 300.4305762616
6.899 300.430244728318
6.9 300.429913461619
6.901 300.429582461292
6.902 300.429251727125
6.903 300.428921258906
6.904 300.428591056426
6.905 300.428261119472
6.906 300.427931447835
6.907 300.427602041304
6.908 300.427272899668
6.909 300.426944022717
6.91 300.426615410242
6.911 300.426287062032
6.912 300.425958977879
6.913 300.425631157572
6.914 300.425303600902
6.915 300.42497630766
6.916 300.424649277637
6.917 300.424322510625
6.918 300.423996006415
6.919 300.423669764797
6.92 300.423343785565
6.921 300.42301806851
6.922 300.422692613424
6.923 300.422367420099
6.924 300.422042488328
6.925 300.421717817904
6.926 300.421393408618
6.927 300.421069260265
6.928 300.420745372637
6.929 300.420421745528
6.93 300.42009837873
6.931 300.419775272038
6.932 300.419452425246
6.933 300.419129838147
6.934 300.418807510536
6.935 300.418485442206
6.936 300.418163632953
6.937 300.417842082571
6.938 300.417520790855
6.939 300.4171997576
6.94 300.416878982601
6.941 300.416558465653
6.942 300.416238206552
6.943 300.415918205094
6.944 300.415598461074
6.945 300.415278974288
6.946 300.414959744533
6.947 300.414640771604
6.948 300.414322055299
6.949 300.414003595414
6.95 300.413685391747
6.951 300.413367444093
6.952 300.41304975225
6.953 300.412732316016
6.954 300.412415135188
6.955 300.412098209563
6.956 300.411781538941
6.957 300.411465123118
6.958 300.411148961892
6.959 300.410833055063
6.96 300.410517402429
6.961 300.410202003788
6.962 300.409886858939
6.963 300.409571967682
6.964 300.409257329815
6.965 300.408942945138
6.966 300.40862881345
6.967 300.408314934551
6.968 300.408001308241
6.969 300.40768793432
6.97 300.407374812588
6.971 300.407061942845
6.972 300.406749324893
6.973 300.406436958531
6.974 300.40612484356
6.975 300.405812979782
6.976 300.405501366998
6.977 300.405190005008
6.978 300.404878893615
6.979 300.404568032621
6.98 300.404257421826
6.981 300.403947061033
6.982 300.403636950044
6.983 300.403327088662
6.984 300.403017476689
6.985 300.402708113927
6.986 300.40239900018
6.987 300.40209013525
6.988 300.401781518941
6.989 300.401473151055
6.99 300.401165031397
6.991 300.40085715977
6.992 300.400549535977
6.993 300.400242159823
6.994 300.399935031112
6.995 300.399628149648
6.996 300.399321515235
6.997 300.399015127679
6.998 300.398708986783
6.999 300.398403092353
7 300.398097444193
7.001 300.39779204211
7.002 300.397486885909
7.003 300.397181975394
7.004 300.396877310372
7.005 300.396572890649
7.006 300.39626871603
7.007 300.395964786322
7.008 300.395661101332
7.009 300.395357660865
7.01 300.395054464729
7.011 300.394751512731
7.012 300.394448804677
7.013 300.394146340374
7.014 300.39384411963
7.015 300.393542142253
7.016 300.393240408049
7.017 300.392938916827
7.018 300.392637668396
7.019 300.392336662561
7.02 300.392035899134
7.021 300.39173537792
7.022 300.39143509873
7.023 300.391135061372
7.024 300.390835265655
7.025 300.390535711388
7.026 300.39023639838
7.027 300.389937326441
7.028 300.389638495379
7.029 300.389339905006
7.03 300.38904155513
7.031 300.388743445561
7.032 300.388445576111
7.033 300.388147946589
7.034 300.387850556805
7.035 300.387553406571
7.036 300.387256495697
7.037 300.386959823995
7.038 300.386663391274
7.039 300.386367197347
7.04 300.386071242025
7.041 300.385775525119
7.042 300.385480046441
7.043 300.385184805804
7.044 300.384889803018
7.045 300.384595037897
7.046 300.384300510253
7.047 300.384006219898
7.048 300.383712166645
7.049 300.383418350306
7.05 300.383124770695
7.051 300.382831427626
7.052 300.38253832091
7.053 300.382245450362
7.054 300.381952815795
7.055 300.381660417023
7.056 300.38136825386
7.057 300.38107632612
7.058 300.380784633617
7.059 300.380493176165
7.06 300.38020195358
7.061 300.379910965675
7.062 300.379620212266
7.063 300.379329693167
7.064 300.379039408194
7.065 300.378749357162
7.066 300.378459539886
7.067 300.378169956183
7.068 300.377880605867
7.069 300.377591488754
7.07 300.377302604662
7.071 300.377013953405
7.072 300.3767255348
7.073 300.376437348664
7.074 300.376149394814
7.075 300.375861673065
7.076 300.375574183236
7.077 300.375286925142
7.078 300.374999898603
7.079 300.374713103434
7.08 300.374426539454
7.081 300.374140206479
7.082 300.373854104329
7.083 300.373568232821
7.084 300.373282591773
7.085 300.372997181004
7.086 300.372712000332
7.087 300.372427049575
7.088 300.372142328553
7.089 300.371857837084
7.09 300.371573574988
7.091 300.371289542083
7.092 300.37100573819
7.093 300.370722163127
7.094 300.370438816714
7.095 300.370155698771
7.096 300.369872809118
7.097 300.369590147575
7.098 300.369307713963
7.099 300.369025508101
7.1 300.368743529811
7.101 300.368461778912
7.102 300.368180255227
7.103 300.367898958575
7.104 300.367617888779
7.105 300.367337045659
7.106 300.367056429037
7.107 300.366776038734
7.108 300.366495874572
7.109 300.366215936373
7.11 300.36593622396
7.111 300.365656737153
7.112 300.365377475776
7.113 300.365098439651
7.114 300.364819628601
7.115 300.364541042449
7.116 300.364262681017
7.117 300.363984544128
7.118 300.363706631606
7.119 300.363428943274
7.12 300.363151478956
7.121 300.362874238474
7.122 300.362597221654
7.123 300.362320428319
7.124 300.362043858292
7.125 300.361767511399
7.126 300.361491387464
7.127 300.36121548631
7.128 300.360939807764
7.129 300.360664351649
7.13 300.360389117791
7.131 300.360114106015
7.132 300.359839316145
7.133 300.359564748008
7.134 300.359290401429
7.135 300.359016276233
7.136 300.358742372247
7.137 300.358468689296
7.138 300.358195227207
7.139 300.357921985805
7.14 300.357648964918
7.141 300.357376164371
7.142 300.357103583991
7.143 300.356831223606
7.144 300.356559083042
7.145 300.356287162126
7.146 300.356015460686
7.147 300.355743978549
7.148 300.355472715542
7.149 300.355201671494
7.15 300.354930846232
7.151 300.354660239583
7.152 300.354389851377
7.153 300.354119681442
7.154 300.353849729605
7.155 300.353579995695
7.156 300.353310479542
7.157 300.353041180973
7.158 300.352772099819
7.159 300.352503235907
7.16 300.352234589068
7.161 300.35196615913
7.162 300.351697945923
7.163 300.351429949277
7.164 300.351162169022
7.165 300.350894604987
7.166 300.350627257003
7.167 300.3503601249
7.168 300.350093208509
7.169 300.349826507659
7.17 300.349560022182
7.171 300.349293751908
7.172 300.349027696668
7.173 300.348761856294
7.174 300.348496230616
7.175 300.348230819466
7.176 300.347965622676
7.177 300.347700640076
7.178 300.3474358715
7.179 300.347171316778
7.18 300.346906975743
7.181 300.346642848227
7.182 300.346378934063
7.183 300.346115233082
7.184 300.345851745118
7.185 300.345588470003
7.186 300.34532540757
7.187 300.345062557652
7.188 300.344799920083
7.189 300.344537494695
7.19 300.344275281322
7.191 300.344013279798
7.192 300.343751489956
7.193 300.34348991163
7.194 300.343228544655
7.195 300.342967388863
7.196 300.342706444091
7.197 300.342445710171
7.198 300.342185186939
7.199 300.341924874229
7.2 300.341664771876
7.201 300.341404879715
7.202 300.341145197581
7.203 300.34088572531
7.204 300.340626462736
7.205 300.340367409695
7.206 300.340108566024
7.207 300.339849931556
7.208 300.339591506129
7.209 300.339333289579
7.21 300.339075281742
7.211 300.338817482454
7.212 300.338559891551
7.213 300.33830250887
7.214 300.338045334248
7.215 300.337788367521
7.216 300.337531608528
7.217 300.337275057104
7.218 300.337018713087
7.219 300.336762576314
7.22 300.336506646624
7.221 300.336250923853
7.222 300.335995407839
7.223 300.335740098421
7.224 300.335484995436
7.225 300.335230098723
7.226 300.33497540812
7.227 300.334720923465
7.228 300.334466644597
7.229 300.334212571355
7.23 300.333958703577
7.231 300.333705041104
7.232 300.333451583773
7.233 300.333198331424
7.234 300.332945283896
7.235 300.332692441029
7.236 300.332439802664
7.237 300.332187368638
7.238 300.331935138793
7.239 300.331683112968
7.24 300.331431291004
7.241 300.331179672741
7.242 300.33092825802
7.243 300.33067704668
7.244 300.330426038563
7.245 300.33017523351
7.246 300.329924631362
7.247 300.329674231959
7.248 300.329424035143
7.249 300.329174040756
7.25 300.328924248638
7.251 300.328674658632
7.252 300.32842527058
7.253 300.328176084322
7.254 300.327927099702
7.255 300.327678316562
7.256 300.327429734743
7.257 300.327181354089
7.258 300.326933174441
7.259 300.326685195643
7.26 300.326437417537
7.261 300.326189839966
7.262 300.325942462774
7.263 300.325695285803
7.264 300.325448308897
7.265 300.325201531899
7.266 300.324954954654
7.267 300.324708577004
7.268 300.324462398794
7.269 300.324216419868
7.27 300.323970640069
7.271 300.323725059242
7.272 300.323479677232
7.273 300.323234493882
7.274 300.322989509038
7.275 300.322744722544
7.276 300.322500134246
7.277 300.322255743988
7.278 300.322011551615
7.279 300.321767556972
7.28 300.321523759906
7.281 300.321280160261
7.282 300.321036757884
7.283 300.32079355262
7.284 300.320550544314
7.285 300.320307732814
7.286 300.320065117965
7.287 300.319822699613
7.288 300.319580477605
7.289 300.319338451788
7.29 300.319096622008
7.291 300.318854988111
7.292 300.318613549946
7.293 300.318372307359
7.294 300.318131260197
7.295 300.317890408308
7.296 300.317649751538
7.297 300.317409289736
7.298 300.31716902275
7.299 300.316928950426
7.3 300.316689072614
7.301 300.31644938916
7.302 300.316209899914
7.303 300.315970604724
7.304 300.315731503437
7.305 300.315492595903
7.306 300.315253881971
7.307 300.315015361489
7.308 300.314777034306
7.309 300.314538900272
7.31 300.314300959235
7.311 300.314063211044
7.312 300.31382565555
7.313 300.313588292602
7.314 300.313351122049
7.315 300.313114143742
7.316 300.312877357529
7.317 300.312640763263
7.318 300.312404360792
7.319 300.312168149967
7.32 300.311932130638
7.321 300.311696302656
7.322 300.311460665872
7.323 300.311225220137
7.324 300.310989965301
7.325 300.310754901215
7.326 300.310520027732
7.327 300.310285344701
7.328 300.310050851975
7.329 300.309816549405
7.33 300.309582436843
7.331 300.309348514141
7.332 300.30911478115
7.333 300.308881237722
7.334 300.30864788371
7.335 300.308414718967
7.336 300.308181743344
7.337 300.307948956694
7.338 300.307716358869
7.339 300.307483949723
7.34 300.307251729109
7.341 300.307019696879
7.342 300.306787852886
7.343 300.306556196985
7.344 300.306324729027
7.345 300.306093448868
7.346 300.30586235636
7.347 300.305631451357
7.348 300.305400733714
7.349 300.305170203283
7.35 300.30493985992
7.351 300.304709703478
7.352 300.304479733812
7.353 300.304249950777
7.354 300.304020354226
7.355 300.303790944015
7.356 300.303561719999
7.357 300.303332682032
7.358 300.30310382997
7.359 300.302875163668
7.36 300.302646682981
7.361 300.302418387764
7.362 300.302190277874
7.363 300.301962353166
7.364 300.301734613495
7.365 300.301507058718
7.366 300.301279688691
7.367 300.301052503269
7.368 300.30082550231
7.369 300.300598685669
7.37 300.300372053204
7.371 300.30014560477
7.372 300.299919340225
7.373 300.299693259426
7.374 300.299467362228
7.375 300.299241648491
7.376 300.29901611807
7.377 300.298790770823
7.378 300.298565606609
7.379 300.298340625283
7.38 300.298115826704
7.381 300.29789121073
7.382 300.297666777219
7.383 300.297442526029
7.384 300.297218457018
7.385 300.296994570044
7.386 300.296770864966
7.387 300.296547341642
7.388 300.296323999931
7.389 300.296100839692
7.39 300.295877860784
7.391 300.295655063065
7.392 300.295432446395
7.393 300.295210010633
7.394 300.294987755638
7.395 300.294765681271
7.396 300.294543787389
7.397 300.294322073854
7.398 300.294100540525
7.399 300.293879187262
7.4 300.293658013925
7.401 300.293437020374
7.402 300.29321620647
7.403 300.292995572073
7.404 300.292775117043
7.405 300.292554841241
7.406 300.292334744528
7.407 300.292114826765
7.408 300.291895087812
7.409 300.291675527531
7.41 300.291456145783
7.411 300.29123694243
7.412 300.291017917332
7.413 300.290799070352
7.414 300.29058040135
7.415 300.29036191019
7.416 300.290143596732
7.417 300.289925460839
7.418 300.289707502373
7.419 300.289489721196
7.42 300.28927211717
7.421 300.289054690159
7.422 300.288837440024
7.423 300.288620366628
7.424 300.288403469835
7.425 300.288186749506
7.426 300.287970205506
7.427 300.287753837697
7.428 300.287537645942
7.429 300.287321630105
7.43 300.28710579005
7.431 300.28689012564
7.432 300.286674636739
7.433 300.286459323211
7.434 300.286244184919
7.435 300.286029221727
7.436 300.285814433501
7.437 300.285599820104
7.438 300.2853853814
7.439 300.285171117254
7.44 300.284957027532
7.441 300.284743112096
7.442 300.284529370813
7.443 300.284315803548
7.444 300.284102410164
7.445 300.283889190529
7.446 300.283676144506
7.447 300.283463271961
7.448 300.283250572761
7.449 300.28303804677
7.45 300.282825693854
7.451 300.28261351388
7.452 300.282401506712
7.453 300.282189672218
7.454 300.281978010263
7.455 300.281766520714
7.456 300.281555203438
7.457 300.281344058299
7.458 300.281133085167
7.459 300.280922283906
7.46 300.280711654384
7.461 300.280501196468
7.462 300.280290910025
7.463 300.280080794923
7.464 300.279870851028
7.465 300.279661078208
7.466 300.27945147633
7.467 300.279242045263
7.468 300.279032784873
7.469 300.278823695029
7.47 300.278614775599
7.471 300.278406026451
7.472 300.278197447452
7.473 300.277989038472
7.474 300.277780799379
7.475 300.277572730041
7.476 300.277364830326
7.477 300.277157100105
7.478 300.276949539244
7.479 300.276742147615
7.48 300.276534925084
7.481 300.276327871522
7.482 300.276120986799
7.483 300.275914270782
7.484 300.275707723342
7.485 300.275501344349
7.486 300.275295133672
7.487 300.275089091181
7.488 300.274883216746
7.489 300.274677510237
7.49 300.274471971523
7.491 300.274266600476
7.492 300.274061396966
7.493 300.273856360863
7.494 300.273651492038
7.495 300.273446790361
7.496 300.273242255704
7.497 300.273037887937
7.498 300.272833686931
7.499 300.272629652557
7.5 300.272425784687
7.501 300.272222083192
7.502 300.272018547943
7.503 300.271815178812
7.504 300.27161197567
7.505 300.27140893839
7.506 300.271206066843
7.507 300.271003360901
7.508 300.270800820436
7.509 300.270598445321
7.51 300.270396235427
7.511 300.270194190627
7.512 300.269992310794
7.513 300.2697905958
7.514 300.269589045518
7.515 300.269387659821
7.516 300.269186438581
7.517 300.268985381672
7.518 300.268784488967
7.519 300.268583760339
7.52 300.268383195662
7.521 300.268182794808
7.522 300.267982557652
7.523 300.267782484067
7.524 300.267582573926
7.525 300.267382827105
7.526 300.267183243476
7.527 300.266983822915
7.528 300.266784565294
7.529 300.266585470489
7.53 300.266386538373
7.531 300.266187768822
7.532 300.26598916171
7.533 300.265790716911
7.534 300.265592434301
7.535 300.265394313754
7.536 300.265196355145
7.537 300.26499855835
7.538 300.264800923244
7.539 300.264603449702
7.54 300.264406137599
7.541 300.264208986812
7.542 300.264011997215
7.543 300.263815168685
7.544 300.263618501097
7.545 300.263421994327
7.546 300.263225648252
7.547 300.263029462747
7.548 300.26283343769
7.549 300.262637572955
7.55 300.26244186842
7.551 300.262246323962
7.552 300.262050939456
7.553 300.26185571478
7.554 300.261660649811
7.555 300.261465744426
7.556 300.261270998501
7.557 300.261076411915
7.558 300.260881984543
7.559 300.260687716264
7.56 300.260493606956
7.561 300.260299656495
7.562 300.26010586476
7.563 300.259912231628
7.564 300.259718756977
7.565 300.259525440686
7.566 300.259332282632
7.567 300.259139282693
7.568 300.258946440748
7.569 300.258753756676
7.57 300.258561230354
7.571 300.258368861662
7.572 300.258176650479
7.573 300.257984596682
7.574 300.257792700151
7.575 300.257600960765
7.576 300.257409378403
7.577 300.257217952944
7.578 300.257026684268
7.579 300.256835572254
7.58 300.256644616782
7.581 300.256453817731
7.582 300.25626317498
7.583 300.256072688411
7.584 300.255882357903
7.585 300.255692183335
7.586 300.255502164588
7.587 300.255312301543
7.588 300.255122594079
7.589 300.254933042076
7.59 300.254743645417
7.591 300.25455440398
7.592 300.254365317647
7.593 300.254176386299
7.594 300.253987609816
7.595 300.25379898808
7.596 300.253610520972
7.597 300.253422208373
7.598 300.253234050164
7.599 300.253046046227
7.6 300.252858196443
7.601 300.252670500694
7.602 300.252482958861
7.603 300.252295570827
7.604 300.252108336474
7.605 300.251921255683
7.606 300.251734328336
7.607 300.251547554316
7.608 300.251360933505
7.609 300.251174465785
7.61 300.25098815104
7.611 300.25080198915
7.612 300.250615980001
7.613 300.250430123473
7.614 300.25024441945
7.615 300.250058867815
7.616 300.249873468451
7.617 300.249688221241
7.618 300.249503126069
7.619 300.249318182818
7.62 300.249133391372
7.621 300.248948751613
7.622 300.248764263426
7.623 300.248579926695
7.624 300.248395741303
7.625 300.248211707134
7.626 300.248027824073
7.627 300.247844092004
7.628 300.24766051081
7.629 300.247477080376
7.63 300.247293800588
7.631 300.247110671328
7.632 300.246927692483
7.633 300.246744863936
7.634 300.246562185572
7.635 300.246379657277
7.636 300.246197278936
7.637 300.246015050433
7.638 300.245832971654
7.639 300.245651042484
7.64 300.245469262809
7.641 300.245287632513
7.642 300.245106151483
7.643 300.244924819605
7.644 300.244743636764
7.645 300.244562602846
7.646 300.244381717737
7.647 300.244200981324
7.648 300.244020393492
7.649 300.243839954127
7.65 300.243659663116
7.651 300.243479520346
7.652 300.243299525703
7.653 300.243119679074
7.654 300.242939980345
7.655 300.242760429403
7.656 300.242581026136
7.657 300.24240177043
7.658 300.242222662172
7.659 300.242043701251
7.66 300.241864887552
7.661 300.241686220963
7.662 300.241507701373
7.663 300.241329328668
7.664 300.241151102736
7.665 300.240973023465
7.666 300.240795090743
7.667 300.240617304458
7.668 300.240439664498
7.669 300.240262170751
7.67 300.240084823105
7.671 300.239907621449
7.672 300.239730565671
7.673 300.23955365566
7.674 300.239376891305
7.675 300.239200272493
7.676 300.239023799114
7.677 300.238847471057
7.678 300.238671288211
7.679 300.238495250464
7.68 300.238319357707
7.681 300.238143609828
7.682 300.237968006717
7.683 300.237792548263
7.684 300.237617234356
7.685 300.237442064885
7.686 300.23726703974
7.687 300.237092158812
7.688 300.236917421989
7.689 300.236742829163
7.69 300.236568380223
7.691 300.236394075059
7.692 300.236219913561
7.693 300.236045895621
7.694 300.235872021128
7.695 300.235698289973
7.696 300.235524702047
7.697 300.23535125724
7.698 300.235177955444
7.699 300.235004796549
7.7 300.234831780447
7.701 300.234658907027
7.702 300.234486176183
7.703 300.234313587804
7.704 300.234141141783
7.705 300.233968838011
7.706 300.233796676379
7.707 300.233624656779
7.708 300.233452779103
7.709 300.233281043243
7.71 300.23310944909
7.711 300.232937996538
7.712 300.232766685477
7.713 300.232595515799
7.714 300.232424487399
7.715 300.232253600167
7.716 300.232082853996
7.717 300.231912248779
7.718 300.231741784409
7.719 300.231571460778
7.72 300.231401277778
7.721 300.231231235304
7.722 300.231061333248
7.723 300.230891571503
7.724 300.230721949962
7.725 300.230552468519
7.726 300.230383127067
7.727 300.2302139255
7.728 300.23004486371
7.729 300.229875941592
7.73 300.22970715904
7.731 300.229538515947
7.732 300.229370012207
7.733 300.229201647714
7.734 300.229033422362
7.735 300.228865336046
7.736 300.22869738866
7.737 300.228529580098
7.738 300.228361910254
7.739 300.228194379023
7.74 300.2280269863
7.741 300.22785973198
7.742 300.227692615957
7.743 300.227525638126
7.744 300.227358798382
7.745 300.22719209662
7.746 300.227025532735
7.747 300.226859106624
7.748 300.22669281818
7.749 300.226526667299
7.75 300.226360653877
7.751 300.226194777809
7.752 300.226029038992
7.753 300.225863437321
7.754 300.225697972691
7.755 300.225532644999
7.756 300.225367454141
7.757 300.225202400013
7.758 300.225037482511
7.759 300.224872701531
7.76 300.22470805697
7.761 300.224543548724
7.762 300.22437917669
7.763 300.224214940764
7.764 300.224050840843
7.765 300.223886876825
7.766 300.223723048605
7.767 300.223559356081
7.768 300.22339579915
7.769 300.223232377709
7.77 300.223069091655
7.771 300.222905940886
7.772 300.222742925299
7.773 300.222580044792
7.774 300.222417299262
7.775 300.222254688607
7.776 300.222092212725
7.777 300.221929871513
7.778 300.22176766487
7.779 300.221605592693
7.78 300.22144365488
7.781 300.221281851331
7.782 300.221120181942
7.783 300.220958646613
7.784 300.220797245242
7.785 300.220635977727
7.786 300.220474843967
7.787 300.220313843861
7.788 300.220152977307
7.789 300.219992244205
7.79 300.219831644454
7.791 300.219671177951
7.792 300.219510844597
7.793 300.219350644292
7.794 300.219190576933
7.795 300.219030642421
7.796 300.218870840654
7.797 300.218711171533
7.798 300.218551634958
7.799 300.218392230827
7.8 300.218232959041
7.801 300.218073819499
7.802 300.217914812102
7.803 300.21775593675
7.804 300.217597193342
7.805 300.217438581779
7.806 300.217280101962
7.807 300.21712175379
7.808 300.216963537165
7.809 300.216805451986
7.81 300.216647498155
7.811 300.216489675571
7.812 300.216331984137
7.813 300.216174423753
7.814 300.216016994319
7.815 300.215859695738
7.816 300.215702527909
7.817 300.215545490735
7.818 300.215388584116
7.819 300.215231807955
7.82 300.215075162151
7.821 300.214918646608
7.822 300.214762261227
7.823 300.214606005909
7.824 300.214449880556
7.825 300.214293885071
7.826 300.214138019354
7.827 300.213982283309
7.828 300.213826676838
7.829 300.213671199841
7.83 300.213515852223
7.831 300.213360633885
7.832 300.21320554473
7.833 300.21305058466
7.834 300.212895753579
7.835 300.212741051387
7.836 300.21258647799
7.837 300.212432033288
7.838 300.212277717186
7.839 300.212123529587
7.84 300.211969470393
7.841 300.211815539507
7.842 300.211661736834
7.843 300.211508062276
7.844 300.211354515737
7.845 300.211201097121
7.846 300.21104780633
7.847 300.21089464327
7.848 300.210741607842
7.849 300.210588699953
7.85 300.210435919504
7.851 300.210283266401
7.852 300.210130740547
7.853 300.209978341848
7.854 300.209826070206
7.855 300.209673925526
7.856 300.209521907713
7.857 300.209370016671
7.858 300.209218252305
7.859 300.20906661452
7.86 300.208915103219
7.861 300.208763718309
7.862 300.208612459694
7.863 300.208461327279
7.864 300.208310320969
7.865 300.208159440669
7.866 300.208008686284
7.867 300.20785805772
7.868 300.207707554883
7.869 300.207557177677
7.87 300.207406926008
7.871 300.207256799782
7.872 300.207106798904
7.873 300.206956923281
7.874 300.206807172819
7.875 300.206657547422
7.876 300.206508046997
7.877 300.206358671451
7.878 300.206209420689
7.879 300.206060294618
7.88 300.205911293144
7.881 300.205762416173
7.882 300.205613663613
7.883 300.205465035369
7.884 300.205316531349
7.885 300.205168151459
7.886 300.205019895605
7.887 300.204871763695
7.888 300.204723755636
7.889 300.204575871335
7.89 300.204428110699
7.891 300.204280473634
7.892 300.20413296005
7.893 300.203985569852
7.894 300.203838302948
7.895 300.203691159246
7.896 300.203544138654
7.897 300.203397241079
7.898 300.203250466428
7.899 300.203103814611
7.9 300.202957285533
7.901 300.202810879105
7.902 300.202664595233
7.903 300.202518433826
7.904 300.202372394793
7.905 300.20222647804
7.906 300.202080683477
7.907 300.201935011013
7.908 300.201789460555
7.909 300.201644032013
7.91 300.201498725295
7.911 300.201353540309
7.912 300.201208476966
7.913 300.201063535173
7.914 300.200918714839
7.915 300.200774015875
7.916 300.200629438188
7.917 300.200484981689
7.918 300.200340646285
7.919 300.200196431888
7.92 300.200052338406
7.921 300.199908365749
7.922 300.199764513826
7.923 300.199620782547
7.924 300.199477171822
7.925 300.199333681561
7.926 300.199190311674
7.927 300.19904706207
7.928 300.19890393266
7.929 300.198760923354
7.93 300.198618034062
7.931 300.198475264695
7.932 300.198332615162
7.933 300.198190085375
7.934 300.198047675244
7.935 300.197905384679
7.936 300.197763213592
7.937 300.197621161892
7.938 300.197479229491
7.939 300.1973374163
7.94 300.19719572223
7.941 300.197054147191
7.942 300.196912691095
7.943 300.196771353853
7.944 300.196630135377
7.945 300.196489035577
7.946 300.196348054366
7.947 300.196207191654
7.948 300.196066447354
7.949 300.195925821377
7.95 300.195785313634
7.951 300.195644924038
7.952 300.1955046525
7.953 300.195364498932
7.954 300.195224463247
7.955 300.195084545357
7.956 300.194944745173
7.957 300.194805062609
7.958 300.194665497575
7.959 300.194526049985
7.96 300.194386719752
7.961 300.194247506788
7.962 300.194108411004
7.963 300.193969432316
7.964 300.193830570634
7.965 300.193691825872
7.966 300.193553197942
7.967 300.193414686759
7.968 300.193276292234
7.969 300.193138014282
7.97 300.192999852814
7.971 300.192861807745
7.972 300.192723878988
7.973 300.192586066457
7.974 300.192448370064
7.975 300.192310789724
7.976 300.192173325349
7.977 300.192035976855
7.978 300.191898744154
7.979 300.191761627161
7.98 300.19162462579
7.981 300.191487739954
7.982 300.191350969568
7.983 300.191214314545
7.984 300.191077774801
7.985 300.19094135025
7.986 300.190805040805
7.987 300.190668846381
7.988 300.190532766894
7.989 300.190396802257
7.99 300.190260952385
7.991 300.190125217193
7.992 300.189989596596
7.993 300.189854090509
7.994 300.189718698847
7.995 300.189583421524
7.996 300.189448258457
7.997 300.189313209559
7.998 300.189178274747
7.999 300.189043453935
8 300.18890874704
8.001 300.188774153977
8.002 300.18863967466
8.003 300.188505309007
8.004 300.188371056932
8.005 300.188236918351
8.006 300.18810289318
8.007 300.187968981336
8.008 300.187835182733
8.009 300.187701497289
8.01 300.187567924919
8.011 300.18743446554
8.012 300.187301119067
8.013 300.187167885417
8.014 300.187034764507
8.015 300.186901756253
8.016 300.186768860571
8.017 300.186636077379
8.018 300.186503406593
8.019 300.186370848129
8.02 300.186238401905
8.021 300.186106067838
8.022 300.185973845844
8.023 300.18584173584
8.024 300.185709737744
8.025 300.185577851474
8.026 300.185446076945
8.027 300.185314414075
8.028 300.185182862783
8.029 300.185051422984
8.03 300.184920094598
8.031 300.184788877542
8.032 300.184657771732
8.033 300.184526777088
8.034 300.184395893526
8.035 300.184265120966
8.036 300.184134459323
8.037 300.184003908518
8.038 300.183873468468
8.039 300.18374313909
8.04 300.183612920304
8.041 300.183482812028
8.042 300.18335281418
8.043 300.183222926678
8.044 300.183093149442
8.045 300.182963482389
8.046 300.182833925438
8.047 300.182704478509
8.048 300.182575141519
8.049 300.182445914388
8.05 300.182316797035
8.051 300.182187789379
8.052 300.182058891338
8.053 300.181930102833
8.054 300.181801423781
8.055 300.181672854104
8.056 300.181544393719
8.057 300.181416042546
8.058 300.181287800506
8.059 300.181159667516
8.06 300.181031643498
8.061 300.180903728371
8.062 300.180775922054
8.063 300.180648224468
8.064 300.180520635531
8.065 300.180393155166
8.066 300.18026578329
8.067 300.180138519825
8.068 300.180011364691
8.069 300.179884317808
8.07 300.179757379096
8.071 300.179630548475
8.072 300.179503825867
8.073 300.179377211191
8.074 300.179250704369
8.075 300.17912430532
8.076 300.178998013966
8.077 300.178871830228
8.078 300.178745754026
8.079 300.178619785282
8.08 300.178493923915
8.081 300.178368169848
8.082 300.178242523002
8.083 300.178116983297
8.084 300.177991550655
8.085 300.177866224998
8.086 300.177741006246
8.087 300.177615894322
8.088 300.177490889146
8.089 300.177365990641
8.09 300.177241198728
8.091 300.177116513329
8.092 300.176991934365
8.093 300.176867461759
8.094 300.176743095433
8.095 300.176618835308
8.096 300.176494681306
8.097 300.17637063335
8.098 300.176246691363
8.099 300.176122855265
8.1 300.17599912498
8.101 300.175875500431
8.102 300.175751981538
8.103 300.175628568226
8.104 300.175505260416
8.105 300.175382058032
8.106 300.175258960996
8.107 300.17513596923
8.108 300.175013082659
8.109 300.174890301204
8.11 300.174767624789
8.111 300.174645053337
8.112 300.174522586771
8.113 300.174400225014
8.114 300.17427796799
8.115 300.174155815622
8.116 300.174033767833
8.117 300.173911824547
8.118 300.173789985687
8.119 300.173668251178
8.12 300.173546620941
8.121 300.173425094903
8.122 300.173303672985
8.123 300.173182355112
8.124 300.173061141209
8.125 300.172940031198
8.126 300.172819025004
8.127 300.172698122552
8.128 300.172577323765
8.129 300.172456628567
8.13 300.172336036884
8.131 300.172215548639
8.132 300.172095163756
8.133 300.171974882162
8.134 300.171854703779
8.135 300.171734628532
8.136 300.171614656347
8.137 300.171494787148
8.138 300.171375020861
8.139 300.171255357409
8.14 300.171135796718
8.141 300.171016338712
8.142 300.170896983318
8.143 300.17077773046
8.144 300.170658580064
8.145 300.170539532054
8.146 300.170420586356
8.147 300.170301742896
8.148 300.170183001599
8.149 300.17006436239
8.15 300.169945825196
8.151 300.169827389942
8.152 300.169709056553
8.153 300.169590824956
8.154 300.169472695076
8.155 300.169354666839
8.156 300.169236740172
8.157 300.169118915
8.158 300.16900119125
8.159 300.168883568848
8.16 300.168766047719
8.161 300.168648627791
8.162 300.168531308989
8.163 300.168414091241
8.164 300.168296974472
8.165 300.16817995861
8.166 300.168063043581
8.167 300.167946229311
8.168 300.167829515728
8.169 300.167712902757
8.17 300.167596390327
8.171 300.167479978364
8.172 300.167363666796
8.173 300.167247455548
8.174 300.167131344549
8.175 300.167015333725
8.176 300.166899423004
8.177 300.166783612314
8.178 300.166667901581
8.179 300.166552290733
8.18 300.166436779697
8.181 300.166321368402
8.182 300.166206056775
8.183 300.166090844743
8.184 300.165975732234
8.185 300.165860719177
8.186 300.165745805499
8.187 300.165630991128
8.188 300.165516275992
8.189 300.165401660019
8.19 300.165287143138
8.191 300.165172725276
8.192 300.165058406362
8.193 300.164944186325
8.194 300.164830065091
8.195 300.164716042591
8.196 300.164602118753
8.197 300.164488293504
8.198 300.164374566774
8.199 300.164260938492
8.2 300.164147408586
8.201 300.164033976985
8.202 300.163920643618
8.203 300.163807408414
8.204 300.163694271302
8.205 300.163581232211
8.206 300.16346829107
8.207 300.163355447808
8.208 300.163242702356
8.209 300.163130054641
8.21 300.163017504594
8.211 300.162905052143
8.212 300.162792697219
8.213 300.162680439751
8.214 300.162568279669
8.215 300.162456216901
8.216 300.162344251379
8.217 300.162232383031
8.218 300.162120611789
8.219 300.162008937581
8.22 300.161897360337
8.221 300.161785879988
8.222 300.161674496464
8.223 300.161563209695
8.224 300.161452019611
8.225 300.161340926143
8.226 300.161229929221
8.227 300.161119028775
8.228 300.161008224736
8.229 300.160897517034
8.23 300.160786905601
8.231 300.160676390366
8.232 300.16056597126
8.233 300.160455648215
8.234 300.160345421161
8.235 300.160235290028
8.236 300.160125254749
8.237 300.160015315253
8.238 300.159905471473
8.239 300.159795723338
8.24 300.159686070781
8.241 300.159576513733
8.242 300.159467052125
8.243 300.159357685887
8.244 300.159248414953
8.245 300.159139239253
8.246 300.159030158719
8.247 300.158921173282
8.248 300.158812282874
8.249 300.158703487428
8.25 300.158594786874
8.251 300.158486181144
8.252 300.158377670172
8.253 300.158269253887
8.254 300.158160932224
8.255 300.158052705112
8.256 300.157944572486
8.257 300.157836534277
8.258 300.157728590417
8.259 300.157620740839
8.26 300.157512985475
8.261 300.157405324257
8.262 300.157297757119
8.263 300.157190283992
8.264 300.157082904809
8.265 300.156975619504
8.266 300.156868428008
8.267 300.156761330254
8.268 300.156654326176
8.269 300.156547415706
8.27 300.156440598778
8.271 300.156333875323
8.272 300.156227245276
8.273 300.15612070857
8.274 300.156014265137
8.275 300.155907914912
8.276 300.155801657827
8.277 300.155695493815
8.278 300.155589422811
8.279 300.155483444748
8.28 300.155377559559
8.281 300.155271767178
8.282 300.155166067539
8.283 300.155060460575
8.284 300.15495494622
8.285 300.154849524408
8.286 300.154744195074
8.287 300.15463895815
8.288 300.154533813572
8.289 300.154428761272
8.29 300.154323801186
8.291 300.154218933247
8.292 300.15411415739
8.293 300.154009473549
8.294 300.153904881658
8.295 300.153800381652
8.296 300.153695973465
8.297 300.153591657033
8.298 300.153487432288
8.299 300.153383299167
8.3 300.153279257603
8.301 300.153175307533
8.302 300.153071448889
8.303 300.152967681608
8.304 300.152864005625
8.305 300.152760420873
8.306 300.15265692729
8.307 300.152553524808
8.308 300.152450213365
8.309 300.152346992894
8.31 300.152243863332
8.311 300.152140824613
8.312 300.152037876673
8.313 300.151935019448
8.314 300.151832252872
8.315 300.151729576883
8.316 300.151626991414
8.317 300.151524496402
8.318 300.151422091783
8.319 300.151319777493
8.32 300.151217553466
8.321 300.15111541964
8.322 300.15101337595
8.323 300.150911422333
8.324 300.150809558723
8.325 300.150707785058
8.326 300.150606101274
8.327 300.150504507307
8.328 300.150403003092
8.329 300.150301588568
8.33 300.150200263669
8.331 300.150099028333
8.332 300.149997882496
8.333 300.149896826094
8.334 300.149795859065
8.335 300.149694981345
8.336 300.14959419287
8.337 300.149493493578
8.338 300.149392883405
8.339 300.149292362288
8.34 300.149191930165
8.341 300.149091586972
8.342 300.148991332646
8.343 300.148891167125
8.344 300.148791090346
8.345 300.148691102246
8.346 300.148591202762
8.347 300.148491391832
8.348 300.148391669393
8.349 300.148292035382
8.35 300.148192489738
8.351 300.148093032397
8.352 300.147993663298
8.353 300.147894382378
8.354 300.147795189575
8.355 300.147696084826
8.356 300.14759706807
8.357 300.147498139244
8.358 300.147399298287
8.359 300.147300545136
8.36 300.14720187973
8.361 300.147103302006
8.362 300.147004811903
8.363 300.146906409359
8.364 300.146808094313
8.365 300.146709866703
8.366 300.146611726467
8.367 300.146513673543
8.368 300.14641570787
8.369 300.146317829388
8.37 300.146220038033
8.371 300.146122333746
8.372 300.146024716464
8.373 300.145927186127
8.374 300.145829742673
8.375 300.145732386042
8.376 300.145635116172
8.377 300.145537933002
8.378 300.145440836471
8.379 300.145343826519
8.38 300.145246903084
8.381 300.145150066107
8.382 300.145053315525
8.383 300.144956651279
8.384 300.144860073307
8.385 300.14476358155
8.386 300.144667175947
8.387 300.144570856437
8.388 300.14447462296
8.389 300.144378475456
8.39 300.144282413864
8.391 300.144186438124
8.392 300.144090548176
8.393 300.14399474396
8.394 300.143899025415
8.395 300.143803392482
8.396 300.143707845101
8.397 300.143612383211
8.398 300.143517006753
8.399 300.143421715668
8.4 300.143326509894
8.401 300.143231389374
8.402 300.143136354046
8.403 300.143041403852
8.404 300.142946538731
8.405 300.142851758625
8.406 300.142757063474
8.407 300.142662453218
8.408 300.142567927798
8.409 300.142473487156
8.41 300.14237913123
8.411 300.142284859964
8.412 300.142190673296
8.413 300.142096571169
8.414 300.142002553523
8.415 300.141908620299
8.416 300.141814771439
8.417 300.141721006883
8.418 300.141627326573
8.419 300.14153373045
8.42 300.141440218454
8.421 300.141346790529
8.422 300.141253446614
8.423 300.141160186652
8.424 300.141067010583
8.425 300.140973918351
8.426 300.140880909895
8.427 300.140787985158
8.428 300.140695144082
8.429 300.140602386608
8.43 300.140509712678
8.431 300.140417122234
8.432 300.140324615217
8.433 300.140232191571
8.434 300.140139851237
8.435 300.140047594156
8.436 300.139955420272
8.437 300.139863329526
8.438 300.13977132186
8.439 300.139679397217
8.44 300.139587555539
8.441 300.139495796769
8.442 300.139404120848
8.443 300.13931252772
8.444 300.139221017327
8.445 300.139129589611
8.446 300.139038244516
8.447 300.138946981983
8.448 300.138855801957
8.449 300.138764704378
8.45 300.138673689191
8.451 300.138582756338
8.452 300.138491905762
8.453 300.138401137407
8.454 300.138310451214
8.455 300.138219847128
8.456 300.138129325092
8.457 300.138038885048
8.458 300.13794852694
8.459 300.137858250711
8.46 300.137768056305
8.461 300.137677943665
8.462 300.137587912735
8.463 300.137497963457
8.464 300.137408095776
8.465 300.137318309635
8.466 300.137228604978
8.467 300.137138981748
8.468 300.13704943989
8.469 300.136959979346
8.47 300.136870600062
8.471 300.136781301981
8.472 300.136692085047
8.473 300.136602949203
8.474 300.136513894395
8.475 300.136424920565
8.476 300.136336027659
8.477 300.13624721562
8.478 300.136158484393
8.479 300.136069833922
8.48 300.135981264152
8.481 300.135892775026
8.482 300.13580436649
8.483 300.135716038488
8.484 300.135627790964
8.485 300.135539623863
8.486 300.135451537129
8.487 300.135363530708
8.488 300.135275604545
8.489 300.135187758583
8.49 300.135099992769
8.491 300.135012307046
8.492 300.13492470136
8.493 300.134837175655
8.494 300.134749729878
8.495 300.134662363972
8.496 300.134575077884
8.497 300.134487871557
8.498 300.134400744939
8.499 300.134313697973
8.5 300.134226730606
8.501 300.134139842782
8.502 300.134053034448
8.503 300.133966305548
8.504 300.133879656029
8.505 300.133793085835
8.506 300.133706594913
8.507 300.133620183208
8.508 300.133533850667
8.509 300.133447597234
8.51 300.133361422855
8.511 300.133275327477
8.512 300.133189311046
8.513 300.133103373507
8.514 300.133017514807
8.515 300.132931734891
8.516 300.132846033707
8.517 300.132760411199
8.518 300.132674867314
8.519 300.132589401999
8.52 300.1325040152
8.521 300.132418706864
8.522 300.132333476936
8.523 300.132248325363
8.524 300.132163252092
8.525 300.132078257069
8.526 300.131993340241
8.527 300.131908501555
8.528 300.131823740957
8.529 300.131739058395
8.53 300.131654453814
8.531 300.131569927162
8.532 300.131485478386
8.533 300.131401107433
8.534 300.13131681425
8.535 300.131232598783
8.536 300.131148460981
8.537 300.13106440079
8.538 300.130980418157
8.539 300.13089651303
8.54 300.130812685356
8.541 300.130728935083
8.542 300.130645262157
8.543 300.130561666526
8.544 300.130478148138
8.545 300.13039470694
8.546 300.13031134288
8.547 300.130228055906
8.548 300.130144845965
8.549 300.130061713005
8.55 300.129978656973
8.551 300.129895677818
8.552 300.129812775487
8.553 300.129729949929
8.554 300.129647201091
8.555 300.129564528921
8.556 300.129481933368
8.557 300.129399414379
8.558 300.129316971903
8.559 300.129234605888
8.56 300.129152316283
8.561 300.129070103035
8.562 300.128987966092
8.563 300.128905905404
8.564 300.128823920919
8.565 300.128742012585
8.566 300.12866018035
8.567 300.128578424164
8.568 300.128496743975
8.569 300.128415139732
8.57 300.128333611383
8.571 300.128252158878
8.572 300.128170782164
8.573 300.128089481192
8.574 300.128008255909
8.575 300.127927106265
8.576 300.127846032209
8.577 300.12776503369
8.578 300.127684110657
8.579 300.127603263059
8.58 300.127522490845
8.581 300.127441793965
8.582 300.127361172368
8.583 300.127280626003
8.584 300.127200154819
8.585 300.127119758767
8.586 300.127039437795
8.587 300.126959191854
8.588 300.126879020892
8.589 300.126798924859
8.59 300.126718903705
8.591 300.12663895738
8.592 300.126559085833
8.593 300.126479289014
8.594 300.126399566873
8.595 300.126319919361
8.596 300.126240346426
8.597 300.126160848019
8.598 300.12608142409
8.599 300.126002074589
8.6 300.125922799466
8.601 300.125843598671
8.602 300.125764472155
8.603 300.125685419867
8.604 300.125606441759
8.605 300.12552753778
8.606 300.125448707881
8.607 300.125369952013
8.608 300.125291270125
8.609 300.125212662168
8.61 300.125134128093
8.611 300.125055667851
8.612 300.124977281392
8.613 300.124898968667
8.614 300.124820729627
8.615 300.124742564222
8.616 300.124664472404
8.617 300.124586454122
8.618 300.124508509329
8.619 300.124430637976
8.62 300.124352840012
8.621 300.12427511539
8.622 300.12419746406
8.623 300.124119885974
8.624 300.124042381083
8.625 300.123964949337
8.626 300.12388759069
8.627 300.123810305091
8.628 300.123733092492
8.629 300.123655952845
8.63 300.123578886101
8.631 300.123501892211
8.632 300.123424971128
8.633 300.123348122803
8.634 300.123271347187
8.635 300.123194644233
8.636 300.123118013891
8.637 300.123041456115
8.638 300.122964970854
8.639 300.122888558063
8.64 300.122812217692
8.641 300.122735949693
8.642 300.122659754019
8.643 300.122583630622
8.644 300.122507579453
8.645 300.122431600465
8.646 300.12235569361
8.647 300.12227985884
8.648 300.122204096109
8.649 300.122128405367
8.65 300.122052786567
8.651 300.121977239662
8.652 300.121901764604
8.653 300.121826361346
8.654 300.12175102984
8.655 300.12167577004
8.656 300.121600581896
8.657 300.121525465363
8.658 300.121450420393
8.659 300.121375446939
8.66 300.121300544954
8.661 300.121225714389
8.662 300.121150955199
8.663 300.121076267337
8.664 300.121001650755
8.665 300.120927105406
8.666 300.120852631243
8.667 300.12077822822
8.668 300.12070389629
8.669 300.120629635406
8.67 300.120555445521
8.671 300.120481326589
8.672 300.120407278562
8.673 300.120333301395
8.674 300.120259395041
8.675 300.120185559452
8.676 300.120111794584
8.677 300.120038100389
8.678 300.11996447682
8.679 300.119890923833
8.68 300.119817441379
8.681 300.119744029413
8.682 300.11967068789
8.683 300.119597416762
8.684 300.119524215983
8.685 300.119451085508
8.686 300.11937802529
8.687 300.119305035283
8.688 300.119232115442
8.689 300.11915926572
8.69 300.119086486072
8.691 300.119013776452
8.692 300.118941136814
8.693 300.118868567112
8.694 300.1187960673
8.695 300.118723637334
8.696 300.118651277166
8.697 300.118578986753
8.698 300.118506766047
8.699 300.118434615005
8.7 300.11836253358
8.701 300.118290521726
8.702 300.118218579399
8.703 300.118146706554
8.704 300.118074903144
8.705 300.118003169125
8.706 300.117931504452
8.707 300.117859909079
8.708 300.117788382961
8.709 300.117716926054
8.71 300.117645538312
8.711 300.11757421969
8.712 300.117502970143
8.713 300.117431789627
8.714 300.117360678097
8.715 300.117289635507
8.716 300.117218661814
8.717 300.117147756972
8.718 300.117076920936
8.719 300.117006153662
8.72 300.116935455106
8.721 300.116864825223
8.722 300.116794263968
8.723 300.116723771297
8.724 300.116653347166
8.725 300.116582991529
8.726 300.116512704344
8.727 300.116442485565
8.728 300.116372335148
8.729 300.116302253049
8.73 300.116232239224
8.731 300.116162293628
8.732 300.116092416219
8.733 300.116022606951
8.734 300.11595286578
8.735 300.115883192663
8.736 300.115813587556
8.737 300.115744050415
8.738 300.115674581195
8.739 300.115605179854
8.74 300.115535846347
8.741 300.115466580631
8.742 300.115397382662
8.743 300.115328252397
8.744 300.115259189791
8.745 300.115190194802
8.746 300.115121267385
8.747 300.115052407498
8.748 300.114983615096
8.749 300.114914890137
8.75 300.114846232578
8.751 300.114777642374
8.752 300.114709119483
8.753 300.114640663861
8.754 300.114572275466
8.755 300.114503954254
8.756 300.114435700182
8.757 300.114367513207
8.758 300.114299393286
8.759 300.114231340376
8.76 300.114163354434
8.761 300.114095435417
8.762 300.114027583282
8.763 300.113959797987
8.764 300.113892079489
8.765 300.113824427745
8.766 300.113756842713
8.767 300.113689324349
8.768 300.113621872611
8.769 300.113554487457
8.77 300.113487168844
8.771 300.11341991673
8.772 300.113352731072
8.773 300.113285611827
8.774 300.113218558954
8.775 300.11315157241
8.776 300.113084652153
8.777 300.11301779814
8.778 300.11295101033
8.779 300.112884288681
8.78 300.112817633149
8.781 300.112751043693
8.782 300.112684520272
8.783 300.112618062842
8.784 300.112551671363
8.785 300.112485345792
8.786 300.112419086087
8.787 300.112352892207
8.788 300.112286764109
8.789 300.112220701752
8.79 300.112154705095
8.791 300.112088774095
8.792 300.112022908711
8.793 300.111957108902
8.794 300.111891374625
8.795 300.11182570584
8.796 300.111760102504
8.797 300.111694564577
8.798 300.111629092017
8.799 300.111563684782
8.8 300.111498342832
8.801 300.111433066125
8.802 300.11136785462
8.803 300.111302708275
8.804 300.11123762705
8.805 300.111172610903
8.806 300.111107659794
8.807 300.111042773681
8.808 300.110977952524
8.809 300.11091319628
8.81 300.110848504911
8.811 300.110783878374
8.812 300.110719316628
8.813 300.110654819634
8.814 300.11059038735
8.815 300.110526019736
8.816 300.11046171675
8.817 300.110397478353
8.818 300.110333304503
8.819 300.11026919516
8.82 300.110205150284
8.821 300.110141169834
8.822 300.110077253769
8.823 300.11001340205
8.824 300.109949614636
8.825 300.109885891486
8.826 300.109822232561
8.827 300.10975863782
8.828 300.109695107222
8.829 300.109631640728
8.83 300.109568238298
8.831 300.109504899891
8.832 300.109441625468
8.833 300.109378414988
8.834 300.109315268412
8.835 300.109252185699
8.836 300.10918916681
8.837 300.109126211705
8.838 300.109063320343
8.839 300.109000492687
8.84 300.108937728694
8.841 300.108875028327
8.842 300.108812391544
8.843 300.108749818308
8.844 300.108687308577
8.845 300.108624862312
8.846 300.108562479475
8.847 300.108500160025
8.848 300.108437903923
8.849 300.10837571113
8.85 300.108313581606
8.851 300.108251515312
8.852 300.108189512209
8.853 300.108127572257
8.854 300.108065695417
8.855 300.108003881651
8.856 300.107942130919
8.857 300.107880443182
8.858 300.1078188184
8.859 300.107757256536
8.86 300.10769575755
8.861 300.107634321403
8.862 300.107572948056
8.863 300.10751163747
8.864 300.107450389607
8.865 300.107389204427
8.866 300.107328081893
8.867 300.107267021965
8.868 300.107206024605
8.869 300.107145089774
8.87 300.107084217433
8.871 300.107023407544
8.872 300.106962660069
8.873 300.106901974969
8.874 300.106841352205
8.875 300.106780791739
8.876 300.106720293534
8.877 300.106659857549
8.878 300.106599483748
8.879 300.106539172092
8.88 300.106478922542
8.881 300.106418735062
8.882 300.106358609611
8.883 300.106298546154
8.884 300.10623854465
8.885 300.106178605063
8.886 300.106118727354
8.887 300.106058911486
8.888 300.10599915742
8.889 300.105939465119
8.89 300.105879834545
8.891 300.105820265659
8.892 300.105760758426
8.893 300.105701312806
8.894 300.105641928761
8.895 300.105582606256
8.896 300.10552334525
8.897 300.105464145708
8.898 300.105405007592
8.899 300.105345930864
8.9 300.105286915486
8.901 300.105227961422
8.902 300.105169068633
8.903 300.105110237084
8.904 300.105051466735
8.905 300.104992757551
8.906 300.104934109493
8.907 300.104875522526
8.908 300.10481699661
8.909 300.10475853171
8.91 300.104700127789
8.911 300.104641784808
8.912 300.104583502732
8.913 300.104525281523
8.914 300.104467121145
8.915 300.10440902156
8.916 300.104350982732
8.917 300.104293004623
8.918 300.104235087198
8.919 300.104177230418
8.92 300.104119434249
8.921 300.104061698652
8.922 300.104004023591
8.923 300.103946409031
8.924 300.103888854933
8.925 300.103831361262
8.926 300.103773927981
8.927 300.103716555053
8.928 300.103659242443
8.929 300.103601990114
8.93 300.103544798029
8.931 300.103487666152
8.932 300.103430594448
8.933 300.103373582879
8.934 300.103316631409
8.935 300.103259740003
8.936 300.103202908625
8.937 300.103146137237
8.938 300.103089425804
8.939 300.103032774291
8.94 300.102976182661
8.941 300.102919650878
8.942 300.102863178906
8.943 300.102806766709
8.944 300.102750414252
8.945 300.102694121499
8.946 300.102637888414
8.947 300.102581714962
8.948 300.102525601105
8.949 300.10246954681
8.95 300.102413552041
8.951 300.102357616761
8.952 300.102301740935
8.953 300.102245924528
8.954 300.102190167505
8.955 300.102134469829
8.956 300.102078831465
8.957 300.102023252379
8.958 300.101967732535
8.959 300.101912271896
8.96 300.10185687043
8.961 300.101801528099
8.962 300.101746244868
8.963 300.101691020704
8.964 300.10163585557
8.965 300.101580749432
8.966 300.101525702254
8.967 300.101470714002
8.968 300.10141578464
8.969 300.101360914134
8.97 300.101306102448
8.971 300.101251349548
8.972 300.1011966554
8.973 300.101142019967
8.974 300.101087443216
8.975 300.101032925112
8.976 300.100978465619
8.977 300.100924064705
8.978 300.100869722333
8.979 300.100815438469
8.98 300.100761213078
8.981 300.100707046127
8.982 300.100652937581
8.983 300.100598887405
8.984 300.100544895565
8.985 300.100490962026
8.986 300.100437086755
8.987 300.100383269716
8.988 300.100329510876
8.989 300.100275810201
8.99 300.100222167655
8.991 300.100168583206
8.992 300.100115056819
8.993 300.100061588459
8.994 300.100008178094
8.995 300.099954825688
8.996 300.099901531208
8.997 300.09984829462
8.998 300.099795115889
8.999 300.099741994983
9 300.099688931867
9.001 300.099635926508
9.002 300.099582978871
9.003 300.099530088924
9.004 300.099477256631
9.005 300.099424481961
9.006 300.099371764878
9.007 300.099319105349
9.008 300.099266503342
9.009 300.099213958822
9.01 300.099161471755
9.011 300.099109042109
9.012 300.09905666985
9.013 300.099004354945
9.014 300.098952097359
9.015 300.098899897061
9.016 300.098847754016
9.017 300.098795668192
9.018 300.098743639555
9.019 300.098691668071
9.02 300.098639753709
9.021 300.098587896434
9.022 300.098536096214
9.023 300.098484353016
9.024 300.098432666806
9.025 300.098381037552
9.026 300.098329465221
9.027 300.098277949779
9.028 300.098226491194
9.029 300.098175089433
9.03 300.098123744464
9.031 300.098072456253
9.032 300.098021224768
9.033 300.097970049976
9.034 300.097918931845
9.035 300.097867870341
9.036 300.097816865433
9.037 300.097765917087
9.038 300.097715025271
9.039 300.097664189954
9.04 300.097613411101
9.041 300.097562688681
9.042 300.097512022661
9.043 300.09746141301
9.044 300.097410859694
9.045 300.097360362681
9.046 300.09730992194
9.047 300.097259537438
9.048 300.097209209143
9.049 300.097158937022
9.05 300.097108721043
9.051 300.097058561176
9.052 300.097008457386
9.053 300.096958409643
9.054 300.096908417914
9.055 300.096858482168
9.056 300.096808602372
9.057 300.096758778495
9.058 300.096709010504
9.059 300.096659298369
9.06 300.096609642057
9.061 300.096560041536
9.062 300.096510496775
9.063 300.096461007742
9.064 300.096411574406
9.065 300.096362196734
9.066 300.096312874695
9.067 300.096263608258
9.068 300.096214397391
9.069 300.096165242063
9.07 300.096116142242
9.071 300.096067097897
9.072 300.096018108997
9.073 300.095969175509
9.074 300.095920297404
9.075 300.095871474649
9.076 300.095822707213
9.077 300.095773995066
9.078 300.095725338175
9.079 300.09567673651
9.08 300.09562819004
9.081 300.095579698733
9.082 300.095531262559
9.083 300.095482881487
9.084 300.095434555484
9.085 300.095386284522
9.086 300.095338068568
9.087 300.095289907592
9.088 300.095241801562
9.089 300.095193750449
9.09 300.095145754222
9.091 300.095097812848
9.092 300.095049926299
9.093 300.095002094543
9.094 300.094954317549
9.095 300.094906595287
9.096 300.094858927727
9.097 300.094811314838
9.098 300.094763756588
9.099 300.094716252949
9.1 300.094668803889
9.101 300.094621409378
9.102 300.094574069385
9.103 300.094526783881
9.104 300.094479552834
9.105 300.094432376215
9.106 300.094385253994
9.107 300.09433818614
9.108 300.094291172622
9.109 300.094244213411
9.11 300.094197308477
9.111 300.09415045779
9.112 300.094103661319
9.113 300.094056919035
9.114 300.094010230907
9.115 300.093963596905
9.116 300.093917017
9.117 300.093870491162
9.118 300.093824019361
9.119 300.093777601566
9.12 300.093731237749
9.121 300.093684927879
9.122 300.093638671927
9.123 300.093592469863
9.124 300.093546321658
9.125 300.093500227281
9.126 300.093454186703
9.127 300.093408199894
9.128 300.093362266826
9.129 300.093316387468
9.13 300.093270561791
9.131 300.093224789766
9.132 300.093179071362
9.133 300.093133406552
9.134 300.093087795305
9.135 300.093042237592
9.136 300.092996733384
9.137 300.092951282652
9.138 300.092905885365
9.139 300.092860541496
9.14 300.092815251016
9.141 300.092770013893
9.142 300.092724830101
9.143 300.09267969961
9.144 300.09263462239
9.145 300.092589598413
9.146 300.09254462765
9.147 300.092499710072
9.148 300.092454845649
9.149 300.092410034354
9.15 300.092365276157
9.151 300.092320571029
9.152 300.092275918942
9.153 300.092231319866
9.154 300.092186773774
9.155 300.092142280637
9.156 300.092097840425
9.157 300.092053453111
9.158 300.092009118666
9.159 300.09196483706
9.16 300.091920608266
9.161 300.091876432256
9.162 300.091832309
9.163 300.09178823847
9.164 300.091744220639
9.165 300.091700255476
9.166 300.091656342955
9.167 300.091612483047
9.168 300.091568675723
9.169 300.091524920956
9.17 300.091481218717
9.171 300.091437568978
9.172 300.091393971711
9.173 300.091350426888
9.174 300.09130693448
9.175 300.09126349446
9.176 300.0912201068
9.177 300.091176771471
9.178 300.091133488446
9.179 300.091090257697
9.18 300.091047079196
9.181 300.091003952914
9.182 300.090960878825
9.183 300.090917856901
9.184 300.090874887112
9.185 300.090831969433
9.186 300.090789103835
9.187 300.09074629029
9.188 300.090703528772
9.189 300.090660819251
9.19 300.090618161701
9.191 300.090575556095
9.192 300.090533002404
9.193 300.090490500601
9.194 300.090448050659
9.195 300.09040565255
9.196 300.090363306247
9.197 300.090321011722
9.198 300.090278768949
9.199 300.090236577899
9.2 300.090194438546
9.201 300.090152350863
9.202 300.090110314822
9.203 300.090068330395
9.204 300.090026397557
9.205 300.089984516279
9.206 300.089942686535
9.207 300.089900908298
9.208 300.08985918154
9.209 300.089817506235
9.21 300.089775882355
9.211 300.089734309874
9.212 300.089692788765
9.213 300.089651319001
9.214 300.089609900555
9.215 300.089568533401
9.216 300.089527217511
9.217 300.089485952858
9.218 300.089444739417
9.219 300.08940357716
9.22 300.089362466062
9.221 300.089321406094
9.222 300.08928039723
9.223 300.089239439445
9.224 300.089198532711
9.225 300.089157677002
9.226 300.089116872291
9.227 300.089076118552
9.228 300.08903541576
9.229 300.088994763886
9.23 300.088954162905
9.231 300.088913612791
9.232 300.088873113517
9.233 300.088832665057
9.234 300.088792267384
9.235 300.088751920473
9.236 300.088711624298
9.237 300.088671378832
9.238 300.088631184049
9.239 300.088591039923
9.24 300.088550946427
9.241 300.088510903537
9.242 300.088470911226
9.243 300.088430969468
9.244 300.088391078238
9.245 300.088351237508
9.246 300.088311447254
9.247 300.08827170745
9.248 300.088232018069
9.249 300.088192379086
9.25 300.088152790475
9.251 300.088113252211
9.252 300.088073764268
9.253 300.088034326619
9.254 300.087994939241
9.255 300.087955602106
9.256 300.087916315189
9.257 300.087877078466
9.258 300.08783789191
9.259 300.087798755495
9.26 300.087759669197
9.261 300.08772063299
9.262 300.087681646849
9.263 300.087642710748
9.264 300.087603824661
9.265 300.087564988565
9.266 300.087526202432
9.267 300.087487466239
9.268 300.087448779959
9.269 300.087410143568
9.27 300.087371557041
9.271 300.087333020352
9.272 300.087294533477
9.273 300.08725609639
9.274 300.087217709066
9.275 300.08717937148
9.276 300.087141083608
9.277 300.087102845424
9.278 300.087064656903
9.279 300.087026518021
9.28 300.086988428753
9.281 300.086950389073
9.282 300.086912398958
9.283 300.086874458382
9.284 300.086836567321
9.285 300.086798725749
9.286 300.086760933643
9.287 300.086723190977
9.288 300.086685497727
9.289 300.086647853868
9.29 300.086610259376
9.291 300.086572714226
9.292 300.086535218393
9.293 300.086497771854
9.294 300.086460374584
9.295 300.086423026558
9.296 300.086385727752
9.297 300.086348478142
9.298 300.086311277703
9.299 300.086274126411
9.3 300.086237024241
9.301 300.086199971171
9.302 300.086162967174
9.303 300.086126012228
9.304 300.086089106308
9.305 300.086052249389
9.306 300.086015441448
9.307 300.085978682461
9.308 300.085941972404
9.309 300.085905311252
9.31 300.085868698982
9.311 300.085832135569
9.312 300.085795620991
9.313 300.085759155222
9.314 300.085722738239
9.315 300.085686370019
9.316 300.085650050537
9.317 300.08561377977
9.318 300.085577557693
9.319 300.085541384284
9.32 300.085505259518
9.321 300.085469183372
9.322 300.085433155822
9.323 300.085397176844
9.324 300.085361246416
9.325 300.085325364513
9.326 300.085289531112
9.327 300.08525374619
9.328 300.085218009722
9.329 300.085182321686
9.33 300.085146682059
9.331 300.085111090815
9.332 300.085075547934
9.333 300.08504005339
9.334 300.085004607162
9.335 300.084969209225
9.336 300.084933859556
9.337 300.084898558133
9.338 300.084863304931
9.339 300.084828099929
9.34 300.084792943102
9.341 300.084757834428
9.342 300.084722773883
9.343 300.084687761445
9.344 300.08465279709
9.345 300.084617880796
9.346 300.08458301254
9.347 300.084548192299
9.348 300.084513420049
9.349 300.084478695768
9.35 300.084444019433
9.351 300.084409391022
9.352 300.084374810511
9.353 300.084340277878
9.354 300.0843057931
9.355 300.084271356155
9.356 300.084236967019
9.357 300.08420262567
9.358 300.084168332086
9.359 300.084134086244
9.36 300.084099888121
9.361 300.084065737695
9.362 300.084031634943
9.363 300.083997579843
9.364 300.083963572372
9.365 300.083929612509
9.366 300.08389570023
9.367 300.083861835514
9.368 300.083828018337
9.369 300.083794248678
9.37 300.083760526515
9.371 300.083726851825
9.372 300.083693224586
9.373 300.083659644775
9.374 300.083626112371
9.375 300.083592627352
9.376 300.083559189695
9.377 300.083525799379
9.378 300.08349245638
9.379 300.083459160678
9.38 300.083425912251
9.381 300.083392711076
9.382 300.083359557131
9.383 300.083326450395
9.384 300.083293390845
9.385 300.08326037846
9.386 300.083227413218
9.387 300.083194495097
9.388 300.083161624076
9.389 300.083128800132
9.39 300.083096023245
9.391 300.083063293391
9.392 300.083030610551
9.393 300.082997974701
9.394 300.082965385821
9.395 300.082932843888
9.396 300.082900348882
9.397 300.08286790078
9.398 300.082835499562
9.399 300.082803145205
9.4 300.082770837689
9.401 300.082738576992
9.402 300.082706363092
9.403 300.082674195969
9.404 300.0826420756
9.405 300.082610001965
9.406 300.082577975043
9.407 300.082545994811
9.408 300.082514061249
9.409 300.082482174336
9.41 300.082450334051
9.411 300.082418540372
9.412 300.082386793278
9.413 300.082355092748
9.414 300.082323438762
9.415 300.082291831297
9.416 300.082260270334
9.417 300.082228755851
9.418 300.082197287827
9.419 300.082165866242
9.42 300.082134491074
9.421 300.082103162303
9.422 300.082071879907
9.423 300.082040643866
9.424 300.08200945416
9.425 300.081978310767
9.426 300.081947213666
9.427 300.081916162838
9.428 300.081885158261
9.429 300.081854199915
9.43 300.081823287779
9.431 300.081792421833
9.432 300.081761602055
9.433 300.081730828426
9.434 300.081700100925
9.435 300.081669419532
9.436 300.081638784225
9.437 300.081608194985
9.438 300.081577651791
9.439 300.081547154623
9.44 300.081516703461
9.441 300.081486298284
9.442 300.081455939071
9.443 300.081425625804
9.444 300.08139535846
9.445 300.081365137021
9.446 300.081334961466
9.447 300.081304831775
9.448 300.081274747928
9.449 300.081244709904
9.45 300.081214717684
9.451 300.081184771247
9.452 300.081154870574
9.453 300.081125015644
9.454 300.081095206438
9.455 300.081065442935
9.456 300.081035725116
9.457 300.081006052961
9.458 300.08097642645
9.459 300.080946845563
9.46 300.080917310281
9.461 300.080887820582
9.462 300.080858376449
9.463 300.080828977861
9.464 300.080799624798
9.465 300.08077031724
9.466 300.080741055169
9.467 300.080711838564
9.468 300.080682667406
9.469 300.080653541675
9.47 300.080624461352
9.471 300.080595426418
9.472 300.080566436852
9.473 300.080537492635
9.474 300.080508593748
9.475 300.080479740172
9.476 300.080450931887
9.477 300.080422168873
9.478 300.080393451112
9.479 300.080364778584
9.48 300.08033615127
9.481 300.08030756915
9.482 300.080279032206
9.483 300.080250540418
9.484 300.080222093767
9.485 300.080193692234
9.486 300.0801653358
9.487 300.080137024445
9.488 300.080108758151
9.489 300.080080536898
9.49 300.080052360668
9.491 300.080024229442
9.492 300.079996143201
9.493 300.079968101925
9.494 300.079940105595
9.495 300.079912154194
9.496 300.079884247702
9.497 300.079856386101
9.498 300.07982856937
9.499 300.079800797493
9.5 300.079773070449
9.501 300.07974538822
9.502 300.079717750788
9.503 300.079690158134
9.504 300.079662610239
9.505 300.079635107085
9.506 300.079607648652
9.507 300.079580234923
9.508 300.079552865879
9.509 300.079525541502
9.51 300.079498261772
9.511 300.079471026672
9.512 300.079443836183
9.513 300.079416690287
9.514 300.079389588964
9.515 300.079362532198
9.516 300.07933551997
9.517 300.07930855226
9.518 300.079281629052
9.519 300.079254750327
9.52 300.079227916066
9.521 300.079201126251
9.522 300.079174380865
9.523 300.079147679888
9.524 300.079121023304
9.525 300.079094411093
9.526 300.079067843238
9.527 300.079041319721
9.528 300.079014840523
9.529 300.078988405628
9.53 300.078962015016
9.531 300.078935668669
9.532 300.078909366571
9.533 300.078883108703
9.534 300.078856895046
9.535 300.078830725585
9.536 300.078804600299
9.537 300.078778519172
9.538 300.078752482187
9.539 300.078726489324
9.54 300.078700540567
9.541 300.078674635898
9.542 300.078648775299
9.543 300.078622958753
9.544 300.078597186241
9.545 300.078571457747
9.546 300.078545773253
9.547 300.078520132741
9.548 300.078494536194
9.549 300.078468983594
9.55 300.078443474924
9.551 300.078418010167
9.552 300.078392589304
9.553 300.078367212319
9.554 300.078341879195
9.555 300.078316589913
9.556 300.078291344457
9.557 300.07826614281
9.558 300.078240984953
9.559 300.078215870871
9.56 300.078190800545
9.561 300.078165773959
9.562 300.078140791096
9.563 300.078115851938
9.564 300.078090956468
9.565 300.078066104669
9.566 300.078041296524
9.567 300.078016532017
9.568 300.077991811129
9.569 300.077967133845
9.57 300.077942500147
9.571 300.077917910018
9.572 300.077893363441
9.573 300.0778688604
9.574 300.077844400878
9.575 300.077819984857
9.576 300.077795612322
9.577 300.077771283255
9.578 300.077746997639
9.579 300.077722755459
9.58 300.077698556696
9.581 300.077674401336
9.582 300.07765028936
9.583 300.077626220752
9.584 300.077602195496
9.585 300.077578213575
9.586 300.077554274973
9.587 300.077530379673
9.588 300.077506527658
9.589 300.077482718913
9.59 300.07745895342
9.591 300.077435231164
9.592 300.077411552127
9.593 300.077387916294
9.594 300.077364323648
9.595 300.077340774174
9.596 300.077317267853
9.597 300.077293804671
9.598 300.077270384611
9.599 300.077247007657
9.6 300.077223673793
9.601 300.077200383002
9.602 300.077177135269
9.603 300.077153930577
9.604 300.07713076891
9.605 300.077107650252
9.606 300.077084574588
9.607 300.0770615419
9.608 300.077038552174
9.609 300.077015605393
9.61 300.076992701542
9.611 300.076969840603
9.612 300.076947022562
9.613 300.076924247403
9.614 300.076901515109
9.615 300.076878825665
9.616 300.076856179055
9.617 300.076833575264
9.618 300.076811014275
9.619 300.076788496073
9.62 300.076766020643
9.621 300.076743587967
9.622 300.076721198032
9.623 300.076698850821
9.624 300.076676546319
9.625 300.07665428451
9.626 300.076632065379
9.627 300.076609888909
9.628 300.076587755087
9.629 300.076565663895
9.63 300.076543615319
9.631 300.076521609343
9.632 300.076499645952
9.633 300.076477725131
9.634 300.076455846864
9.635 300.076434011135
9.636 300.076412217931
9.637 300.076390467234
9.638 300.076368759031
9.639 300.076347093305
9.64 300.076325470042
9.641 300.076303889227
9.642 300.076282350844
9.643 300.076260854878
9.644 300.076239401314
9.645 300.076217990137
9.646 300.076196621332
9.647 300.076175294885
9.648 300.076154010779
9.649 300.076132769
9.65 300.076111569533
9.651 300.076090412363
9.652 300.076069297475
9.653 300.076048224855
9.654 300.076027194487
9.655 300.076006206357
9.656 300.075985260449
9.657 300.07596435675
9.658 300.075943495244
9.659 300.075922675916
9.66 300.075901898753
9.661 300.075881163738
9.662 300.075860470858
9.663 300.075839820098
9.664 300.075819211444
9.665 300.07579864488
9.666 300.075778120392
9.667 300.075757637966
9.668 300.075737197587
9.669 300.075716799241
9.67 300.075696442913
9.671 300.075676128589
9.672 300.075655856254
9.673 300.075635625894
9.674 300.075615437494
9.675 300.075595291041
9.676 300.075575186519
9.677 300.075555123915
9.678 300.075535103215
9.679 300.075515124404
9.68 300.075495187467
9.681 300.075475292391
9.682 300.075455439161
9.683 300.075435627764
9.684 300.075415858185
9.685 300.075396130411
9.686 300.075376444426
9.687 300.075356800217
9.688 300.07533719777
9.689 300.075317637071
9.69 300.075298118106
9.691 300.075278640861
9.692 300.075259205322
9.693 300.075239811475
9.694 300.075220459306
9.695 300.075201148802
9.696 300.075181879949
9.697 300.075162652732
9.698 300.075143467138
9.699 300.075124323153
9.7 300.075105220764
9.701 300.075086159957
9.702 300.075067140718
9.703 300.075048163033
9.704 300.075029226889
9.705 300.075010332272
9.706 300.074991479168
9.707 300.074972667564
9.708 300.074953897447
9.709 300.074935168803
9.71 300.074916481618
9.711 300.074897835878
9.712 300.074879231572
9.713 300.074860668684
9.714 300.074842147202
9.715 300.074823667112
9.716 300.074805228401
9.717 300.074786831055
9.718 300.074768475062
9.719 300.074750160408
9.72 300.074731887079
9.721 300.074713655063
9.722 300.074695464346
9.723 300.074677314915
9.724 300.074659206756
9.725 300.074641139858
9.726 300.074623114206
9.727 300.074605129787
9.728 300.074587186589
9.729 300.074569284599
9.73 300.074551423802
9.731 300.074533604187
9.732 300.07451582574
9.733 300.074498088449
9.734 300.0744803923
9.735 300.074462737281
9.736 300.074445123378
9.737 300.074427550579
9.738 300.074410018871
9.739 300.074392528241
9.74 300.074375078676
9.741 300.074357670164
9.742 300.074340302691
9.743 300.074322976246
9.744 300.074305690815
9.745 300.074288446386
9.746 300.074271242946
9.747 300.074254080483
9.748 300.074236958983
9.749 300.074219878434
9.75 300.074202838825
9.751 300.074185840141
9.752 300.074168882371
9.753 300.074151965502
9.754 300.074135089522
9.755 300.074118254418
9.756 300.074101460178
9.757 300.074084706789
9.758 300.07406799424
9.759 300.074051322517
9.76 300.074034691608
9.761 300.074018101502
9.762 300.074001552185
9.763 300.073985043646
9.764 300.073968575872
9.765 300.073952148852
9.766 300.073935762572
9.767 300.073919417022
9.768 300.073903112188
9.769 300.073886848058
9.77 300.073870624621
9.771 300.073854441864
9.772 300.073838299776
9.773 300.073822198344
9.774 300.073806137557
9.775 300.073790117401
9.776 300.073774137867
9.777 300.07375819894
9.778 300.07374230061
9.779 300.073726442865
9.78 300.073710625693
9.781 300.073694849082
9.782 300.07367911302
9.783 300.073663417495
9.784 300.073647762496
9.785 300.073632148011
9.786 300.073616574028
9.787 300.073601040535
9.788 300.073585547521
9.789 300.073570094975
9.79 300.073554682884
9.791 300.073539311236
9.792 300.073523980022
9.793 300.073508689228
9.794 300.073493438843
9.795 300.073478228856
9.796 300.073463059255
9.797 300.073447930029
9.798 300.073432841166
9.799 300.073417792656
9.8 300.073402784485
9.801 300.073387816645
9.802 300.073372889121
9.803 300.073358001904
9.804 300.073343154983
9.805 300.073328348345
9.806 300.07331358198
9.807 300.073298855876
9.808 300.073284170022
9.809 300.073269524408
9.81 300.073254919021
9.811 300.073240353851
9.812 300.073225828886
9.813 300.073211344116
9.814 300.073196899529
9.815 300.073182495115
9.816 300.073168130862
9.817 300.073153806759
9.818 300.073139522796
9.819 300.073125278961
9.82 300.073111075243
9.821 300.073096911632
9.822 300.073082788117
9.823 300.073068704686
9.824 300.073054661329
9.825 300.073040658035
9.826 300.073026694793
9.827 300.073012771593
9.828 300.072998888424
9.829 300.072985045275
9.83 300.072971242135
9.831 300.072957478994
9.832 300.07294375584
9.833 300.072930072664
9.834 300.072916429455
9.835 300.072902826201
9.836 300.072889262894
9.837 300.072875739521
9.838 300.072862256073
9.839 300.072848812539
9.84 300.072835408908
9.841 300.07282204517
9.842 300.072808721315
9.843 300.072795437331
9.844 300.07278219321
9.845 300.07276898894
9.846 300.072755824511
9.847 300.072742699912
9.848 300.072729615134
9.849 300.072716570166
9.85 300.072703564998
9.851 300.07269059962
9.852 300.072677674021
9.853 300.072664788191
9.854 300.072651942121
9.855 300.072639135799
9.856 300.072626369216
9.857 300.072613642362
9.858 300.072600955226
9.859 300.072588307799
9.86 300.072575700071
9.861 300.072563132031
9.862 300.07255060367
9.863 300.072538114977
9.864 300.072525665943
9.865 300.072513256558
9.866 300.072500886811
9.867 300.072488556694
9.868 300.072476266196
9.869 300.072464015307
9.87 300.072451804017
9.871 300.072439632317
9.872 300.072427500198
9.873 300.072415407648
9.874 300.072403354659
9.875 300.072391341221
9.876 300.072379367324
9.877 300.072367432958
9.878 300.072355538114
9.879 300.072343682783
9.88 300.072331866954
9.881 300.072320090618
9.882 300.072308353766
9.883 300.072296656388
9.884 300.072284998474
9.885 300.072273380016
9.886 300.072261801003
9.887 300.072250261427
9.888 300.072238761277
9.889 300.072227300545
9.89 300.07221587922
9.891 300.072204497295
9.892 300.072193154759
9.893 300.072181851603
9.894 300.072170587818
9.895 300.072159363395
9.896 300.072148178324
9.897 300.072137032596
9.898 300.072125926202
9.899 300.072114859133
9.9 300.072103831379
9.901 300.072092842933
9.902 300.072081893783
9.903 300.072070983922
9.904 300.072060113341
9.905 300.07204928203
9.906 300.07203848998
9.907 300.072027737182
9.908 300.072017023628
9.909 300.072006349309
9.91 300.071995714215
9.911 300.071985118337
9.912 300.071974561668
9.913 300.071964044197
9.914 300.071953565917
9.915 300.071943126817
9.916 300.071932726891
9.917 300.071922366128
9.918 300.07191204452
9.919 300.071901762059
9.92 300.071891518735
9.921 300.07188131454
9.922 300.071871149466
9.923 300.071861023503
9.924 300.071850936643
9.925 300.071840888878
9.926 300.071830880199
9.927 300.071820910597
9.928 300.071810980064
9.929 300.071801088592
9.93 300.071791236172
9.931 300.071781422794
9.932 300.071771648452
9.933 300.071761913137
9.934 300.07175221684
9.935 300.071742559552
9.936 300.071732941267
9.937 300.071723361974
9.938 300.071713821667
9.939 300.071704320336
9.94 300.071694857974
9.941 300.071685434572
9.942 300.071676050122
9.943 300.071666704616
9.944 300.071657398045
9.945 300.071648130402
9.946 300.071638901678
9.947 300.071629711866
9.948 300.071620560957
9.949 300.071611448943
9.95 300.071602375816
9.951 300.071593341568
9.952 300.071584346192
9.953 300.071575389678
9.954 300.07156647202
9.955 300.071557593209
9.956 300.071548753237
9.957 300.071539952097
9.958 300.07153118978
9.959 300.07152246628
9.96 300.071513781587
9.961 300.071505135695
9.962 300.071496528594
9.963 300.071487960279
9.964 300.07147943074
9.965 300.071470939971
9.966 300.071462487963
9.967 300.071454074709
9.968 300.071445700202
9.969 300.071437364432
9.97 300.071429067394
9.971 300.07142080908
9.972 300.071412589481
9.973 300.07140440859
9.974 300.071396266401
9.975 300.071388162905
9.976 300.071380098094
9.977 300.071372071962
9.978 300.071364084501
9.979 300.071356135704
9.98 300.071348225563
9.981 300.071340354071
9.982 300.07133252122
9.983 300.071324727004
9.984 300.071316971415
9.985 300.071309254445
9.986 300.071301576088
9.987 300.071293936336
9.988 300.071286335182
9.989 300.071278772619
9.99 300.071271248639
9.991 300.071263763236
9.992 300.071256316403
9.993 300.071248908131
9.994 300.071241538415
9.995 300.071234207247
9.996 300.07122691462
9.997 300.071219660527
9.998 300.071212444961
9.999 300.071205267916
};
\end{axis}

\end{tikzpicture}

\end{figure}

\sloppy\subsection{График зависимости температуры от времени при нескольких фиксированных значениях координаты}
На рисунке представлены графики зависимости температуры от времени при фиксированных $x = 0, 0.2, 0.4, \ldots, 3.2$.
\begin{figure}[H]
    \caption{Зависимость температуры от времени}
    % This file was created by tikzplotlib v0.9.2.
\begin{tikzpicture}[scale=0.875]

\definecolor{color0}{rgb}{0.83921568627451,0.152941176470588,0.156862745098039}
\definecolor{color1}{rgb}{0.12156862745098,0.466666666666667,0.705882352941177}

\begin{groupplot}[group style={group size=2 by 3, vertical sep=2.5cm, horizontal sep=2.5cm}]
\nextgroupplot[
    x label style={at={(axis description cs:0.5,-0.05)},anchor=north},
    y label style={at={(axis description cs:-0.005,.5)},rotate=0,anchor=south},
tick align=outside,
tick pos=left,
x grid style={white!69.0196078431373!black},
xlabel={Время},
xmin=-2.995e-05, xmax=0.00062895,
xtick style={color=black},
xtick={-0.0002,0,0.0002,0.0004,0.0006,0.0008},
xticklabels={−0.0002,0.0000,0.0002,0.0004,0.0006,0.0008},
y grid style={white!69.0196078431373!black},
ylabel={$I$},
ymin=-39.2367438907993, ymax=830.571621706785,
ytick style={color=black}
]
\addplot [semithick, color0]
table {%
0 0.3
1e-06 6.92799899470395
2e-06 13.6256669742019
3e-06 20.2955945127448
4e-06 26.9098943911363
5e-06 33.4750390025119
6e-06 39.9957082071605
7e-06 46.475726677361
8e-06 52.9088844372231
9e-06 59.2789291552555
1e-05 65.580329688496
1.1e-05 71.8186088998594
1.2e-05 77.9986422720176
1.3e-05 84.1243307345589
1.4e-05 90.1987916588464
1.5e-05 96.2250016679866
1.6e-05 102.205438933802
1.7e-05 108.142145918165
1.8e-05 114.037214699241
1.9e-05 119.892749463643
2e-05 125.710555006264
2.1e-05 131.491880252722
2.2e-05 137.237990356808
2.3e-05 142.950134928437
2.4e-05 148.629271909057
2.5e-05 154.27620556462
2.6e-05 159.891890955236
2.7e-05 165.477235195717
2.8e-05 171.032888973696
2.9e-05 176.55993121512
3e-05 182.061719658813
3.1e-05 187.541294561413
3.2e-05 192.999451714487
3.3e-05 198.436744760735
3.4e-05 203.846785792438
3.5e-05 209.220708543005
3.6e-05 214.55648803037
3.7e-05 219.854821864674
3.8e-05 225.116362352211
3.9e-05 230.341717464285
4e-05 235.531529985967
4.1e-05 240.686478654734
4.2e-05 245.807155117384
4.3e-05 250.894063992494
4.4e-05 255.947688093568
4.5e-05 260.968505619619
4.6e-05 265.956986123797
4.7e-05 270.913564591047
4.8e-05 275.838647149301
4.9e-05 280.732623364404
5e-05 285.595867092908
5.1e-05 290.428746370075
5.2e-05 295.231700804855
5.3e-05 300.005170640729
5.4e-05 304.749496256925
5.5e-05 309.464980956636
5.6e-05 314.15191643872
5.7e-05 318.810613705024
5.8e-05 323.441374530706
5.9e-05 328.044481790136
6e-05 332.620216764204
6.1e-05 337.168829861547
6.2e-05 341.690566847361
6.3e-05 346.185676344516
6.4e-05 350.654384032501
6.5e-05 355.096896214893
6.6e-05 359.513411909206
6.7e-05 363.904123160067
6.8e-05 368.269215336123
6.9e-05 372.608867411717
7e-05 376.92325223427
7.1e-05 381.212536778243
7.2e-05 385.476882386505
7.3e-05 389.716444999861
7.4e-05 393.931375375437
7.5e-05 398.121938432494
7.6e-05 402.287648134583
7.7e-05 406.426646910408
7.8e-05 410.53785914277
7.9e-05 414.621508093876
8e-05 418.677783838527
8.1e-05 422.70685910249
8.2e-05 426.708913244844
8.3e-05 430.684124525976
8.4e-05 434.632648598894
8.5e-05 438.554634316044
8.6e-05 442.450237605885
8.7e-05 446.31961070353
8.8e-05 450.162891725776
8.9e-05 453.980215555736
9e-05 457.77171394626
9.1e-05 461.537515619614
9.2e-05 465.277764176021
9.3e-05 468.992603241097
9.4e-05 472.682155864478
9.5e-05 476.3465391371
9.6e-05 479.985867567381
9.7e-05 483.600253161665
9.8e-05 487.189805501886
9.9e-05 490.754631820576
0.0001 494.294837073326
0.000101 497.810524008801
0.000102 501.301800340571
0.000103 504.768774321695
0.000104 508.211545069522
0.000105 511.63020694407
0.000106 515.024852415937
0.000107 518.395572123613
0.000108 521.742454928944
0.000109 525.065587970826
0.00011 528.365056717195
0.000111 531.640945015385
0.000112 534.8933351409
0.000113 538.122307844678
0.000114 541.327942398887
0.000115 544.510316641321
0.000116 547.669507018436
0.000117 550.805588627088
0.000118 553.918635255003
0.000119 557.008725110465
0.00012 560.075946674238
0.000121 563.120381445012
0.000122 566.142105965814
0.000123 569.141202765479
0.000124 572.117745206367
0.000125 575.071798190756
0.000126 578.003425599508
0.000127 580.912690323051
0.000128 583.799654291484
0.000129 586.66437850382
0.00013 589.506923056414
0.000131 592.32734717059
0.000132 595.125709219495
0.000133 597.902066754218
0.000134 600.65647652918
0.000135 603.388994526827
0.000136 606.099675981657
0.000137 608.788575403588
0.000138 611.455746600697
0.000139 614.101242701344
0.00014 616.725116175712
0.000141 619.327418856764
0.000142 621.908201960654
0.000143 624.467516106591
0.000144 627.005411336189
0.000145 629.521943435827
0.000146 632.017167673313
0.000147 634.491132484617
0.000148 636.943885620774
0.000149 639.375474345661
0.00015 641.785945452522
0.000151 644.175345280075
0.000152 646.543720933478
0.000153 648.891124326212
0.000154 651.217605716136
0.000155 653.523209605953
0.000156 655.807980119529
0.000157 658.071961015709
0.000158 660.315195701797
0.000159 662.537727246696
0.00016 664.739598393728
0.000161 666.920851573145
0.000162 669.081528914325
0.000163 671.221672257676
0.000164 673.341323166256
0.000165 675.440522937103
0.000166 677.519312612298
0.000167 679.577732989762
0.000168 681.615824633792
0.000169 683.633627885349
0.00017 685.631182872097
0.000171 687.608529518208
0.000172 689.56570755393
0.000173 691.502756524937
0.000174 693.419715801449
0.000175 695.316624587148
0.000176 697.193521927875
0.000177 699.050446720134
0.000178 700.887437719391
0.000179 702.704533548182
0.00018 704.50177270403
0.000181 706.279193567189
0.000182 708.036834408192
0.000183 709.774733395244
0.000184 711.492928601428
0.000185 713.191458011755
0.000186 714.870359530052
0.000187 716.529670985688
0.000188 718.169431793704
0.000189 719.789682861186
0.00019 721.390463515394
0.000191 722.971814368201
0.000192 724.533775826813
0.000193 726.076385204353
0.000194 727.59967978362
0.000195 729.103696822672
0.000196 730.588473560287
0.000197 732.054047221295
0.000198 733.500455021796
0.000199 734.927734174249
0.0002 736.33592189246
0.000201 737.725055396445
0.000202 739.095171917185
0.000203 740.446308701281
0.000204 741.778503015496
0.000205 743.091792151193
0.000206 744.386213428679
0.000207 745.661804201444
0.000208 746.918601860311
0.000209 748.156643837481
0.00021 749.375967610499
0.000211 750.576626364789
0.000212 751.758801958387
0.000213 752.922667344281
0.000214 754.06826554376
0.000215 755.1956343366
0.000216 756.304810353354
0.000217 757.395828138375
0.000218 758.468721823508
0.000219 759.523525065521
0.00022 760.560269863807
0.000221 761.578990101956
0.000222 762.579722282766
0.000223 763.562501250104
0.000224 764.527359308786
0.000225 765.474328893834
0.000226 766.403442572891
0.000227 767.314733048576
0.000228 768.208234554418
0.000229 769.083981944775
0.00023 769.942008817921
0.000231 770.782348348742
0.000232 771.605033858561
0.000233 772.410098817159
0.000234 773.197576844752
0.000235 773.967501713912
0.000236 774.719907351444
0.000237 775.454827840208
0.000238 776.172297420906
0.000239 776.872350493811
0.00024 777.55502162046
0.000241 778.220348044449
0.000242 778.868367435121
0.000243 779.499115105664
0.000244 780.112626109653
0.000245 780.708935670663
0.000246 781.288079183708
0.000247 781.850092216626
0.000248 782.395010511431
0.000249 782.922869985627
0.00025 783.433706733485
0.000251 783.927557027283
0.000252 784.404457318514
0.000253 784.864444239049
0.000254 785.307554602281
0.000255 785.733825404219
0.000256 786.143293824563
0.000257 786.535997227737
0.000258 786.911973163894
0.000259 787.271259369886
0.00026 787.613893770214
0.000261 787.939914477929
0.000262 788.249359795524
0.000263 788.54226821578
0.000264 788.818678422596
0.000265 789.078629291778
0.000266 789.322159891815
0.000267 789.549309484613
0.000268 789.760117526217
0.000269 789.954623667491
0.00027 790.132867754786
0.000271 790.294889830576
0.000272 790.440730134067
0.000273 790.570429101786
0.000274 790.684027368148
0.000275 790.781565765988
0.000276 790.863085691795
0.000277 790.928628672348
0.000278 790.978236070163
0.000279 791.011949350695
0.00028 791.029810181063
0.000281 791.031860430448
0.000282 791.018142170475
0.000283 790.988697675573
0.000284 790.943569423309
0.000285 790.882800094712
0.000286 790.806432574572
0.000287 790.7145103566
0.000288 790.607077327676
0.000289 790.484177173132
0.00029 790.345853751918
0.000291 790.19215112711
0.000292 790.023113566106
0.000293 789.838785540794
0.000294 789.639211727713
0.000295 789.424437008187
0.000296 789.194506468451
0.000297 788.949465399749
0.000298 788.689359298431
0.000299 788.41423386602
0.0003 788.124135009269
0.000301 787.819108840205
0.000302 787.499201676151
0.000303 787.164460039738
0.000304 786.814930658901
0.000305 786.45066046686
0.000306 786.071696602085
0.000307 785.678086408248
0.000308 785.269877434162
0.000309 784.847117433706
0.00031 784.409854365731
0.000311 783.958136393963
0.000312 783.492011886882
0.000313 783.011529417597
0.000314 782.516737763697
0.000315 782.007685907105
0.000316 781.484423033902
0.000317 780.946998534152
0.000318 780.395462001712
0.000319 779.82986323402
0.00032 779.250252231887
0.000321 778.656679199268
0.000322 778.04919454302
0.000323 777.427849826802
0.000324 776.792698667129
0.000325 776.143793919753
0.000326 775.481186908608
0.000327 774.8049291585
0.000328 774.115072394808
0.000329 773.411668543166
0.00033 772.694769729137
0.000331 771.964428277876
0.000332 771.220696713783
0.000333 770.463627760148
0.000334 769.693274338782
0.000335 768.909689569643
0.000336 768.112926770448
0.000337 767.303040035186
0.000338 766.480084877155
0.000339 765.644116421754
0.00034 764.795188822923
0.000341 763.933356429955
0.000342 763.058673787057
0.000343 762.171195632889
0.000344 761.270976900117
0.000345 760.358076226804
0.000346 759.432552935788
0.000347 758.494463030726
0.000348 757.543862308007
0.000349 756.580807925326
0.00035 755.605358151969
0.000351 754.617570273132
0.000352 753.617504983008
0.000353 752.605224171024
0.000354 751.580788910707
0.000355 750.544266617744
0.000356 749.495838506067
0.000357 748.43569199497
0.000358 747.363898338934
0.000359 746.280518025881
0.00036 745.185611703141
0.000361 744.079240176746
0.000362 742.961464410695
0.000363 741.832345526228
0.000364 740.691944801074
0.000365 739.540323668701
0.000366 738.377543717551
0.000367 737.203666690265
0.000368 736.0187544829
0.000369 734.82286914414
0.00037 733.616072874486
0.000371 732.398428025453
0.000372 731.169997098743
0.000373 729.930842745416
0.000374 728.681027765054
0.000375 727.420615104904
0.000376 726.149667859028
0.000377 724.868249267428
0.000378 723.576422715177
0.000379 722.274251731526
0.00038 720.96180070823
0.000381 719.639136093648
0.000382 718.306323747538
0.000383 716.963428560201
0.000384 715.610517505714
0.000385 714.247656918423
0.000386 712.874911361857
0.000387 711.492345535387
0.000388 710.100024273248
0.000389 708.698012543563
0.00039 707.286375447343
0.000391 705.865178217495
0.000392 704.434486217802
0.000393 702.994364941914
0.000394 701.544880012311
0.000395 700.08609717927
0.000396 698.618082319816
0.000397 697.140901436668
0.000398 695.65462065717
0.000399 694.15930623222
0.0004 692.655024535185
0.000401 691.141842060806
0.000402 689.619825424099
0.000403 688.089041359238
0.000404 686.54955671844
0.000405 685.001438470829
0.000406 683.4447537013
0.000407 681.879569609366
0.000408 680.305953508003
0.000409 678.723972822476
0.00041 677.133695089167
0.000411 675.535187954383
0.000412 673.928519173162
0.000413 672.313756608062
0.000414 670.690968227949
0.000415 669.060222106766
0.000416 667.421586422301
0.000417 665.775129454937
0.000418 664.1209195864
0.000419 662.459025298487
0.00042 660.789515171799
0.000421 659.112457884445
0.000422 657.427922210753
0.000423 655.735977019959
0.000424 654.036691274896
0.000425 652.330134030659
0.000426 650.616374433275
0.000427 648.895481718351
0.000428 647.167525209717
0.000429 645.432576198907
0.00043 643.690708989179
0.000431 641.941996069637
0.000432 640.186507113875
0.000433 638.424311875959
0.000434 636.65548018899
0.000435 634.880081963674
0.000436 633.098187186867
0.000437 631.309865920113
0.000438 629.515188298175
0.000439 627.714224527548
0.00044 625.907047877749
0.000441 624.093733339035
0.000442 622.274352966601
0.000443 620.44897724209
0.000444 618.617676710492
0.000445 616.780521978577
0.000446 614.937583713329
0.000447 613.088932640363
0.000448 611.234639542328
0.000449 609.3747752573
0.00045 607.509410677163
0.000451 605.638616745975
0.000452 603.762464458325
0.000453 601.881024857673
0.000454 599.994369034682
0.000455 598.102568125531
0.000456 596.205693310223
0.000457 594.303815810872
0.000458 592.397006889982
0.000459 590.48533784871
0.00046 588.568880025115
0.000461 586.6477047924
0.000462 584.721883557128
0.000463 582.791487757438
0.000464 580.856588861235
0.000465 578.917258364378
0.000466 576.97356778884
0.000467 575.025588680866
0.000468 573.073392609111
0.000469 571.117051162764
0.00047 569.156635949658
0.000471 567.192218594364
0.000472 565.223870736274
0.000473 563.251670756814
0.000474 561.275700330239
0.000475 559.296034387276
0.000476 557.312744573169
0.000477 555.325906397717
0.000478 553.335602726915
0.000479 551.341912510005
0.00048 549.344907349981
0.000481 547.344658842683
0.000482 545.341238574708
0.000483 543.334718121301
0.000484 541.325169044228
0.000485 539.312662889636
0.000486 537.297271185897
0.000487 535.279065441431
0.000488 533.258117142519
0.000489 531.234497751094
0.00049 529.208278702518
0.000491 527.179531403343
0.000492 525.148327229051
0.000493 523.114737521784
0.000494 521.078833588051
0.000495 519.040686696416
0.000496 517.000368075181
0.000497 514.957948910034
0.000498 512.913500341696
0.000499 510.867093463538
0.0005 508.818799319187
0.000501 506.768688900116
0.000502 504.716833143209
0.000503 502.663302928314
0.000504 500.608169075777
0.000505 498.551503054001
0.000506 496.493381376147
0.000507 494.433879734778
0.000508 492.373068644735
0.000509 490.311018547575
0.00051 488.247799808992
0.000511 486.183482716223
0.000512 484.118137475426
0.000513 482.051834209053
0.000514 479.98464295319
0.000515 477.916633654893
0.000516 475.847876169493
0.000517 473.778440257893
0.000518 471.708395583836
0.000519 469.637811711165
0.00052 467.566758101056
0.000521 465.495304109236
0.000522 463.423518983185
0.000523 461.351476197234
0.000524 459.279256382608
0.000525 457.206935644924
0.000526 455.13458265647
0.000527 453.062265964559
0.000528 450.990053988616
0.000529 448.918015017241
0.00053 446.84621720526
0.000531 444.774728570756
0.000532 442.703616992079
0.000533 440.632950204846
0.000534 438.562795798913
0.000535 436.493226824247
0.000536 434.424316266095
0.000537 432.356131339105
0.000538 430.288738952861
0.000539 428.22220585458
0.00054 426.156598625989
0.000541 424.091983680179
0.000542 422.028434260355
0.000543 419.966027685441
0.000544 417.904834041124
0.000545 415.844918890598
0.000546 413.786347610024
0.000547 411.729190257368
0.000548 409.673517344941
0.000549 407.619394302169
0.00055 405.56690494687
0.000551 403.516135367118
0.000552 401.46715682917
0.000553 399.419636039219
0.000554 397.372282887795
0.000555 395.324632396883
0.000556 393.277172805281
0.000557 391.22997495684
0.000558 389.183109678534
0.000559 387.136647778075
0.00056 385.090660041491
0.000561 383.045217230661
0.000562 381.000390080825
0.000563 378.956249298045
0.000564 376.912865556644
0.000565 374.870309496587
0.000566 372.828651720845
0.000567 370.787962792704
0.000568 368.748313233039
0.000569 366.709773517552
0.00057 364.672414073959
0.000571 362.636305279149
0.000572 360.601517456286
0.000573 358.568120871879
0.000574 356.536185732802
0.000575 354.505782183275
0.000576 352.476980301793
0.000577 350.449850098017
0.000578 348.424461509615
0.000579 346.400884399056
0.00058 344.379188550357
0.000581 342.359443665782
0.000582 340.341719362493
0.000583 338.326090156876
0.000584 336.312637232167
0.000585 334.301436648684
0.000586 332.29255755968
0.000587 330.286069012506
0.000588 328.28203994492
0.000589 326.280539181355
0.00059 324.281640238869
0.000591 322.28542559818
0.000592 320.29197273628
0.000593 318.30134984962
0.000594 316.313624997755
0.000595 314.328866885799
0.000596 312.347156998234
0.000597 310.368575762161
0.000598 308.393190477107
0.000599 306.421068280725
};
\addplot [semithick, color1]
table {%
0 0.3
1e-06 6.74429176632002
2e-06 13.4679610037353
3e-06 20.1384744786995
4e-06 26.7535533170123
5e-06 33.3195017287996
6e-06 39.8408672783424
7e-06 46.3214130610813
8e-06 52.7616550979209
9e-06 59.1330077940708
1e-05 65.4355841132537
1.1e-05 71.6750970491574
1.2e-05 77.8561274959426
1.3e-05 83.9828397613205
1.4e-05 90.0582298106118
1.5e-05 96.085446908376
1.6e-05 102.066836338006
1.7e-05 108.004365785259
1.8e-05 113.900325260018
1.9e-05 119.756630956457
2e-05 125.57521350544
2.1e-05 131.357317331844
2.2e-05 137.104165629235
2.3e-05 142.816991751298
2.4e-05 148.496851743888
2.5e-05 154.14447822213
2.6e-05 159.760831030155
2.7e-05 165.34684489966
2.8e-05 170.903086403937
2.9e-05 176.430669887259
3e-05 181.932889405543
3.1e-05 187.4129060089
3.2e-05 192.871510843083
3.3e-05 198.309266210053
3.4e-05 203.721948047906
3.5e-05 209.096740795432
3.6e-05 214.433368948376
3.7e-05 219.732543665845
3.8e-05 224.994911656626
3.9e-05 230.221081492053
4e-05 235.411664831398
4.1e-05 240.567398159616
4.2e-05 245.688848187323
4.3e-05 250.776520010468
4.4e-05 255.830896894034
4.5e-05 260.852448759238
4.6e-05 265.84166166546
4.7e-05 270.798963604185
4.8e-05 275.724761067341
4.9e-05 280.6194439663
5e-05 285.483386485838
5.1e-05 290.316950338352
5.2e-05 295.120566680879
5.3e-05 299.89470767038
5.4e-05 304.639697787953
5.5e-05 309.355840592564
5.6e-05 314.043428026857
5.7e-05 318.702768640017
5.8e-05 323.334168217876
5.9e-05 327.937900345683
6e-05 332.514261808651
6.1e-05 337.063496344905
6.2e-05 341.585841007164
6.3e-05 346.081561144763
6.4e-05 350.55087497181
6.5e-05 354.993988953144
6.6e-05 359.411102260679
6.7e-05 363.802407086938
6.8e-05 368.168088942285
6.9e-05 372.508326936913
7e-05 376.823294048512
7.1e-05 381.113157376509
7.2e-05 385.37807838369
7.3e-05 389.618213125969
7.4e-05 393.833712471001
7.5e-05 398.024753831776
7.6e-05 402.191966279699
7.7e-05 406.331613183356
7.8e-05 410.443462700403
7.9e-05 414.52774941443
8e-05 418.584660789427
8.1e-05 422.614368087025
8.2e-05 426.617047651994
8.3e-05 430.592882959586
8.4e-05 434.54202773453
8.5e-05 438.464630750774
8.6e-05 442.360847837176
8.7e-05 446.230831606606
8.8e-05 450.074720292034
8.9e-05 453.892648847976
9e-05 457.684749096402
9.1e-05 461.451149826488
9.2e-05 465.191991631816
9.3e-05 468.907423316678
9.4e-05 472.597565958023
9.5e-05 476.262536705604
9.6e-05 479.902450124813
9.7e-05 483.517418277186
9.8e-05 487.10755079813
9.9e-05 490.672954971986
0.0001 494.213735804549
0.000101 497.729996093134
0.000102 501.221841319466
0.000103 504.689383863214
0.000104 508.13272108947
0.000105 511.551947401163
0.000106 514.947155310486
0.000107 518.318435496251
0.000108 521.665876859391
0.000109 524.989566576697
0.00011 528.289590152847
0.000111 531.566031470791
0.000112 534.818972840567
0.000113 538.048495046595
0.000114 541.254677393503
0.000115 544.437597750559
0.000116 547.597332594733
0.000117 550.733957052463
0.000118 553.847544940157
0.000119 556.938170546016
0.00012 560.005929433722
0.000121 563.050900029994
0.000122 566.073153422573
0.000123 569.072782393617
0.000124 572.049855589742
0.000125 575.004437936183
0.000126 577.936593336045
0.000127 580.846384701309
0.000128 583.733873982952
0.000129 586.599122200212
0.00013 589.442189469034
0.000131 592.263135029711
0.000132 595.062017273764
0.000133 597.838893770065
0.000134 600.593821290252
0.000135 603.326855833435
0.000136 606.03805265024
0.000137 608.727466266184
0.000138 611.395150504437
0.000139 614.041158507955
0.00014 616.665542761033
0.000141 619.268355110278
0.000142 621.849646785026
0.000143 624.409468417226
0.000144 626.947870060794
0.000145 629.464907124594
0.000146 631.960635352625
0.000147 634.435103120588
0.000148 636.888358190172
0.000149 639.32044783552
0.00015 641.731418859773
0.000151 644.121317611184
0.000152 646.490190341677
0.000153 648.838090382293
0.000154 651.165067457246
0.000155 653.471166077387
0.000156 655.756430374408
0.000157 658.020904114672
0.000158 660.264630712694
0.000159 662.487653244293
0.00016 664.690014459419
0.000161 666.871756794665
0.000162 669.032922385478
0.000163 671.173553078068
0.000164 673.293690441028
0.000165 675.393375776677
0.000166 677.472650132127
0.000167 679.531554310086
0.000168 681.570128879398
0.000169 683.58841418534
0.00017 685.586450359663
0.000171 687.564277330405
0.000172 689.521934831464
0.000173 691.45946241195
0.000174 693.376899445313
0.000175 695.274285138264
0.000176 697.151658539475
0.000177 699.009058548088
0.000178 700.846523922017
0.000179 702.664093286062
0.00018 704.461805139835
0.000181 706.239697865497
0.000182 707.99780973532
0.000183 709.736178919077
0.000184 711.454843491257
0.000185 713.153841438117
0.000186 714.833210664572
0.000187 716.492989000927
0.000188 718.133214729999
0.000189 719.753930869548
0.00019 721.355175648884
0.000191 722.936989773725
0.000192 724.499413684374
0.000193 726.042484694013
0.000194 727.566240085368
0.000195 729.070717116301
0.000196 730.555953025265
0.000197 732.021985036645
0.000198 733.468850365974
0.000199 734.896586225034
0.0002 736.305229826833
0.000201 737.694818390481
0.000202 739.065389145944
0.000203 740.416979338702
0.000204 741.74962623429
0.000205 743.063367122744
0.000206 744.358239322944
0.000207 745.634280186857
0.000208 746.891527103683
0.000209 748.130017503914
0.00021 749.349788863288
0.000211 750.550882597162
0.000212 751.733494798846
0.000213 752.897797186961
0.000214 754.043832138724
0.000215 755.17163540407
0.000216 756.281246635035
0.000217 757.372698524924
0.000218 758.446025905274
0.000219 759.501262135909
0.00022 760.538439212807
0.000221 761.557589795452
0.000222 762.558752616266
0.000223 763.541961507518
0.000224 764.507248770343
0.000225 765.454646836021
0.000226 766.3841882684
0.000227 767.295905766246
0.000228 768.189833100123
0.000229 769.066005946569
0.00023 769.924457534985
0.000231 770.765221036188
0.000232 771.588329767385
0.000233 772.393817194193
0.000234 773.181716932621
0.000235 773.952062750985
0.000236 774.704888571792
0.000237 775.440228473565
0.000238 776.158116692621
0.000239 776.858587624813
0.00024 777.541675827215
0.000241 778.207418131296
0.000242 778.855852879646
0.000243 779.487015114879
0.000244 780.100939885939
0.000245 780.697662411738
0.000246 781.277218082589
0.000247 781.839642461599
0.000248 782.384971286019
0.000249 782.913240468558
0.00025 783.424486098663
0.000251 783.918744443759
0.000252 784.396051950456
0.000253 784.856445245716
0.000254 785.299961137995
0.000255 785.72663661834
0.000256 786.136508861462
0.000257 786.529615226773
0.000258 786.905993259388
0.000259 787.265680691102
0.00026 787.608715441331
0.000261 787.935135618025
0.000262 788.244979518551
0.000263 788.538285630544
0.000264 788.815092632738
0.000265 789.075439395756
0.000266 789.319364982883
0.000267 789.546908650806
0.000268 789.758109850331
0.000269 789.953008227069
0.00027 790.1316436221
0.000271 790.294056072612
0.000272 790.44028581251
0.000273 790.570373273009
0.000274 790.684359083193
0.000275 790.782284070555
0.000276 790.864189541032
0.000277 790.930117172175
0.000278 790.980108254329
0.000279 791.01420424754
0.00028 791.032446813509
0.000281 791.034877815986
0.000282 791.021539321157
0.000283 790.992473598003
0.000284 790.947723118636
0.000285 790.887330558619
0.000286 790.811338797269
0.000287 790.71979128491
0.000288 790.612731948638
0.000289 790.490204456569
0.00029 790.352252662171
0.000291 790.198920623037
0.000292 790.030252601074
0.000293 789.846293062677
0.000294 789.647086678886
0.000295 789.432678325522
0.000296 789.203113083315
0.000297 788.958436238
0.000298 788.698693280414
0.000299 788.423929906568
0.0003 788.134192017699
0.000301 787.829525720315
0.000302 787.509977326222
0.000303 787.175593352531
0.000304 786.826420521659
0.000305 786.462505761303
0.000306 786.083896204415
0.000307 785.690639189148
0.000308 785.282782258796
0.000309 784.860373161721
0.00031 784.423459851259
0.000311 783.972090485623
0.000312 783.506313427781
0.000313 783.026177245331
0.000314 782.53173071036
0.000315 782.023022799284
0.000316 781.500102692686
0.000317 780.963019775134
0.000318 780.411823634993
0.000319 779.846564064213
0.00032 779.267291058123
0.000321 778.674054815198
0.000322 778.066905736822
0.000323 777.445894728873
0.000324 776.811076567017
0.000325 776.162503586486
0.000326 775.500227105778
0.000327 774.824298644272
0.000328 774.134769921925
0.000329 773.431692858956
0.00033 772.715119575523
0.000331 771.985102391379
0.000332 771.241693825534
0.000333 770.484946595893
0.000334 769.714913618893
0.000335 768.931648009123
0.000336 768.135203078944
0.000337 767.325632519432
0.000338 766.50299254355
0.000339 765.667337962652
0.00034 764.818722925365
0.000341 763.957201775681
0.000342 763.082829052509
0.000343 762.19565948923
0.000344 761.295748013234
0.000345 760.383152827164
0.000346 759.457933906515
0.000347 758.520147019548
0.000348 757.569847957444
0.000349 756.607093206073
0.00035 755.631942258402
0.000351 754.64445183105
0.000352 753.644682362989
0.000353 752.632695463852
0.000354 751.608551977511
0.000355 750.572318752648
0.000356 749.524167618839
0.000357 748.464302839947
0.000358 747.392789438407
0.000359 746.309687897692
0.00036 745.2150588607
0.000361 744.108963129048
0.000362 742.991461662339
0.000363 741.862615577435
0.000364 740.722486147705
0.000365 739.571134802276
0.000366 738.408623125268
0.000367 737.235012855018
0.000368 736.0503658833
0.000369 734.854744254533
0.00037 733.648210164976
0.000371 732.430825961919
0.000372 731.20265414286
0.000373 729.963757354678
0.000374 728.714198392791
0.000375 727.454040200306
0.000376 726.183345867165
0.000377 724.902178629275
0.000378 723.61060186763
0.000379 722.308679107428
0.00038 720.996474237358
0.000381 719.674054567124
0.000382 718.341485585473
0.000383 716.998831647214
0.000384 715.646160648919
0.000385 714.283538525464
0.000386 712.91102983651
0.000387 711.528699277583
0.000388 710.136611679099
0.000389 708.734832005385
0.00039 707.323425353684
0.000391 705.902456953155
0.000392 704.471992163866
0.000393 703.03209647577
0.000394 701.582835507681
0.000395 700.124275006235
0.000396 698.656480844842
0.000397 697.179519022632
0.000398 695.69345566339
0.000399 694.198357014479
0.0004 692.69428944576
0.000401 691.181319448496
0.000402 689.659513634251
0.000403 688.128938733779
0.000404 686.589661595903
0.000405 685.041749186382
0.000406 683.485268586775
0.000407 681.920286993289
0.000408 680.346871715622
0.000409 678.765090175794
0.00041 677.17500990697
0.000411 675.57669855227
0.000412 673.970223863576
0.000413 672.355653700322
0.000414 670.733056028279
0.000415 669.102498918331
0.000416 667.464050545232
0.000417 665.817779186369
0.000418 664.163753220501
0.000419 662.502041126494
0.00042 660.832711482044
0.000421 659.155832962396
0.000422 657.471474339044
0.000423 655.779704478423
0.000424 654.0805923406
0.000425 652.374206977938
0.000426 650.660617533767
0.000427 648.939893241031
0.000428 647.212103420934
0.000429 645.477318070557
0.00043 643.735613831946
0.000431 641.987062141882
0.000432 640.2317326715
0.000433 638.469695172436
0.000434 636.701019475407
0.000435 634.925775488766
0.000436 633.144033197057
0.000437 631.355862659548
0.000438 629.561334008767
0.000439 627.760517449011
0.00044 625.953484960442
0.000441 624.14031387094
0.000442 622.32107518095
0.000443 620.495839370097
0.000444 618.664676981393
0.000445 616.82765861967
0.000446 614.98485495002
0.000447 613.136336696202
0.000448 611.282174639054
0.000449 609.422439614882
0.00045 607.557202513844
0.000451 605.686534278315
0.000452 603.810505901242
0.000453 601.92918842449
0.000454 600.04265293717
0.000455 598.150970573953
0.000456 596.254212513379
0.000457 594.352449976144
0.000458 592.445754223382
0.000459 590.534196554921
0.00046 588.617848307544
0.000461 586.696780853219
0.000462 584.771065597326
0.000463 582.840773976865
0.000464 580.905977458651
0.000465 578.966747537502
0.000466 577.023155734399
0.000467 575.075273594643
0.000468 573.123172685997
0.000469 571.166924596806
0.00047 569.206600934109
0.000471 567.242273321736
0.000472 565.274013398388
0.000473 563.301896927939
0.000474 561.326010310486
0.000475 559.346426367512
0.000476 557.363216743829
0.000477 555.376454282793
0.000478 553.38622661394
0.000479 551.392610589471
0.00048 549.39567781222
0.000481 547.395499877927
0.000482 545.392148373146
0.000483 543.385694873132
0.000484 541.376210939723
0.000485 539.363768119192
0.000486 537.348437940098
0.000487 535.330291911104
0.000488 533.309401518796
0.000489 531.285838225469
0.00049 529.25967346691
0.000491 527.230978650154
0.000492 525.199825151229
0.000493 523.166284312885
0.000494 521.130427442297
0.000495 519.092325808765
0.000496 517.052050641384
0.000497 515.0096731267
0.000498 512.965264406358
0.000499 510.918895574715
0.0005 508.870637676453
0.000501 506.820561704161
0.000502 504.768738595909
0.000503 502.715239232795
0.000504 500.660134436484
0.000505 498.603495155082
0.000506 496.545398648501
0.000507 494.485920388278
0.000508 492.425130890877
0.000509 490.36310059955
0.00051 488.299899881758
0.000511 486.235599026572
0.000512 484.170268242059
0.000513 482.10397765265
0.000514 480.036797296485
0.000515 477.968797122745
0.000516 475.900046988961
0.000517 473.83061665831
0.000518 471.760575796883
0.000519 469.689993970947
0.00052 467.618940644178
0.000521 465.54748517488
0.000522 463.475696813185
0.000523 461.403646046678
0.000524 459.331418900626
0.000525 457.259089084703
0.000526 455.186725274229
0.000527 453.114396019633
0.000528 451.042169743529
0.000529 448.970114737793
0.00053 446.898299160605
0.000531 444.826791033486
0.000532 442.755658238306
0.000533 440.684968514282
0.000534 438.614789454957
0.000535 436.5451937684
0.000536 434.476254840058
0.000537 432.408039839548
0.000538 430.340615680513
0.000539 428.274049114314
0.00054 426.208406726911
0.000541 424.143754935709
0.000542 422.080163618636
0.000543 420.017716270044
0.000544 417.956480182684
0.000545 415.896520924487
0.000546 413.837903876439
0.000547 411.78069839017
0.000548 409.724976121602
0.000549 407.670802079856
0.00055 405.618257328162
0.000551 403.567432359609
0.000552 401.518396130649
0.000553 399.471098309331
0.000554 397.423712158326
0.000555 395.376046931834
0.000556 393.328570797558
0.000557 391.281354599474
0.000558 389.234469164748
0.000559 387.18798530135
0.00056 385.141973795631
0.000561 383.096505409864
0.000562 381.051650879749
0.000563 379.007480911883
0.000564 376.964066181187
0.000565 374.921477328308
0.000566 372.879784956962
0.000567 370.839059631261
0.000568 368.79937187298
0.000569 366.760792158797
0.00057 364.723390917486
0.000571 362.687238527068
0.000572 360.652405311924
0.000573 358.61896153986
0.000574 356.586977419132
0.000575 354.556523095424
0.000576 352.527668648784
0.000577 350.50048409051
0.000578 348.475039359998
0.000579 346.451404321533
0.00058 344.42964876104
0.000581 342.409842382785
0.000582 340.392054806025
0.000583 338.376357124047
0.000584 336.362836775315
0.000585 334.351566997221
0.000586 332.342616945596
0.000587 330.336055670465
0.000588 328.331952112371
0.000589 326.330375098633
0.00059 324.331394823953
0.000591 322.3350997145
0.000592 320.341564637064
0.000593 318.350857791531
0.000594 316.363047241006
0.000595 314.378201052543
0.000596 312.396401290458
0.000597 310.417728455872
0.000598 308.442249852444
0.000599 306.470032622079
};

\nextgroupplot[
    x label style={at={(axis description cs:0.5,-0.05)},anchor=north},
    y label style={at={(axis description cs:-0.005,.5)},rotate=0,anchor=south},
tick align=outside,
tick pos=left,
x grid style={white!69.0196078431373!black},
xlabel={Время},
xmin=-2.995e-05, xmax=0.00062895,
xtick style={color=black},
xtick={-0.0002,0,0.0002,0.0004,0.0006,0.0008},
xticklabels={−0.0002,0.0000,0.0002,0.0004,0.0006,0.0008},
y grid style={white!69.0196078431373!black},
ylabel={$U$},
ymin=51.8262612233686, ymax=1464.19874946555,
ytick style={color=black}
,ytick={250, 500, ..., 1250}
]
\addplot [semithick, color0]
table {%
0 1400
1e-06 1399.98663761581
2e-06 1399.94829535565
3e-06 1399.88495370042
4e-06 1399.79683539444
5e-06 1399.68413336427
6e-06 1399.54702150136
7e-06 1399.38565780585
8e-06 1399.20018763692
9e-06 1398.99081490998
1e-05 1398.7578069694
1.1e-05 1398.50140897876
1.2e-05 1398.22184679291
1.3e-05 1397.91932922654
1.4e-05 1397.59405299338
1.5e-05 1397.24620367651
1.6e-05 1396.87595575741
1.7e-05 1396.48347633956
1.8e-05 1396.0689246657
1.9e-05 1395.63245221672
2e-05 1395.17420254616
2.1e-05 1394.694313653
2.2e-05 1394.1929197746
2.3e-05 1393.67014947745
2.4e-05 1393.12612766482
2.5e-05 1392.56097575553
2.6e-05 1391.97481265377
2.7e-05 1391.36775265504
2.8e-05 1390.73990788438
2.9e-05 1390.09138779065
3e-05 1389.42229646574
3.1e-05 1388.73271827253
3.2e-05 1388.02273460245
3.3e-05 1387.29242389732
3.4e-05 1386.54186341504
3.5e-05 1385.771179125
3.6e-05 1384.98051471595
3.7e-05 1384.17001118342
3.8e-05 1383.33980697629
3.9e-05 1382.49003822231
4e-05 1381.62083883347
4.1e-05 1380.73234014293
4.2e-05 1379.82467099359
4.3e-05 1378.89795830429
4.4e-05 1377.95232715284
4.5e-05 1376.98790085501
4.6e-05 1376.00480092046
4.7e-05 1375.00314716
4.8e-05 1373.9830578374
4.9e-05 1372.94464973278
5e-05 1371.88803820274
5.1e-05 1370.81333723742
5.2e-05 1369.72065944776
5.3e-05 1368.61011554296
5.4e-05 1367.48181484676
5.5e-05 1366.33586552925
5.6e-05 1365.17237465094
5.7e-05 1363.99144820506
5.8e-05 1362.79319093186
5.9e-05 1361.57770657126
6e-05 1360.34509775367
6.1e-05 1359.09546611892
6.2e-05 1357.82891242906
6.3e-05 1356.54553649037
6.4e-05 1355.24543720878
6.5e-05 1353.92871270662
6.6e-05 1352.59546035033
6.7e-05 1351.24577677704
6.8e-05 1349.87975792004
6.9e-05 1348.49749903312
7e-05 1347.09909471392
7.1e-05 1345.68463892634
7.2e-05 1344.25422502194
7.3e-05 1342.80794576057
7.4e-05 1341.34589333006
7.5e-05 1339.8681593652
7.6e-05 1338.37483408784
7.7e-05 1336.8660127005
7.8e-05 1335.3417993622
7.9e-05 1333.80229728849
8e-05 1332.24760895928
8.1e-05 1330.67783617208
8.2e-05 1329.09308011794
8.3e-05 1327.49344125639
8.4e-05 1325.87901944729
8.5e-05 1324.24991398899
8.6e-05 1322.60622363069
8.7e-05 1320.94804650688
8.8e-05 1319.27548023066
8.9e-05 1317.58862190588
9e-05 1315.88756813905
9.1e-05 1314.1724150508
9.2e-05 1312.44325828702
9.3e-05 1310.7001928971
9.4e-05 1308.9433134543
9.5e-05 1307.17271409011
9.6e-05 1305.388488504
9.7e-05 1303.5907299729
9.8e-05 1301.77953136043
9.9e-05 1299.95498512575
0.0001 1298.11718333223
0.000101 1296.26621765577
0.000102 1294.40217939296
0.000103 1292.52515941607
0.000104 1290.63524821348
0.000105 1288.73253591791
0.000106 1286.81711231359
0.000107 1284.88906684316
0.000108 1282.94848861445
0.000109 1280.99546640694
0.00011 1279.03008867817
0.000111 1277.05244356982
0.000112 1275.06261891369
0.000113 1273.06070223746
0.000114 1271.04678077031
0.000115 1269.02094144835
0.000116 1266.98327091987
0.000117 1264.93385555049
0.000118 1262.87278142809
0.000119 1260.80013436761
0.00012 1258.71599987357
0.000121 1256.62046310014
0.000122 1254.51360894376
0.000123 1252.39552198772
0.000124 1250.26628651208
0.000125 1248.12598655285
0.000126 1245.97470590577
0.000127 1243.81252813012
0.000128 1241.63953655234
0.000129 1239.45581426953
0.00013 1237.2614441529
0.000131 1235.05650885108
0.000132 1232.84109079333
0.000133 1230.61527219267
0.000134 1228.37913504887
0.000135 1226.1327611514
0.000136 1223.87623208225
0.000137 1221.6096292187
0.000138 1219.33303373596
0.000139 1217.04652660974
0.00014 1214.75018861877
0.000141 1212.44410034719
0.000142 1210.12834218694
0.000143 1207.80299433993
0.000144 1205.46813682034
0.000145 1203.12384945666
0.000146 1200.77021184688
0.000147 1198.40730340675
0.000148 1196.03520337294
0.000149 1193.65399080487
0.00015 1191.26374458649
0.000151 1188.86454342801
0.000152 1186.45646586756
0.000153 1184.03959026394
0.000154 1181.61399476768
0.000155 1179.17975736257
0.000156 1176.73695586704
0.000157 1174.28566793561
0.000158 1171.82597106017
0.000159 1169.35794257129
0.00016 1166.88165963945
0.000161 1164.39719927623
0.000162 1161.90463833548
0.000163 1159.40405351439
0.000164 1156.89552135454
0.000165 1154.37911824297
0.000166 1151.85492041308
0.000167 1149.3230039456
0.000168 1146.78344476949
0.000169 1144.23631866276
0.00017 1141.68170125329
0.000171 1139.11966801966
0.000172 1136.55029429182
0.000173 1133.97365525182
0.000174 1131.3898259345
0.000175 1128.79888122812
0.000176 1126.20089587494
0.000177 1123.59594447185
0.000178 1120.98410147083
0.000179 1118.36544117953
0.00018 1115.74003776173
0.000181 1113.10796523778
0.000182 1110.46929748503
0.000183 1107.82410823824
0.000184 1105.17247108991
0.000185 1102.51445949067
0.000186 1099.85014674957
0.000187 1097.17960603437
0.000188 1094.5029103718
0.000189 1091.82013263556
0.00019 1089.13134554696
0.000191 1086.43662168635
0.000192 1083.73603347184
0.000193 1081.02965318226
0.000194 1078.3175529573
0.000195 1075.59980479756
0.000196 1072.87648056468
0.000197 1070.14765198136
0.000198 1067.41339063143
0.000199 1064.67376795987
0.0002 1061.9288552728
0.000201 1059.1787237375
0.000202 1056.42344438232
0.000203 1053.66308809669
0.000204 1050.89772563101
0.000205 1048.12742759658
0.000206 1045.35226446553
0.000207 1042.57230657065
0.000208 1039.78762410529
0.000209 1036.99828712323
0.00021 1034.20436553846
0.000211 1031.40592912507
0.000212 1028.60304740423
0.000213 1025.79578878583
0.000214 1022.98422150239
0.000215 1020.16841364358
0.000216 1017.34843316015
0.000217 1014.52434786818
0.000218 1011.69622546084
0.000219 1008.86413349938
0.00022 1006.02813942585
0.000221 1003.1883105627
0.000222 1000.34471409928
0.000223 997.497417085327
0.000224 994.646486450298
0.000225 991.791989002862
0.000226 988.933991430421
0.000227 986.072560298603
0.000228 983.20776205076
0.000229 980.339662997126
0.00023 977.468329320788
0.000231 974.593827081323
0.000232 971.716222214246
0.000233 968.835580530467
0.000234 965.951967715727
0.000235 963.065449330039
0.000236 960.176090807111
0.000237 957.283957453768
0.000238 954.38911444937
0.000239 951.491626845216
0.00024 948.591559563945
0.000241 945.688977398935
0.000242 942.783944995042
0.000243 939.876526864222
0.000244 936.966787387951
0.000245 934.054790816594
0.000246 931.140601268763
0.000247 928.224282730682
0.000248 925.305899055533
0.000249 922.385513962805
0.00025 919.46319103764
0.000251 916.538993730169
0.000252 913.612985354844
0.000253 910.685229089769
0.000254 907.755787976025
0.000255 904.824724916989
0.000256 901.892102677652
0.000257 898.957983883932
0.000258 896.022431021979
0.000259 893.085506437486
0.00026 890.147272334987
0.000261 887.207790777155
0.000262 884.267123684097
0.000263 881.325332832648
0.000264 878.382479855656
0.000265 875.438626241269
0.000266 872.493833332218
0.000267 869.548162325094
0.000268 866.601674269628
0.000269 863.654430067965
0.00027 860.706490473934
0.000271 857.757916092318
0.000272 854.808767378122
0.000273 851.859104635837
0.000274 848.908988018701
0.000275 845.958477527964
0.000276 843.007633012144
0.000277 840.056514163715
0.000278 837.105180520981
0.000279 834.153691467898
0.00028 831.202106233326
0.000281 828.250483890273
0.000282 825.298883355143
0.000283 822.347363386975
0.000284 819.395982586691
0.000285 816.444799396337
0.000286 813.493872098319
0.000287 810.543258814645
0.000288 807.593017502994
0.000289 804.643205958157
0.00029 801.693881811656
0.000291 798.745102530974
0.000292 795.796925418804
0.000293 792.849407612278
0.000294 789.902606082206
0.000295 786.956577632314
0.000296 784.011378898479
0.000297 781.06706634796
0.000298 778.12369627864
0.000299 775.181324818254
0.0003 772.240007923628
0.000301 769.29980137991
0.000302 766.360760799806
0.000303 763.422941622812
0.000304 760.48639911445
0.000305 757.551188365501
0.000306 754.617364291239
0.000307 751.684981630666
0.000308 748.754094945743
0.000309 745.824758620631
0.00031 742.89702686092
0.000311 739.970953692868
0.000312 737.046592962634
0.000313 734.123998335517
0.000314 731.203223295192
0.000315 728.284321142945
0.000316 725.367344996913
0.000317 722.452347791322
0.000318 719.539382275729
0.000319 716.628501014255
0.00032 713.719756384834
0.000321 710.813200578446
0.000322 707.908885598369
0.000323 705.006863259415
0.000324 702.107185179947
0.000325 699.209902774881
0.000326 696.315067268008
0.000327 693.422729691237
0.000328 690.532940883847
0.000329 687.645751491736
0.00033 684.761211966679
0.000331 681.879372565576
0.000332 679.000283349717
0.000333 676.123994184029
0.000334 673.250554736342
0.000335 670.380014476648
0.000336 667.512422676357
0.000337 664.647828407568
0.000338 661.786280537954
0.000339 658.927827725443
0.00034 656.072518426263
0.000341 653.220400894211
0.000342 650.371523179926
0.000343 647.525933130159
0.000344 644.683678387049
0.000345 641.844806387403
0.000346 639.009364335679
0.000347 636.177399226146
0.000348 633.348957845237
0.000349 630.524086770831
0.00035 627.702832362768
0.000351 624.885240764174
0.000352 622.071357907332
0.000353 619.261229482358
0.000354 616.454900959782
0.000355 613.652417573879
0.000356 610.853824291913
0.000357 608.059164994435
0.000358 605.268483253581
0.000359 602.481822416146
0.00036 599.699225602926
0.000361 596.920735708124
0.000362 594.14639539876
0.000363 591.376247114082
0.000364 588.61033306498
0.000365 585.848695233407
0.000366 583.091375371795
0.000367 580.338415002479
0.000368 577.589855417126
0.000369 574.845737676162
0.00037 572.106102608203
0.000371 569.370990809491
0.000372 566.640442643333
0.000373 563.914498239542
0.000374 561.193197493878
0.000375 558.476580067499
0.000376 555.764685386413
0.000377 553.057552640925
0.000378 550.355220785102
0.000379 547.657728536227
0.00038 544.965114374268
0.000381 542.277416535939
0.000382 539.594673005403
0.000383 536.916921527827
0.000384 534.244199603097
0.000385 531.576544476651
0.000386 528.913993153285
0.000387 526.256582396632
0.000388 523.604348728658
0.000389 520.957328429158
0.00039 518.31555753526
0.000391 515.679071840931
0.000392 513.047906896487
0.000393 510.422098008105
0.000394 507.801680237342
0.000395 505.186688400656
0.000396 502.577157068931
0.000397 499.973120567003
0.000398 497.374612973199
0.000399 494.781668118869
0.0004 492.194319587929
0.000401 489.612600716406
0.000402 487.036544591987
0.000403 484.466184053571
0.000404 481.901551690828
0.000405 479.342679843762
0.000406 476.789600602273
0.000407 474.24234580573
0.000408 471.700947042545
0.000409 469.165435649751
0.00041 466.635842712588
0.000411 464.112199064084
0.000412 461.594535284657
0.000413 459.082881701701
0.000414 456.577268389194
0.000415 454.077725167303
0.000416 451.584281601991
0.000417 449.096967004635
0.000418 446.615810431642
0.000419 444.140840684077
0.00042 441.672086307292
0.000421 439.209575590555
0.000422 436.753336566694
0.000423 434.303397011736
0.000424 431.859784444559
0.000425 429.422526126544
0.000426 426.99164906123
0.000427 424.567179993985
0.000428 422.149145411664
0.000429 419.737571542291
0.00043 417.332484340673
0.000431 414.933909480399
0.000432 412.541872375249
0.000433 410.156398178879
0.000434 407.777511784519
0.000435 405.405237824679
0.000436 403.039600670867
0.000437 400.680624433307
0.000438 398.328332960657
0.000439 395.982749839746
0.00044 393.643898395306
0.000441 391.311801667367
0.000442 388.986482421062
0.000443 386.667963158667
0.000444 384.356266119341
0.000445 382.051413278905
0.000446 379.753426349603
0.000447 377.462326779889
0.000448 375.178135754205
0.000449 372.900874192775
0.00045 370.6305627514
0.000451 368.36722182126
0.000452 366.110871528722
0.000453 363.861531735156
0.000454 361.619222036755
0.000455 359.38396176436
0.000456 357.155769983296
0.000457 354.934665493213
0.000458 352.720666827926
0.000459 350.513792255275
0.00046 348.314059776979
0.000461 346.121487128502
0.000462 343.936091778927
0.000463 341.757890930834
0.000464 339.586901520187
0.000465 337.423140216222
0.000466 335.266623421351
0.000467 333.117367271065
0.000468 330.975387633849
0.000469 328.8407001111
0.00047 326.713320037055
0.000471 324.593262478722
0.000472 322.480542235828
0.000473 320.375173840761
0.000474 318.277171508319
0.000475 316.186549161385
0.000476 314.103320455536
0.000477 312.027498779011
0.000478 309.959097223873
0.000479 307.89812856003
0.00048 305.844605290256
0.000481 303.798539650161
0.000482 301.759943608222
0.000483 299.72882886582
0.000484 297.705206857288
0.000485 295.689088749965
0.000486 293.680485444257
0.000487 291.679407573708
0.000488 289.685865505082
0.000489 287.699869338442
0.00049 285.721428907252
0.000491 283.750553778477
0.000492 281.787253252697
0.000493 279.831536364224
0.000494 277.883411881232
0.000495 275.942888305897
0.000496 274.009973874537
0.000497 272.084676557774
0.000498 270.167004060689
0.000499 268.256963823003
0.0005 266.35456301925
0.000501 264.459808558975
0.000502 262.572707086925
0.000503 260.693264983263
0.000504 258.821488363784
0.000505 256.957383080142
0.000506 255.100954714781
0.000507 253.252208548406
0.000508 251.411149598511
0.000509 249.577782619609
0.00051 247.752112103512
0.000511 245.934142279615
0.000512 244.123877115199
0.000513 242.321320315731
0.000514 240.526475325191
0.000515 238.73934532639
0.000516 236.959933241312
0.000517 235.188241731462
0.000518 233.424273198221
0.000519 231.668029783212
0.00052 229.919513368685
0.000521 228.178725577897
0.000522 226.445667775516
0.000523 224.720341068027
0.000524 223.002746271815
0.000525 221.292883891635
0.000526 219.590754175833
0.000527 217.896357116742
0.000528 216.209692451162
0.000529 214.530759660833
0.00053 212.859557972938
0.000531 211.196086360601
0.000532 209.540343543405
0.000533 207.892327987919
0.000534 206.252037908238
0.000535 204.61947126653
0.000536 202.994625731788
0.000537 201.377498721416
0.000538 199.768087402854
0.000539 198.166388694182
0.00054 196.572399264725
0.000541 194.986115535684
0.000542 193.407533680767
0.000543 191.836649574653
0.000544 190.273458813938
0.000545 188.71795675027
0.000546 187.170138491005
0.000547 185.62999889991
0.000548 184.097532561568
0.000549 182.572733813443
0.00055 181.055596751877
0.000551 179.546115089036
0.000552 178.044282278431
0.000553 176.550091500598
0.000554 175.063538691215
0.000555 173.58462666199
0.000556 172.11335482974
0.000557 170.649722349973
0.000558 169.193728113878
0.000559 167.745370748401
0.00056 166.304648616318
0.000561 164.871559816321
0.000562 163.44610218311
0.000563 162.028273287502
0.000564 160.618070436543
0.000565 159.215490673628
0.000566 157.820530778638
0.000567 156.433187268082
0.000568 155.053456395248
0.000569 153.68133415037
0.00057 152.316816260799
0.000571 150.959898191189
0.000572 149.610575143689
0.000573 148.268842058153
0.000574 146.934693612354
0.000575 145.608124222215
0.000576 144.289128042044
0.000577 142.97769896479
0.000578 141.673830622301
0.000579 140.377516385602
0.00058 139.088749365181
0.000581 137.807522411286
0.000582 136.533828114239
0.000583 135.267658804757
0.000584 134.009006517133
0.000585 132.75786297617
0.000586 131.514219648495
0.000587 130.278067742899
0.000588 129.049398210731
0.000589 127.82820174632
0.00059 126.614468787401
0.000591 125.408189479742
0.000592 124.209353645481
0.000593 123.017950852115
0.000594 121.833970412933
0.000595 120.657401387526
0.000596 119.488232576501
0.000597 118.326452431381
0.000598 117.172049152388
0.000599 116.025010688923
};
\addplot [semithick, color1]
table {%
0 1400
1e-06 1399.98689675113
2e-06 1399.94919062626
3e-06 1399.88647343051
4e-06 1399.79897218402
5e-06 1399.68688106227
6e-06 1399.5503746432
7e-06 1399.38961197543
8e-06 1399.20473879965
9e-06 1398.99595778642
1e-05 1398.76353321434
1.1e-05 1398.50771085037
1.2e-05 1398.22871724427
1.3e-05 1397.92676245599
1.4e-05 1397.60204347355
1.5e-05 1397.25474594038
1.6e-05 1396.88504490047
1.7e-05 1396.4931077257
1.8e-05 1396.07909380225
1.9e-05 1395.64315466578
2e-05 1395.18543449014
2.1e-05 1394.70607155878
2.2e-05 1394.20519985825
2.3e-05 1393.68294847545
2.4e-05 1393.13944237543
2.5e-05 1392.57480311332
2.6e-05 1391.9891493465
2.7e-05 1391.38259582716
2.8e-05 1390.75525473693
2.9e-05 1390.10723528469
3e-05 1389.43863951094
3.1e-05 1388.74955474641
3.2e-05 1388.04006235291
3.3e-05 1387.3102410303
3.4e-05 1386.56017243694
3.5e-05 1385.78997633904
3.6e-05 1384.99979646855
3.7e-05 1384.18977389512
3.8e-05 1383.36004714464
3.9e-05 1382.51075241145
4e-05 1381.64202365046
4.1e-05 1380.75399226947
4.2e-05 1379.84678719951
4.3e-05 1378.92053541403
4.4e-05 1377.97536204288
4.5e-05 1377.01139044746
4.6e-05 1376.02874219423
4.7e-05 1375.02753713937
4.8e-05 1374.00789359001
4.9e-05 1372.96992836767
5e-05 1371.91375686862
5.1e-05 1370.83949312098
5.2e-05 1369.74724971828
5.3e-05 1368.63713748175
5.4e-05 1367.50926576851
5.5e-05 1366.36374278043
5.6e-05 1365.20067560855
5.7e-05 1364.02017024206
5.8e-05 1362.82233148902
5.9e-05 1361.60726310469
6e-05 1360.3750677632
6.1e-05 1359.12584712925
6.2e-05 1357.85970198657
6.3e-05 1356.57673217162
6.4e-05 1355.27703661239
6.5e-05 1353.96071345239
6.6e-05 1352.62786007843
6.7e-05 1351.27857314726
6.8e-05 1349.91294861104
6.9e-05 1348.53108174176
7e-05 1347.13306715463
7.1e-05 1345.71899883043
7.2e-05 1344.28897013705
7.3e-05 1342.84307385005
7.4e-05 1341.38140217245
7.5e-05 1339.90404675368
7.6e-05 1338.41109721932
7.7e-05 1336.90264902495
7.8e-05 1335.37880632487
7.9e-05 1333.8396723877
8e-05 1332.2853497052
8.1e-05 1330.71594009814
8.2e-05 1329.13154475535
8.3e-05 1327.53226417395
8.4e-05 1325.91819822879
8.5e-05 1324.28944623269
8.6e-05 1322.64610693692
8.7e-05 1320.98827850401
8.8e-05 1319.31605856047
8.9e-05 1317.62954422325
9e-05 1315.92883211161
9.1e-05 1314.21401835862
9.2e-05 1312.48519860553
9.3e-05 1310.74246793446
9.4e-05 1308.98592093022
9.5e-05 1307.21565173553
9.6e-05 1305.43175406084
9.7e-05 1303.63432119379
9.8e-05 1301.82344600846
9.9e-05 1299.99922097424
0.0001 1298.16173816448
0.000101 1296.31108926485
0.000102 1294.44736557672
0.000103 1292.57065798835
0.000104 1290.68105699722
0.000105 1288.77865274496
0.000106 1286.86353502451
0.000107 1284.93579328701
0.000108 1282.99551664863
0.000109 1281.04279389701
0.00011 1279.07771349767
0.000111 1277.10036360012
0.000112 1275.11083204379
0.000113 1273.10920636389
0.000114 1271.09557379695
0.000115 1269.07002128627
0.000116 1267.03263548722
0.000117 1264.98350277232
0.000118 1262.92270923625
0.000119 1260.85034070062
0.00012 1258.76648267114
0.000121 1256.67122031711
0.000122 1254.5646385412
0.000123 1252.44682193565
0.000124 1250.31785478671
0.000125 1248.17782113626
0.000126 1246.02680478585
0.000127 1243.86488930043
0.000128 1241.69215801203
0.000129 1239.50869402325
0.00013 1237.3145802107
0.000131 1235.10989922832
0.000132 1232.8947335106
0.000133 1230.6691652757
0.000134 1228.43327652844
0.000135 1226.18714906329
0.000136 1223.93086446714
0.000137 1221.6645041221
0.000138 1219.38814920814
0.000139 1217.10188070568
0.00014 1214.80577939806
0.000141 1212.499925874
0.000142 1210.18440052991
0.000143 1207.85928357216
0.000144 1205.52465501931
0.000145 1203.18059469693
0.000146 1200.82718221579
0.000147 1198.46449699586
0.000148 1196.09261827794
0.000149 1193.71162512556
0.00015 1191.3215964267
0.000151 1188.9226108956
0.000152 1186.51474707433
0.000153 1184.09808332133
0.000154 1181.67269779656
0.000155 1179.23866848764
0.000156 1176.79607321678
0.000157 1174.34498964225
0.000158 1171.88549525964
0.000159 1169.41766740319
0.00016 1166.94158324702
0.000161 1164.45731980633
0.000162 1161.96495393853
0.000163 1159.46456234437
0.000164 1156.95622156896
0.000165 1154.44000800281
0.000166 1151.9159978828
0.000167 1149.3842672931
0.000168 1146.84489216607
0.000169 1144.29794828314
0.00017 1141.74351127556
0.000171 1139.18165662524
0.000172 1136.61245966548
0.000173 1134.03599558163
0.000174 1131.45233941185
0.000175 1128.86156604765
0.000176 1126.26375023457
0.000177 1123.65896657273
0.000178 1121.04728951735
0.000179 1118.4287933793
0.00018 1115.80355232556
0.000181 1113.17164037967
0.000182 1110.53313142217
0.000183 1107.88809919097
0.000184 1105.23661728175
0.000185 1102.57875914829
0.000186 1099.91459810278
0.000187 1097.2442073161
0.000188 1094.56765981811
0.000189 1091.88502848612
0.00019 1089.1963860447
0.000191 1086.50180507411
0.000192 1083.8013579994
0.000193 1081.09511710248
0.000194 1078.3831545261
0.000195 1075.66554227395
0.000196 1072.94235221072
0.000197 1070.2136560622
0.000198 1067.47952541529
0.000199 1064.74003171803
0.0002 1061.99524627962
0.000201 1059.24524027038
0.000202 1056.49008472176
0.000203 1053.72985052623
0.000204 1050.96460843728
0.000205 1048.19442906928
0.000206 1045.41938289743
0.000207 1042.6395402576
0.000208 1039.85497134624
0.000209 1037.06574622018
0.00021 1034.27193479652
0.000211 1031.47360685245
0.000212 1028.67083179031
0.000213 1025.86367817117
0.000214 1023.05221423312
0.000215 1020.23650807112
0.000216 1017.41662763965
0.000217 1014.59264075734
0.000218 1011.76461511876
0.000219 1008.93261828935
0.00022 1006.09671771373
0.000221 1003.25698071693
0.000222 1000.41347449041
0.000223 997.566266087889
0.000224 994.715422441409
0.000225 991.861010362258
0.000226 989.003096540457
0.000227 986.141747544274
0.000228 983.277029818794
0.000229 980.409009678231
0.00023 977.537753308342
0.000231 974.663326771383
0.000232 971.785796005568
0.000233 968.905226824515
0.000234 966.021684916689
0.000235 963.13523584484
0.000236 960.245945045431
0.000237 957.353877828056
0.000238 954.459099374858
0.000239 951.561674739933
0.00024 948.661668848736
0.000241 945.759146495042
0.000242 942.854172329611
0.000243 939.946810867247
0.000244 937.037126492289
0.000245 934.125183457978
0.000246 931.211045885824
0.000247 928.29477776496
0.000248 925.376442951499
0.000249 922.456105167875
0.00025 919.533828002194
0.000251 916.609674907565
0.000252 913.683709201439
0.000253 910.755994064936
0.000254 907.826592542169
0.000255 904.895567539565
0.000256 901.962981825185
0.000257 899.028898028031
0.000258 896.093378637361
0.000259 893.156486001989
0.00026 890.21828232959
0.000261 887.278829685998
0.000262 884.338189994497
0.000263 881.39642503512
0.000264 878.45359644393
0.000265 875.50976571231
0.000266 872.564994186242
0.000267 869.619343065592
0.000268 866.672873403383
0.000269 863.725646105068
0.00027 860.777721927806
0.000271 857.829161479731
0.000272 854.880025219214
0.000273 851.930373454136
0.000274 848.980266341142
0.000275 846.029763884908
0.000276 843.078925937081
0.000277 840.127812194085
0.000278 837.176482197694
0.000279 834.224995335354
0.00028 831.273410839434
0.000281 828.321787786473
0.000282 825.370185096422
0.000283 822.418661531894
0.000284 819.467275697399
0.000285 816.516086038592
0.000286 813.565150841511
0.000287 810.61452823139
0.000288 807.664276170009
0.000289 804.714452455868
0.00029 801.765114724212
0.000291 798.816320446276
0.000292 795.868126928517
0.000293 792.920591311856
0.000294 789.973770570912
0.000295 787.02772151324
0.000296 784.082500778564
0.000297 781.138164838015
0.000298 778.194769993365
0.000299 775.252372376261
0.0003 772.311027947459
0.000301 769.37079249606
0.000302 766.431721638741
0.000303 763.493870818993
0.000304 760.55729530635
0.000305 757.622050195628
0.000306 754.688190406156
0.000307 751.755770681011
0.000308 748.82484558625
0.000309 745.895469510152
0.00031 742.967696662443
0.000311 740.041581073538
0.000312 737.117176593778
0.000313 734.194536892659
0.000314 731.273715458076
0.000315 728.354765595556
0.000316 725.437740427497
0.000317 722.522692892409
0.000318 719.609675744147
0.000319 716.698741551159
0.00032 713.78994269572
0.000321 710.883331373177
0.000322 707.978959591191
0.000323 705.07687916898
0.000324 702.177141728627
0.000325 699.27979869066
0.000326 696.384901283348
0.000327 693.492500543098
0.000328 690.602647313707
0.000329 687.715392245614
0.00033 684.830785795151
0.000331 681.9488782238
0.000332 679.069719597448
0.000333 676.193359785646
0.000334 673.319848460863
0.000335 670.449235097751
0.000336 667.581568972404
0.000337 664.71689916162
0.000338 661.855274537247
0.000339 658.99674376281
0.00034 656.141355299304
0.000341 653.289157405314
0.000342 650.440198136284
0.000343 647.594525343794
0.000344 644.752186674829
0.000345 641.913229567446
0.000346 639.077701235297
0.000347 636.24564867757
0.000348 633.417118685633
0.000349 630.592157842083
0.00035 627.770812512327
0.000351 624.953128844497
0.000352 622.139152771213
0.000353 619.328929993063
0.000354 616.522505985287
0.000355 613.719925986014
0.000356 610.921234852471
0.000357 608.126476611937
0.000358 605.335694842014
0.000359 602.54893289492
0.00036 599.766233896889
0.000361 596.98764074758
0.000362 594.213196119488
0.000363 591.442942457348
0.000364 588.676921977558
0.000365 585.915176667594
0.000366 583.157748285427
0.000367 580.40467835895
0.000368 577.656008185402
0.000369 574.911778830798
0.00037 572.172031129359
0.000371 569.436805682951
0.000372 566.706142860519
0.000373 563.980082797528
0.000374 561.258665395411
0.000375 558.541930321013
0.000376 555.829917006041
0.000377 553.122664646522
0.000378 550.420212202254
0.000379 547.722598396273
0.00038 545.02986171431
0.000381 542.342040397721
0.000382 539.659172438086
0.000383 536.981295586385
0.000384 534.308447347227
0.000385 531.640664973498
0.000386 528.977985475858
0.000387 526.320445623818
0.000388 523.668081945236
0.000389 521.020930725816
0.00039 518.379028008609
0.000391 515.742409593519
0.000392 513.111111036813
0.000393 510.485167650635
0.000394 507.864614502523
0.000395 505.249486414929
0.000396 502.639817964747
0.000397 500.035643482836
0.000398 497.436997053561
0.000399 494.843912514322
0.0004 492.2564234551
0.000401 489.674563218002
0.000402 487.098364896805
0.000403 484.527861336516
0.000404 481.963085132924
0.000405 479.404068632164
0.000406 476.850843930283
0.000407 474.303442872808
0.000408 471.761897054323
0.000409 469.226237818047
0.00041 466.696496255415
0.000411 464.172703205669
0.000412 461.654889255445
0.000413 459.143084738376
0.000414 456.637319734687
0.000415 454.137624070804
0.000416 451.644027318961
0.000417 449.156558796821
0.000418 446.675247567086
0.000419 444.200122437129
0.00042 441.731211958621
0.000421 439.268544427161
0.000422 436.812147881919
0.000423 434.362050105277
0.000424 431.918278622478
0.000425 429.480860701277
0.000426 427.049823351602
0.000427 424.625193325217
0.000428 422.206997115387
0.000429 419.795260956555
0.00043 417.390010809267
0.000431 414.991272355139
0.000432 412.599071014404
0.000433 410.21343194718
0.000434 407.834380053166
0.000435 405.461939971357
0.000436 403.09613607975
0.000437 400.736992495069
0.000438 398.384533072483
0.000439 396.038781405339
0.00044 393.699760823505
0.000441 391.367494375803
0.000442 389.042004833922
0.000443 386.723314706692
0.000444 384.411446239839
0.000445 382.106421415753
0.000446 379.808261953259
0.000447 377.516989307399
0.000448 375.232624669212
0.000449 372.955188965524
0.00045 370.684702858748
0.000451 368.421186746682
0.000452 366.164660762318
0.000453 363.915144773661
0.000454 361.67265838354
0.000455 359.437220929446
0.000456 357.208851483357
0.000457 354.987568851579
0.000458 352.773391574597
0.000459 350.56633792692
0.00046 348.366425916945
0.000461 346.17367328682
0.000462 343.988097512318
0.000463 341.809715802714
0.000464 339.638545100669
0.000465 337.474602082127
0.000466 335.317903156209
0.000467 333.168464465122
0.000468 331.026301884068
0.000469 328.891431021169
0.00047 326.763867217388
0.000471 324.64362554647
0.000472 322.530720814873
0.000473 320.425167558082
0.000474 318.326980003174
0.000475 316.236172079797
0.000476 314.152757450278
0.000477 312.076749509605
0.000478 310.008161353651
0.000479 307.947005764167
0.00048 305.89329525068
0.000481 303.847042055559
0.000482 301.808258154036
0.000483 299.776955254255
0.000484 297.753144797307
0.000485 295.736837957294
0.000486 293.728045641385
0.000487 291.726778489886
0.000488 289.733046876324
0.000489 287.746860907526
0.00049 285.768230423721
0.000491 283.797164998635
0.000492 281.83367393961
0.000493 279.877766287722
0.000494 277.929450817905
0.000495 275.988736039093
0.000496 274.055630194366
0.000497 272.130141261099
0.000498 270.212276951131
0.000499 268.302044710933
0.0005 266.399451721792
0.000501 264.504504899998
0.000502 262.617210897045
0.000503 260.737576099839
0.000504 258.865606630912
0.000505 257.001308348654
0.000506 255.144686837501
0.000507 253.295747390762
0.000508 251.45449503265
0.000509 249.620934524394
0.00051 247.795070364517
0.000511 245.976906789119
0.000512 244.16644777218
0.000513 242.363697025865
0.000514 240.568658000839
0.000515 238.781333886599
0.000516 237.001727611803
0.000517 235.229841844626
0.000518 233.465678993111
0.000519 231.709241205539
0.00052 229.960530370805
0.000521 228.219548118807
0.000522 226.486295820846
0.000523 224.760774590032
0.000524 223.042985247139
0.000525 221.332928308012
0.000526 219.630604027582
0.000527 217.936012404763
0.000528 216.249153182922
0.000529 214.570025850358
0.00053 212.8986296408
0.000531 211.234963533909
0.000532 209.579026255795
0.000533 207.930816279541
0.000534 206.290331825743
0.000535 204.657570856631
0.000536 203.032531055249
0.000537 201.415209845462
0.000538 199.805604401157
0.000539 198.203711646845
0.00054 196.609528258275
0.000541 195.023050663052
0.000542 193.444275038671
0.000543 191.87319727097
0.000544 190.309812962932
0.000545 188.75411747254
0.000546 187.206105913467
0.000547 185.665773150956
0.000548 184.133113781807
0.000549 182.60812214975
0.00055 181.090792338149
0.000551 179.581118088982
0.000552 178.07909285725
0.000553 176.584709832521
0.000554 175.097965541196
0.000555 173.618861560822
0.000556 172.147397791584
0.000557 170.683573395739
0.000558 169.22738727123
0.000559 167.778838051753
0.00056 166.337924106838
0.000561 164.904643541928
0.000562 163.478994198479
0.000563 162.06097365406
0.000564 160.650579222467
0.000565 159.247807953847
0.000566 157.852656634833
0.000567 156.465121788681
0.000568 155.085199675429
0.000569 153.712886292054
0.00057 152.348177372653
0.000571 150.991068388619
0.000572 149.641554548842
0.000573 148.299630799911
0.000574 146.965291826333
0.000575 145.638532050758
0.000576 144.319345634222
0.000577 143.007726476395
0.000578 141.703668215842
0.000579 140.407164230301
0.00058 139.118207636968
0.000581 137.836791292794
0.000582 136.562907794797
0.000583 135.296549480385
0.000584 134.037708389762
0.000585 132.786376257462
0.000586 131.542544556779
0.000587 130.30620450316
0.000588 129.077347054605
0.000589 127.855962912082
0.00059 126.642042519959
0.000591 125.435576027002
0.000592 124.236553268307
0.000593 123.044963817968
0.000594 121.860796995852
0.000595 120.684041868126
0.000596 119.514687228561
0.000597 118.35272155136
0.000598 117.198133043267
0.000599 116.050909660193
};

\nextgroupplot[
    x label style={at={(axis description cs:0.5,-0.05)},anchor=north},
    y label style={at={(axis description cs:-0.005,.5)},rotate=0,anchor=south},
tick align=outside,
tick pos=left,
x grid style={white!69.0196078431373!black},
xlabel={Время},
xmin=-3e-05, xmax=0.00063,
xtick style={color=black},
xtick={-0.0002,0,0.0002,0.0004,0.0006,0.0008},
xticklabels={−0.0002,0.0000,0.0002,0.0004,0.0006,0.0008},
y grid style={white!69.0196078431373!black},
ylabel={$R_p$},
ymin=-27.9594057188797, ymax=604.712856857802,
ytick style={color=black},
%ytick={-500,0,500,1000},
ytick={0,100,...,500},
%yticklabels={−250,0,250,500}
]
\addplot [semithick, color0]
table {%
0 575.95502674068
1e-06 21.3560343834861
1e-06 21.1034732383368
2e-06 10.5462911419143
2e-06 10.5448445110109
3e-06 7.53363151989684
3e-06 7.54298464747173
4e-06 5.9789836441199
4e-06 5.98367182722134
5e-06 5.01537859468212
5e-06 5.01817837569026
6e-06 4.35206839853523
6e-06 4.35389303582995
7e-06 3.86401178808444
7e-06 3.86529794806061
8e-06 3.55335413029898
8e-06 3.55419665752291
9e-06 3.36680519267889
9e-06 3.36778496343724
1e-05 3.20323476461533
1e-05 3.20404961171972
1.1e-05 3.05744750853959
1.1e-05 3.05811372771038
1.2e-05 2.9260686885785
1.2e-05 2.92663662500457
1.3e-05 2.80846060632809
1.3e-05 2.80894405869112
1.4e-05 2.70204768870091
1.4e-05 2.70247197786928
1.5e-05 2.60448355182258
1.5e-05 2.60484432592912
1.6e-05 2.51580530852081
1.6e-05 2.51612585551525
1.7e-05 2.43416749316232
1.7e-05 2.43445455491214
1.8e-05 2.35837650462373
1.8e-05 2.35862838714645
1.9e-05 2.28737386108891
1.9e-05 2.2875994247293
2e-05 2.22133370700917
2e-05 2.22153658538738
2.1e-05 2.16020240875786
2.1e-05 2.16038691941711
2.2e-05 2.10259872209797
2.2e-05 2.10276608664981
2.3e-05 2.04873364868048
2.3e-05 2.04888757548216
2.4e-05 1.99818952296347
2.4e-05 1.99832999410642
2.5e-05 1.95086911774761
2.5e-05 1.95099943160645
2.6e-05 1.90573833180189
2.6e-05 1.90585815423889
2.7e-05 1.86327111759028
2.7e-05 1.86338232004536
2.8e-05 1.82303087428989
2.8e-05 1.82313547690294
2.9e-05 1.78401252007179
2.9e-05 1.78411941051225
3e-05 1.74211893576916
3e-05 1.74220345579395
3.1e-05 1.70199135664722
3.1e-05 1.70206962555501
3.2e-05 1.66348275421007
3.2e-05 1.66355507143654
3.3e-05 1.62688642931112
3.3e-05 1.62695431752518
3.4e-05 1.60433900289298
3.4e-05 1.60439269235245
3.5e-05 1.58743078511461
3.5e-05 1.58749079780716
3.6e-05 1.5709869456234
3.6e-05 1.57104474847985
3.7e-05 1.55497784614721
3.7e-05 1.55503341077479
3.8e-05 1.53941028119444
3.8e-05 1.53946378534772
3.9e-05 1.52426544584048
3.9e-05 1.52431701060793
4e-05 1.50942673980709
4e-05 1.50947697376023
4.1e-05 1.49488221016307
4.1e-05 1.4949305105477
4.2e-05 1.48071831243117
4.2e-05 1.48076496091257
4.3e-05 1.46691988159655
4.3e-05 1.4669649694496
4.4e-05 1.45347242104667
4.4e-05 1.45351603304655
4.5e-05 1.440339314161
4.5e-05 1.44038161326209
4.6e-05 1.42751563902552
4.6e-05 1.42755658057593
4.7e-05 1.41500548467334
4.7e-05 1.41504516712234
4.8e-05 1.40279716222201
4.8e-05 1.40283564971593
4.9e-05 1.39087955718692
4.9e-05 1.39091690963765
5e-05 1.37924211982727
5e-05 1.37927839325029
5.1e-05 1.36786312416966
5.1e-05 1.36789861356968
5.2e-05 1.35663608840154
5.2e-05 1.35667060155
5.3e-05 1.34563416746498
5.3e-05 1.34566767873141
5.4e-05 1.33488033400693
5.4e-05 1.33491294564704
5.5e-05 1.32436618701667
5.5e-05 1.32439794046743
5.6e-05 1.31408366586843
5.6e-05 1.31411460007496
5.7e-05 1.30398947548492
5.7e-05 1.3040196745374
5.8e-05 1.2941107726141
5.8e-05 1.29414021673472
5.9e-05 1.28441827173771
5.9e-05 1.28444706127401
6e-05 1.27491877689499
6e-05 1.27494685454374
6.1e-05 1.26561800179929
6.1e-05 1.26564541920554
6.2e-05 1.25649361113928
6.2e-05 1.25652045120149
6.3e-05 1.24754383055164
6.3e-05 1.24757003865063
6.4e-05 1.23877585154468
6.4e-05 1.23880147614094
6.5e-05 1.23018421633
6.5e-05 1.23020928086362
6.6e-05 1.22176367261031
6.6e-05 1.22178819927822
6.7e-05 1.21350918514794
6.7e-05 1.21353319498436
6.8e-05 1.20541592441657
6.8e-05 1.2054394373673
6.9e-05 1.19747925596875
6.9e-05 1.19750229095998
7e-05 1.18969473046653
7e-05 1.18971730546846
7.1e-05 1.18205807432688
7.1e-05 1.18208020641281
7.2e-05 1.17456518093775
7.2e-05 1.17458688633877
7.3e-05 1.16721210240397
7.3e-05 1.16723339655983
7.4e-05 1.15999504178586
7.4e-05 1.16001593939218
7.5e-05 1.15279842976804
7.5e-05 1.15282129332605
7.6e-05 1.14642270717285
7.6e-05 1.14643845071834
7.7e-05 1.14133022404314
7.7e-05 1.14134733741261
7.8e-05 1.13628459406195
7.8e-05 1.13630153993797
7.9e-05 1.13130504756804
7.9e-05 1.13132175894973
8e-05 1.12639466678514
8e-05 1.12641112300978
8.1e-05 1.12155916475185
8.1e-05 1.12157538274123
8.2e-05 1.11678050154613
8.2e-05 1.11679653024055
8.3e-05 1.11207119629652
8.3e-05 1.11208699513056
8.4e-05 1.10743260310648
8.4e-05 1.10744818229099
8.5e-05 1.10286314686401
8.5e-05 1.10287854368351
8.6e-05 1.09835253660733
8.6e-05 1.09836771264506
8.7e-05 1.09390861159324
8.7e-05 1.09392358507904
8.8e-05 1.08953005884864
8.8e-05 1.08954483522031
8.9e-05 1.08521559075729
8.9e-05 1.08523017525636
9e-05 1.08096395626988
9e-05 1.08097835395033
9.1e-05 1.07677393958805
9.1e-05 1.07678815532453
9.2e-05 1.07263004125877
9.2e-05 1.07264411740584
9.3e-05 1.06854313254641
9.3e-05 1.06855703161804
9.4e-05 1.06451470644535
9.4e-05 1.06452843650216
9.5e-05 1.06054367150907
9.5e-05 1.06055723680783
9.6e-05 1.05662896239138
9.6e-05 1.05664236704369
9.7e-05 1.05276954247579
9.7e-05 1.05278279045418
9.8e-05 1.04896440289528
9.8e-05 1.04897749803899
9.9e-05 1.0452125615928
9.9e-05 1.0452255076133
0.0001 1.04151306242037
0.0001 1.04152586290657
0.000101 1.03786497427504
0.000101 1.03787763269827
0.000102 1.03426209028715
0.000102 1.03427462830341
0.000103 1.03070687327259
0.000103 1.03071927266209
0.000104 1.02720054204322
0.000104 1.02721280886316
0.000105 1.02374225751188
0.000105 1.02375439481718
0.000106 1.02033119900374
0.000106 1.02034320975297
0.000107 1.01696656647765
0.000107 1.01697845353678
0.000108 1.01364757987192
0.000108 1.01365934601791
0.000109 1.01037347847534
0.000109 1.01038512639954
0.00011 1.00714352032235
0.00011 1.00715505263377
0.000111 1.00395698161119
0.000111 1.0039684008396
0.000112 1.00081315614409
0.000112 1.00082446474298
0.000113 0.997711354788423
0.000113 0.997722555137865
0.000114 0.994650904958004
0.000114 0.994661999367368
0.000115 0.99163115011359
0.000115 0.991642140824127
0.000116 0.988651449281753
0.000116 0.988662338469084
0.000117 0.985711176591332
0.000117 0.985721966367827
0.000118 0.982809720826739
0.000118 0.982820413243789
0.000119 0.97994266428023
0.000119 0.97995329942923
0.00012 0.977105516762834
0.00012 0.97711604713054
0.000121 0.974305610690312
0.000121 0.974316049170752
0.000122 0.971537100538961
0.000122 0.971547479639024
0.000123 0.968800020376693
0.000123 0.968810304122243
0.000124 0.966098699143044
0.000124 0.966108896083491
0.000125 0.963432616906691
0.000125 0.963442728739046
0.000126 0.960801258727025
0.000126 0.96081128710184
0.000127 0.958204121128673
0.000127 0.958214067651614
0.000128 0.95564071178299
0.000128 0.955650578016352
0.000129 0.953110549200362
0.000129 0.95312033666452
0.00013 0.950613162432892
0.00013 0.950622872607697
0.000131 0.94814809078706
0.000131 0.948157725113179
0.000132 0.945714883545958
0.000132 0.945724443426159
0.000133 0.943313099700729
0.000133 0.943322586501122
0.000134 0.940942307690856
0.000134 0.940951722742076
0.000135 0.938602085152944
0.000135 0.938611429751299
0.000136 0.936292018677692
0.000136 0.936301294086259
0.000137 0.934011703574722
0.000137 0.934020911024404
0.000138 0.931760743644968
0.000138 0.931769884335514
0.000139 0.929538750960358
0.000139 0.929547826061338
0.00014 0.927345345650493
0.00014 0.927354356302243
0.000141 0.925180155696072
0.000141 0.925189103010598
0.000142 0.923042816728815
0.000142 0.923051701790667
0.000143 0.920932971837633
0.000143 0.920941795704748
0.000144 0.918850271380837
0.000144 0.918859035085348
0.000145 0.916790627935618
0.000145 0.916799343543866
0.000146 0.914757409914316
0.000146 0.914766067113742
0.000147 0.912750383687735
0.000147 0.912758983587888
0.000148 0.910769226116307
0.000148 0.910777769655654
0.000149 0.908813620348694
0.000149 0.908822108443477
0.00015 0.906883255813365
0.00015 0.906891689358248
0.000151 0.904977828066668
0.000151 0.90498620793537
0.000152 0.903096341454762
0.000152 0.903104684304138
0.000153 0.901236165955782
0.000153 0.901244452460604
0.000154 0.899400098189513
0.000154 0.899408333438616
0.000155 0.897587860175112
0.000155 0.89759604496613
0.000156 0.895799175113038
0.000156 0.895807310225443
0.000157 0.894033771529959
0.000157 0.894041857725581
0.000158 0.892291383154846
0.000158 0.892299421178357
0.000159 0.890571748798612
0.000159 0.890579739378014
0.00016 0.888874612237197
0.00016 0.888882556084273
0.000161 0.887199722097968
0.000161 0.887207619908737
0.000162 0.885546831749337
0.000162 0.885554684204481
0.000163 0.883915699193483
0.000163 0.883923506958765
0.000164 0.882306086962075
0.000164 0.88231385068874
0.000165 0.880717762014896
0.000165 0.880725482340067
0.000166 0.879150495641289
0.000166 0.879158173188337
0.000167 0.877604063364304
0.000167 0.877611698743217
0.000168 0.876078244847487
0.000168 0.876085838655224
0.000169 0.874572823804209
0.000169 0.874580376625042
0.00017 0.873087587909465
0.00017 0.873095100315308
0.000171 0.871622328714046
0.000171 0.871629801264777
0.000172 0.870176841561036
0.000172 0.870184274804807
0.000173 0.868750925504533
0.000173 0.868758319978071
0.000174 0.867344383230545
0.000174 0.867351739459444
0.000175 0.86595702097998
0.000175 0.865964339478981
0.000176 0.864588648473671
0.000176 0.864595929746938
0.000177 0.863239078839373
0.000177 0.863246323380757
0.000178 0.861908128540667
0.000178 0.861915336833961
0.000179 0.860595617307718
0.000179 0.86060278982691
0.00018 0.859301368069826
0.00018 0.859308505279337
0.000181 0.858025206889719
0.000181 0.858032309244638
0.000182 0.856766962899528
0.000182 0.85677403084584
0.000183 0.855526468238411
0.000183 0.855533502213216
0.000184 0.854303557991751
0.000184 0.854310558423477
0.000185 0.853098070131901
0.000185 0.853105037440515
0.000186 0.851909845460426
0.000186 0.851916780057629
0.000187 0.850738727551792
0.000187 0.850745629841221
0.000188 0.849583701590366
0.000188 0.849590580965111
0.000189 0.848444655308181
0.000189 0.848451501967464
0.00019 0.84732221263502
0.00019 0.847329038041656
0.000191 0.846214751067037
0.000191 0.846221543494688
0.000192 0.845123699202219
0.000192 0.845130461169298
0.000193 0.84404891773332
0.000193 0.844055649599995
0.000194 0.842990267629887
0.000194 0.842996969749453
0.000195 0.841947612243094
0.000195 0.841954284962127
0.000196 0.84092081726048
0.000196 0.840927460918989
0.000197 0.839909750661769
0.000197 0.839916365593342
0.000198 0.838914282675763
0.000198 0.838920869207706
0.000199 0.837934285738248
0.000199 0.83794084419173
0.0002 0.836969634450918
0.0002 0.836976165141099
0.000201 0.836020205541255
0.000201 0.836026708777419
0.000202 0.835085877823366
0.000202 0.835092353909045
0.000203 0.834166532159731
0.000203 0.834172981392835
0.000204 0.833262051423855
0.000204 0.833268474096786
0.000205 0.832372320463781
0.000205 0.832378716863552
0.000206 0.831497226066454
0.000206 0.831503596474802
0.000207 0.830636656922907
0.000207 0.830643001616403
0.000208 0.829790503594245
0.000208 0.829796822844402
0.000209 0.828958658478415
0.000209 0.828964952551793
0.00021 0.828141015777729
0.00021 0.828147284936036
0.000211 0.827329669105966
0.000211 0.827336435186444
0.000212 0.826467748084375
0.000212 0.826474446523515
0.000213 0.825616956598035
0.000213 0.825623642638389
0.000214 0.824779617626422
0.000214 0.824786285643889
0.000215 0.823955904191995
0.000215 0.823962564434284
0.000216 0.823146027897007
0.000216 0.823152661933553
0.000217 0.822350673723888
0.000217 0.822357292579557
0.000218 0.821569178515623
0.000218 0.82157577198371
0.000219 0.820802303629296
0.000219 0.820808873254194
0.00022 0.820049949584197
0.00022 0.820056495609372
0.000221 0.819311150411142
0.000221 0.819317690862169
0.000222 0.818585425506733
0.000222 0.818591940863564
0.000223 0.817873966674674
0.000223 0.817880459110004
0.000224 0.817176681615494
0.000224 0.817183151356995
0.000225 0.816493477741965
0.000225 0.816499925013367
0.000226 0.815824264093108
0.000226 0.815830689114278
0.000227 0.815168951307803
0.000227 0.815175354294809
0.000228 0.814526773117804
0.000228 0.814533162682081
0.000229 0.813898073198901
0.000229 0.813904440729478
0.00023 0.813283033786682
0.00023 0.813289379898016
0.000231 0.812681571938721
0.000231 0.812687896832715
0.000232 0.812093605783515
0.000232 0.812099909658604
0.000233 0.811519054906348
0.000233 0.811525337957562
0.000234 0.810957840326635
0.000234 0.81096410274566
0.000235 0.810409884475754
0.000235 0.810416126450987
0.000236 0.809875111175354
0.000236 0.809881332891964
0.000237 0.80935344561614
0.000237 0.809359647256118
0.000238 0.808844814337098
0.000238 0.808850996079318
0.000239 0.808349145205181
0.000239 0.808355307225446
0.00024 0.807866367395416
0.00024 0.807872509866512
0.000241 0.807395200720894
0.000241 0.807401340236972
0.000242 0.806936634140937
0.000242 0.806942754125524
0.000243 0.80649078903174
0.000243 0.806496889953434
0.000244 0.806057599725384
0.000244 0.806063681745976
0.000245 0.805637001470839
0.000245 0.805643064749335
0.000246 0.805228930720369
0.000246 0.805234975413032
0.000247 0.804833325112325
0.000247 0.804839351372721
0.000248 0.804450123454309
0.000248 0.804456131433346
0.000249 0.804079265706673
0.000249 0.804085255552645
0.00025 0.803720692966379
0.00025 0.803726664825003
0.000251 0.803374347451182
0.000251 0.803380301465637
0.000252 0.803040172484143
0.000252 0.803046108795108
0.000253 0.802718112478466
0.000253 0.802724031224159
0.000254 0.802408112922647
0.000254 0.802414014238856
0.000255 0.802110120365921
0.000255 0.802116004386044
0.000256 0.801824082404019
0.000256 0.801829949259096
0.000257 0.80154994766521
0.000257 0.801555797483953
0.000258 0.801287665796625
0.000258 0.801293498705456
0.000259 0.801037187450869
0.000259 0.801043003573948
0.00026 0.800798464272899
0.00026 0.800804263732156
0.000261 0.800571448887171
0.000261 0.800577231802332
0.000262 0.80035609488504
0.000262 0.800361861373664
0.000263 0.80015235681243
0.000263 0.80015810698993
0.000264 0.799960190157737
0.000264 0.799965924137411
0.000265 0.799779551339984
0.000265 0.799785269233041
0.000266 0.799610397697212
0.000266 0.799616099612797
0.000267 0.799452687475103
0.000267 0.799458373520326
0.000268 0.799306379815832
0.000268 0.799312050095792
0.000269 0.799171434747143
0.000269 0.799177089364947
0.00027 0.799047813171636
0.00027 0.799053452228429
0.000271 0.798935476856277
0.000271 0.798941100451261
0.000272 0.798834388422106
0.000272 0.798839996652565
0.000273 0.79874451133416
0.000273 0.798750104295481
0.000274 0.798665809891586
0.000274 0.798671387677277
0.000275 0.798598249217953
0.000275 0.798603811919666
0.000276 0.798541622781174
0.000276 0.798547189821034
0.000277 0.798496081361526
0.000277 0.798501633430225
0.000278 0.79846161944378
0.000278 0.798467156696546
0.000279 0.798438205418213
0.000279 0.79844372793998
0.00028 0.798425808390624
0.00028 0.79843131626456
0.000281 0.798424398242217
0.000281 0.798429891549743
0.000282 0.798433945621145
0.000282 0.79843942444195
0.000283 0.798454421934215
0.000283 0.798459886346273
0.000284 0.798485799338757
0.000284 0.798491249418342
0.000285 0.798528050734652
0.000285 0.798533486556354
0.000286 0.798581149756515
0.000286 0.798586571393255
0.000287 0.798644879261292
0.000287 0.798650267475235
0.000288 0.798719333494431
0.000288 0.798724707634294
0.000289 0.798804521774651
0.000289 0.798809881929012
0.00029 0.798900420571034
0.00029 0.798905766805811
0.000291 0.799007007024426
0.000291 0.799012339403938
0.000292 0.799124258961552
0.000292 0.799129577548534
0.000293 0.79925215488821
0.000293 0.799257459743821
0.000294 0.799390673982595
0.000294 0.799395965166436
0.000295 0.799539796088754
0.000295 0.79954507365887
0.000296 0.799699501710155
0.000296 0.799704765723057
0.000297 0.799869772003396
0.000297 0.799875022514062
0.000298 0.800050588772016
0.000298 0.800055825833906
0.000299 0.800241934460436
0.000299 0.800247158125497
0.0003 0.800443792148003
0.0003 0.800449002466683
0.000301 0.800656145543156
0.000301 0.800661342564411
0.000302 0.800878978977698
0.000302 0.800884162749001
0.000303 0.80111227740118
0.000303 0.801117447968526
0.000304 0.801356026375378
0.000304 0.801361183783296
0.000305 0.801610212068886
0.000305 0.801615356360445
0.000306 0.801874821251809
0.000306 0.801879952468625
0.000307 0.802149841290545
0.000307 0.802154959472787
0.000308 0.802435260142678
0.000308 0.802440365329074
0.000309 0.802731066351949
0.000309 0.802736158579796
0.00031 0.803037249043339
0.00031 0.803042328348506
0.000311 0.803353797918225
0.000311 0.803358864335156
0.000312 0.803680703249637
0.000312 0.803685756811363
0.000313 0.804017955877598
0.000313 0.804022996615734
0.000314 0.804365547204546
0.000314 0.804370575149301
0.000315 0.804723469190847
0.000315 0.804728484371027
0.000316 0.805091714350381
0.000316 0.805096716793394
0.000317 0.805470275746213
0.000317 0.80547526547807
0.000318 0.805859146986344
0.000318 0.805864124031666
0.000319 0.806258322219531
0.000319 0.806263286601548
0.00032 0.806667796131189
0.00032 0.806672747871749
0.000321 0.807087563939365
0.000321 0.807092503058927
0.000322 0.807517621390775
0.000322 0.807522547908422
0.000323 0.807957505744564
0.000323 0.807962404414106
0.000324 0.808406791116898
0.000324 0.808411675970348
0.000325 0.808866319224799
0.000325 0.808871191505277
0.000326 0.809336088931555
0.000326 0.809340948652361
0.000327 0.809816098377591
0.000327 0.809820945550657
0.000328 0.810306346205348
0.000328 0.81031118084124
0.000329 0.810806831555763
0.000329 0.810811653663677
0.00033 0.811317554064815
0.00033 0.81132236365258
0.000331 0.811838513860125
0.000331 0.811843310934205
0.000332 0.812369711557617
0.000332 0.812374496123108
0.000333 0.812911148258225
0.000333 0.812915920318854
0.000334 0.813462825544663
0.000334 0.81346758510279
0.000335 0.814024745478238
0.000335 0.814029492534853
0.000336 0.814596910595721
0.000336 0.814601645150444
0.000337 0.815179041737222
0.000337 0.815183756415165
0.000338 0.815770831628191
0.000338 0.815775532946871
0.000339 0.81637285727372
0.000339 0.816377546069553
0.00034 0.816985123802251
0.00034 0.816989800069253
0.000341 0.817607635959475
0.000341 0.817612299690285
0.000342 0.818240398946493
0.000342 0.818245050132369
0.000343 0.818883418416985
0.000343 0.818888057047804
0.000344 0.819536700474435
0.000344 0.819541326538687
0.000345 0.820198524287183
0.000345 0.820203121748327
0.000346 0.82087037400758
0.000346 0.820874958560935
0.000347 0.821552467789746
0.000347 0.821557039709396
0.000348 0.822244813528706
0.000348 0.822249372797395
0.000349 0.822946840696354
0.000349 0.822951379861345
0.00035 0.823658679609694
0.00035 0.823663205434736
0.000351 0.824380775454869
0.000351 0.824385288556585
0.000352 0.825111106886129
0.000352 0.825115591346425
0.000353 0.825851201307078
0.000353 0.825855665951612
0.000354 0.826600007343513
0.000354 0.826604444431268
0.000355 0.82735552097735
0.000355 0.827359898089258
0.000356 0.82806303969357
0.000356 0.828066987490023
0.000357 0.828773756177533
0.000357 0.828777683189197
0.000358 0.829493512042
0.000358 0.82949742557759
0.000359 0.830222317801549
0.000359 0.830226217827466
0.00036 0.830960176900884
0.00036 0.830964063382315
0.000361 0.831707093053012
0.000361 0.831710965953929
0.000362 0.832463070236326
0.000362 0.832466929519482
0.000363 0.8332281126917
0.000363 0.833231958318627
0.000364 0.834002224919594
0.000364 0.834006056850599
0.000365 0.83478541167716
0.000365 0.834789229871322
0.000366 0.835577677975362
0.000366 0.835581482390532
0.000367 0.8363790290761
0.000367 0.836382819668894
0.000368 0.837189470489337
0.000368 0.837193247215134
0.000369 0.838009007970231
0.000369 0.838012770783171
0.00037 0.838837647516277
0.00037 0.838841396369255
0.000371 0.839675395364438
0.000371 0.8396791302091
0.000372 0.840522257988293
0.000372 0.840525978775035
0.000373 0.841378242095172
0.000373 0.841381948773136
0.000374 0.842243354623308
0.000374 0.842247047140375
0.000375 0.84311760273897
0.000375 0.843121281041759
0.000376 0.844000993833607
0.000376 0.844004657867469
0.000377 0.844893535520983
0.000377 0.844897185229997
0.000378 0.845795235634319
0.000378 0.84579887096129
0.000379 0.846706102223416
0.000379 0.846709723109864
0.00038 0.84762577045827
0.00038 0.847629371042966
0.000381 0.848553627099486
0.000381 0.848557211800581
0.000382 0.84949065420265
0.000382 0.849494224276969
0.000383 0.850436461760596
0.000383 0.850440011343604
0.000384 0.851390442025542
0.000384 0.8513939755025
0.000385 0.852353598993227
0.000385 0.85235711764598
0.000386 0.853325943202355
0.000386 0.853329446963596
0.000387 0.854307484033888
0.000387 0.854310972835001
0.000388 0.855298231055943
0.000388 0.855301704826996
0.000389 0.856298194020834
0.000389 0.856301652690578
0.00039 0.857307382862101
0.00039 0.857310826357961
0.000391 0.858325807691525
0.000391 0.858329235939596
0.000392 0.859353478796121
0.000392 0.859356891721166
0.000393 0.86039040663514
0.000393 0.860393804160577
0.000394 0.861436601837022
0.000394 0.861439983884925
0.000395 0.86249207519636
0.000395 0.862495441687452
0.000396 0.863556837670833
0.000396 0.86356018852448
0.000397 0.864630900378127
0.000397 0.86463423551233
0.000398 0.865714274592832
0.000398 0.865717593924224
0.000399 0.866806971743323
0.000399 0.866810275187165
0.0004 0.867909003408624
0.0004 0.867912290878793
0.000401 0.869020381315242
0.000401 0.869023652724233
0.000402 0.87014111733399
0.000402 0.870144372592901
0.000403 0.871271223476774
0.000403 0.87127446249531
0.000404 0.872410711893376
0.000404 0.872413934579833
0.000405 0.873559594868187
0.000405 0.873562801129455
0.000406 0.874717884816946
0.000406 0.874721074558496
0.000407 0.875885594283422
0.000407 0.875888767409305
0.000408 0.877062735936095
0.000408 0.877065892348932
0.000409 0.878249322564796
0.000409 0.878252462165772
0.00041 0.879445367077321
0.00041 0.879448489766183
0.000411 0.88065088249602
0.000411 0.880653988171065
0.000412 0.881865881954356
0.000412 0.881868970512429
0.000413 0.883090378693426
0.000413 0.883093450029913
0.000414 0.884324386058468
0.000414 0.884327440067286
0.000415 0.885567917495316
0.000415 0.885570954068912
0.000416 0.886820986546836
0.000416 0.886824005576177
0.000417 0.888083606849325
0.000417 0.888086608223894
0.000418 0.889355792128878
0.000418 0.889358775736666
0.000419 0.890637556197715
0.000419 0.890640521925215
0.00042 0.891928912950475
0.00042 0.891931860682676
0.000421 0.893229876360482
0.000421 0.893232805980863
0.000422 0.894540460475956
0.000422 0.89454337186648
0.000423 0.895860679416208
0.000423 0.895863572457313
0.000424 0.897190547367775
0.000424 0.89719342193837
0.000425 0.898530078580533
0.000425 0.898532934557993
0.000426 0.899879287363764
0.000426 0.899882124623919
0.000427 0.901238188082173
0.000427 0.901241006499308
0.000428 0.902606795151883
0.000428 0.902609594598728
0.000429 0.90398403317112
0.000429 0.903986806429518
0.00043 0.905369293970416
0.00043 0.905372045696077
0.000431 0.9067642664024
0.000431 0.906766998772406
0.000432 0.908168967066741
0.000432 0.908171679948083
0.000433 0.909583410282916
0.000433 0.909586103541012
0.000434 0.911007610394459
0.000434 0.911010283893154
0.000435 0.912441581764646
0.000435 0.912444235366197
0.000436 0.913885338772117
0.000436 0.913887972337195
0.000437 0.915338895806464
0.000437 0.915341509194144
0.000438 0.916802267263762
0.000438 0.916804860331521
0.000439 0.918275467542059
0.000439 0.918278040145769
0.00044 0.919756722759143
0.00044 0.919759267524886
0.000441 0.921246819948694
0.000441 0.921249342617861
0.000442 0.922746752608654
0.000442 0.922749254381769
0.000443 0.924256536165189
0.000443 0.924259016892265
0.000444 0.925776184672142
0.000444 0.925778644201574
0.000445 0.927305712155111
0.000445 0.927308150333673
0.000446 0.928845132606588
0.000446 0.928847549279429
0.000447 0.930394459981038
0.000447 0.930396854991675
0.000448 0.931953708189923
0.000448 0.931956081380241
0.000449 0.933522891096675
0.000449 0.933525242306918
0.00045 0.935102022511613
0.00045 0.935104351580385
0.000451 0.936691116186801
0.000451 0.936693422951059
0.000452 0.93829018581085
0.000452 0.938292470105901
0.000453 0.939899245003667
0.000453 0.939901506663164
0.000454 0.941518307311142
0.000454 0.941520546167085
0.000455 0.943147386199779
0.000455 0.943149602082502
0.000456 0.944786495051262
0.000456 0.944788687789442
0.000457 0.946435647156975
0.000457 0.946437816577622
0.000458 0.948094855712449
0.000458 0.948097001640905
0.000459 0.949764133811753
0.000459 0.949766256071692
0.00046 0.95144349444183
0.00046 0.95144559285525
0.000461 0.953132950476757
0.000461 0.953135024863984
0.000462 0.954832514671965
0.000462 0.954834564851649
0.000463 0.95654219965837
0.000463 0.956544225447484
0.000464 0.95826201793647
0.000464 0.958264019150308
0.000465 0.959991981870353
0.000465 0.959993958322529
0.000466 0.961732103681658
0.000466 0.961734055184106
0.000467 0.963482395443469
0.000467 0.963484321806443
0.000468 0.965242869074137
0.000468 0.965244770106209
0.000469 0.967013536331046
0.000469 0.967015411839108
0.00047 0.968794408804313
0.00047 0.968796258593576
0.000471 0.970585497910415
0.000471 0.970587321784411
0.000472 0.972386814885764
0.000472 0.972388612646344
0.000473 0.974193902628575
0.000473 0.974195661389696
0.000474 0.976008975750871
0.000474 0.976010705022832
0.000475 0.977834225315011
0.000475 0.977835927925014
0.000476 0.979669664170339
0.000476 0.979671339915549
0.000477 0.981512701342033
0.000477 0.98151433969652
0.000478 0.983360932689708
0.000478 0.983362537004136
0.000479 0.98521928311699
0.000479 0.985220860016208
0.00048 0.987087767662591
0.00048 0.987089316939619
0.000481 0.988966394288281
0.000481 0.988967915734566
0.000482 0.990855170702173
0.000482 0.990856664107593
0.000483 0.992754104351696
0.000483 0.992755569504568
0.000484 0.994663202416503
0.000484 0.994664639103589
0.000485 0.996582471801322
0.000485 0.996583879807835
0.000486 0.998511919128734
0.000486 0.998513298238345
0.000487 1.00045155073189
0.000487 1.00045290072674
0.000488 1.0024013726472
0.000488 1.00240269330789
0.000489 1.00436139060686
0.000489 1.00436268171249
0.00049 1.00633161003145
0.00049 1.00633287135961
0.000491 1.0083120360224
0.000491 1.00831326734917
0.000492 1.01030267335436
0.000492 1.01030387445433
0.000493 1.01230352646757
0.000493 1.01230469711387
0.000494 1.01431459946015
0.000494 1.01431573942443
0.000495 1.01633589608035
0.000495 1.01633700513279
0.000496 1.01836741971864
0.000496 1.018368497628
0.000497 1.02040917339993
0.000497 1.02041021993352
0.000498 1.0224611597755
0.000498 1.02246217469923
0.000499 1.02452338111508
0.000499 1.02452436419343
0.0005 1.02659583929874
0.0005 1.02659679029482
0.000501 1.02867853580876
0.000501 1.0286794544843
0.000502 1.03077147172149
0.000502 1.03077235783685
0.000503 1.03287464769904
0.000503 1.03287550101325
0.000504 1.03498806398106
0.000504 1.0349888842518
0.000505 1.03711118771914
0.000505 1.03711197235444
0.000506 1.03924069279797
0.000506 1.039241438665
0.000507 1.04138037975365
0.000507 1.04138109190285
0.000508 1.04353025198549
0.000508 1.04353093017109
0.000509 1.04569030713279
0.000509 1.04569095110782
0.00051 1.04786054236129
0.00051 1.04786115187759
0.000511 1.05004095435457
0.000511 1.0500415291628
0.000512 1.05223153930542
0.000512 1.05223207915512
0.000513 1.05443229290716
0.000513 1.05443279754673
0.000514 1.05664321034493
0.000514 1.0566436795217
0.000515 1.05886428628693
0.000515 1.05886471974714
0.000516 1.06109551487562
0.000516 1.06109591236446
0.000517 1.06333688971887
0.000517 1.06333725098053
0.000518 1.06558840388109
0.000518 1.06558872865874
0.000519 1.06785004987431
0.000519 1.06785033791017
0.00052 1.07012181964923
0.00052 1.07012207068459
0.000521 1.07240370458626
0.000521 1.07240391836149
0.000522 1.07469569548647
0.000522 1.07469587174107
0.000523 1.07699426597512
0.000523 1.07699439907009
0.000524 1.0792969249878
0.000524 1.07929701205634
0.000525 1.08160954250435
0.000525 1.08160959142963
0.000526 1.08393211437134
0.000526 1.08393212489365
0.000527 1.08626462767017
0.000527 1.08626459952916
0.000528 1.08860706885029
0.000528 1.08860700178499
0.000529 1.09095942372022
0.000529 1.09095931746907
0.00053 1.0933216774385
0.00053 1.09332153173937
0.000531 1.09569381450469
0.000531 1.09569362909497
0.000532 1.09807581875035
0.000532 1.09807559336693
0.000533 1.10046767333
0.000533 1.10046740770937
0.000534 1.10286936071211
0.000534 1.10286905459034
0.000535 1.10527605678934
0.000535 1.10527570407648
0.000536 1.1076923635245
0.000536 1.10769196970697
0.000537 1.11011839070692
0.000537 1.11011795569277
0.000538 1.11255411782177
0.000538 1.11255364134813
0.000539 1.11499952345417
0.000539 1.11499900525806
0.00054 1.1174545854511
0.00054 1.11745402526943
0.000541 1.11991928091243
0.000541 1.11991867848206
0.000542 1.12238738110475
0.000542 1.12238673040253
0.000543 1.12486113449525
0.000543 1.1248604360115
0.000544 1.12734433189952
0.000544 1.1273435905924
0.000545 1.12983695196305
0.000545 1.12983616757432
0.000546 1.13233896732867
0.000546 1.13233813960034
0.000547 1.13484592421736
0.000547 1.13484504870372
0.000548 1.13736152172477
0.000548 1.13736060159953
0.000549 1.13988637151714
0.000549 1.13988540739439
0.00055 1.14240264782712
0.00055 1.14240162385693
0.000551 1.14492538922975
0.000551 1.14492431785772
0.000552 1.14745264942338
0.000552 1.1474515301317
0.000553 1.1503647684489
0.000553 1.15036375187601
0.000554 1.15419095771138
0.000554 1.15419160922725
0.000555 1.15765045432456
0.000555 1.15765033975454
0.000556 1.16113031584446
0.000556 1.16113014004843
0.000557 1.16463132276381
0.000557 1.16463108499488
0.000558 1.16815354944787
0.000558 1.16815324895081
0.000559 1.17169706970601
0.000559 1.17169670571723
0.00056 1.17526195676432
0.00056 1.1752615285118
0.000561 1.17884828323767
0.000561 1.17884778994095
0.000562 1.18245612110123
0.000562 1.18245556197137
0.000563 1.18608554166142
0.000563 1.18608491590096
0.000564 1.18973661552633
0.000564 1.18973592232927
0.000565 1.19340941257554
0.000565 1.19340865112732
0.000566 1.19710400192943
0.000566 1.19710317140691
0.000567 1.20082045191788
0.000567 1.20081955148928
0.000568 1.20455883004837
0.000568 1.2045578588733
0.000569 1.20831920297356
0.000569 1.20831816020298
0.00057 1.21210163645824
0.00057 1.21210052123443
0.000571 1.21590619534569
0.000571 1.21590500680228
0.000572 1.2197329435235
0.000572 1.21973168078541
0.000573 1.22358194388868
0.000573 1.22358060607218
0.000574 1.22745325831233
0.000574 1.22745184452498
0.000575 1.23134694760355
0.000575 1.23134545694422
0.000576 1.23526307147282
0.000576 1.23526150303171
0.000577 1.23920168849479
0.000577 1.23920004135342
0.000578 1.2431628560704
0.000578 1.24316112930164
0.000579 1.24714663038842
0.000579 1.24714482305648
0.00058 1.25115306638639
0.00058 1.25115117754687
0.000581 1.25518221771087
0.000581 1.25518024641076
0.000582 1.25923413667719
0.000582 1.2592320819549
0.000583 1.26330336057145
0.000583 1.26330121821402
0.000584 1.26738778376119
0.000584 1.26738554246379
0.000585 1.27149489178129
0.000585 1.27149256438637
0.000586 1.27562474247253
0.000586 1.27562232799552
0.000587 1.27977737830063
0.000587 1.27977487574885
0.000588 1.28395284013146
0.000588 1.28395024850419
0.000589 1.2881511671869
0.000589 1.28814848547545
0.00059 1.29236684981369
0.00059 1.29236407740995
0.000591 1.29659476736442
0.000591 1.29659188380159
0.000592 1.30084537867328
0.000592 1.30084240233062
0.000593 1.30511873250073
0.000593 1.30511566235261
0.000594 1.30941485632624
0.000594 1.30941169133976
0.000595 1.31373283983554
0.000595 1.31372958957323
0.000596 1.31805812855743
0.000596 1.31805475344344
0.000597 1.3224059822285
0.000597 1.3224025095678
0.000598 1.32677644619524
0.000598 1.32677287494153
0.000599 1.33116953533234
0.000599 1.33116586443296
0.0006 1.33558526233861
};
\addplot [semithick, color1]
table {%
0 575.95502674068
5e-07 58.2285634401653
5e-07 61.2473358154187
1e-06 22.23775777823
1e-06 21.8189774978713
1.5e-06 13.628228136359
1.5e-06 13.6011168440849
2e-06 10.6400543746958
2e-06 10.6491931537977
2.5e-06 8.82528985820176
2.5e-06 8.83228920407612
3e-06 7.5921848091781
3e-06 7.59205503976863
3.5e-06 6.69318620850872
3.5e-06 6.69633376631804
4e-06 6.01217081784217
4e-06 6.01213204848363
4.5e-06 5.47255932126204
4.5e-06 5.4743125427106
5e-06 5.0369671050664
5e-06 5.036949352724
5.5e-06 4.67377904182128
5.5e-06 4.67488186338911
6e-06 4.36729651833037
6e-06 4.36728288285531
6.5e-06 4.10487951978232
6.5e-06 4.10563336520867
7e-06 3.87553008786899
7e-06 3.87551685115054
7.5e-06 3.67414177954716
7.5e-06 3.67470772178713
8e-06 3.55864781358862
8e-06 3.55873542256361
8.5e-06 3.46170889435197
8.5e-06 3.46225085751954
9e-06 3.3717741484072
9e-06 3.37176939098877
9.5e-06 3.28704828972805
9.5e-06 3.28748421134303
1e-05 3.20763584929663
1e-05 3.20763171117188
1.05e-05 3.13176499776013
1.05e-05 3.13212595287647
1.1e-05 3.06126833282124
1.1e-05 3.06126647655892
1.15e-05 2.99375972094949
1.15e-05 2.99406682508937
1.2e-05 2.92952725569822
1.2e-05 2.92952413756718
1.25e-05 2.86890353099634
1.25e-05 2.86916376302419
1.3e-05 2.81153692205562
1.3e-05 2.81153512625145
1.35e-05 2.75696280654011
1.35e-05 2.75718388863632
1.4e-05 2.70486882251485
1.4e-05 2.70486699467863
1.45e-05 2.65451580505072
1.45e-05 2.65470782513613
1.5e-05 2.60699577831787
1.5e-05 2.60699515355278
1.55e-05 2.56139209645624
1.55e-05 2.56156008317702
1.6e-05 2.51810855077024
1.6e-05 2.51810791289477
1.65e-05 2.47641627781626
1.65e-05 2.47656469984657
1.7e-05 2.43630752354379
1.7e-05 2.43630643463484
1.75e-05 2.3973713238902
1.75e-05 2.397503118927
1.8e-05 2.36032607769168
1.8e-05 2.36032575044844
1.85e-05 2.32412152916917
1.85e-05 2.32424075558952
1.9e-05 2.28919534409729
1.9e-05 2.28919455645629
1.95e-05 2.25546137623294
1.95e-05 2.25556796835592
2e-05 2.2230212435431
2e-05 2.22302075225748
2.05e-05 2.19177499349815
2.05e-05 2.1918709242621
2.1e-05 2.16176699503316
2.1e-05 2.1617666851704
2.15e-05 2.13239147326926
2.15e-05 2.13247896209224
2.2e-05 2.10406614389363
2.2e-05 2.1040657874504
2.25e-05 2.07668298060736
2.25e-05 2.07676258432442
2.3e-05 2.05011233040199
2.3e-05 2.05011186878212
2.35e-05 2.02432003820846
2.35e-05 2.02439296847056
2.4e-05 1.99947333414354
2.4e-05 1.99947316443854
2.45e-05 1.97537820836544
2.45e-05 1.97544545438374
2.5e-05 1.95207698345879
2.5e-05 1.95207676982149
2.55e-05 1.92912536356639
2.55e-05 1.9291875358493
2.6e-05 1.90688378997564
2.6e-05 1.90688357075201
2.65e-05 1.88526346869338
2.65e-05 1.88532094942537
2.7e-05 1.86434764833199
2.7e-05 1.86434751689834
2.75e-05 1.84399009913273
2.75e-05 1.84404361136709
2.8e-05 1.82406127042704
2.8e-05 1.82406092091504
2.85e-05 1.80434146361821
2.85e-05 1.80439129003964
2.9e-05 1.78510726461783
2.9e-05 1.78510699850512
2.95e-05 1.76393373060103
2.95e-05 1.7639775928893
3e-05 1.74316819676491
3e-05 1.74316778973725
3.05e-05 1.72284086172394
3.05e-05 1.72288116955088
3.1e-05 1.70299263929608
3.1e-05 1.70299238274912
3.15e-05 1.68346540246423
3.15e-05 1.68350281729279
3.2e-05 1.66443692391848
3.2e-05 1.66443674753963
3.25e-05 1.64585287962809
3.25e-05 1.64588783949906
3.3e-05 1.6277917887017
3.3e-05 1.62779169924378
3.35e-05 1.61342265582501
3.35e-05 1.61344482067274
3.4e-05 1.6047866025206
3.4e-05 1.60478968672044
3.45e-05 1.59625783620166
3.45e-05 1.59628829917661
3.5e-05 1.58787671887699
3.5e-05 1.58787668066229
3.55e-05 1.57958427289268
3.55e-05 1.57961355356575
3.6e-05 1.57142091186716
3.6e-05 1.57142086675504
3.65e-05 1.56333939068682
3.65e-05 1.56336759144398
3.7e-05 1.55539915695118
3.7e-05 1.55539912402894
3.75e-05 1.54754168799669
3.75e-05 1.54756883470221
3.8e-05 1.53981956993683
3.8e-05 1.53981953882215
3.85e-05 1.53217652352007
3.85e-05 1.53220267849757
3.9e-05 1.5246632527622
3.9e-05 1.5246632233236
3.95e-05 1.5172122998942
3.95e-05 1.51723764377703
4e-05 1.50981876917827
4e-05 1.50981870022744
4.05e-05 1.50248007508599
4.05e-05 1.50250456481129
4.1e-05 1.49526329236939
4.1e-05 1.49526326478462
4.15e-05 1.48811767777508
4.15e-05 1.48814132347395
4.2e-05 1.4810891314398
4.2e-05 1.48108910527287
4.25e-05 1.47412873446869
4.25e-05 1.47415158304934
4.3e-05 1.46728087097522
4.3e-05 1.4672808461293
4.35e-05 1.46049830595702
4.35e-05 1.46052040096884
4.4e-05 1.45382398911322
4.4e-05 1.45382396549968
4.45e-05 1.44720985570215
4.45e-05 1.44723129755454
4.5e-05 1.44068285683301
4.5e-05 1.44068282382231
4.55e-05 1.43421624777412
4.55e-05 1.43423698043981
4.6e-05 1.42785042758999
4.6e-05 1.42785040600956
4.65e-05 1.42154260590214
4.65e-05 1.42156269623625
4.7e-05 1.41533190447945
4.7e-05 1.41533188391794
4.75e-05 1.40917689627719
4.75e-05 1.40919637718255
4.8e-05 1.40311553968218
4.8e-05 1.40311552007565
4.85e-05 1.39710769035748
4.85e-05 1.39712659255369
4.9e-05 1.39119020112041
4.9e-05 1.39119018240995
4.95e-05 1.38532414477788
4.95e-05 1.38534249697655
5e-05 1.37954532265812
5e-05 1.37954530478937
5.05e-05 1.3738159634851
5.05e-05 1.37383379254864
5.1e-05 1.36816138026411
5.1e-05 1.36816135728832
5.15e-05 1.36252051412341
5.15e-05 1.36253800303155
5.2e-05 1.35692901846071
5.2e-05 1.3569289812432
5.25e-05 1.35138452626278
5.25e-05 1.3514014729056
5.3e-05 1.34592023121973
5.3e-05 1.34592021491319
5.35e-05 1.34050126863398
5.35e-05 1.34051775710192
5.4e-05 1.3351598533896
5.4e-05 1.33515983777746
5.45e-05 1.3298621024942
5.45e-05 1.32987815399719
5.5e-05 1.32463939354714
5.5e-05 1.32463937859025
5.55e-05 1.31945875996318
5.55e-05 1.31947439442963
5.6e-05 1.31435077971762
5.6e-05 1.31435076537962
5.65e-05 1.30926748349974
5.65e-05 1.30928274794078
5.7e-05 1.30425162556541
5.7e-05 1.30425160967524
5.75e-05 1.29927515549402
5.75e-05 1.29929003201014
5.8e-05 1.29436720312387
5.8e-05 1.29436718975959
5.85e-05 1.28949185789333
5.85e-05 1.28950641447917
5.9e-05 1.28467010811545
5.9e-05 1.28467008720239
5.95e-05 1.2798852862507
5.95e-05 1.27989946800308
6e-05 1.27516521376924
6e-05 1.27516520129562
6.05e-05 1.27048082129036
6.05e-05 1.27049466728134
6.1e-05 1.2658592450079
6.1e-05 1.26585923301795
6.15e-05 1.26127113376848
6.15e-05 1.26128469951384
6.2e-05 1.25673057367869
6.2e-05 1.25673055379767
6.25e-05 1.25222348882392
6.25e-05 1.25223672027265
6.3e-05 1.24777587156928
6.3e-05 1.24777586037043
6.35e-05 1.24336063425429
6.35e-05 1.24337356913094
6.4e-05 1.2390031569108
6.4e-05 1.23900314613098
6.45e-05 1.23467697928869
6.45e-05 1.23468962956091
6.5e-05 1.23040693324204
6.5e-05 1.23040692286089
6.55e-05 1.2261671523744
6.55e-05 1.22617952937247
6.6e-05 1.22198194189827
6.6e-05 1.22198193189673
6.65e-05 1.21782600531892
6.65e-05 1.21783811977709
6.7e-05 1.21372314162323
6.7e-05 1.2137231319834
6.75e-05 1.20964860154009
6.75e-05 1.2096604636349
6.8e-05 1.20562569719774
6.8e-05 1.20562568790283
6.85e-05 1.2016302050619
6.85e-05 1.20164182444753
6.9e-05 1.197684968786
6.9e-05 1.19768495982024
6.95e-05 1.19376627020077
6.95e-05 1.19377765604178
7e-05 1.18989650194656
7e-05 1.18989649329509
7.05e-05 1.18605243189166
7.05e-05 1.18606359289336
7.1e-05 1.18225601825956
7.1e-05 1.18225600990842
7.15e-05 1.1784844966077
7.15e-05 1.17849544104429
7.2e-05 1.17475940652595
7.2e-05 1.17475939846198
7.25e-05 1.17105843383093
7.25e-05 1.17106916957159
7.3e-05 1.16740271449786
7.3e-05 1.16740270670867
7.35e-05 1.16377036803489
7.35e-05 1.1637809025681
7.4e-05 1.1601821411027
7.4e-05 1.1601821335766
7.45e-05 1.1566165711434
7.45e-05 1.15662691159945
7.5e-05 1.15300516132266
7.5e-05 1.1530050935035
7.55e-05 1.14912294473289
7.55e-05 1.14912994547938
7.6e-05 1.14655593882724
7.6e-05 1.1465564992632
7.65e-05 1.14399718613157
7.65e-05 1.14400579579764
7.7e-05 1.14146378250347
7.7e-05 1.14146378005175
7.75e-05 1.13893142147413
7.75e-05 1.13893995214081
7.8e-05 1.13641711216103
7.8e-05 1.13641710629336
7.85e-05 1.1339138266771
7.85e-05 1.13392221627121
7.9e-05 1.13143577376987
7.9e-05 1.13143577136893
7.95e-05 1.12896597239215
7.95e-05 1.12897424816145
8e-05 1.12652338399521
8e-05 1.12652338273438
8.05e-05 1.12409130923248
8.05e-05 1.12409946458647
8.1e-05 1.12168592761428
8.1e-05 1.12168592641729
8.15e-05 1.1192843963319
8.15e-05 1.11929245863292
8.2e-05 1.11690583942879
8.2e-05 1.11690583674279
8.25e-05 1.11453733313446
8.25e-05 1.11454527671786
8.3e-05 1.11219464766461
8.3e-05 1.11219464653295
8.35e-05 1.1098617596241
8.35e-05 1.10986959221148
8.4e-05 1.10755421338828
8.4e-05 1.10755421231647
8.45e-05 1.10525621325286
8.45e-05 1.10526393791771
8.5e-05 1.10298309125368
8.5e-05 1.10298309024006
8.55e-05 1.10071476156401
8.55e-05 1.10072239062834
8.6e-05 1.09847080754285
8.6e-05 1.09847080651026
8.65e-05 1.09623595614432
8.65e-05 1.09624348272859
8.7e-05 1.09402514970455
8.7e-05 1.09402514877631
8.75e-05 1.09182321553585
8.75e-05 1.09183064254288
8.8e-05 1.08964489971367
8.8e-05 1.0896448988396
8.85e-05 1.0874752324015
8.85e-05 1.08748256248677
8.9e-05 1.08532876885241
8.9e-05 1.08532876803109
8.95e-05 1.08319073659155
8.95e-05 1.08319797231525
9e-05 1.08107550505816
9e-05 1.08107550428821
9.05e-05 1.07896849389441
9.05e-05 1.07897563772567
9.1e-05 1.07688389155705
9.1e-05 1.07688389083715
9.15e-05 1.07480183688533
9.15e-05 1.07480891261652
9.2e-05 1.07273885572018
9.2e-05 1.07273885361632
9.25e-05 1.07068374974244
9.25e-05 1.0706907336963
9.3e-05 1.06865039925038
9.3e-05 1.0686503985802
9.35e-05 1.06662473441329
9.35e-05 1.06663163302085
9.4e-05 1.06462046041876
9.4e-05 1.06462045979593
9.45e-05 1.06262368249586
9.45e-05 1.06263049791254
9.5e-05 1.06064794189317
9.5e-05 1.06064794131652
9.55e-05 1.05867951310233
9.55e-05 1.05868624740986
9.6e-05 1.05673177750961
9.6e-05 1.05673177697802
9.65e-05 1.05479117460877
9.65e-05 1.05479782981827
9.7e-05 1.05287092986286
9.7e-05 1.05287092937526
9.75e-05 1.05095764365257
9.75e-05 1.05096422170752
9.8e-05 1.04906438932501
9.8e-05 1.04906438888036
9.85e-05 1.0471779241717
9.85e-05 1.04718442695075
9.9e-05 1.04531117310473
9.9e-05 1.04531117270204
9.95e-05 1.04345104648368
9.95e-05 1.04345747580326
0.0001 1.04161032434528
0.0001 1.04161032398358
0.0001005 1.03977606640225
0.0001005 1.03978242401917
0.000101 1.03796091125935
0.000101 1.03796091093772
0.0001015 1.03615062630382
0.0001015 1.03615692484162
0.000102 1.03435691599065
0.000102 1.03435691465122
0.0001025 1.03256938331184
0.0001025 1.03257561021126
0.000103 1.03080041415159
0.000103 1.03080041388825
0.0001035 1.02903747844114
0.0001035 1.02904363844104
0.000104 1.02729282484837
0.000104 1.02729282462285
0.0001045 1.02555406029049
0.0001045 1.02556015493645
0.000105 1.02383330445537
0.000105 1.02383330426686
0.0001055 1.02211829705018
0.0001055 1.02212432783891
0.000106 1.02042103171863
0.000106 1.02042103156634
0.0001065 1.01872937789962
0.0001065 1.01873534628089
0.000107 1.01705520603695
0.000107 1.01705520592012
0.0001075 1.01538651233937
0.0001075 1.01539241971781
0.000108 1.0137350468069
0.000108 1.0137350467248
0.0001085 1.01208892954895
0.0001085 1.01209477728579
0.000109 1.01045979279312
0.000109 1.01045979274504
0.0001095 1.0088358777694
0.0001095 1.00884166718412
0.00011 1.00722870152279
0.00011 1.00722870150804
0.0001105 1.00562662370944
0.0001105 1.00563235608134
0.000111 1.00404104870313
0.000111 1.00404104872105
0.0001115 1.00246045197414
0.0001115 1.00246612854389
0.000112 1.00089612766092
0.000112 1.00089612771085
0.0001125 0.999336664515131
0.0001125 0.999342286486178
0.000113 0.997793248803076
0.000113 0.997793248884416
0.0001135 0.996254580101416
0.0001135 0.996260148641369
0.000114 0.994731739097402
0.000114 0.994731739209543
0.0001145 0.993213533809801
0.0001145 0.993219050051756
0.000115 0.991710941572488
0.000115 0.991710941714847
0.0001155 0.990212876534193
0.0001155 0.990218341577993
0.000116 0.988730214836078
0.000116 0.98873021500809
0.0001165 0.987251974512873
0.0001165 0.987257389426315
0.000117 0.985788932611029
0.000117 0.985788932812149
0.0001175 0.984330208873004
0.0001175 0.984335574692992
0.000118 0.982886483288123
0.000118 0.982886483517822
0.0001185 0.981446975191616
0.0001185 0.98145229292527
0.000119 0.98001875901385
0.000119 0.980018757655898
0.0001195 0.978592358719179
0.0001195 0.978597642833276
0.00012 0.977180631493908
0.00012 0.977180631751054
0.0001205 0.975772960634109
0.0001205 0.975778198419683
0.000121 0.974379772975437
0.000121 0.974379773260121
0.0001215 0.972990547362791
0.0001215 0.972995739748641
0.000122 0.971610593938421
0.000122 0.971610591929792
0.0001225 0.970234499013275
0.0001225 0.970239658787003
0.000123 0.968872577145202
0.000123 0.96887257745917
0.0001235 0.967514462942817
0.0001235 0.967519578955556
0.000124 0.966170342407375
0.000124 0.966170342747867
0.0001245 0.964829941176483
0.0001245 0.964835014285688
0.000125 0.963503360369864
0.000125 0.963503360736434
0.0001255 0.962180412610759
0.0001255 0.96218544365036
0.000126 0.960871115777574
0.000126 0.960871116169788
0.0001265 0.959565367779611
0.0001265 0.959570357560812
0.000127 0.958273104849675
0.000127 0.958273105267116
0.0001275 0.956984308530089
0.0001275 0.95698925784214
0.000128 0.955708834960782
0.000128 0.95570883540304
0.0001285 0.954436747708994
0.0001285 0.95444165731992
0.000129 0.953177824332947
0.000129 0.953177824799628
0.0001295 0.951922208860134
0.0001295 0.951927079517451
0.00013 0.950679601738061
0.00013 0.95067960222878
0.0001305 0.949440225931783
0.0001305 0.949445058363173
0.000131 0.948213706210228
0.000131 0.948213706724612
0.0001315 0.946990342993908
0.0001315 0.946995137907875
0.000132 0.945779686767736
0.000132 0.945779687305423
0.0001325 0.944572113964813
0.0001325 0.944576872051303
0.000133 0.94337710214424
0.000133 0.943377102704875
0.0001335 0.942185102346792
0.0001335 0.942189824277806
0.000134 0.941005520528792
0.000134 0.941005521112035
0.0001345 0.939828880970488
0.0001345 0.939833567400652
0.000135 0.938664519314402
0.000135 0.93866451991992
0.0001355 0.937503031747583
0.0001355 0.937507683314713
0.000136 0.936353684854773
0.000136 0.936353685482241
0.0001365 0.935207145431524
0.0001365 0.935211762757154
0.000137 0.934072612228907
0.000137 0.934072612878011
0.0001375 0.932940821385963
0.0001375 0.932945405075869
0.000138 0.931820905013294
0.000138 0.931820905683729
0.0001385 0.930703667360636
0.0001385 0.930708218005325
0.000139 0.929598175061383
0.000139 0.92959817575285
0.0001395 0.92849529927438
0.0001395 0.92849981744957
0.00014 0.92740404229007
0.00014 0.927404043002279
0.0001405 0.92631534100504
0.0001405 0.926319827272123
0.000141 0.925238134472934
0.000141 0.925238135205603
0.0001415 0.924163424185996
0.0001415 0.924167879092475
0.000142 0.92310008703998
0.000142 0.923100087792835
0.0001425 0.922039188009075
0.0001425 0.922043612088994
0.000143 0.920989542883641
0.000143 0.920989543656415
0.0001435 0.919942279033601
0.0001435 0.919946672807954
0.000144 0.918906152170812
0.000144 0.918906152963244
0.0001445 0.917870615544721
0.0001445 0.917874985279746
0.000145 0.916845925342908
0.000145 0.916845926056944
0.0001455 0.915823514536661
0.0001455 0.915827854656529
0.000146 0.914812032939305
0.000146 0.914812033761748
0.0001465 0.913802777733178
0.0001465 0.913807088992493
0.000147 0.912804341055649
0.000147 0.91280434189713
0.0001475 0.911808078577803
0.0001475 0.9118123614502
0.000148 0.910822526141777
0.000148 0.910822527002058
0.0001485 0.909839096855392
0.0001485 0.909843351803267
0.000149 0.908866271177399
0.000149 0.90886627205625
0.0001495 0.907895518712417
0.0001495 0.907899746187261
0.00015 0.906935265426232
0.00015 0.906935266323429
0.0001505 0.905977036502594
0.0001505 0.905981236945307
0.000151 0.905029204283952
0.000151 0.905029205199277
0.0001515 0.904083348637124
0.0001515 0.904087522478328
0.000152 0.903147194094701
0.000152 0.903147194757532
0.0001525 0.902211685854965
0.0001525 0.902215839270555
0.000153 0.901286394519667
0.000153 0.901286395462347
0.0001535 0.900362994810224
0.0001535 0.900367122414822
0.000154 0.899449714595821
0.000154 0.899449715556119
0.0001545 0.898538281063145
0.0001545 0.898542383259197
0.000155 0.897636871661677
0.000155 0.897636872639392
0.0001555 0.896737264963724
0.0001555 0.896741342144498
0.000156 0.895847588775098
0.000156 0.895847589770036
0.0001565 0.894959672254829
0.0001565 0.894963724804674
0.000157 0.894081594323568
0.000157 0.894081595335539
0.0001575 0.893205233946364
0.0001575 0.893209262240962
0.000158 0.892338621900179
0.000158 0.892338622928998
0.0001585 0.891473686193056
0.0001585 0.891477690599661
0.000159 0.890618410183174
0.000159 0.890618411228661
0.0001595 0.889764770175735
0.0001595 0.889768751053411
0.00016 0.88892070281894
0.00016 0.88892070388092
0.0001605 0.888078231986013
0.0001605 0.888082189685855
0.000161 0.887245248308316
0.000161 0.887245249386616
0.0001615 0.88641382251421
0.0001615 0.886417757379566
0.000162 0.885591799896124
0.000162 0.885591800990576
0.0001625 0.884771297340462
0.0001625 0.884775209707143
0.000163 0.883960115463805
0.000163 0.883960116574247
0.0001635 0.883150416628875
0.0001635 0.883154306825357
0.000164 0.882349957425062
0.000164 0.882349958551332
0.0001645 0.881550945024631
0.0001645 0.881554813372254
0.000165 0.880761092624407
0.000165 0.880761093766352
0.0001655 0.879972651553964
0.0001655 0.879976498367125
0.000166 0.879193292238525
0.000166 0.879193293395992
0.0001665 0.878415309526893
0.0001665 0.878419135113228
0.000167 0.877646331680353
0.000167 0.877646332853193
0.0001675 0.876878696442632
0.0001675 0.876882501103196
0.000168 0.876119990505795
0.000168 0.876119991693863
0.0001685 0.875362593897596
0.0001685 0.875366377927034
0.000169 0.874614052322983
0.000169 0.874614053526137
0.0001695 0.8738667874959
0.0001695 0.873870551182619
0.00017 0.873128304704005
0.00017 0.873128305922109
0.0001705 0.872391066762303
0.0001705 0.872394810388631
0.000171 0.871662539099022
0.000171 0.871662540331939
0.0001715 0.870935225057483
0.0001715 0.870938948899825
0.000172 0.870216550752692
0.000172 0.870216552000291
0.0001725 0.8694990594956
0.0001725 0.869502763824593
0.000173 0.868790138622838
0.000173 0.868790139884989
0.0001735 0.868082370864051
0.0001735 0.86808605594471
0.000174 0.867383105301281
0.000174 0.86738310657786
0.0001745 0.866684963545375
0.0001745 0.866688629637236
0.000175 0.865995256936777
0.000175 0.86599525822766
0.0001755 0.865306645441206
0.0001755 0.865310292798463
0.000176 0.864626403159984
0.000176 0.86462640446505
0.0001765 0.863947227898243
0.0001765 0.863950856769885
0.000177 0.863276357010408
0.000177 0.863276358329542
0.0001775 0.862606525636162
0.0001775 0.862610136266099
0.000178 0.861944934865255
0.000178 0.86194493619834
0.0001785 0.861284356677397
0.0001785 0.861287949304589
0.000179 0.860631956370138
0.000179 0.860631957717062
0.0001795 0.859980542278764
0.0001795 0.859984117137342
0.00018 0.859337244371582
0.00018 0.859337245732236
0.0001805 0.858694906864834
0.0001805 0.858698464184219
0.000181 0.858060624851266
0.000181 0.858060626225543
0.0001815 0.857427277963039
0.0001815 0.857430817968057
0.000182 0.856801926861963
0.000182 0.856801928249757
0.0001825 0.856177486140429
0.0001825 0.85618100905142
0.000183 0.855560982465102
0.000183 0.855560983866313
0.0001835 0.854945364942048
0.0001835 0.854948870974975
0.000184 0.854337626669943
0.000184 0.854337628084469
0.0001845 0.85373075083087
0.0001845 0.853734240197427
0.000185 0.85313169737427
0.000185 0.853131698802013
0.0001855 0.852533483129266
0.0001855 0.852536956036974
0.000186 0.851943035306591
0.000186 0.851943036747456
0.0001865 0.851353403961926
0.0001865 0.851356860614239
0.000187 0.850771483969795
0.000187 0.850771485423689
0.0001875 0.850190358200228
0.0001875 0.850193798796622
0.000188 0.849616076291412
0.000188 0.849616077393942
0.0001885 0.849042532119589
0.0001885 0.84904596077103
0.000189 0.848476602184125
0.000189 0.848476603660345
0.0001895 0.847911425540316
0.0001895 0.847914838508246
0.00019 0.847353783321874
0.00019 0.847353784800982
0.0001905 0.846796082066592
0.0001905 0.846799483421553
0.000191 0.84624590018492
0.000191 0.846245901683334
0.0001915 0.845696432208028
0.0001915 0.845699818238425
0.000192 0.845154432554013
0.000192 0.845154434065115
0.0001925 0.844613126690394
0.0001925 0.844616497577748
0.000193 0.844079238900387
0.000193 0.844079240424092
0.0001935 0.843546025213692
0.0001935 0.843549381136056
0.000194 0.84302018013106
0.000194 0.843020181667287
0.0001945 0.842494989883786
0.0001945 0.842498331015823
0.000195 0.841977119535887
0.000195 0.841977121084558
0.0001955 0.841459885166558
0.0001955 0.841463211679619
0.000196 0.840949922742268
0.000196 0.840949924303304
0.0001965 0.840440577843156
0.0001965 0.840443889905351
0.000197 0.839938457670945
0.000197 0.83993845924427
0.0001975 0.839436936966336
0.0001975 0.839440234742609
0.000198 0.838942594492858
0.000198 0.838942596078399
0.0001985 0.838448833817832
0.0001985 0.838452117470029
0.000199 0.837962205587033
0.000199 0.837962207184716
0.0001995 0.83747614186675
0.0001995 0.837479411553689
0.0002 0.836997165499469
0.0002 0.836997167109223
0.0002005 0.83651873672895
0.0002005 0.836521992606483
0.000201 0.836047350903001
0.000201 0.836047352524756
0.0002015 0.83557649612738
0.0002015 0.83557973834846
0.000202 0.835112640558106
0.000202 0.835112642191794
0.0002025 0.834649299853352
0.0002025 0.834652528568095
0.000203 0.834192915274633
0.000203 0.834192916920188
0.0002035 0.833737029728717
0.0002035 0.833740245084462
0.000204 0.833288057874425
0.000204 0.833288059531781
0.0002045 0.832839569568933
0.0002045 0.832842771710302
0.000205 0.83239795315481
0.000205 0.832397954823904
0.0002055 0.83195680514698
0.0002055 0.831959994215934
0.000206 0.831522487852947
0.000206 0.831522489533716
0.0002065 0.831088624158118
0.0002065 0.831091800294015
0.000207 0.830661550610985
0.000207 0.830661552303367
0.0002075 0.830234916185458
0.0002075 0.830238079525104
0.000208 0.829815031942027
0.000208 0.829815033645963
0.0002085 0.829395572666317
0.0002085 0.829398723344019
0.000209 0.828982824196885
0.000209 0.828982825912317
0.0002095 0.828570486859344
0.0002095 0.828573625006968
0.00021 0.828164821531574
0.00021 0.828164823258443
0.0002105 0.827759553812407
0.0002105 0.827762679559418
0.000211 0.827355103967158
0.000211 0.82735510303156
0.0002115 0.82692130849915
0.0002115 0.82692465843198
0.000212 0.826492924396636
0.000212 0.826492925139088
0.0002125 0.826064090483299
0.0002125 0.826067437872508
0.000213 0.825641814545602
0.000213 0.825641815858062
0.0002135 0.8252194772747
0.0002135 0.825222815222491
0.000214 0.824804127331452
0.000214 0.824804128880544
0.0002145 0.824389152767532
0.0002145 0.824392487555478
0.000215 0.823980104609186
0.000215 0.823980105655448
0.0002155 0.823571460553404
0.0002155 0.823574781274731
0.000216 0.823169849632654
0.000216 0.823169851255536
0.0002165 0.822768630886936
0.0002165 0.822771939367769
0.000217 0.822374155555654
0.000217 0.822374157071386
0.0002175 0.821979738843942
0.0002175 0.821983039137079
0.000218 0.821592287214161
0.000218 0.821592288861648
0.0002185 0.821205201441786
0.0002185 0.821208489738416
0.000219 0.820825043174563
0.000219 0.820825044835774
0.0002195 0.820445237528915
0.0002195 0.820448513951722
0.00022 0.820072322496128
0.00022 0.820072324170996
0.0002205 0.819699747167238
0.0002205 0.819703011836763
0.000221 0.81933322350468
0.000221 0.819333224802633
0.0002215 0.818966760240928
0.0002215 0.818970021200184
0.000222 0.818607133600447
0.000222 0.818607135296805
0.0002225 0.818247823599147
0.0002225 0.818251073026114
0.000223 0.817895314490484
0.000223 0.817895316200395
0.0002235 0.817543109959372
0.0002235 0.817546347968769
0.000224 0.817197671506837
0.000224 0.817197673230239
0.0002245 0.816852525869822
0.0002245 0.816855752574384
0.000225 0.816514112020292
0.000225 0.816514113757126
0.0002255 0.816175979519069
0.0002255 0.816179195029579
0.000226 0.815844545028577
0.000226 0.815844546778784
0.0002265 0.815513380709328
0.0002265 0.815516585134658
0.000227 0.815188881129946
0.000227 0.815188882893471
0.0002275 0.814864459280624
0.0002275 0.814867657168082
0.000228 0.814546382341266
0.000228 0.814546383970098
0.0002285 0.814228554217102
0.0002285 0.814231740787314
0.000229 0.813917334639906
0.000229 0.813917336427369
0.0002295 0.81360635409782
0.0002295 0.813609529892579
0.00023 0.81330195011458
0.00023 0.813301951915238
0.0002305 0.812997775076911
0.0002305 0.81300094019795
0.000231 0.812700145230285
0.000231 0.812700147044087
0.0002315 0.812402734508834
0.0002315 0.812405889056135
0.000232 0.812111838077952
0.000232 0.812111839904845
0.0002325 0.811821151216932
0.0002325 0.811824295288763
0.000233 0.811536948205888
0.000233 0.811536950045823
0.0002335 0.811252945470513
0.0002335 0.811256079163456
0.000234 0.810975396597105
0.000234 0.810975398450034
0.0002345 0.810698038962423
0.0002345 0.810701162371405
0.000235 0.810427105647142
0.000235 0.810427107513018
0.0002355 0.810156354787114
0.0002355 0.81015946800543
0.000236 0.809891999142363
0.000236 0.809892001021139
0.0002365 0.809627817419167
0.0002365 0.809630920538519
0.000237 0.80937000223872
0.000237 0.809370004130351
0.0002375 0.809112352692302
0.0002375 0.809115445802814
0.000238 0.808861041440979
0.000238 0.808861043345421
0.0002385 0.80860988777881
0.0002385 0.808612970969063
0.000239 0.808365044582384
0.000239 0.808365046499595
0.0002395 0.808120351169449
0.0002395 0.808123424526503
0.00024 0.80788194080476
0.00024 0.807881942734698
0.0002405 0.807643197498108
0.0002405 0.807646269523076
0.000241 0.807410493864241
0.000241 0.807410495688307
0.0002415 0.807177925439056
0.0002415 0.807180987447502
0.000242 0.806951603467714
0.000242 0.806951605419031
0.0002425 0.806725410015314
0.0002425 0.80672846243536
0.000243 0.806505436627245
0.000243 0.806505438591251
0.0002435 0.806285584836851
0.0002435 0.806288627750016
0.000244 0.806071927224973
0.000244 0.806071929201627
0.0002445 0.80585838451391
0.0002445 0.805861418000314
0.000245 0.805651010478878
0.000245 0.805651012468144
0.0002455 0.805443744869209
0.0002455 0.80544676900759
0.000246 0.805242622810681
0.000246 0.805242624812521
0.0002465 0.805041602920537
0.0002465 0.805044617788276
0.000247 0.804846701828623
0.000247 0.804846703843003
0.0002475 0.804651896863731
0.0002475 0.804654902536869
0.000248 0.804463186310623
0.000248 0.804463188337509
0.0002485 0.804274566055992
0.0002485 0.804277562609254
0.000249 0.804092016187767
0.000249 0.804092018227125
0.0002495 0.803909550999562
0.0002495 0.803912538506376
0.00025 0.803733132528156
0.00025 0.803733134579953
0.0002505 0.803556793325736
0.0002505 0.803559771858252
0.000251 0.803386477521084
0.000251 0.803386479585288
0.0002515 0.803216235779203
0.0002515 0.803219205408313
0.000252 0.803051994461543
0.000252 0.803051996538123
0.0002525 0.802887822202716
0.0002525 0.802890782998072
0.000253 0.80272962773505
0.000253 0.802729629823975
0.0002535 0.802571497522073
0.0002535 0.802574449552105
0.000254 0.802419322802784
0.000254 0.802419324904026
0.0002545 0.802267207731412
0.0002545 0.802270151063347
0.000255 0.802121026187035
0.000255 0.802121028300564
0.0002555 0.801974899878807
0.0002555 0.801977834578683
0.000256 0.801834685456945
0.000256 0.801834687582734
0.0002565 0.801694522052157
0.0002565 0.801697448184844
0.000257 0.801560249214543
0.000257 0.801560251352564
0.0002575 0.801426023365367
0.0002575 0.80142894099458
0.000258 0.801297667081068
0.000258 0.801297669231296
0.0002585 0.801169353944808
0.0002585 0.801172263133124
0.000259 0.801046889683568
0.000259 0.801046891845977
0.0002595 0.800924464916053
0.0002595 0.800927365724928
0.00026 0.800807868641773
0.00026 0.800807870816336
0.0002605 0.80069130839089
0.0002605 0.800694200880669
0.000261 0.800580556555231
0.000261 0.800580558741925
0.0002615 0.800469837454582
0.0002615 0.800472721684522
0.000262 0.800364906990711
0.000262 0.800364909189512
0.0002625 0.800260006153403
0.0002625 0.80026288218168
0.000263 0.800160874469857
0.000263 0.800160876680743
0.0002635 0.800061769482414
0.0002635 0.800064637366141
0.000264 0.799968414457089
0.000264 0.799968416680036
0.0002645 0.799875083373491
0.0002645 0.799877943168729
0.000265 0.799787483347749
0.000265 0.799787485582737
0.0002655 0.799699904683588
0.0002655 0.799702756445361
0.000266 0.799618038456492
0.000266 0.799618040703499
0.0002665 0.799536191183241
0.0002665 0.79953903496555
0.000267 0.799460038005895
0.000267 0.799460040264901
0.0002675 0.799383901545304
0.0002675 0.799386737401136
0.000268 0.79931344111531
0.000268 0.799313443386293
0.0002685 0.799242995333899
0.0002685 0.799245823315242
0.000269 0.799178207789927
0.000269 0.79917821007287
0.0002695 0.799113432993604
0.0002695 0.799116253151457
0.00027 0.799054298910064
0.00027 0.799054301204947
0.0002705 0.798995175838833
0.0002705 0.798997988223221
0.000271 0.798941676220664
0.000271 0.798941678527469
0.0002715 0.798888186043455
0.0002715 0.798890990703437
0.000272 0.798840302321004
0.000272 0.798840304639712
0.0002725 0.798792426630597
0.0002725 0.798795223614277
0.000273 0.798750140654605
0.000273 0.798750142985201
0.0002735 0.798707861462655
0.0002735 0.798710650817194
0.000274 0.798671155499349
0.000274 0.798671157841816
0.0002745 0.798634455231507
0.0002745 0.798637237003134
0.000275 0.798603311957779
0.000275 0.798603314312099
0.0002755 0.798572108229865
0.0002755 0.798574892162975
0.000276 0.798546420192268
0.000276 0.798546422535159
0.0002765 0.798520736067978
0.0002765 0.798523512443517
0.000277 0.798500596515164
0.000277 0.798500598891826
0.0002775 0.798480460340074
0.0002775 0.798483229267627
0.000278 0.798465853213581
0.000278 0.798465855602142
0.0002785 0.798451248996575
0.0002785 0.798454010518994
0.000279 0.798442158567187
0.000279 0.798442160967634
0.0002795 0.798433070732669
0.0002795 0.798435824891918
0.00028 0.798429481661764
0.00028 0.798429484074083
0.0002805 0.798425895023454
0.0002805 0.798428641860618
0.000281 0.798427792358711
0.000281 0.798427794782888
0.0002815 0.798429692115389
0.0002815 0.798432431670686
0.000282 0.798437061286584
0.000282 0.798437063722607
0.0002825 0.798444433017922
0.0002825 0.798447165330705
0.000283 0.7984572598328
0.000283 0.798457262280655
0.0002835 0.798470089495264
0.0002835 0.798472814604032
0.000284 0.7984883601355
0.000284 0.798488362595175
0.0002845 0.79850663405834
0.0002845 0.79850935200075
0.000285 0.798530335075572
0.000285 0.798530337547056
0.0002855 0.798554039956892
0.0002855 0.798556750769759
0.000286 0.798583158268834
0.000286 0.798583160752115
0.0002865 0.798612205045177
0.0002865 0.798614899139636
0.000287 0.798646606203746
0.000287 0.798646608686388
0.0002875 0.798681012752714
0.0002875 0.798683699745936
0.000288 0.798720786704701
0.000288 0.798720789211852
0.0002885 0.79876056710579
0.0002885 0.798763247070509
0.000289 0.798805701859788
0.000289 0.798805704378611
0.0002895 0.798850844211494
0.0002895 0.798853517181009
0.00029 0.798901328086274
0.00029 0.798901330616758
0.0002905 0.798951820847689
0.0002905 0.798954486854496
0.000291 0.79900764250724
0.000291 0.799007645049376
0.0002915 0.799063474481739
0.0002915 0.799066133557541
0.000292 0.799124622931817
0.000292 0.799124625485597
0.0002925 0.799185783263635
0.0002925 0.799188435439344
0.000293 0.799252247848379
0.000293 0.799252250413792
0.0002935 0.799318726019249
0.0002935 0.799321371324992
0.000294 0.799390496417858
0.000294 0.799390498994895
0.0002945 0.799462282243719
0.0002945 0.799464920708847
0.000295 0.7995393484672
0.000295 0.799539351055852
0.0002955 0.799616432094964
0.0002955 0.799619063748055
0.000296 0.799698784482931
0.000296 0.79969878708319
0.0002965 0.799781156387318
0.0002965 0.799783781256179
0.000297 0.799868785604859
0.000297 0.799868788216717
0.0002975 0.799956436585281
0.0002975 0.799959054696961
0.000298 0.800049333619886
0.000298 0.800049336243335
0.0002985 0.800142254797399
0.0002985 0.800144866178187
0.000299 0.800240410955943
0.000299 0.800240413590975
0.0002995 0.800338593770254
0.0002995 0.800341198445686
0.0003 0.800442000676031
0.0003 0.800442003322638
0.0003005 0.800545436882564
0.0003005 0.800548034877428
0.000301 0.800654086472391
0.000301 0.800654089130566
0.0003015 0.800762768139401
0.0003015 0.800765359477741
0.000302 0.800876652660763
0.000302 0.8008766553305
0.0003025 0.800990572166513
0.0003025 0.800993156871633
0.000303 0.801109684174772
0.000303 0.80110968685606
0.0003035 0.801228834204756
0.0003035 0.801231412299224
0.000304 0.801353166560401
0.000304 0.801353169253237
0.0003045 0.801477540104627
0.0003045 0.801480111610279
0.000305 0.801607085970587
0.000305 0.801607088674963
0.0003055 0.801736676320904
0.0003055 0.801739241258846
0.000306 0.801871429159903
0.000306 0.801871431875814
0.0003065 0.802006229907379
0.0003065 0.802008788297995
0.000307 0.802146183479345
0.000307 0.802146186206783
0.0003075 0.802286188511694
0.0003075 0.802288740374642
0.000308 0.802431336871213
0.000308 0.802431339610171
0.0003085 0.802576540370273
0.0003085 0.802579085724496
0.000309 0.80272687786409
0.000309 0.802726880614564
0.0003095 0.802877274303346
0.0003095 0.802879813167071
0.00031 0.803032795567917
0.00031 0.803032798329899
0.0003105 0.803188379710065
0.0003105 0.803190912100805
0.000311 0.803349079669146
0.000311 0.803349082442631
0.0003115 0.803509846563711
0.0003115 0.803512372498269
0.000312 0.803675720426
0.000312 0.80367572321098
0.0003125 0.803841665406988
0.0003125 0.803844184901458
0.000313 0.804012708663801
0.000313 0.804012711460273
0.0003135 0.804183827347402
0.0003135 0.804186340417178
0.000314 0.804360035770401
0.000314 0.804360038578358
0.0003145 0.804536324052731
0.0003145 0.804538830712499
0.000315 0.804717693691686
0.000315 0.804717696511123
0.0003155 0.804899147746566
0.0003155 0.804901648010313
0.000316 0.805085674927161
0.000316 0.805085677758072
0.0003165 0.805272291203945
0.0003165 0.80527478508496
0.000317 0.805463972525622
0.000317 0.805463975368
0.0003175 0.805655747747055
0.0003175 0.805658235257929
0.000318 0.805852580080899
0.000318 0.80585258293474
0.0003185 0.806049511241022
0.0003185 0.806051992393649
0.000319 0.806251491727679
0.000319 0.806251494592977
0.0003195 0.806453576089763
0.0003195 0.806456050895346
0.00032 0.806660702137405
0.00032 0.806660705014155
0.0003205 0.806867937231933
0.0003205 0.806870405700981
0.000321 0.807080206514246
0.000321 0.807080209402441
0.0003215 0.807292590136916
0.0003215 0.807295052279247
0.000322 0.807510000591136
0.000322 0.80751000349077
0.0003225 0.807727530800915
0.0003225 0.807729986625657
0.000323 0.80794964502096
0.000323 0.807949647710008
0.0003235 0.808171658002458
0.0003235 0.808174099270885
0.000324 0.808398666510859
0.000324 0.808398669430844
0.0003245 0.80862580520312
0.0003245 0.808628240160333
0.000325 0.808857932205084
0.000325 0.808857935136403
0.0003255 0.80909019513237
0.0003255 0.809092623785291
0.000326 0.809327439505561
0.000326 0.809327442448209
0.0003265 0.809564825665358
0.0003265 0.809567248020229
0.000327 0.809807186539487
0.000327 0.809807189493458
0.0003275 0.810049695181116
0.0003275 0.810052111243492
0.000328 0.810297171936157
0.000328 0.810297174901443
0.0003285 0.810544802559005
0.0003285 0.810547212333756
0.000329 0.810797394823443
0.000329 0.810797397800038
0.0003295 0.81105014717523
0.0003295 0.811052550666543
0.00033 0.811307854824342
0.00033 0.811307857812238
0.0003305 0.811565728899411
0.0003305 0.811568126110789
0.000331 0.81182855205357
0.000331 0.811828555052762
0.0003315 0.812091548091212
0.0003315 0.812093939025474
0.000332 0.812359487114226
0.000332 0.812359490124706
0.0003325 0.812627605597022
0.0003325 0.812629990256305
0.000333 0.812900661094494
0.000333 0.812900664116255
0.0003335 0.813173902746699
0.0003335 0.813176281132453
0.000334 0.813452075564414
0.000334 0.813452078597447
0.0003345 0.81373044135035
0.0003345 0.81373281346334
0.000335 0.81401373257269
0.000335 0.814013735616989
0.0003355 0.814297223695181
0.0003355 0.814299589535489
0.000336 0.814585634643566
0.000336 0.814585637699122
0.0003365 0.814874252542385
0.0003365 0.814876612109407
0.000337 0.815167519605339
0.000337 0.815167522537322
0.0003375 0.815460832507036
0.0003375 0.815463181692312
0.000338 0.815759045654439
0.000338 0.815759048730699
0.0003385 0.816057479558727
0.0003385 0.816059822461756
0.000339 0.816360808276769
0.000339 0.816360811364214
0.0003395 0.816664365132053
0.0003395 0.816666701750113
0.00034 0.816972811614467
0.00034 0.816972814713091
0.0003405 0.817281493732109
0.0003405 0.817283824061793
0.000341 0.817595060401099
0.000341 0.817595063510889
0.0003415 0.817908870320896
0.0003415 0.817911194358105
0.000342 0.818227559825698
0.000342 0.818227562946646
0.0003425 0.81854650031447
0.0003425 0.818548818054414
0.000343 0.818870315529951
0.000343 0.818870318662045
0.0003435 0.819194389580138
0.0003435 0.819196701017334
0.000344 0.819523333605401
0.000344 0.819523336748633
0.0003445 0.81985183230471
0.0003445 0.819854129655315
0.000345 0.820184930907053
0.000345 0.820184933935468
0.0003455 0.820518302314316
0.0003455 0.820520592970466
0.000346 0.820856516714648
0.000346 0.820856519875948
0.0003465 0.821195012556738
0.0003465 0.821197296877586
0.000347 0.821538346751442
0.000347 0.821538349923752
0.0003475 0.821881970653256
0.0003475 0.821884248630448
0.000348 0.822230428489076
0.000348 0.822230431672382
0.0003485 0.822579138588821
0.0003485 0.822581406860744
0.000349 0.822932212445991
0.000349 0.822932215413127
0.0003495 0.823285591107028
0.0003495 0.823287852333357
0.00035 0.823643786426853
0.00035 0.823643789629889
0.0003505 0.824002296038273
0.0003505 0.824004550885687
0.000351 0.824365617915921
0.000351 0.824365621129867
0.0003515 0.82472833179901
0.0003515 0.824730572444575
0.000352 0.825095726153096
0.000352 0.825095729319339
0.0003525 0.825463450209869
0.0003525 0.82546568427889
0.000353 0.825835578643997
0.000353 0.825835581683139
0.0003535 0.82620789773742
0.0003535 0.826210121437598
0.000354 0.826584164333299
0.000354 0.826584167159476
0.0003545 0.826960232344194
0.0003545 0.826962438284051
0.000355 0.827339558731454
0.000355 0.82733956123126
0.0003555 0.827695177577369
0.0003555 0.827697155879511
0.000356 0.828048013091605
0.000356 0.828048013048827
0.0003565 0.828401139209335
0.0003565 0.828403101330741
0.000357 0.828758484106445
0.000357 0.828758487209187
0.0003575 0.829116139659319
0.0003575 0.829118095032512
0.000358 0.829478004843523
0.000358 0.829478007954093
0.0003585 0.829840188251031
0.0003585 0.829842136859474
0.000359 0.830206575331215
0.000359 0.830206578449579
0.0003595 0.83057328828682
0.0003595 0.830575230113373
0.00036 0.830944199006851
0.00036 0.830944202132976
0.0003605 0.831315443338875
0.0003605 0.831317378365787
0.000361 0.831690879576136
0.000361 0.831690882709988
0.0003615 0.832066657246294
0.0003615 0.832068585455209
0.000362 0.832446621010232
0.000362 0.832446624151775
0.0003625 0.83282693411218
0.0003625 0.832828855484131
0.000363 0.83321142754285
0.000363 0.833211430692048
0.0003635 0.833596278300739
0.0003635 0.833598192816148
0.000364 0.833985303667355
0.000364 0.833985306824171
0.0003645 0.834374694434383
0.0003645 0.834376602073058
0.000365 0.834768254133875
0.000365 0.834768257298272
0.0003655 0.835162187390845
0.0003655 0.835164088131984
0.000366 0.835560283946419
0.000366 0.835560287118357
0.0003665 0.835958762300302
0.0003665 0.835960656122483
0.000367 0.836361398359998
0.000367 0.836361401539441
0.0003675 0.836764424542501
0.0003675 0.836766311423686
0.000368 0.837171602877758
0.000368 0.837171606064663
0.0003685 0.837579179743882
0.0003685 0.837581059661416
0.000369 0.837990903248106
0.000369 0.837990906442435
0.0003695 0.838403033774723
0.0003695 0.838404906705329
0.00037 0.838819305461855
0.00037 0.838819308663566
0.0003705 0.839235992746275
0.0003705 0.839237858666052
0.000371 0.839656815749354
0.000371 0.839656818958405
0.0003715 0.840078063007894
0.0003715 0.840079921892319
0.000372 0.840503440577638
0.000372 0.840503443793985
0.0003725 0.840929251144197
0.0003725 0.840931102968119
0.000373 0.84135918664756
0.000373 0.84135918987116
0.0003735 0.841789563972191
0.0003735 0.841791408709833
0.000374 0.842224060890944
0.000374 0.842224064121751
0.0003745 0.842659008538423
0.0003745 0.842660846163376
0.000375 0.843098070467718
0.000375 0.843098073705686
0.0003755 0.843537592116119
0.0003755 0.843539422601342
0.000376 0.843981222763058
0.000376 0.84398122600814
0.0003765 0.844425322202321
0.0003765 0.844427145520139
0.000377 0.844873525384526
0.000377 0.844873528636676
0.0003775 0.845322206515028
0.0003775 0.845324022637129
0.000378 0.845774986159207
0.000378 0.845774989418375
0.0003785 0.846228252990326
0.0003785 0.846230061887761
0.000379 0.846685613130835
0.000379 0.846685616396971
0.0003795 0.847143469779517
0.0003795 0.847145271422695
0.00038 0.847605071911802
0.00038 0.847605075025768
0.0003805 0.848066842362785
0.0003805 0.848068633166075
0.000381 0.848532687382852
0.000381 0.848532690659271
0.0003815 0.848999046320611
0.0003815 0.849000829807032
0.000382 0.849469474497661
0.000382 0.849469477780849
0.0003825 0.849940426126826
0.0003825 0.84994220226491
0.000383 0.850415073462415
0.000383 0.850415076581657
0.0003835 0.850889917580425
0.0003835 0.850891682768737
0.000384 0.851368812130261
0.000384 0.851368815423199
0.0003845 0.85184824786698
0.0003845 0.851850005640648
0.000385 0.852331728709741
0.000385 0.852331732009241
0.0003855 0.852815760483424
0.0003855 0.852817510809092
0.000386 0.853303832219143
0.000386 0.853303835525147
0.0003865 0.853792464710902
0.0003865 0.853794207554564
0.000387 0.854285132034005
0.000387 0.854285135346455
0.0003875 0.854778370019262
0.0003875 0.854780105346257
0.000388 0.855275637717096
0.000388 0.855275641035931
0.0003885 0.855773486064107
0.0003885 0.855775213839111
0.000389 0.85627535901545
0.000389 0.856275362340612
0.0003895 0.856777822683817
0.0003895 0.856779542870854
0.00039 0.857284305857401
0.00039 0.857284309188826
0.0003905 0.857791389896588
0.0003905 0.857793102459012
0.000391 0.858302488349593
0.000391 0.858302491687217
0.0003915 0.858814197897425
0.0003915 0.858815902797929
0.000392 0.859329916773982
0.000392 0.859329920117741
0.0003925 0.859846257055146
0.0003925 0.859847954255753
0.000393 0.860366601584824
0.000393 0.860366604934653
0.0003935 0.860887577909355
0.0003935 0.860889267371416
0.000394 0.861412553405643
0.000394 0.861412556761474
0.0003945 0.861938171167405
0.0003945 0.8619398528516
0.000395 0.862467783026191
0.000395 0.862467786387956
0.0003955 0.862998047701347
0.0003955 0.862999721567678
0.000396 0.863532301399376
0.000396 0.863532304767005
0.0003965 0.864067218544856
0.0003965 0.864068884552644
0.000397 0.864606119638187
0.000397 0.864606123011611
0.0003975 0.865145694890138
0.0003975 0.865147352998027
0.000398 0.865689249012597
0.000398 0.865689252391741
0.0003985 0.866233488084828
0.0003985 0.866235138250771
0.000399 0.866781700946436
0.000399 0.866781704331228
0.0003995 0.867330609628855
0.0003995 0.86733225181012
0.0004 0.86788348701426
0.0004 0.867883490404624
0.0004005 0.868437071171294
0.0004005 0.868438705324459
0.000401 0.868994618938184
0.000401 0.868994622334044
0.0004015 0.869552884507195
0.0004015 0.869554510588145
0.000402 0.870115108584707
0.000402 0.870115111985985
0.0004025 0.870678061574393
0.0004025 0.870679679538312
0.000403 0.8712449679615
0.000403 0.871244971368116
0.0004035 0.871812614450286
0.0004035 0.871814224251666
0.000404 0.872384209214184
0.000404 0.872384212626058
0.0004045 0.872956555348603
0.0004045 0.872958156941225
0.000405 0.873532844623078
0.000405 0.873532848040127
0.0004055 0.874109896616132
0.0004055 0.874111489953075
0.000406 0.874690886599917
0.000406 0.874690890022057
0.0004065 0.875272650729435
0.0004065 0.87527423576307
0.000407 0.875858347684556
0.000407 0.875858351111701
0.0004075 0.876444830291532
0.0004075 0.876446406973517
0.000408 0.877035240541638
0.000408 0.877035243973701
0.0004085 0.877626448028558
0.0004085 0.877628016309839
0.000409 0.878221577957239
0.000409 0.878221581394131
0.0004095 0.878817516786399
0.0004095 0.8788190766172
0.00041 0.879417372835484
0.00041 0.879417376277115
0.0004105 0.880018049527282
0.0004105 0.88001960085711
0.000411 0.880622638195137
0.000411 0.880622641641415
0.0004115 0.881228059326363
0.0004115 0.881229602103998
0.000412 0.881837387166156
0.000412 0.881837390616986
0.0004125 0.882447559368258
0.0004125 0.882449093541754
0.000413 0.883061632986223
0.000413 0.883061636441511
0.0004135 0.883676562943564
0.0004135 0.883678088460244
0.000414 0.884295388997243
0.000414 0.884295392456892
0.0004145 0.884915083445336
0.0004145 0.88491660025179
0.000415 0.885538668641806
0.000415 0.885538672105715
0.0004155 0.886163134365538
0.0004155 0.88616464240762
0.000416 0.886791485459622
0.000416 0.886791488927692
0.0004165 0.887420729291461
0.0004165 0.887422228514285
0.000417 0.888053853083921
0.000417 0.888053856556049
0.0004175 0.888687881902102
0.0004175 0.88868937225004
0.000418 0.889325785237821
0.000418 0.889325788713903
0.0004185 0.88996460596452
0.0004185 0.889966087381196
0.000419 0.890607295730654
0.000419 0.890607299210582
0.0004195 0.891250915330141
0.0004195 0.89125238775843
0.00042 0.891898398454266
0.00042 0.891898401937934
0.0004205 0.892546823931042
0.0004205 0.892548287313069
0.000421 0.893199107379279
0.000421 0.893199110866575
0.0004215 0.893852345776189
0.0004215 0.893853800053321
0.000422 0.894509436551305
0.000422 0.894509440042118
0.0004225 0.895167494947644
0.0004225 0.895168940060491
0.000423 0.895829400087141
0.000423 0.895829403581356
0.0004235 0.896492285596728
0.0004235 0.896493721485138
0.000424 0.897159012170908
0.000424 0.89715901566841
0.0004245 0.897826731940149
0.0004245 0.897828158543205
0.000425 0.898498287050162
0.000425 0.898498290550833
0.0004255 0.899170848256089
0.0004255 0.899172265512107
0.000426 0.899847239031963
0.000426 0.899847242535681
0.0004265 0.900524648880252
0.0004265 0.900526056726775
0.000427 0.901205882478895
0.000427 0.901205885985539
0.0004275 0.901888148201871
0.0004275 0.901889546575671
0.000428 0.902574231805058
0.000428 0.902574235314504
0.0004285 0.903261360659669
0.0004285 0.903262749496739
0.000429 0.903951277662228
0.000429 0.903951280699632
0.0004295 0.904641881359961
0.0004295 0.904643255909771
0.00043 0.905336277894618
0.00043 0.905336281402339
0.0004305 0.90603174184672
0.0004305 0.906033106732372
0.000431 0.906730992066321
0.000431 0.906730995576393
0.0004315 0.907431322978817
0.0004315 0.907432678134222
0.000432 0.908135433999103
0.000432 0.908135437511392
0.0004325 0.908840639060993
0.0004325 0.908841984419281
0.000433 0.909549618011011
0.000433 0.909549621525383
0.0004335 0.910259704424307
0.0004335 0.910261039917822
0.000434 0.910973558444258
0.000434 0.910973561960574
0.0004345 0.911688533421829
0.0004345 0.911689858982126
0.000435 0.912407269660903
0.000435 0.912407273179023
0.0004355 0.9131271404243
0.0004355 0.913128455982142
0.000436 0.913850766038484
0.000436 0.913850769558264
0.0004365 0.914575539815734
0.0004365 0.914576845301094
0.000437 0.915304061965598
0.000437 0.915304065486896
0.0004375 0.916033745988988
0.0004375 0.916035041331039
0.000438 0.916767171837445
0.000438 0.916767175360112
0.0004385 0.917501773341267
0.0004385 0.917503058468383
0.000439 0.918240110051308
0.000439 0.918240113575195
0.0004395 0.918979303469533
0.0004395 0.918980575472118
0.00044 0.919721168419736
0.00044 0.91972117145905
0.0004405 0.92046423042485
0.0004405 0.920465490629822
0.000441 0.921211001645172
0.000441 0.921211005164174
0.0004415 0.921958985757138
0.0004415 0.921960235532503
0.000442 0.922710671695832
0.000442 0.922710675215531
0.0004425 0.923463584514024
0.0004425 0.92346482378524
0.000443 0.924220192169924
0.000443 0.92422019569016
0.0004435 0.924978040761896
0.0004435 0.924979269453614
0.000444 0.925739577121183
0.000444 0.925739580641794
0.0004445 0.926502368541434
0.0004445 0.926503586577497
0.000445 0.927268840575226
0.000445 0.927268844096047
0.0004455 0.928036581862778
0.0004455 0.928037789166222
0.000446 0.928807996524689
0.000446 0.928808000045553
0.0004465 0.929580694700633
0.0004465 0.929581891193679
0.000447 0.930357058924305
0.000447 0.93035706244504
0.0004475 0.931134720989323
0.0004475 0.931135906593382
0.000448 0.931916041685934
0.000448 0.931916045206369
0.0004485 0.932698674617796
0.0004485 0.932699849253461
0.000449 0.933484958673537
0.000449 0.933484962193496
0.0004495 0.934272569424564
0.0004495 0.934273733011611
0.00045 0.935063823698091
0.00045 0.935063827217394
0.0004505 0.935856419192597
0.0004505 0.93585757164998
0.000451 0.93665265051245
0.000451 0.936652654030918
0.0004515 0.937450237644151
0.0004515 0.937451378890005
0.000452 0.938251452806155
0.000452 0.938251456323602
0.0004525 0.939054038435547
0.0004525 0.939055168387184
0.000453 0.939860244200173
0.000453 0.939860247716413
0.0004535 0.940667835151894
0.0004535 0.940668953725799
0.000454 0.941479038241596
0.000454 0.941479041756439
0.0004545 0.942291641301739
0.0004545 0.942292748413575
0.000455 0.943107848398266
0.000455 0.943107851911519
0.0004555 0.943925470311684
0.0004555 0.943926565876282
0.000456 0.944746688053354
0.000456 0.944746691564821
0.0004565 0.945569335520915
0.0004565 0.945570419452278
0.000457 0.946395570499865
0.000457 0.946395574009347
0.0004575 0.947223250175695
0.0004575 0.947224322386997
0.000458 0.9480545089351
0.000458 0.948054512442395
0.0004585 0.948887227423785
0.0004585 0.948888287827366
0.000459 0.949723516455047
0.000459 0.94972351995995
0.0004595 0.950561280308806
0.0004595 0.950562328816175
0.00046 0.95140260604871
0.00046 0.951402609551014
0.0004605 0.952245421764541
0.0004605 0.952246458286373
0.000461 0.953091790592386
0.000461 0.953091794091878
0.0004615 0.953939664609176
0.0004615 0.953940689055311
0.000462 0.95479108284387
0.000462 0.954791086340336
0.0004625 0.955644021539477
0.0004625 0.955645033818922
0.000463 0.956500495436602
0.000463 0.956500498929824
0.0004635 0.957358505124911
0.0004635 0.957359505145834
0.000464 0.958220040873755
0.000464 0.958220044363512
0.0004645 0.959083127801689
0.0004645 0.959084115471424
0.000465 0.959949731522253
0.000465 0.959949735008322
0.0004655 0.960817901866766
0.0004655 0.960818877091808
0.000466 0.961689579606732
0.000466 0.961689583088883
0.0004665 0.962562839471758
0.0004665 0.962563802157767
0.000467 0.963439597203427
0.000467 0.963439600681431
0.0004675 0.96431795261681
0.0004675 0.96431890266861
0.000468 0.965199796234012
0.000468 0.965199799707634
0.0004685 0.96608325314439
0.0004685 0.966084190465965
0.000469 0.966970188459356
0.000469 0.966970191928357
0.0004695 0.967858752733023
0.0004695 0.967859677227522
0.00047 0.968750785473226
0.00047 0.968750788937366
0.0004705 0.969644462890957
0.0004705 0.969645374460692
0.000471 0.970541598695924
0.000471 0.970541602154957
0.0004715 0.971440394949764
0.0004715 0.971441293496212
0.000472 0.972342639367851
0.000472 0.97234264282153
0.0004725 0.973245589985781
0.0004725 0.973246470819238
0.000473 0.974149611221705
0.000473 0.974149613574681
0.0004735 0.975055307988386
0.0004735 0.975056172375261
0.000474 0.975964405179318
0.000474 0.975964408605729
0.0004745 0.976875197964681
0.0004745 0.976876049055499
0.000475 0.977789379297098
0.000475 0.977789382717288
0.0004755 0.978705272011301
0.0004755 0.978706109705201
0.000476 0.979624542321387
0.000476 0.979624545735093
0.0004765 0.980545539847225
0.0004765 0.980546364042533
0.000477 0.981467488934647
0.000477 0.98146749121985
0.0004775 0.982389800993165
0.0004775 0.982390603061689
0.000478 0.983315438110994
0.000478 0.983315441493104
0.0004785 0.984242819838035
0.0004785 0.984243608235908
0.000479 0.985173513951773
0.000479 0.985173517326486
0.0004795 0.986105968431857
0.0004795 0.986106743056149
0.00048 0.987041723567049
0.00048 0.987041726934088
0.0004805 0.98797925486673
0.0004805 0.987980015613727
0.000481 0.988920074924417
0.000481 0.988920078283504
0.0004815 0.989862686984776
0.0004815 0.989863433749982
0.000482 0.990808575738002
0.000482 0.990808579088855
0.0004825 0.991756272371141
0.0004825 0.991757005049283
0.000483 0.992707233461439
0.000483 0.992707236803772
0.0004835 0.99366001834694
0.0004835 0.99366073683197
0.000484 0.994616055280781
0.000484 0.994616058614301
0.0004845 0.995573931962138
0.0004845 0.995574636147235
0.000485 0.996535048107344
0.000485 0.996535051431759
0.0004855 0.997498019988364
0.0004855 0.99749870976594
0.000486 0.9984642185705
0.000486 0.998464221885511
0.0004865 0.999432288911666
0.0004865 0.99943296417337
0.000487 1.00040357301039
0.000487 1.0004035763157
0.0004875 1.0013767449252
0.0004875 1.00137740556192
0.000488 1.0023531174706
0.000488 1.0023531207659
0.0004885 1.00333139392187
0.0004885 1.00333203982373
0.000489 1.00431285769073
0.000489 1.00431286097571
0.0004895 1.00529624148687
0.0004895 1.00529687254325
0.00049 1.00628279909896
0.00049 1.0062828023733
0.0004905 1.00727129289021
0.0004905 1.00727190898975
0.000491 1.00826294680451
0.000491 1.0082629500679
0.0004915 1.00925655307914
0.0004915 1.00925715410973
0.000492 1.01025330559004
0.000492 1.01025330884215
0.0004925 1.01125202667057
0.0004925 1.01125261251935
0.000493 1.01225387990402
0.000493 1.01225388314453
0.0004935 1.01325771794333
0.0004935 1.01325828849673
0.000494 1.01426467385301
0.000494 1.01426467708159
0.0004945 1.01527363083047
0.0004945 1.01527418597418
0.000495 1.01628569119388
0.000495 1.0162856944102
0.0004955 1.01729976891145
0.0004955 1.01730030853045
0.000496 1.018316935326
0.000496 1.01831693852972
0.0004965 1.01933613540427
0.0004965 1.01933665938283
0.000497 1.02035840928334
0.000497 1.02035841247411
0.0004975 1.02138273315758
0.0004975 1.02138324137923
0.000498 1.0224101157265
0.000498 1.02241011890397
0.0004985 1.02343956464264
0.0004985 1.02344005699025
0.000499 1.02447205693473
0.000499 1.02447206009856
0.0004995 1.02550663194535
0.0004995 1.02550710830107
0.0005 1.02654423479786
0.0005 1.02654423794768
0.0005005 1.0275839367581
0.0005005 1.02758439700341
0.000501 1.02862665080814
0.000501 1.0286266539436
0.0005015 1.02967148037166
0.0005015 1.02967192438735
0.000502 1.03071930605211
0.000502 1.03071930917286
0.0005025 1.03176926366694
0.0005025 1.03176969133314
0.000503 1.03282220120235
0.000503 1.032822204308
0.0005035 1.03387728710674
0.0005035 1.03387769830291
0.000504 1.03493533650916
0.000504 1.03493533959934
0.0005045 1.03599555072745
0.0005045 1.03599594533239
0.000505 1.03705828792494
0.000505 1.0370582907985
0.0005055 1.03812144095306
0.0005055 1.038121815058
0.000506 1.03918750761236
0.000506 1.03918751066079
0.0005065 1.04025576295444
0.0005065 1.04025612025381
0.000507 1.04132691455868
0.000507 1.04132691759047
0.0005075 1.04240027119979
0.0005075 1.04240061157132
0.000508 1.04347650668875
0.000508 1.0434765097035
0.0005085 1.04455496357407
0.0005085 1.04455528689488
0.000509 1.04563628165375
0.000509 1.04563628465108
0.0005095 1.04671983749315
0.0005095 1.04672014363978
0.00051 1.04780623663155
0.00051 1.04780623961105
0.0005105 1.0488948898953
0.0005105 1.0488951787437
0.000511 1.04998636831808
0.000511 1.04998637127936
0.0005115 1.05108011723254
0.0005115 1.05108038865809
0.000512 1.05217667291874
0.000512 1.05217667586139
0.0005125 1.05327551546203
0.0005125 1.05327576933956
0.000513 1.05437714613973
0.000513 1.05437714906333
0.0005135 1.05548108003735
0.0005135 1.05548131624113
0.000514 1.05658778317928
0.000514 1.05658778608342
0.0005145 1.05769680589975
0.0005145 1.05769702430353
0.000515 1.05880857871895
0.000515 1.05880858160321
0.0005155 1.05992268746941
0.0005155 1.0599228879464
0.000516 1.06103952691481
0.000516 1.06103952977876
0.0005165 1.06215871863658
0.0005165 1.06215890105949
0.000517 1.06328062138859
0.000517 1.06328062423181
0.0005175 1.06440489275274
0.0005175 1.06440505699378
0.000518 1.06553185521882
0.000518 1.06553185804086
0.0005185 1.06666120262171
0.0005185 1.0666613485526
0.000519 1.06779322093188
0.000519 1.06779322373231
0.0005195 1.06892764049071
0.0005195 1.06892776798271
0.00052 1.07006471049313
0.00052 1.07006471327151
0.0005205 1.07120419804139
0.0005205 1.07120430696531
0.000521 1.07234631529785
0.000521 1.07234631805373
0.0005215 1.07349086638085
0.0005215 1.07349095660707
0.000522 1.07463802616227
0.000522 1.07463802889519
0.0005225 1.07578763603263
0.0005225 1.07578770743107
0.000523 1.07693655428784
0.000523 1.07693655549123
0.0005235 1.07808647362368
0.0005235 1.07808651938813
0.000524 1.07923892671724
0.000524 1.07923892938313
0.0005245 1.08039384191448
0.0005245 1.08039386867116
0.000525 1.081551267443
0.000525 1.08155127008454
0.0005255 1.08271117101685
0.0005255 1.08271117863653
0.000526 1.08387356269932
0.000526 1.08387356531603
0.0005265 1.08503844834132
0.0005265 1.08503843669441
0.000527 1.08620579958405
0.000527 1.08620580217547
0.0005275 1.08737566067116
0.0005275 1.08737562962778
0.000528 1.08854796456337
0.000528 1.08854796712902
0.0005285 1.08972279415349
0.0005285 1.08972274358345
0.000529 1.09090004346277
0.000529 1.09090004600217
0.0005295 1.09207983429019
0.0005295 1.09207976406303
0.00053 1.09326202145805
0.00053 1.09326202397071
0.0005305 1.09444676592896
0.0005305 1.09444667591395
0.000531 1.09563388306627
0.000531 1.09563388555171
0.0005315 1.09682357325424
0.0005315 1.09682346332043
0.000532 1.09801561213677
0.000532 1.0980156145945
0.0005325 1.09921023977822
0.0005325 1.09921010979444
0.000533 1.10040719184214
0.000533 1.10040719427165
0.0005335 1.1016067483318
0.0005335 1.10160659816666
0.000534 1.10280860466913
0.000534 1.10280860706992
0.0005345 1.10401086945291
0.0005345 1.10401069631477
0.000535 1.10521514553622
0.000535 1.10521514778571
0.0005355 1.10642204598356
0.0005355 1.10642185211005
0.000536 1.10763117959378
0.000536 1.10763118192583
0.0005365 1.10884295341948
0.0005365 1.10884273901826
0.000537 1.11005693476705
0.000537 1.11005693706886
0.0005375 1.1112735716811
0.0005375 1.11127333662131
0.000538 1.11249239027756
0.000538 1.11249239254862
0.0005385 1.11371387973886
0.0005385 1.11371362388957
0.000539 1.1149375247301
0.000539 1.11493752696988
0.0005395 1.1161638558297
0.0005395 1.11616357905993
0.00054 1.11739231599151
0.00054 1.11739231819949
0.0005405 1.1186234774482
0.0005405 1.11862317962692
0.000541 1.11985674118185
0.000541 1.11985674335749
0.0005415 1.12109179443705
0.0005415 1.12109147547402
0.000542 1.12232482421327
0.000542 1.12232482450116
0.0005425 1.12356055592749
0.0005425 1.12356021022203
0.000543 1.124798304221
0.000543 1.12479830631116
0.0005435 1.12603877655535
0.0005435 1.12603840951195
0.000544 1.1272812358397
0.000544 1.12728123789596
0.0005445 1.12852643384491
0.0005445 1.12852604533496
0.000545 1.12977359068379
0.000545 1.12977359270562
0.0005455 1.13102350085428
0.0005455 1.13102309074932
0.000546 1.13227534141747
0.000546 1.13227534340432
0.0005465 1.13352818948132
0.0005465 1.13352775615668
0.000547 1.13478215865822
0.000547 1.13478216025665
0.0005475 1.13603889752863
0.0005475 1.13603844137251
0.000548 1.13729749303688
0.000548 1.13729749494272
0.0005485 1.13855887387837
0.0005485 1.1385583957994
0.000549 1.13982208157463
0.000549 1.13982208344387
0.0005495 1.14108097061179
0.0005495 1.14108046500927
0.00055 1.14233859625206
0.00055 1.14233859671373
0.0005505 1.14359898521024
0.0005505 1.14359845364101
0.000551 1.14486107893443
0.000551 1.14486108069211
0.0005515 1.14612419124038
0.0005515 1.14612363636606
0.000552 1.14738821006918
0.000552 1.14738821144673
0.0005525 1.14865391295879
0.0005525 1.14865333577654
0.000553 1.15026591312684
0.000553 1.15026614839828
0.0005535 1.15220519646958
0.0005535 1.15220562527358
0.000554 1.15410503628784
0.000554 1.15410501262835
0.0005545 1.15583152161772
0.0005545 1.15583146847629
0.000555 1.15756322978644
0.000555 1.15756323333586
0.0005555 1.15930028222796
0.0005555 1.15930019862644
0.000556 1.16104251868584
0.000556 1.16104252219748
0.0005565 1.16279014869363
0.0005565 1.16279003426109
0.000557 1.16454295064393
0.000557 1.16454295411668
0.0005575 1.16630119565937
0.0005575 1.16630105002074
0.000558 1.16806460003586
0.000558 1.16806460346856
0.0005585 1.1698334972282
0.0005585 1.16983332000425
0.000559 1.17160754068158
0.000559 1.17160754407307
0.0005595 1.1733871269343
0.0005595 1.17338691774162
0.00056 1.17517184581849
0.00056 1.17517184916757
0.0005605 1.1769621577154
0.0005605 1.17696191616638
0.000561 1.17875758807347
0.000561 1.17875759137894
0.0005615 1.18055866188461
0.0005615 1.18055838758743
0.000562 1.18236483943447
0.000562 1.18236484269509
0.0005625 1.18417671110161
0.0005625 1.18417640366022
0.000563 1.18599367122144
0.000563 1.18599367443596
0.0005635 1.18781637634342
0.0005635 1.18781603535753
0.000564 1.18964415405679
0.000564 1.18964415722392
0.0005645 1.19147772787448
0.0005645 1.19147735293953
0.000565 1.19331635783522
0.000565 1.19331636095366
0.0005655 1.19516083521627
0.0005655 1.19516042592346
0.000566 1.19701035169303
0.000566 1.19701035476146
0.0005665 1.19886576711632
0.0005665 1.19886532305254
0.000567 1.20072620397684
0.000567 1.2007262069939
0.0005675 1.2025925915166
0.0005675 1.20259211226447
0.000568 1.20446398221174
0.000568 1.20446398517606
0.0005685 1.20634137552143
0.0005685 1.20634086065927
0.000569 1.20822375306885
0.000569 1.20822375597902
0.0005695 1.21011218536472
0.0005695 1.21011163446653
0.00057 1.21200558233229
0.00057 1.21200558518689
0.0005705 1.21390508637667
0.0005705 1.21390449901215
0.000571 1.2158095348656
0.000571 1.21580953766318
0.0005715 1.21772014294983
0.0005715 1.21771951868438
0.000572 1.2196356745775
0.000572 1.21963567731659
0.0005725 1.22155741850466
0.0005725 1.22155675689933
0.000573 1.22348406438713
0.000573 1.22348406706623
0.0005735 1.22541697545437
0.0005735 1.22541627606592
0.000574 1.22735476618861
0.000574 1.22735476880618
0.0005745 1.22929887516924
0.0005745 1.2292981375501
0.000575 1.23124784081505
0.000575 1.23124784336955
0.0005755 1.23320317794032
0.0005755 1.2332024016386
0.000576 1.23516334800195
0.000576 1.23516335049179
0.0005765 1.2371299429425
0.0005765 1.23712912750201
0.000577 1.23910134634998
0.000577 1.23910134877355
0.0005775 1.24107922819698
0.0005775 1.24107837315722
0.000578 1.24306189328711
0.000578 1.24306189564277
0.0005785 1.24505109053316
0.0005785 1.24505019542932
0.000579 1.2470450450302
0.000579 1.2470450473163
0.0005795 1.24904558554983
0.0005795 1.24904464991283
0.00058 1.25105085654594
0.00058 1.25105085876078
0.0005805 1.25306276757583
0.0005805 1.2530617909323
0.000581 1.25507938151113
0.000581 1.25507938365299
0.0005815 1.25710268963008
0.0005815 1.25710167150238
0.000582 1.25913067227244
0.000582 1.25913067433957
0.0005825 1.26116540338093
0.0005825 1.26116434328719
0.000583 1.26319959902925
0.000583 1.26319959785514
0.0005835 1.26523921105103
0.0005835 1.26523810010094
0.000584 1.26728337910123
0.000584 1.2672833809815
0.0005845 1.26933435844882
0.0005845 1.26933320468483
0.000585 1.2713898592823
0.000585 1.27138986108239
0.0005855 1.2734522252861
0.0005855 1.27345102821911
0.000586 1.27551908049815
0.000586 1.2755190822162
0.0005865 1.27759285485115
0.0005865 1.27759161398804
0.000587 1.27967108525214
0.000587 1.27967108688626
0.0005875 1.28175628885549
0.0005875 1.28175500369912
0.000588 1.28384591444903
0.000588 1.28384591599731
0.0005885 1.28594256739
0.0005885 1.28594123743927
0.000589 1.28804360735083
0.000589 1.28804360881133
0.0005895 1.2901517288805
0.0005895 1.29015035363037
0.00059 1.29225906639631
0.00059 1.29225906462911
0.0005905 1.29437057693536
0.0005905 1.29436914655589
0.000591 1.29648633114624
0.000591 1.29648633238801
0.0005915 1.29860921986796
0.0005915 1.29860774335386
0.000592 1.30073631084211
0.000592 1.30073631199014
0.0005925 1.30287058983472
0.0005925 1.30286906667641
0.000593 1.30500903179184
0.000593 1.3050090328441
0.0005935 1.30715471531079
0.0005935 1.30715314499508
0.000594 1.30930452152191
0.000594 1.3093045224763
0.0005945 1.31146162286573
0.0005945 1.31146000487586
0.000595 1.31362228877038
0.000595 1.31362228930663
0.0005955 1.3157826192199
0.0005955 1.3157809443573
0.000596 1.31794691897829
0.000596 1.31794691969597
0.0005965 1.32011856772054
0.0005965 1.32011684435377
0.000597 1.32229414176754
0.000597 1.3222941423811
0.0005975 1.32447711756628
0.0005975 1.32447534517542
0.000598 1.32666397390918
0.000598 1.32666397441645
0.0005985 1.32885828475636
0.0005985 1.3288564628183
0.000599 1.331056430332
0.000599 1.33105643073077
0.0005995 1.33326208313998
0.0005995 1.33326021112851
0.0006 1.33547152379031
};

\nextgroupplot[
    x label style={at={(axis description cs:0.5,-0.05)},anchor=north},
    y label style={at={(axis description cs:-0.005,.5)},rotate=0,anchor=south},
tick align=outside,
tick pos=left,
x grid style={white!69.0196078431373!black},
xlabel={Время},
xmin=-3e-05, xmax=0.00063,
xtick style={color=black},
xtick={-0.0002,0,0.0002,0.0004,0.0006,0.0008},
xticklabels={−0.0002,0.0000,0.0002,0.0004,0.0006,0.0008},
y grid style={white!69.0196078431373!black},
ylabel={$I \cdot R_p$},
ymin=112.799342986641, ymax=656.289549192947,
ytick style={color=black},
ytick={0,150,...,600}
%yticklabels={0,200,400,600}
]
\addplot [semithick, color0]
table {%
0 172.786508022204
1e-06 146.550189074281
1e-06 146.204841379959
2e-06 143.676895915286
2e-06 143.680539601775
3e-06 153.124527852231
3e-06 153.089357820946
4e-06 161.048812229841
4e-06 161.019976941744
5e-06 168.007127097309
5e-06 167.983716847793
6e-06 174.156370849617
6e-06 174.137035426243
7e-06 179.658126805758
7e-06 179.642530960629
8e-06 188.10136293504
8e-06 188.048580220044
9e-06 199.701703890202
9e-06 199.638686257731
1e-05 210.174890058131
1e-05 210.122629874877
1.1e-05 219.674525545063
1.1e-05 219.629473781723
1.2e-05 228.311550201573
1.2e-05 228.273683173916
1.3e-05 236.334131817385
1.3e-05 236.300539008206
1.4e-05 243.788821761215
1.4e-05 243.759706895702
1.5e-05 250.677484556486
1.5e-05 250.651149607375
1.6e-05 257.185450657133
1.6e-05 257.161747475625
1.7e-05 263.288146317827
1.7e-05 263.26713970845
1.8e-05 268.990649081992
1.8e-05 268.971411780745
1.9e-05 274.283615505379
1.9e-05 274.266584702243
2e-05 279.286229930435
2e-05 279.270597115767
2.1e-05 284.087984883009
2.1e-05 284.07333810754
2.2e-05 288.592689220235
2.2e-05 288.579391922268
2.3e-05 292.901075966022
2.3e-05 292.888755368373
2.4e-05 297.021993916484
2.4e-05 297.010332058066
2.5e-05 301.003793410876
2.5e-05 300.992789366974
2.6e-05 304.741312933824
2.6e-05 304.731264173713
2.7e-05 308.356963830153
2.7e-05 308.347354433687
2.8e-05 311.824986274002
2.8e-05 311.816127605146
2.9e-05 315.010014890252
2.9e-05 315.004000399605
3e-05 317.192853988356
3e-05 317.188557157373
3.1e-05 319.212209536696
3.1e-05 319.208341010247
3.2e-05 321.068727544617
3.2e-05 321.065216684107
3.3e-05 322.850871529043
3.3e-05 322.84751864412
3.4e-05 327.066443053622
3.4e-05 327.050293484923
3.5e-05 332.154007432153
3.5e-05 332.135949522714
3.6e-05 337.09517535474
3.6e-05 337.077843772393
3.7e-05 341.898274360723
3.7e-05 341.881593519507
3.8e-05 346.574568939211
3.8e-05 346.558487330444
3.9e-05 351.129315246686
3.9e-05 351.113798183457
4e-05 355.544193027308
4e-05 355.529421108334
4.1e-05 359.823780915095
4.1e-05 359.809560417248
4.2e-05 363.99637294994
4.2e-05 363.982622439423
4.3e-05 368.06610901335
4.3e-05 368.052802919835
4.4e-05 372.03695371177
4.4e-05 372.024068265198
4.5e-05 375.906678701542
4.5e-05 375.894237134984
4.6e-05 379.680701703217
4.6e-05 379.668645691168
4.7e-05 383.366628681855
4.7e-05 383.354930282448
4.8e-05 386.967646512317
4.8e-05 386.956287790454
4.9e-05 390.486788733633
4.9e-05 390.475752924486
5e-05 393.926937266411
5e-05 393.91620868283
5.1e-05 397.287432489318
5.1e-05 397.277079500406
5.2e-05 400.542123199592
5.2e-05 400.532169127552
5.3e-05 403.716937083887
5.3e-05 403.707261583529
5.4e-05 406.823473476299
5.4e-05 406.814047732783
5.5e-05 409.863970736106
5.5e-05 409.85478342576
5.6e-05 412.840579561453
5.6e-05 412.831620033652
5.7e-05 415.743999279142
5.7e-05 415.735312722694
5.8e-05 418.586968430721
5.8e-05 418.578490536145
5.9e-05 421.364001180583
5.9e-05 421.355770602497
6e-05 424.0811341073
6e-05 424.073099121179
6.1e-05 426.744038711592
6.1e-05 426.73618501316
6.2e-05 429.34882744044
6.2e-05 429.341185226339
6.3e-05 431.898347110245
6.3e-05 431.890877617422
6.4e-05 434.398478725118
6.4e-05 434.391168554754
6.5e-05 436.850654477617
6.5e-05 436.843497329425
6.6e-05 439.256254248942
6.6e-05 439.249244152919
6.7e-05 441.61660195116
6.7e-05 441.609733246417
6.8e-05 443.93296841928
6.8e-05 443.926235734475
6.9e-05 446.206574122323
6.9e-05 446.199972357534
7e-05 448.438591706628
7e-05 448.432116016566
7.1e-05 450.630148383537
7.1e-05 450.623794161976
7.2e-05 452.78232817262
7.2e-05 452.776091037941
7.3e-05 454.896174010648
7.3e-05 454.890049792412
7.4e-05 456.972689735727
7.4e-05 456.966674462191
7.5e-05 458.968284376055
7.5e-05 458.963447965222
7.6e-05 461.20632539666
7.6e-05 461.198028070534
7.7e-05 463.882958728403
7.7e-05 463.87397130473
7.8e-05 466.50357452195
7.8e-05 466.494801546766
7.9e-05 469.078953903032
7.9e-05 469.070333835152
8e-05 471.611800948518
8e-05 471.603312672801
8.1e-05 474.105972503088
8.1e-05 474.097607285217
8.2e-05 476.555240164142
8.2e-05 476.547033734559
8.3e-05 478.966302471748
8.3e-05 478.958213894528
8.4e-05 481.341112613055
8.4e-05 481.333136655164
8.5e-05 483.680348998514
8.5e-05 483.672496420131
8.6e-05 485.980795256051
8.6e-05 485.973055438438
8.7e-05 488.247184991013
8.7e-05 488.239548631889
8.8e-05 490.480189366266
8.8e-05 490.472653687657
8.9e-05 492.680466566269
8.9e-05 492.673028890473
9e-05 494.84865608151
9e-05 494.841313826651
9.1e-05 496.985379381287
9.1e-05 496.97813005711
9.2e-05 499.084578691715
9.2e-05 499.077456703149
9.3e-05 501.152376223044
9.3e-05 501.145343970122
9.4e-05 503.190542944536
9.4e-05 503.183596344884
9.5e-05 505.199632414225
9.5e-05 505.192769310214
9.6e-05 507.18018494633
9.6e-05 507.173403253916
9.7e-05 509.132726282245
9.7e-05 509.126023987883
9.8e-05 511.05776808776
9.8e-05 511.051143245472
9.9e-05 512.955808429656
9.9e-05 512.949259158239
0.0001 514.827332232683
0.0001 514.820856713055
0.000101 516.672811717862
0.000101 516.666408190541
0.000102 518.490049867552
0.000102 518.483733215072
0.000103 520.281150621556
0.000103 520.274903931392
0.000104 522.047588490161
0.000104 522.041408707553
0.000105 523.789787297377
0.000105 523.783672880214
0.000106 525.508161561814
0.000106 525.502111016624
0.000107 527.203115378532
0.000107 527.197127258586
0.000108 528.875042749544
0.000108 528.869115653051
0.000109 530.524327901566
0.000109 530.518460469952
0.00011 532.15134559159
0.00011 532.145536507851
0.000111 533.756461400816
0.000111 533.750709387948
0.000112 535.340032017477
0.000112 535.334335836978
0.000113 536.902405509047
0.000113 536.896763959477
0.000114 538.443921584284
0.000114 538.4383334999
0.000115 539.96491184558
0.000115 539.959376095022
0.000116 541.46570003202
0.000116 541.460215517058
0.000117 542.94660225355
0.000117 542.941167907882
0.000118 544.407927216652
0.000118 544.402542004758
0.000119 545.847848225294
0.000119 545.842537982869
0.00012 547.264452934486
0.00012 547.259195107227
0.000121 548.662437109448
0.000121 548.657225257031
0.000122 550.03907775429
0.000122 550.033936168616
0.000123 551.394955990386
0.000123 551.389861739723
0.000124 552.73309440414
0.000124 552.728043251099
0.000125 554.053751367988
0.000125 554.048742469772
0.000126 555.357182762652
0.000126 555.352215299536
0.000127 556.643638770345
0.000127 556.638711944893
0.000128 557.913364033086
0.000128 557.908477069403
0.000129 559.166597805636
0.000129 559.161749948642
0.00013 560.403574103272
0.00013 560.398764618013
0.000131 561.624521844611
0.000131 561.61975001559
0.000132 562.829664989665
0.000132 562.824930120205
0.000133 564.019222673326
0.000133 564.014524084955
0.000134 565.193409334454
0.000134 565.188746366317
0.000135 566.352434840738
0.000135 566.347806849023
0.000136 567.496504609488
0.000136 567.491910966888
0.000137 568.625819724526
0.000137 568.621259819709
0.000138 569.740577049311
0.000138 569.736050286417
0.000139 570.840969336449
0.000139 570.836475134601
0.00014 571.927185333712
0.00014 571.922723126553
0.000141 572.99940988672
0.000141 572.994979121958
0.000142 574.057824038377
0.000142 574.053424177355
0.000143 575.102605125212
0.000143 575.098235642488
0.000144 576.133926870718
0.000144 576.129587253663
0.000145 577.149601958934
0.000145 577.145304488425
0.000146 578.152127417956
0.000146 578.147858820883
0.000147 579.141721454391
0.000147 579.137481182187
0.000148 580.11854395301
0.000148 580.114331541495
0.000149 581.08275168589
0.000149 581.078566681872
0.00015 582.034498316859
0.00015 582.030340277818
0.000151 582.973934476701
0.000151 582.969802970472
0.000152 583.900757064529
0.000152 583.896662982451
0.000153 584.813592414664
0.000153 584.809526049923
0.000154 585.714582609014
0.000154 585.710541463036
0.000155 586.603864555817
0.000155 586.599848235875
0.000156 587.481574572953
0.000156 587.477582695256
0.000157 588.347846354154
0.000157 588.3438785436
0.000158 589.202811030061
0.000158 589.198866919987
0.000159 590.046597227511
0.000159 590.042676459464
0.00016 590.879331127148
0.00016 590.87543335065
0.000161 591.70113651938
0.000161 591.697261391718
0.000162 592.512134858768
0.000162 592.508282044776
0.000163 593.312445316878
0.000163 593.308614488732
0.000164 594.102184833652
0.000164 594.098375670671
0.000165 594.881468167349
0.000165 594.877680355807
0.000166 595.650407943111
0.000166 595.646641176046
0.000167 596.409114700173
0.000167 596.405368677209
0.000168 597.157696937791
0.000168 597.153971364968
0.000169 597.896261159912
0.000169 597.892555749513
0.00017 598.62491191863
0.00017 598.621226389017
0.000171 599.343751856458
0.000171 599.340085931921
0.000172 600.052881747484
0.000172 600.04923515808
0.000173 600.752400537399
0.000173 600.748773018809
0.000174 601.442405382474
0.000174 601.43879667586
0.000175 602.122991687501
0.000175 602.119401539364
0.000176 602.794253142727
0.000176 602.790681304773
0.000177 603.456281759819
0.000177 603.452727988831
0.000178 604.109167906891
0.000178 604.105631964601
0.000179 604.753000342617
0.000179 604.749481995583
0.00018 605.387866249455
0.00018 605.384365268944
0.000181 606.013851266027
0.000181 606.010367427896
0.000182 606.631039518647
0.000182 606.627572603236
0.000183 607.239513652065
0.000183 607.236063444085
0.000184 607.839354859411
0.000184 607.835921147841
0.000185 608.430642911389
0.000185 608.427225489373
0.000186 609.013456184733
0.000186 609.010054849482
0.000187 609.587871689948
0.000187 609.584486242642
0.000188 610.153346668429
0.000188 610.149984788996
0.000189 610.709983374669
0.000189 610.706637524258
0.00019 611.258410802184
0.00019 611.255087502923
0.000191 611.797631858033
0.000191 611.794324657814
0.000192 612.328856425485
0.000192 612.325564097247
0.000193 612.852152605327
0.000193 612.848874972876
0.000194 613.367588357763
0.000194 613.364325248264
0.000195 613.875230477678
0.000195 613.871981721576
0.000196 614.37514461678
0.000196 614.371910047732
0.000197 614.867395305207
0.000197 614.864174760007
0.000198 615.352045972623
0.000198 615.348839291133
0.000199 615.829158968802
0.000199 615.825965993886
0.0002 616.298795583731
0.0002 616.295616161187
0.000201 616.761016067228
0.000201 616.757850045728
0.000202 617.21587964811
0.000202 617.212726879132
0.000203 617.663444552893
0.000203 617.660304890667
0.000204 618.103768024075
0.000204 618.100641325521
0.000205 618.536906337977
0.000205 618.533792462647
0.000206 618.962914822179
0.000206 618.959813632206
0.000207 619.381847872551
0.000207 619.37875923259
0.000208 619.793758969891
0.000208 619.790682747068
0.000209 620.198700696181
0.000209 620.195636760046
0.00021 620.596724750479
0.00021 620.59367297295
0.000211 620.982025621773
0.000211 620.979390390911
0.000212 621.311986208253
0.000212 621.309439767739
0.000213 621.633275693997
0.000213 621.630755237797
0.000214 621.94766681036
0.000214 621.945163859768
0.000215 622.255409983553
0.000215 622.252931517561
0.000216 622.556786565688
0.000216 622.554317875515
0.000217 622.852434848617
0.000217 622.849982638925
0.000218 623.141968438566
0.000218 623.139525657646
0.000219 623.426082743947
0.000219 623.423648819084
0.00022 623.704814763769
0.00022 623.702389604232
0.000221 623.977541062886
0.000221 623.975139579477
0.000222 624.244007174997
0.000222 624.241614926647
0.000223 624.505232893177
0.000223 624.502849081618
0.000224 624.761252237466
0.000224 624.758876778595
0.000225 625.012099330305
0.000225 625.009732141473
0.000226 625.257807693814
0.000226 625.255448693797
0.000227 625.498410259527
0.000227 625.4960593685
0.000228 625.733418157066
0.000228 625.731082890028
0.000229 625.963195394785
0.000229 625.960868198762
0.00023 626.187978259018
0.00023 626.18565890896
0.000231 626.407797173677
0.000231 626.40548559532
0.000232 626.622682168307
0.000232 626.620378288661
0.000233 626.832662737105
0.000233 626.830366484429
0.000234 627.037767847243
0.000234 627.035479151022
0.000235 627.238025947007
0.000235 627.235744737936
0.000236 627.433464973769
0.000236 627.431191183726
0.000237 627.624112361777
0.000237 627.621845923805
0.000238 627.809995049784
0.000238 627.807735898072
0.000239 627.991139488515
0.000239 627.988887558379
0.00024 628.167571647969
0.00024 628.165326875831
0.000241 628.338374861182
0.000241 628.336152010771
0.000242 628.50440128534
0.000242 628.502185519347
0.000243 628.665820905826
0.000243 628.663612054172
0.000244 628.822657574773
0.000244 628.820455578469
0.000245 628.974934809838
0.000245 628.972739610904
0.000246 629.122675690534
0.000246 629.120487231988
0.000247 629.265902864489
0.000247 629.263721090332
0.000248 629.404638553583
0.000248 629.402463408778
0.000249 629.538904559937
0.000249 629.536735990403
0.00025 629.668722271804
0.00025 629.666560224393
0.000251 629.794112669309
0.000251 629.791957091799
0.000252 629.915096330095
0.000252 629.912947171171
0.000253 630.031693434837
0.000253 630.029550644078
0.000254 630.143923772644
0.000254 630.141787300516
0.000255 630.251806746359
0.000255 630.249676544194
0.000256 630.355361377738
0.000256 630.353237397728
0.000257 630.454606312534
0.000257 630.452488507715
0.000258 630.549559825467
0.000258 630.54744814971
0.000259 630.6402398251
0.000259 630.638134233099
0.00026 630.726663858614
0.00026 630.724564305872
0.000261 630.808849116485
0.000261 630.806755559307
0.000262 630.886812437065
0.000262 630.884724832545
0.000263 630.960570311073
0.000263 630.958488617085
0.000264 631.030138885997
0.000264 631.028063061183
0.000265 631.095533970399
0.000265 631.093463974163
0.000266 631.156771038146
0.000266 631.154706830641
0.000267 631.213865232545
0.000267 631.211806774666
0.000268 631.266831370402
0.000268 631.264778623774
0.000269 631.315683945996
0.000269 631.313636872967
0.00027 631.360437134975
0.00027 631.358395698611
0.000271 631.40110479818
0.000271 631.399068962248
0.000272 631.43770048538
0.000272 631.435670214349
0.000273 631.470237438947
0.000273 631.468212697975
0.000274 631.498728597449
0.000274 631.496709352377
0.000275 631.523186599179
0.000275 631.52117281652
0.000276 631.543487383606
0.000276 631.541494612374
0.000277 631.559789332544
0.000277 631.557801921598
0.000278 631.572125270322
0.000278 631.570143163792
0.000279 631.580506517456
0.000279 631.578529684639
0.00028 631.584944136544
0.00028 631.582972547371
0.000281 631.585448910842
0.000281 631.583482535874
0.000282 631.582031347314
0.000282 631.580070157735
0.000283 631.574701679614
0.000283 631.572745647224
0.000284 631.56346987101
0.000284 631.561518968221
0.000285 631.548345617253
0.000285 631.546399817082
0.000286 631.529338349392
0.000286 631.527397625459
0.000287 631.506305809467
0.000287 631.504355192848
0.000288 631.47935221563
0.000288 631.477406692151
0.000289 631.448512697092
0.000289 631.446572234422
0.00029 631.413795759453
0.00029 631.411860333469
0.000291 631.37520966354
0.000291 631.373279250702
0.000292 631.332762420314
0.000292 631.330836997659
0.000293 631.286461793344
0.000293 631.28454133848
0.000294 631.236315301236
0.000294 631.234399792338
0.000295 631.182330220014
0.000295 631.180419635823
0.000296 631.124513585458
0.000296 631.122607905275
0.000297 631.062872195398
0.000297 631.060971399081
0.000298 630.997412611962
0.000298 630.99551667992
0.000299 630.928141163779
0.000299 630.926250076973
0.0003 630.855063948151
0.0003 630.853177688087
0.000301 630.77818683317
0.000301 630.776305381896
0.000302 630.697515459805
0.000302 630.695638799911
0.000303 630.613055243946
0.000303 630.611183358558
0.000304 630.524811378414
0.000304 630.522944251189
0.000305 630.432788834918
0.000305 630.43092645005
0.000306 630.336992366004
0.000306 630.335134708211
0.000307 630.237426506936
0.000307 630.235573561465
0.000308 630.134095577566
0.000308 630.132247330186
0.000309 630.027003684155
0.000309 630.025160121159
0.00031 629.916154721169
0.00031 629.914315829369
0.000311 629.801552373036
0.000311 629.79971813976
0.000312 629.683200115874
0.000312 629.681370528966
0.000313 629.561101219186
0.000313 629.559276267005
0.000314 629.435258747524
0.000314 629.43343841894
0.000315 629.305675562123
0.000315 629.303859846519
0.000316 629.172354322509
0.000316 629.170543209774
0.000317 629.035297488066
0.000317 629.033490968599
0.000318 628.89450731959
0.000318 628.892705384296
0.000319 628.749985880808
0.000319 628.748188521097
0.00032 628.601735039862
0.00032 628.59994224765
0.000321 628.449756470787
0.000321 628.44796823849
0.000322 628.294051654935
0.000322 628.292267975475
0.000323 628.13426503023
0.000323 628.132474804552
0.000324 627.970072597721
0.000324 627.968287411023
0.000325 627.802135969476
0.000325 627.800355367297
0.000326 627.630455497353
0.000326 627.628679474671
0.000327 627.455031796149
0.000327 627.453260348446
0.000328 627.275865295976
0.000328 627.274098419239
0.000329 627.092956243554
0.000329 627.091193934268
0.00033 626.906304703474
0.00033 626.90454695863
0.000331 626.715910559453
0.000331 626.714157376541
0.000332 626.521773515555
0.000332 626.520024892572
0.000333 626.323893097402
0.000333 626.322149032844
0.000334 626.122268653363
0.000334 626.120529146228
0.000335 625.916899355716
0.000335 625.915164405508
0.000336 625.707784201805
0.000336 625.70605380853
0.000337 625.494705504153
0.000337 625.492974484658
0.000338 625.277425860968
0.000338 625.275699733824
0.000339 625.056386468995
0.000339 625.054664926982
0.00034 624.831585365172
0.00034 624.829868410366
0.000341 624.603020725829
0.000341 624.601308360814
0.000342 624.370690558959
0.000342 624.368982786829
0.000343 624.134592705261
0.000343 624.132889529619
0.000344 623.894724839163
0.000344 623.893026264124
0.000345 623.649771029888
0.000345 623.648067767777
0.000346 623.400863914832
0.000346 623.39916542099
0.000347 623.148159510708
0.000347 623.146465683491
0.000348 622.891654803574
0.000348 622.889965649275
0.000349 622.630908991014
0.000349 622.629219858758
0.00035 622.366015521606
0.00035 622.364331339113
0.000351 622.097302881767
0.000351 622.095623419485
0.000352 621.823236596
0.000352 621.821553273072
0.000353 621.544971661197
0.000353 621.543288606423
0.000354 621.261707243405
0.000354 621.26002046275
0.000355 620.971939385352
0.000355 620.970227940333
0.000356 620.634682882818
0.000356 620.632761128027
0.000357 620.288710480976
0.000357 620.286798827695
0.000358 619.938334751428
0.000358 619.936429641778
0.000359 619.583550468666
0.000359 619.581651918949
0.00036 619.224355847765
0.00036 619.222463874878
0.000361 618.860748972231
0.000361 618.858863593667
0.000362 618.492727795422
0.000362 618.490849029269
0.000363 618.120290141973
0.000363 618.118418006919
0.000364 617.743433709214
0.000364 617.741568224545
0.000365 617.362156068581
0.000365 617.360297254184
0.000366 616.976454667028
0.000366 616.974602543391
0.000367 616.586326828432
0.000367 616.584481416651
0.000368 616.191769755004
0.000368 616.189931076778
0.000369 615.792780528684
0.000369 615.79094860632
0.00037 615.389356112549
0.00037 615.387530968963
0.000371 614.981493352206
0.000371 614.979675010924
0.000372 614.569188977197
0.000372 614.56737746236
0.000373 614.152439602395
0.000373 614.150634938756
0.000374 613.731241729399
0.000374 613.72944394233
0.000375 613.30559174794
0.000375 613.303800863431
0.000376 612.875485937273
0.000376 612.873701981934
0.000377 612.440920467581
0.000377 612.439143468646
0.000378 612.001891401378
0.000378 612.000121386706
0.000379 611.55839469491
0.000379 611.556631692985
0.00038 611.110157211473
0.00038 611.108397680321
0.000381 610.656730613041
0.000381 610.654978826205
0.000382 610.19881794184
0.000382 610.197073285156
0.000383 609.736127317996
0.000383 609.734386317686
0.000384 609.268216528477
0.000384 609.266483410591
0.000385 608.795800000532
0.000385 608.794074136381
0.000386 608.318872444003
0.000386 608.317153866636
0.000387 607.837429135881
0.000387 607.835717878993
0.000388 607.351465260668
0.000388 607.3497613586
0.000389 606.860975911834
0.000389 606.859279399581
0.00039 606.365956093281
0.00039 606.364267006489
0.000391 605.866400720817
0.000391 605.864719095789
0.000392 605.362304623631
0.000392 605.360630497327
0.000393 604.853662545779
0.000393 604.851995955823
0.000394 604.340469147681
0.000394 604.338810132357
0.000395 603.822719007617
0.000395 603.821067605879
0.000396 603.30040662324
0.000396 603.298762874711
0.000397 602.773526413096
0.000397 602.77189035807
0.000398 602.242072718148
0.000398 602.240444397594
0.000399 601.706039803315
0.000399 601.704419258882
0.0004 601.165421859024
0.0004 601.163809133039
0.000401 600.62021300276
0.000401 600.618608138236
0.000402 600.070407280641
0.000402 600.068810321278
0.000403 599.515998668997
0.000403 599.514409659183
0.000404 598.956981075959
0.000404 598.955400060774
0.000405 598.393348343061
0.000405 598.391775368284
0.000406 597.825094246862
0.000406 597.823529358968
0.000407 597.252212500569
0.000407 597.250655746735
0.000408 596.674696755679
0.000408 596.673148183787
0.000409 596.092540603634
0.000409 596.091000262274
0.00041 595.50573757749
0.00041 595.504205515962
0.000411 594.914281153597
0.000411 594.912757421918
0.000412 594.318164753302
0.000412 594.316649402202
0.000413 593.717381744653
0.000413 593.715874825584
0.000414 593.111925444134
0.000414 593.110427009272
0.000415 592.501789118401
0.000415 592.500299220647
0.000416 591.886965986046
0.000416 591.885484679031
0.000417 591.26744921937
0.000417 591.265976557459
0.000418 590.643231946177
0.000418 590.641767984469
0.000419 590.014307251579
0.000419 590.012852045914
0.00042 589.380668179829
0.00042 589.379221786786
0.000421 588.742307736159
0.000421 588.740870213066
0.000422 588.099218888649
0.000422 588.09779029358
0.000423 587.451394570106
0.000423 587.449974961887
0.000424 586.798827679962
0.000424 586.797417118173
0.000425 586.141511086197
0.000425 586.140109631177
0.000426 585.47943762728
0.000426 585.478045340127
0.000427 584.812600114122
0.000427 584.8112170567
0.000428 584.140991332066
0.000428 584.139617567005
0.000429 583.463900604536
0.000429 583.462533323627
0.00043 582.780930674284
0.00043 582.779573993091
0.000431 582.093164316674
0.000431 582.091817162032
0.000432 581.400593208029
0.000432 581.399255645704
0.000433 580.70321014857
0.000433 580.701882245105
0.000434 580.00100792638
0.000434 579.999689749104
0.000435 579.29397931955
0.000435 579.292670936573
0.000436 578.582117098336
0.000436 578.58081857856
0.000437 577.865414027357
0.000437 577.864125440469
0.000438 577.143862867795
0.000438 577.14258428428
0.000439 576.417456379643
0.000439 576.416187870778
0.00044 575.685068022815
0.00044 575.683807894702
0.000441 574.94719077192
0.000441 574.945941570513
0.000442 574.204434089791
0.000442 574.203195220829
0.000443 573.456790199609
0.000443 573.455561737585
0.000444 572.704252003951
0.000444 572.703034024167
0.000445 571.946812419231
0.000445 571.945604997791
0.000446 571.184464378117
0.000446 571.183267591939
0.000447 570.417200831984
0.000447 570.416014758796
0.000448 569.645014753388
0.000448 569.643839471732
0.000449 568.867899138563
0.000449 568.866734727795
0.00045 568.085847009954
0.00045 568.08469355025
0.000451 567.298851418777
0.000451 567.297708991132
0.000452 566.506905447597
0.000452 566.505774133828
0.000453 565.710002212951
0.000453 565.708882095696
0.000454 564.908134867984
0.000454 564.907026030709
0.000455 564.101296605123
0.000455 564.100199132117
0.000456 563.289480658779
0.000456 563.28839463516
0.000457 562.472680308075
0.000457 562.471605819791
0.000458 561.650888879609
0.000458 561.649826013439
0.000459 560.824099750244
0.000459 560.823048593797
0.00046 559.992306349927
0.00046 559.991266991646
0.000461 559.155502164542
0.000461 559.154474693703
0.000462 558.313680738793
0.000462 558.312665245506
0.000463 557.466835679112
0.000463 557.465832254325
0.000464 556.614960656612
0.000464 556.613969392105
0.000465 555.758049410052
0.000465 555.757070398445
0.000466 554.896095748855
0.000466 554.895129083603
0.000467 554.029093556139
0.000467 554.028139331535
0.000468 553.157036791795
0.000468 553.156095102967
0.000469 552.279919495582
0.000469 552.278990438497
0.00047 551.397735790272
0.00047 551.396819461735
0.000471 550.510479884813
0.000471 550.509576382462
0.000472 549.618146077529
0.000472 549.617255499842
0.000473 548.71821205692
0.000473 548.717333921786
0.000474 547.811955413198
0.000474 547.8110919915
0.000475 546.900606889413
0.000475 546.899756769862
0.000476 545.984159943822
0.000476 545.983323228008
0.000477 545.061165481118
0.000477 545.060340334327
0.000478 544.130310124399
0.000478 544.129502112252
0.000479 543.194347419837
0.000479 543.193553206088
0.00048 542.253269671134
0.00048 542.252489360351
0.000481 541.307072746547
0.000481 541.306306444095
0.000482 540.355752642402
0.000482 540.355000454438
0.000483 539.399305486643
0.000483 539.398567520116
0.000484 538.43772754242
0.000484 538.437003905066
0.000485 537.471015211709
0.000485 537.470306012048
0.000486 536.499165038959
0.000486 536.498470386293
0.000487 535.522173714773
0.000487 535.521493719179
0.000488 534.540038079626
0.000488 534.539372851955
0.000489 533.552755127605
0.000489 533.552104779476
0.00049 532.560322010185
0.00049 532.559686653982
0.000491 531.562736040044
0.000491 531.56211578891
0.000492 530.559994694892
0.000492 530.559389662723
0.000493 529.552095621347
0.000493 529.551505922791
0.000494 528.539036638838
0.000494 528.538462389282
0.000495 527.520815743529
0.000495 527.520257059102
0.000496 526.497431112286
0.000496 526.496888109848
0.000497 525.468881106669
0.000497 525.468353903803
0.000498 524.435164276952
0.000498 524.434652991962
0.000499 523.396279366176
0.000499 523.395784118077
0.0005 522.352225314229
0.0005 522.351746222741
0.000501 521.303001261957
0.000501 521.302538447497
0.000502 520.248606555299
0.000502 520.248160138973
0.000503 519.18904074946
0.000503 519.188610853055
0.000504 518.124303613099
0.000504 518.123890359076
0.000505 517.054129575104
0.000505 517.053732652605
0.000506 515.976873264169
0.000506 515.976495948998
0.000507 514.894453814272
0.000507 514.894093551964
0.000508 513.806869398751
0.000508 513.806526314032
0.000509 512.714121105447
0.000509 512.713795323656
0.00051 511.616210262466
0.00051 511.615901909547
0.000511 510.51313844255
0.000511 510.51284764504
0.000512 509.404907467452
0.000512 509.404634352471
0.000513 508.291519412349
0.000513 508.291264107586
0.000514 507.172976610263
0.000514 507.172739243966
0.000515 506.049281656521
0.000515 506.049062357483
0.000516 504.920437413211
0.000516 504.920236310761
0.000517 503.786447013687
0.000517 503.78626423767
0.000518 502.647313867067
0.000518 502.647149547834
0.000519 501.503041662764
0.000519 501.502895931162
0.00052 500.353634375033
0.00052 500.353507362383
0.000521 499.199096267531
0.000521 499.198988105618
0.000522 498.039431897892
0.000522 498.039342718948
0.000523 496.87302373908
0.000523 496.87295586714
0.000524 495.698773514407
0.000524 495.698729113207
0.000525 494.519431811753
0.000525 494.519406861698
0.000526 493.335000657454
0.000526 493.334995291411
0.000527 492.145486548588
0.000527 492.145500899765
0.000528 490.950896305474
0.000528 490.950930507399
0.000529 489.751237076274
0.000529 489.751291262777
0.00053 488.546516341603
0.00053 488.546590646799
0.000531 487.336741919153
0.000531 487.33683647742
0.000532 486.121921968306
0.000532 486.122036914266
0.000533 484.90206499476
0.000533 484.902200463259
0.000534 483.677179855156
0.000534 483.677335981242
0.000535 482.445178028786
0.000535 482.445358602783
0.000536 481.208124953372
0.000536 481.208326573396
0.000537 479.966081939813
0.000537 479.9663046534
0.000538 478.719059410925
0.000538 478.719303353099
0.000539 477.467068251412
0.000539 477.46733355727
0.00054 476.210119725046
0.00054 476.210406529741
0.000541 474.948225479228
0.000541 474.94853391794
0.000542 473.678778833826
0.000542 473.679114466377
0.000543 472.402808730144
0.000543 472.403169012265
0.000544 471.121953759108
0.000544 471.12233613384
0.000545 469.836224863599
0.000545 469.836629464607
0.000546 468.545636083932
0.000546 468.546063044753
0.000547 467.248379714785
0.000547 467.248832970367
0.000548 465.946041790712
0.000548 465.946518146838
0.000549 464.63890019677
0.000549 464.639399335982
0.00055 463.319753033511
0.00055 463.320290793935
0.000551 461.99487336961
0.000551 461.99543602978
0.000552 460.663513498208
0.000552 460.664103401255
0.000553 459.477660588452
0.000553 459.477871087026
0.000554 458.643942313297
0.000554 458.643754648568
0.000555 457.647662019995
0.000555 457.64769500759
0.000556 456.645928141932
0.000556 456.645978737245
0.000557 455.638521813149
0.000557 455.638590216503
0.000558 454.625427593958
0.000558 454.625514007759
0.000559 453.606630235238
0.000559 453.60673486398
0.00056 452.582114685703
0.00056 452.58223773598
0.000561 451.551866099312
0.000561 451.552007779816
0.000562 450.515869842799
0.000562 450.516030364332
0.000563 449.47411150334
0.000563 449.474291078815
0.000564 448.426576896356
0.000564 448.426775740801
0.000565 447.373252073441
0.000565 447.373470404004
0.000566 446.314123330436
0.000566 446.314361366386
0.000567 445.249177215629
0.000567 445.24943517836
0.000568 444.178400538101
0.000568 444.178678651132
0.000569 443.101780376203
0.000569 443.102078865179
0.00057 442.019304086175
0.00057 442.019623178864
0.000571 440.930959310899
0.000571 440.931299237197
0.000572 439.836733988803
0.000572 439.837094980726
0.000573 438.736616362891
0.000573 438.736998654578
0.000574 437.630594989926
0.000574 437.630998817629
0.000575 436.518658749745
0.000575 436.519084351832
0.000576 435.400796854728
0.000576 435.401244471671
0.000577 434.27699885939
0.000577 434.277468733763
0.000578 433.147254670138
0.000578 433.147747046607
0.000579 432.011554555152
0.000579 432.01206968047
0.00058 430.86988915442
0.00058 430.870427277413
0.000581 429.722249489906
0.000581 429.722810861466
0.000582 428.568626975871
0.000582 428.569211848942
0.000583 427.407148021159
0.000583 427.407761848767
0.000584 426.23713227624
0.000584 426.237774175919
0.000585 425.061124691925
0.000585 425.061790962481
0.000586 423.879114948311
0.000586 423.879805849865
0.000587 422.691097137461
0.000587 422.691812932056
0.000588 421.497065815156
0.000588 421.497806766742
0.000589 420.297016012166
0.000589 420.297782386575
0.00059 419.089144402225
0.00059 419.089942808291
0.000591 417.871837037569
0.000591 417.872667098142
0.000592 416.648522862707
0.000592 416.649379261476
0.000593 415.419194028181
0.000593 415.420077036716
0.000594 414.183848810569
0.000594 414.184758702123
0.000595 412.942191870832
0.000595 412.9431332849
0.000596 411.689677855659
0.000596 411.690655006067
0.000597 410.431178525785
0.000597 410.432183478865
0.000598 409.1666869083
0.000598 409.167719941704
0.000599 407.896204845571
0.000599 407.897266238383
0.0006 406.6197347978
};
\addplot [semithick, color1]
table {%
0 172.786508022204
5e-07 208.523412695146
5e-07 213.347192401634
1e-06 147.681934485615
1e-06 147.153550288415
1.5e-06 137.503443268746
1.5e-06 137.549970802624
2e-06 143.445706861644
2e-06 143.422918116593
2.5e-06 148.429498973579
2.5e-06 148.408608738736
3e-06 152.8918712417
3e-06 152.892406659263
3.5e-06 157.016938919031
3.5e-06 157.001393026441
4e-06 160.845649783224
4e-06 160.845895308025
4.5e-06 164.441534874861
4.5e-06 164.428798898374
5e-06 167.828497547092
5e-06 167.828642665964
5.5e-06 171.018099077616
5.5e-06 171.007585094016
6e-06 173.996195490353
6e-06 173.996337702815
6.5e-06 176.883848313928
6.5e-06 176.874898355908
7e-06 179.51925833406
7e-06 179.519416887325
7.5e-06 182.061567864695
7.5e-06 182.055848886688
8e-06 187.770240604918
8e-06 187.764770950055
8.5e-06 193.736376131586
8.5e-06 193.703100007066
9e-06 199.382560660546
9e-06 199.382865677148
9.5e-06 204.786402135225
9.5e-06 204.758131590448
1e-05 209.892990002364
1e-05 209.893254640727
1.05e-05 214.747134252943
1.05e-05 214.723649388404
1.1e-05 219.41644659322
1.1e-05 219.416571800693
1.15e-05 223.872860933374
1.15e-05 223.85286690702
1.2e-05 228.081197343537
1.2e-05 228.081404756872
1.25e-05 232.188455250753
1.25e-05 232.170993990411
1.3e-05 236.120579472456
1.3e-05 236.1207039913
1.35e-05 239.947115610107
1.35e-05 239.931371291177
1.4e-05 243.595408231919
1.4e-05 243.595533413907
1.45e-05 247.091912245349
1.45e-05 247.078407075676
1.5e-05 250.49424889062
1.5e-05 250.494294417089
1.55e-05 253.801742346346
1.55e-05 253.789366708496
1.6e-05 257.015261133983
1.6e-05 257.015308226867
1.65e-05 260.138589843756
1.65e-05 260.127305724931
1.7e-05 263.131651766139
1.7e-05 263.131731331281
1.75e-05 266.018431476831
1.75e-05 266.008518265625
1.8e-05 268.841845736491
1.8e-05 268.841870695673
1.85e-05 271.546157338568
1.85e-05 271.537396647113
1.9e-05 274.146168290342
1.9e-05 274.146227685065
1.95e-05 276.689253500154
1.95e-05 276.681244958304
2e-05 279.156267779939
2e-05 279.156305591756
2.05e-05 281.588967265868
2.05e-05 281.58141297698
2.1e-05 283.963847890333
2.1e-05 283.963872461338
2.15e-05 286.250974468715
2.15e-05 286.244172650632
2.2e-05 288.476155926025
2.2e-05 288.476184217407
2.25e-05 290.671267429162
2.25e-05 290.664823154983
2.3e-05 292.790772939241
2.3e-05 292.790809853095
2.35e-05 294.872755930276
2.35e-05 294.866813836456
2.4e-05 296.915455988896
2.4e-05 296.915470065513
2.45e-05 298.931355601238
2.45e-05 298.925681351243
2.5e-05 300.901837108226
2.5e-05 300.901855133675
2.55e-05 302.796196300625
2.55e-05 302.791068712742
2.6e-05 304.645285570378
2.6e-05 304.645303941092
2.65e-05 306.471022680569
2.65e-05 306.466131564841
2.7e-05 308.263968366107
2.7e-05 308.263979715656
2.75e-05 310.034403522263
2.75e-05 310.029716846017
2.8e-05 311.737611200508
2.8e-05 311.737641173188
2.85e-05 313.367055376763
2.85e-05 313.362967358988
2.9e-05 314.947607921851
2.9e-05 314.947623566691
2.95e-05 316.074863476557
2.95e-05 316.072603254669
3e-05 317.139532029604
3e-05 317.139552705573
3.05e-05 318.171470314314
3.05e-05 318.169435140691
3.1e-05 319.162738691876
3.1e-05 319.162751362033
3.15e-05 320.106717018321
3.15e-05 320.104931016507
3.2e-05 321.022421644476
3.2e-05 321.022430200715
3.25e-05 321.921539299019
3.25e-05 321.919839978526
3.3e-05 322.806173004891
3.3e-05 322.806177419848
3.35e-05 324.335020737948
3.35e-05 324.328358063142
3.4e-05 326.931808871757
3.4e-05 326.930881185876
3.45e-05 329.497952155903
3.45e-05 329.488783556428
3.5e-05 332.019827213049
3.5e-05 332.019838711555
3.55e-05 334.515728770431
3.55e-05 334.506913015694
3.6e-05 336.96505696808
3.6e-05 336.965070494061
3.65e-05 339.388886196481
3.65e-05 339.380425564575
3.7e-05 341.771796055349
3.7e-05 341.771805938506
3.75e-05 344.131298120272
3.75e-05 344.123143835139
3.8e-05 346.451551752651
3.8e-05 346.451561104436
3.85e-05 348.749418500849
3.85e-05 348.741552721511
3.9e-05 351.009607326202
3.9e-05 351.009616184717
3.95e-05 353.245279655287
3.95e-05 353.237727183591
4e-05 355.428913538886
4e-05 355.428933814119
4.05e-05 357.587537419667
4.05e-05 357.580331760677
4.1e-05 359.711585051681
4.1e-05 359.711593172889
4.15e-05 361.81594841741
4.15e-05 361.808982745778
4.2e-05 363.887068624322
4.2e-05 363.887076337284
4.25e-05 365.939320143244
4.25e-05 365.932581326567
4.3e-05 367.959577138137
4.3e-05 367.959584470322
4.35e-05 369.961736353843
4.35e-05 369.955212189387
4.4e-05 371.933082043252
4.4e-05 371.933089019825
4.45e-05 373.886433998295
4.45e-05 373.880131054889
4.5e-05 375.805632770159
4.5e-05 375.805642479423
4.55e-05 377.708167464065
4.55e-05 377.702065981488
4.6e-05 379.582118188699
4.6e-05 379.582124543284
4.65e-05 381.440047971587
4.65e-05 381.434128818494
4.7e-05 383.270401259563
4.7e-05 383.270407320936
4.75e-05 385.085355141203
4.75e-05 385.079609123946
4.8e-05 386.873685736467
4.8e-05 386.873691522738
4.85e-05 388.647206807278
4.85e-05 388.641625335173
4.9e-05 390.395009911386
4.9e-05 390.395015439257
4.95e-05 392.12856375576
4.95e-05 392.123138793139
5e-05 393.837260137012
5e-05 393.837265421907
5.05e-05 395.532239784466
5.05e-05 395.526963807182
5.1e-05 397.200426116389
5.1e-05 397.200432818724
5.15e-05 398.845454158099
5.15e-05 398.840412983838
5.2e-05 400.457639156437
5.2e-05 400.457649890199
5.25e-05 402.057136244913
5.25e-05 402.052246093763
5.3e-05 403.634344691086
5.3e-05 403.634349399048
5.35e-05 405.199303341804
5.35e-05 405.194540321443
5.4e-05 406.742684966879
5.4e-05 406.742689479139
5.45e-05 408.274256548672
5.45e-05 408.269614866433
5.5e-05 409.784924118393
5.5e-05 409.784928445798
5.55e-05 411.284200895084
5.55e-05 411.279675104219
5.6e-05 412.763215836908
5.6e-05 412.76321998954
5.65e-05 414.226216846762
5.65e-05 414.221828461892
5.7e-05 415.668594436108
5.7e-05 415.668599006696
5.75e-05 417.100367323536
5.75e-05 417.096086147927
5.8e-05 418.513134821523
5.8e-05 418.513138669428
5.85e-05 419.913902485906
5.85e-05 419.909743110711
5.9e-05 421.292005055418
5.9e-05 421.292011034058
5.95e-05 422.660232847294
5.95e-05 422.656176547976
6e-05 424.0106120234
6e-05 424.010615592892
6.05e-05 425.351441177028
6.05e-05 425.347477024524
6.1e-05 426.674935527076
6.1e-05 426.674938961509
6.15e-05 427.98886273124
6.15e-05 427.985002112033
6.2e-05 429.281357477755
6.2e-05 429.281363138375
6.25e-05 430.564947659724
6.25e-05 430.561178510828
6.3e-05 431.832214524072
6.3e-05 431.832217715747
6.35e-05 433.090860167262
6.35e-05 433.087171956802
6.4e-05 434.333633893851
6.4e-05 434.33363696904
6.45e-05 435.568066896982
6.45e-05 435.564456410011
6.5e-05 436.787058617697
6.5e-05 436.787061581952
6.55e-05 437.997978472008
6.55e-05 437.994442662821
6.6e-05 439.193870227114
6.6e-05 439.19387308564
6.65e-05 440.381947940797
6.65e-05 440.378483922047
6.7e-05 441.555394194973
6.7e-05 441.55539695266
6.75e-05 442.721273770783
6.75e-05 442.717878803409
6.8e-05 443.872902833476
6.8e-05 443.872905494913
6.85e-05 445.017202653476
6.85e-05 445.01387413739
6.9e-05 446.147618010633
6.9e-05 446.147620580141
6.95e-05 447.27093213494
6.95e-05 447.267667600391
7e-05 448.38071369859
7e-05 448.38071618023
7.05e-05 449.483613084334
7.05e-05 449.480410183912
7.1e-05 450.573318365953
7.1e-05 450.573320763551
7.15e-05 451.656352012208
7.15e-05 451.653208513416
7.2e-05 452.726517225296
7.2e-05 452.726519542459
7.25e-05 453.790213249241
7.25e-05 453.787127027598
7.3e-05 454.841354346124
7.3e-05 454.841356586253
7.35e-05 455.886220995259
7.35e-05 455.883190027892
7.4e-05 456.918834642694
7.4e-05 456.918836808998
7.45e-05 457.945361247564
7.45e-05 457.942383607247
7.5e-05 458.924554162804
7.5e-05 458.924568508515
7.55e-05 459.780470483152
7.55e-05 459.776755152471
7.6e-05 461.136108256332
7.6e-05 461.135812889434
7.65e-05 462.482343583859
7.65e-05 462.477822000021
7.7e-05 463.812817851232
7.7e-05 463.812819138799
7.75e-05 465.13328960183
7.75e-05 465.128873179228
7.8e-05 466.434969141294
7.8e-05 466.434972179021
7.85e-05 467.730927983469
7.85e-05 467.726584686661
7.9e-05 469.011522674093
7.9e-05 469.011523912543
7.95e-05 470.285490066618
7.95e-05 470.281221300169
8e-05 471.545407382883
8e-05 471.545408033229
8.05e-05 472.799888577569
8.05e-05 472.795682010724
8.1e-05 474.040588367546
8.1e-05 474.040588984953
8.15e-05 475.273279994501
8.15e-05 475.269152183143
8.2e-05 476.491069201303
8.2e-05 476.49107057649
8.25e-05 477.703700045244
8.25e-05 477.699633100456
8.3e-05 478.903098683459
8.3e-05 478.903099262841
8.35e-05 480.09746891968
8.35e-05 480.093458885432
8.4e-05 481.278852697191
8.4e-05 481.278853245919
8.45e-05 482.45533707571
8.45e-05 482.451382375869
8.5e-05 483.619072867528
8.5e-05 483.619073386454
8.55e-05 484.776048085015
8.55e-05 484.772157200958
8.6e-05 485.920476765639
8.6e-05 485.920477292264
8.65e-05 487.060251358306
8.65e-05 487.056412818723
8.7e-05 488.187751463595
8.7e-05 488.187751936992
8.75e-05 489.310715268736
8.75e-05 489.30692759079
8.8e-05 490.421622617115
8.8e-05 490.421623062874
8.85e-05 491.528108082693
8.85e-05 491.524369909792
8.9e-05 492.622748973689
8.9e-05 492.622749392541
8.95e-05 493.713079055241
8.95e-05 493.709389079473
9e-05 494.791770541772
9e-05 494.791770934416
9.05e-05 495.866259068945
9.05e-05 495.862616028935
9.1e-05 496.929309289315
9.1e-05 496.929309656426
9.15e-05 497.985727878935
9.15e-05 497.982147793201
9.2e-05 499.029522750131
9.2e-05 499.029523814606
9.25e-05 500.069322516409
9.25e-05 500.065788938617
9.3e-05 501.098104485509
9.3e-05 501.098104824585
9.35e-05 502.122987516808
9.35e-05 502.119497191032
9.4e-05 503.137037653551
9.4e-05 503.137037968668
9.45e-05 504.147285037531
9.45e-05 504.143836871247
9.5e-05 505.146878791235
9.5e-05 505.14687908298
9.55e-05 506.142763276096
9.55e-05 506.139356214139
9.6e-05 507.128168627558
9.6e-05 507.128168896499
9.65e-05 508.109955586232
9.65e-05 508.106588609219
9.7e-05 509.081433303945
9.7e-05 509.081433550627
9.75e-05 510.049380986415
9.75e-05 510.046053109258
9.8e-05 511.007184872104
9.8e-05 511.00718509705
9.85e-05 511.961544647236
9.85e-05 511.958254917718
9.9e-05 512.905921771226
9.9e-05 512.905921974942
9.95e-05 513.846938358615
9.95e-05 513.843685856042
0.0001 514.778129285537
0.0001 514.778129468512
0.0001005 515.706040977857
0.0001005 515.702824811775
0.000101 516.624279983151
0.000101 516.624280145856
0.0001015 517.538600561228
0.0001015 517.535427325269
0.000102 518.442276668197
0.000102 518.442277343004
0.0001025 519.342831640473
0.0001025 519.339694557791
0.000103 520.234025638536
0.000103 520.234025771206
0.0001035 521.122171361755
0.0001035 521.119068043299
0.000104 522.001098217686
0.000104 522.001098331298
0.0001045 522.877049768677
0.0001045 522.873979434239
0.000105 523.743920516914
0.000105 523.74392061188
0.0001055 524.607887004869
0.0001055 524.604848898909
0.000106 525.462907347364
0.000106 525.46290742408
0.0001065 526.315092610825
0.0001065 526.312086001506
0.000107 527.158463086985
0.000107 527.158463145836
0.0001075 527.999065861388
0.0001075 527.996090039658
0.000108 528.830982011432
0.000108 528.830982052787
0.0001085 529.660196090162
0.0001085 529.65725036888
0.000109 530.480848612177
0.000109 530.480848636397
0.0001095 531.29886300144
0.0001095 531.295946714539
0.00011 532.108437902437
0.00011 532.108437909868
0.0001105 532.915436970215
0.0001105 532.912549471905
0.000111 533.714115711442
0.000111 533.714115702418
0.0001115 534.510279330805
0.0001115 534.507419994811
0.000112 535.298238967574
0.000112 535.29823894242
0.0001125 536.083742654603
0.0001125 536.080910873433
0.000113 536.861155970883
0.000113 536.861155929912
0.0001135 537.636171017429
0.0001135 537.633366201682
0.000114 538.403206655423
0.000114 538.403206598939
0.0001145 539.16790025694
0.0001145 539.165121834637
0.000115 539.924722841878
0.000115 539.924722770176
0.0001155 540.679258220542
0.0001155 540.676505636491
0.000116 541.426028480884
0.000116 541.426028394247
0.0001165 542.170565004196
0.0001165 542.167837719383
0.000117 542.907439887455
0.000117 542.907439786159
0.0001175 543.642133182534
0.0001175 543.639430673537
0.000118 544.369265966897
0.000118 544.36926585121
0.0001185 545.094268030635
0.0001185 545.09158978907
0.000119 545.809853311615
0.000119 545.809853989655
0.0001195 546.522067111671
0.0001195 546.519428720697
0.00012 547.226948036775
0.00012 547.226947908381
0.0001205 547.929797152
0.0001205 547.92718194115
0.000121 548.625408447273
0.000121 548.625408305133
0.0001215 549.319035181569
0.0001215 549.316442686001
0.000122 550.002670677437
0.000122 550.002671672469
0.0001225 550.684355895346
0.0001225 550.681799876556
0.000123 551.359013595095
0.000123 551.359013439565
0.0001235 552.031778603135
0.0001235 552.02924431124
0.000124 552.697605212674
0.000124 552.697605044009
0.0001245 553.361583057151
0.0001245 553.359070066307
0.000125 554.018708571456
0.000125 554.018708389876
0.0001255 554.674028201222
0.0001255 554.671536097267
0.000126 555.32257970845
0.000126 555.32257951417
0.0001265 555.969367195057
0.0001265 555.966895575114
0.000127 556.609468957672
0.000127 556.609468750901
0.0001275 557.247847575482
0.0001275 557.245396047578
0.000128 557.879621108613
0.000128 557.879620889552
0.0001285 558.509711412184
0.0001285 558.507279594882
0.000129 559.133275559323
0.000129 559.133275328169
0.0001295 559.755195458178
0.0001295 559.752782980231
0.00013 560.370666464337
0.00013 560.370666221282
0.0001305 560.984531295224
0.0001305 560.982137795232
0.000131 561.592022877634
0.000131 561.592022622862
0.0001315 562.197945474362
0.0001315 562.19557060045
0.000132 562.797568890823
0.000132 562.797568624514
0.0001325 563.39565965179
0.0001325 563.393303061297
0.000133 563.987523766762
0.000133 563.987523489091
0.0001335 564.577890720248
0.0001335 564.575552079426
0.000134 565.162102068763
0.000134 565.162101779901
0.0001345 565.744850936083
0.0001345 565.74252991981
0.000135 566.321513785574
0.000135 566.321513485686
0.0001355 566.896748042167
0.0001355 566.894444333672
0.000136 567.465964452286
0.000136 567.465964141532
0.0001365 568.033785386827
0.0001365 568.031498677421
0.000137 568.59565526733
0.000137 568.595654945866
0.0001375 569.156162038938
0.0001375 569.153892027758
0.000138 569.710783205704
0.000138 569.710782873684
0.0001385 570.264072899312
0.0001385 570.261819293078
0.000139 570.811541128592
0.000139 570.811540786161
0.0001395 571.357708808552
0.0001395 571.355471321326
0.00014 571.898117889473
0.00014 571.898117536777
0.0001405 572.43725665147
0.0001405 572.435035004431
0.000141 572.970698436897
0.000141 572.970698074075
0.0001415 573.502899458218
0.0001415 573.500693379438
0.000142 574.029463914013
0.000142 574.029463541201
0.0001425 574.554816502248
0.0001425 574.552625726484
0.000143 575.074591754995
0.000143 575.074591372325
0.0001435 575.593183395216
0.0001435 575.591007663707
0.000144 576.106255778461
0.000144 576.106255386064
0.0001445 576.617081944733
0.0001445 576.614927307081
0.000145 577.122336045073
0.000145 577.122335692997
0.0001455 577.626461873777
0.0001455 577.624321875736
0.000146 578.125194489823
0.000146 578.125194084301
0.0001465 578.622825073985
0.0001465 578.620699342731
0.000147 579.115117195327
0.000147 579.115116780426
0.0001475 579.606333121403
0.0001475 579.604221422667
0.000148 580.092264249127
0.000148 580.092263824964
0.0001485 580.577144453044
0.0001485 580.575046558117
0.000149 581.056792506913
0.000149 581.056792073601
0.0001495 581.535414357187
0.0001495 581.53333004275
0.00015 582.008855714051
0.00015 582.008855271701
0.0001505 582.481295049777
0.0001505 582.479224097748
0.000151 582.948604580841
0.000151 582.948604129561
0.0001515 583.414935748555
0.0001515 583.412877945936
0.000152 583.87580217062
0.000152 583.875801845348
0.0001525 584.334881916882
0.0001525 584.33284372952
0.000153 584.788944181924
0.000153 584.78894371933
0.0001535 585.242074439372
0.0001535 585.24004895158
0.000154 585.690235175732
0.000154 585.690234704501
0.0001545 586.13748605068
0.0001545 586.135473064306
0.000155 586.579814357492
0.000155 586.579813877723
0.0001555 587.021254473055
0.0001555 587.019253794467
0.000156 587.457818115329
0.000156 587.457817627116
0.0001565 587.893514771765
0.0001565 587.891526211718
0.000157 588.324380211544
0.000157 588.32437971498
0.0001575 588.754399417085
0.0001575 588.752422790602
0.000158 589.179631843712
0.000158 589.179631338889
0.0001585 589.604038344515
0.0001585 589.602073470764
0.000159 590.02370170403
0.000159 590.023701191037
0.0001595 590.442559013263
0.0001595 590.440605715447
0.00016 590.856716036962
0.00016 590.856715515885
0.0001605 591.270086463069
0.0001605 591.268144568309
0.000161 591.67879869525
0.000161 591.678798166173
0.0001615 592.08674336941
0.0001615 592.084812708642
0.000162 592.490071194338
0.000162 592.490070657344
0.0001625 592.892650097154
0.0001625 592.890730505025
0.000163 593.290652765272
0.000163 593.290652220441
0.0001635 593.687924752709
0.0001635 593.686016067473
0.000164 594.080660406105
0.000164 594.080659853515
0.0001645 594.472683234708
0.0001645 594.470785298132
0.000165 594.860208931886
0.000165 594.860208371615
0.0001655 595.247039283294
0.0001655 595.245151940561
0.000166 595.629411023253
0.000166 595.629410455375
0.0001665 596.011104528034
0.0001665 596.009227627659
0.000167 596.388377273689
0.000167 596.388376698277
0.0001675 596.764988534527
0.0001675 596.763121928258
0.000168 597.137216235478
0.000168 597.137215652603
0.0001685 597.50879884972
0.0001685 597.506942392463
0.000169 597.876034464412
0.000169 597.876033874144
0.0001695 598.242641046002
0.0001695 598.240794595733
0.00017 598.604936563277
0.00017 598.604935965684
0.0001705 598.966618764099
0.0001705 598.964782181786
0.000171 599.324025224166
0.000171 599.324024619314
0.0001715 599.680833754807
0.0001715 599.679006904332
0.000172 600.033401269651
0.000172 600.033400657606
0.0001725 600.385385919611
0.0001725 600.383568667693
0.000173 600.733163692853
0.000173 600.733163073677
0.0001735 601.08037335021
0.0001735 601.07856556634
0.000174 601.423409696445
0.000174 601.4234090702
0.0001745 601.765892367005
0.0001745 601.764093923366
0.000175 602.104234730616
0.000175 602.104234097363
0.0001755 602.44203755655
0.0001755 602.440248327955
0.000176 602.775732530035
0.000176 602.775731889833
0.0001765 603.108901808029
0.0001765 603.107121671854
0.000177 603.437995149848
0.000177 603.437994502754
0.0001775 603.766576348774
0.0001775 603.764805184891
0.000178 604.091113000719
0.000178 604.091112346791
0.0001785 604.415150778848
0.0001785 604.41338846957
0.000179 604.735174882977
0.000179 604.735174222268
0.0001795 605.054713104744
0.0001795 605.052959534758
0.00018 605.37026801986
0.00018 605.370267352425
0.0001805 605.685349772197
0.0001805 605.683604828507
0.000181 605.996478089915
0.000181 605.996477415806
0.0001815 606.307145698161
0.0001815 606.305409270036
0.000182 606.613889258557
0.000182 606.613888577827
0.0001825 606.920184301964
0.0001825 606.918456280877
0.000183 607.222584208825
0.000183 607.222583521523
0.0001835 607.524547535665
0.0001835 607.522827815244
0.000184 607.822644171353
0.000184 607.822643477527
0.0001845 608.120315913645
0.0001845 608.118604389619
0.000185 608.414148953582
0.000185 608.414148253282
0.0001855 608.707568541445
0.0001855 608.705865111597
0.000186 608.997176968238
0.000186 608.99717626151
0.0001865 609.286383143882
0.0001865 609.284687707997
0.000187 609.571805261087
0.000187 609.571804547978
0.0001875 609.856836092458
0.0001875 609.855148552279
0.000188 610.137525483999
0.000188 610.137524945203
0.0001885 610.417809818537
0.0001885 610.416134281818
0.000189 610.694371456781
0.000189 610.694370735377
0.0001895 610.970563276229
0.0001895 610.968895424321
0.00019 611.243038992039
0.00019 611.243038271859
0.0001905 611.514583678829
0.0001905 611.512927564631
0.000191 611.782465500873
0.000191 611.782464771302
0.0001915 612.049998054171
0.0001915 612.04834942207
0.000192 612.313892688664
0.000192 612.313891952925
0.0001925 612.577447895828
0.0001925 612.575806657024
0.000193 612.837389738009
0.000193 612.837388996143
0.0001935 613.097001785187
0.0001935 613.095367852575
0.000194 613.353024639707
0.000194 613.353023891752
0.0001945 613.608727126454
0.0001945 613.607100414584
0.000195 613.86086421864
0.000195 613.860863464637
0.0001955 614.112690169091
0.0001955 614.111070594134
0.000196 614.360974155939
0.000196 614.360973395925
0.0001965 614.608956029697
0.0001965 614.607343509405
0.000197 614.853419010601
0.000197 614.853418244612
0.0001975 615.097588713376
0.0001975 615.09598316705
0.000198 615.338262240596
0.000198 615.338261468669
0.0001985 615.578651134584
0.0001985 615.577052483038
0.000199 615.815566223472
0.000199 615.815565445642
0.0001995 616.052205137487
0.0001995 616.050613303016
0.0002 616.285392276463
0.0002 616.285391492765
0.0002005 616.518311515829
0.0002005 616.516726422176
0.000201 616.747800676124
0.000201 616.747799886592
0.0002015 616.977030032338
0.0002015 616.975451604663
0.000202 617.202850677509
0.000202 617.202849882176
0.0002025 617.428419437668
0.0002025 617.426847602518
0.000203 617.650600532887
0.000203 617.650599731786
0.0002035 617.872537488903
0.0002035 617.87097217418
0.000204 618.091107510033
0.000204 618.091106703195
0.0002045 618.309440967625
0.0002045 618.307882102562
0.000205 618.524427910079
0.000205 618.524427097536
0.0002055 618.739185697566
0.0002055 618.737633212693
0.000206 618.950617084967
0.000206 618.950616266748
0.0002065 619.161826561846
0.0002065 619.160280388966
0.000207 619.369729454482
0.000207 619.369728630618
0.0002075 619.577417519819
0.0002075 619.575877591985
0.000208 619.781818522907
0.000208 619.781817693428
0.0002085 619.986011623531
0.0002085 619.984477875011
0.000209 620.186936895292
0.000209 620.186936060225
0.0002095 620.387661033795
0.0002095 620.386133400056
0.00021 620.585136293343
0.00021 620.585135452716
0.0002105 620.78241703591
0.0002105 620.78089545359
0.000211 620.972102435786
0.000211 620.972102801603
0.0002115 621.139274112617
0.0002115 621.137995337223
0.000212 621.302415323574
0.000212 621.302415041327
0.0002125 621.464717884104
0.0002125 621.463456002328
0.000213 621.62390491974
0.000213 621.623904424978
0.0002135 621.782557793987
0.0002135 621.781304833568
0.000214 621.938466686405
0.000214 621.938466104928
0.0002145 622.094185429103
0.0002145 622.092944453064
0.000215 622.246404317587
0.000215 622.246403928244
0.0002155 622.398472084945
0.0002155 622.397236355591
0.000216 622.547921903828
0.000216 622.547921299913
0.0002165 622.697225066406
0.0002165 622.695993903906
0.000217 622.843735099875
0.000217 622.843734538315
0.0002175 622.989861631934
0.0002175 622.988638916865
0.000218 623.133407011904
0.000218 623.133406401534
0.0002185 623.276816177892
0.0002185 623.275597918736
0.000219 623.417658160982
0.000219 623.417657545534
0.0002195 623.558368858275
0.0002195 623.557155009313
0.00022 623.69652608713
0.00022 623.696525466628
0.0002205 623.834556839902
0.0002205 623.833347356212
0.000221 623.969436396603
0.000221 623.969435920028
0.0002215 624.103991901267
0.0002215 624.102794566929
0.000222 624.236036597559
0.000222 624.236035974706
0.0002225 624.367964455245
0.0002225 624.366771365694
0.000223 624.497394667285
0.000223 624.497394039461
0.0002235 624.626712491685
0.0002235 624.625523604544
0.000224 624.753545495548
0.000224 624.753544862776
0.0002245 624.880270452171
0.0002245 624.879085725796
0.000225 625.004523220287
0.000225 625.004522582588
0.0002255 625.128672172762
0.0002255 625.127491566226
0.000226 625.250361378863
0.000226 625.250360736259
0.0002265 625.371950893845
0.0002265 625.370774366928
0.000227 625.491092917807
0.000227 625.49109227032
0.0002275 625.609999885472
0.0002275 625.608831112163
0.000228 625.726251349606
0.000228 625.726250754298
0.0002285 625.842411419669
0.0002285 625.841246791776
0.000229 625.956155750146
0.000229 625.956155096866
0.0002295 626.06981228022
0.0002295 626.068651599485
0.00023 626.181064798581
0.00023 626.181064140484
0.0002305 626.292233222796
0.0002305 626.291076451802
0.000231 626.401009135473
0.000231 626.401008472578
0.0002315 626.509704561388
0.0002315 626.508551663359
0.000232 626.616018804173
0.000232 626.616018136497
0.0002325 626.722256070198
0.0002325 626.721107008986
0.000233 626.826123312464
0.000233 626.826122640027
0.0002335 626.92991699205
0.0002335 626.928771732129
0.000234 627.031351640895
0.000234 627.031350963713
0.0002345 627.132716046637
0.0002345 627.131574553085
0.000235 627.231732250923
0.000235 627.231731569014
0.0002355 627.330681438572
0.0002355 627.329543677065
0.000236 627.427293092885
0.000236 627.427292406267
0.0002365 627.523840865273
0.0002365 627.522706802077
0.000237 627.618061613801
0.000237 627.618060922489
0.0002375 627.712221524682
0.0002375 627.711091126637
0.000238 627.804064765
0.000238 627.804064069011
0.0002385 627.895850122812
0.0002385 627.894723357327
0.000239 627.985329009593
0.000239 627.985328308942
0.0002395 628.074752881136
0.0002395 628.073629716177
0.00024 628.16188032978
0.00024 628.161879624483
0.0002405 628.248585930277
0.0002405 628.247473680834
0.000241 628.332837882123
0.000241 628.332837221708
0.0002415 628.417040631891
0.0002415 628.415932015556
0.000242 628.498981577722
0.000242 628.498980871239
0.0002425 628.580875756906
0.0002425 628.579770618467
0.000243 628.660517712483
0.000243 628.660517001411
0.0002435 628.74011542024
0.0002435 628.739013729971
0.000244 628.81747030151
0.000244 628.817469585861
0.0002445 628.894783372117
0.0002445 628.893685100806
0.000245 628.96986287374
0.000245 628.969862153529
0.0002455 629.044902921281
0.0002455 629.043808040215
0.000246 629.117718519808
0.000246 629.117717795048
0.0002465 629.190496941322
0.0002465 629.189405422284
0.000247 629.261059898306
0.000247 629.26105916901
0.0002475 629.331587876877
0.0002475 629.330499692137
0.000248 629.39990924192
0.000248 629.399908508101
0.0002485 629.468197749698
0.0002485 629.467112872004
0.000249 629.534288363432
0.000249 629.534287625101
0.0002495 629.600348164591
0.0002495 629.599266567161
0.00025 629.664218661598
0.00025 629.664217918767
0.0002505 629.728060315234
0.0002505 629.726981971754
0.000251 629.789721126911
0.000251 629.789720379591
0.0002515 629.851354989879
0.0002515 629.85027987449
0.000252 629.910816347232
0.000252 629.910815595435
0.0002525 629.970252576925
0.0002525 629.969180664221
0.000253 630.027524513319
0.000253 630.027523757055
0.0002535 630.084773070389
0.0002535 630.083704335406
0.000254 630.139865424227
0.000254 630.139864663508
0.0002545 630.194936075251
0.0002545 630.193870493464
0.000255 630.247858492611
0.000255 630.247857727446
0.0002555 630.300760812701
0.0002555 630.299698360016
0.000256 630.351522749912
0.000256 630.351521980311
0.0002565 630.402266125268
0.0002565 630.401206778019
0.000257 630.450876851434
0.000257 630.450876077408
0.0002575 630.499470481856
0.0002575 630.498414216793
0.000258 630.545939081328
0.000258 630.545938302885
0.0002585 630.592391982664
0.0002585 630.591338776953
0.000259 630.636727357464
0.000259 630.636726574615
0.0002595 630.681048364015
0.0002595 630.67999819523
0.00026 630.72325923621
0.00026 630.723258448962
0.0002605 630.765457003085
0.0002605 630.764409849203
0.000261 630.80555191711
0.000261 630.805551125473
0.0002615 630.845634922534
0.0002615 630.844590761928
0.000262 630.883622247472
0.000262 630.883621451454
0.0002625 630.921598795043
0.0002625 630.920557606479
0.000263 630.957486726857
0.000263 630.957485926466
0.0002635 630.993364947764
0.0002635 630.992326710395
0.000264 631.027161511483
0.000264 631.027160706727
0.0002645 631.060949366675
0.0002645 631.059914060037
0.000265 631.092662418537
0.000265 631.092661609425
0.0002655 631.124367700849
0.0002655 631.123335304854
0.000266 631.154004930404
0.000266 631.154004116943
0.0002665 631.183635266641
0.0002665 631.182605761574
0.000267 631.211204198804
0.000267 631.211203381001
0.0002675 631.238767051782
0.0002675 631.237740418296
0.000268 631.264275048857
0.000268 631.264274226719
0.0002685 631.289777719405
0.0002685 631.288753938517
0.000269 631.313231983054
0.000269 631.313231156588
0.0002695 631.336681611976
0.0002695 631.335660665062
0.00027 631.35808918516
0.00027 631.358088354373
0.0002705 631.379492755155
0.0002705 631.378474623947
0.000271 631.398860524036
0.000271 631.398859688934
0.0002715 631.418224861583
0.0002715 631.417209528163
0.000272 631.435559557378
0.000272 631.435558717967
0.0002725 631.452891334584
0.0002725 631.451878781381
0.000273 631.468199535393
0.000273 631.46819869168
0.0002735 631.483505271802
0.0002735 631.482495481587
0.000274 631.496793404396
0.000274 631.496792556387
0.0002745 631.510079468762
0.0002745 631.509072424648
0.000275 631.521353810337
0.000275 631.521352958037
0.0002755 631.53257484536
0.0002755 631.531578311451
0.000276 631.541770107817
0.000276 631.541769269159
0.0002765 631.550963966275
0.0002765 631.549970138357
0.000277 631.558173094311
0.000277 631.558172243564
0.0002775 631.565381011297
0.0002775 631.564389849991
0.000278 631.570609756574
0.000278 631.570608901568
0.0002785 631.575837459341
0.0002785 631.574848949157
0.000279 631.57909145476
0.000279 631.5790905955
0.0002795 631.582344520924
0.0002795 631.581358646692
0.00028 631.583629258679
0.00028 631.583628395169
0.0002805 631.584913125605
0.0002805 631.583929872469
0.000281 631.584233958723
0.000281 631.584233090969
0.0002815 631.583553925056
0.0002815 631.582573278476
0.000282 631.580916068915
0.000282 631.580915196921
0.0002825 631.578277296087
0.0002825 631.577299241829
0.000283 631.573685829895
0.000283 631.573684953665
0.0002835 631.569093343589
0.0002835 631.568117867729
0.000284 631.562553211843
0.000284 631.562552331381
0.0002845 631.556011903443
0.0002845 631.555038992361
0.000285 631.547527917354
0.000285 631.547527032664
0.0002855 631.539042545358
0.0002855 631.538072185735
0.000286 631.528619384248
0.000286 631.528618495335
0.0002865 631.518134377105
0.0002865 631.51715907376
0.000287 631.505680629657
0.000287 631.505679730902
0.0002875 631.493224925192
0.0002875 631.492252191727
0.000288 631.478826130582
0.000288 631.478825222954
0.0002885 631.464424993799
0.0002885 631.463454803752
0.000289 631.448085487176
0.000289 631.448084575322
0.0002895 631.43174322041
0.0002895 631.430775561596
0.00029 631.413467223838
0.00029 631.413466307761
0.0002905 631.395187998088
0.0002905 631.394222858613
0.000291 631.374979607869
0.000291 631.374978687572
0.0002915 631.354767468711
0.0002915 631.353804836972
0.000292 631.332630656638
0.000292 631.332629732124
0.0002925 631.310489525472
0.0002925 631.309529390152
0.000293 631.286428140062
0.000293 631.286427211336
0.0002935 631.262361815333
0.0002935 631.261404165401
0.000294 631.236379583036
0.000294 631.2363786501
0.0002945 631.210391741431
0.0002945 631.209436566139
0.000295 631.182492267815
0.000295 631.182491330671
0.0002955 631.154586465441
0.0002955 631.153633754323
0.000296 631.124773236351
0.000296 631.124772295004
0.0002965 631.09495290989
0.0002965 631.094002652756
0.000297 631.063229292593
0.000297 631.063228347045
0.0002975 631.031497760427
0.0002975 631.030549947368
0.000298 630.997867004728
0.000298 630.997866054981
0.0002985 630.964227468055
0.0002985 630.963282089436
0.000299 630.928692707396
0.000299 630.928691753454
0.0002995 630.893148251316
0.0002995 630.892205297777
0.0003 630.855712503851
0.0003 630.855711545716
0.0003005 630.818266098437
0.0003005 630.817325560889
0.000301 630.778932268091
0.000301 630.778931305765
0.0003015 630.739586769432
0.0003015 630.738648639058
0.000302 630.698357646935
0.000302 630.698356680422
0.0003025 630.657115798172
0.0003025 630.656180066423
0.000303 630.61399406208
0.000303 630.61399309138
0.0003035 630.570858494405
0.0003035 630.569925153002
0.000304 630.525846712094
0.000304 630.525845737211
0.0003045 630.480819945751
0.0003045 630.479888986679
0.000305 630.433920574399
0.000305 630.433919595335
0.0003055 630.387005019649
0.0003055 630.386076435158
0.000306 630.338220407195
0.000306 630.338219423953
0.0003065 630.289418365272
0.0003065 630.288492147878
0.000307 630.238750751363
0.000307 630.238749763944
0.0003075 630.188064415407
0.0003075 630.187140557887
0.000308 630.135515932322
0.000308 630.135514940728
0.0003085 630.0829473883
0.0003085 630.082025883693
0.000309 630.028520061857
0.000309 630.028519066091
0.0003095 629.974071289464
0.0003095 629.97315213107
0.00031 629.917767039915
0.00031 629.917766039978
0.0003105 629.861439913459
0.0003105 629.860523094837
0.000311 629.803260556361
0.000311 629.803259552256
0.0003115 629.745056845631
0.0003115 629.744142360599
0.000312 629.685004092713
0.000312 629.685003084441
0.0003125 629.62492546383
0.0003125 629.624013306461
0.000313 629.563000923828
0.000313 629.562999911391
0.0003135 629.501048940084
0.0003135 629.50013910471
0.000314 629.437254119573
0.000314 629.437253102974
0.0003145 629.373430242253
0.0003145 629.372522723461
0.000315 629.307766546465
0.000315 629.307765525705
0.0003155 629.242072135649
0.0003155 629.241166928281
0.000316 629.174540869262
0.000316 629.174539844343
0.0003165 629.106977184624
0.0003165 629.106074283777
0.000317 629.037579552554
0.000317 629.037578523478
0.0003175 628.968147754133
0.0003175 628.967247155158
0.000318 628.8968848623
0.000318 628.896883829069
0.0003185 628.825586011268
0.0003185 628.824687709767
0.000319 628.752458867353
0.000319 628.752457829969
0.0003195 628.679293926763
0.0003195 628.678397918592
0.00032 628.604303440951
0.00032 628.604302399416
0.0003205 628.529273276473
0.0003205 628.528379557741
0.000321 628.452420262182
0.000321 628.452419216497
0.0003215 628.375525642829
0.0003215 628.374634209895
0.000322 628.296810817426
0.000322 628.296809767593
0.0003225 628.218052416261
0.0003225 628.217163265736
0.000323 628.137137742502
0.000323 628.137136759785
0.0003235 628.056002682728
0.0003235 628.055110515345
0.000324 627.973041763036
0.000324 627.973040695918
0.0003245 627.890033029378
0.0003245 627.889143163344
0.000325 627.805201052534
0.000325 627.805199981266
0.0003255 627.720319152734
0.0003255 627.719431585436
0.000326 627.633616496938
0.000326 627.633615421524
0.0003265 627.546861765182
0.0003265 627.545976494263
0.000327 627.458288715915
0.000327 627.458287636357
0.0003275 627.369661393743
0.0003275 627.368778417095
0.000328 627.279218144414
0.000328 627.279217060714
0.0003285 627.188718381368
0.0003285 627.187837697132
0.000329 627.09640503396
0.000329 627.09640394612
0.0003295 627.00403288822
0.0003295 627.003154494794
0.00033 626.909849453922
0.00033 626.909848361945
0.0003305 626.815604892937
0.0003305 626.814728788966
0.000331 626.719551292764
0.000331 626.719550196652
0.0003315 626.623434193863
0.0003315 626.622560378245
0.000332 626.52551025927
0.000332 626.525509159025
0.0003325 626.427520410274
0.0003325 626.426648882157
0.000333 626.327725883753
0.000333 626.327724779379
0.0003335 626.227862983567
0.0003335 626.226993742352
0.000334 626.126197519245
0.000334 626.126196410743
0.0003345 626.024461178445
0.0003345 626.023594223783
0.000335 625.920924342661
0.000335 625.920923230034
0.0003355 625.817314084075
0.0003355 625.816449415871
0.000336 625.711905355955
0.000336 625.711904239206
0.0003365 625.606420615229
0.0003365 625.605558233638
0.000337 625.498935916741
0.000337 625.498934840249
0.0003375 625.391244323043
0.0003375 625.390381803988
0.000338 625.281753176033
0.000338 625.28175204656
0.0003385 625.172180574498
0.0003385 625.1713203557
0.000339 625.060810387857
0.000339 625.060809254268
0.0003395 624.949356023243
0.0003395 624.948498105638
0.00034 624.836105951308
0.00034 624.836104813607
0.0003405 624.722768934539
0.0003405 624.721913319322
0.000341 624.607638047198
0.000341 624.607636905388
0.0003415 624.492417404763
0.0003415 624.491564093378
0.000342 624.375404687981
0.000342 624.375403542066
0.0003425 624.25829936245
0.0003425 624.257448356599
0.000343 624.139403718791
0.000343 624.139402568773
0.0003435 624.020412569333
0.0003435 624.019563870973
0.000344 623.899632818467
0.000344 623.899631664352
0.0003445 623.778212803285
0.0003445 623.777361687062
0.000345 623.654807089155
0.000345 623.65480596719
0.0003455 623.531299808715
0.0003455 623.530451165792
0.000346 623.405997789883
0.000346 623.405996618679
0.0003465 623.280590989864
0.0003465 623.27974468713
0.000347 623.153391141653
0.000347 623.153389966361
0.0003475 623.026083434781
0.0003475 623.025239475281
0.000348 622.896984287406
0.000348 622.89698310803
0.0003485 622.767739466594
0.0003485 622.766895391895
0.000349 622.636352513503
0.000349 622.63635140936
0.0003495 622.504851579061
0.0003495 622.504010119059
0.00035 622.371557679041
0.00035 622.371556487104
0.0003505 622.238146251148
0.0003505 622.237307157712
0.000351 622.102943461919
0.000351 622.102942265912
0.0003515 621.966917786892
0.0003515 621.966076724072
0.000352 621.829010030437
0.000352 621.829008841932
0.0003525 621.690977910925
0.0003525 621.690139309385
0.000353 621.550860997815
0.000353 621.550859852139
0.0003535 621.410506110981
0.0003535 621.409667830363
0.000354 621.267730040656
0.000354 621.267728966271
0.0003545 621.124415645814
0.0003545 621.123573568366
0.000355 620.978173842761
0.000355 620.978172869145
0.0003555 620.813754729213
0.0003555 620.81279171231
0.000356 620.641997708031
0.000356 620.641997728855
0.0003565 620.470098492511
0.0003565 620.469143344096
0.000357 620.296144862115
0.000357 620.296143351713
0.0003575 620.122039257714
0.0003575 620.121087386104
0.000358 619.945883656849
0.000358 619.945882142623
0.0003585 619.769572379547
0.0003585 619.768623792763
0.000359 619.591213971349
0.000359 619.591212453316
0.0003595 619.412696142892
0.0003595 619.411750849255
0.00036 619.232134024308
0.00036 619.232132502484
0.0003605 619.051408700313
0.0003605 619.050466708439
0.000361 618.868641902811
0.000361 618.868640377211
0.0003615 618.685708073478
0.0003615 618.684769392281
0.000362 618.500735563764
0.000362 618.500734034407
0.0003625 618.31559215459
0.0003625 618.314656793282
0.000363 618.128412835317
0.000363 618.128411302219
0.0003635 617.9410587078
0.0003635 617.940126675894
0.000364 617.751671418277
0.000364 617.751669881457
0.0003645 617.562105370627
0.0003645 617.561176677935
0.000365 617.370508887526
0.000365 617.370507347001
0.0003655 617.17872965537
0.0003655 617.177804312004
0.000366 616.984922693431
0.000366 616.98492114922
0.0003665 616.790928950513
0.0003665 616.790006966888
0.000367 616.594910163249
0.000367 616.59490861537
0.0003675 616.398700522138
0.0003675 616.397781908969
0.000368 616.200468502533
0.000368 616.200466951006
0.0003685 616.002041515319
0.0003685 616.001126283627
0.000369 615.801594796536
0.000369 615.80159324138
0.0003695 615.600948955535
0.0003695 615.600037116643
0.00037 615.398286011613
0.00037 615.398284452848
0.0003705 615.195419750062
0.0003705 615.194511315597
0.000371 614.990538996616
0.000371 614.990537434262
0.0003715 614.785450689375
0.0003715 614.784545671272
0.000372 614.578350484297
0.000372 614.578348918376
0.0003725 614.371038448549
0.0003725 614.370136859048
0.000373 614.161717092707
0.000373 614.16171552324
0.0003735 613.952179588652
0.0003735 613.951281440302
0.000374 613.740635326592
0.000374 613.7406337536
0.0003745 613.52887055815
0.0003745 613.527975863807
0.000375 613.315101578791
0.000375 613.315100002296
0.0003755 613.101107694299
0.0003755 613.100216467131
0.000376 612.885112131638
0.000376 612.885110551663
0.0003765 612.668887224551
0.0003765 612.667999478034
0.000377 612.450663158361
0.000377 612.450661574929
0.0003775 612.232205267951
0.0003775 612.231321015874
0.000378 612.011750724483
0.000378 612.011749137618
0.0003785 611.791057836541
0.0003785 611.790177093006
0.000379 611.568370789229
0.000379 611.568369198954
0.0003795 611.345440836765
0.0003795 611.344563616187
0.00038 611.120272161
0.00038 611.12027063927
0.0003805 610.894613995876
0.0003805 610.893738863215
0.000381 610.666963520641
0.000381 610.666961919508
0.0003815 610.439060749311
0.0003815 610.438189183449
0.000382 610.209168233069
0.000382 610.209166628611
0.0003825 609.979018739521
0.0003825 609.978150755848
0.000383 609.74661785415
0.000383 609.746616324225
0.0003835 609.51371567744
0.0003835 609.512849883363
0.000384 609.278825668968
0.000384 609.278824053831
0.0003845 609.04366904138
0.0003845 609.042806875436
0.000385 608.80652715547
0.000385 608.806525537098
0.0003855 608.569113849462
0.0003855 608.568255328028
0.000386 608.329717769255
0.000386 608.329716147677
0.0003865 608.090045428006
0.0003865 608.08919056778
0.000387 607.848392789993
0.000387 607.848391165237
0.0003875 607.606459010075
0.0003875 607.605607828081
0.000388 607.362547404831
0.000388 607.362545776926
0.0003885 607.118349736939
0.0003885 607.117502250522
0.000389 606.872176709848
0.000389 606.872175078824
0.0003895 606.62571265953
0.0003895 606.624868886363
0.00039 606.37727571152
0.00039 606.377274077407
0.0003905 606.128542739913
0.0003905 606.127702697993
0.000391 605.877839328193
0.000391 605.877837691021
0.0003915 605.626834852756
0.0003915 605.625998560413
0.000392 605.373862391559
0.000392 605.37386075136
0.0003925 605.120583786819
0.0003925 605.119751262709
0.000393 604.865339648142
0.000393 604.865338004949
0.0003935 604.609784246436
0.0003935 604.608955509546
0.000394 604.352265760791
0.000394 604.352264114636
0.0003945 604.094430853016
0.0003945 604.093605922668
0.000395 603.834635310183
0.000395 603.8346336611
0.0003955 603.574518146548
0.0003955 603.573697042394
0.000396 603.312442796329
0.000396 603.312441144351
0.0003965 603.050040587114
0.0003965 603.049223329143
0.000397 602.785682640095
0.000397 602.785680985257
0.0003975 602.520992556411
0.0003975 602.52017916495
0.000398 602.254349184729
0.000398 602.254347527067
0.0003985 601.987368359287
0.0003985 601.986558854998
0.000399 601.718436697399
0.000399 601.718435036948
0.0003995 601.44916222528
0.0003995 601.448356629165
0.0004 601.177939370739
0.0004 601.177937707537
0.0004005 600.906368310172
0.0004005 600.905566643575
0.000401 600.632851324408
0.000401 600.632849658492
0.0004015 600.35898069755
0.0004015 600.358182982157
0.000402 600.083166606659
0.000402 600.083164938066
0.0004025 599.806993400385
0.0004025 599.806199658224
0.000403 599.528879195914
0.000403 599.528877524684
0.0004035 599.25040036261
0.0004035 599.249610616056
0.000404 598.969983002361
0.000404 598.969981328534
0.0004045 598.689195460722
0.0004045 598.688409732496
0.000405 598.406471869556
0.000405 598.406470193171
0.0004055 598.123372505392
0.0004055 598.122590818561
0.000406 597.838339576032
0.000406 597.838337897131
0.0004065 597.552925243081
0.0004065 597.552147621065
0.000407 597.265579836936
0.000407 597.26557815556
0.0004075 596.977847357678
0.0004075 596.977073824246
0.000408 596.688186305664
0.000408 596.688184621855
0.0004085 596.398132472149
0.0004085 596.397363051424
0.000409 596.106152575513
0.000409 596.106150889316
0.0004095 595.813774150195
0.0004095 595.813008866652
0.00041 595.519472181358
0.00041 595.519470492817
0.0004105 595.224765897927
0.0004105 595.224004776399
0.000411 594.928138601326
0.000411 594.928136910486
0.0004115 594.631101165558
0.0004115 594.630344231234
0.000412 594.332145258496
0.000412 594.332143565402
0.0004125 594.032773349106
0.0004125 594.032020627533
0.000413 593.731485522609
0.000413 593.731483827308
0.0004135 593.42977579211
0.0004135 593.429027309198
0.000414 593.126152711797
0.000414 593.126151014338
0.0004145 592.82210178737
0.0004145 592.821357569389
0.000415 592.516140094325
0.000415 592.516138394754
0.0004155 592.209744578695
0.0004155 592.209004652278
0.000416 591.901440890346
0.000416 591.901439188714
0.0004165 591.592697362667
0.0004165 591.591961754812
0.000417 591.282048273682
0.000417 591.282046570039
0.0004175 590.970953290429
0.0004175 590.970222028501
0.000418 590.657955373611
0.000418 590.657953668008
0.0004185 590.344505469481
0.0004185 590.343778581214
0.000419 590.029155276675
0.000419 590.029153569165
0.0004195 589.7133469655
0.0004195 589.712624478997
0.00042 589.395641028512
0.00042 589.395639319147
0.0004205 589.077470804176
0.0004205 589.07675274791
0.000421 588.757405635696
0.000421 588.757403924529
0.0004215 588.436869973061
0.0004215 588.436156375879
0.000422 588.114442067596
0.000422 588.114440354684
0.0004225 587.791537423446
0.0004225 587.790828314568
0.000423 587.466743258267
0.000423 587.466741543664
0.0004235 587.141466072247
0.0004235 587.14076148127
0.000424 586.814302108339
0.000424 586.814300392103
0.0004245 586.486648803916
0.0004245 586.485948760815
0.000425 586.157111486944
0.000425 586.157109769132
0.0004255 585.827078472374
0.0004255 585.8263830075
0.000426 585.495164233651
0.000426 585.495162514324
0.0004265 585.162747902952
0.0004265 585.162057047039
0.000427 584.828453160429
0.000427 584.828451439644
0.0004275 584.493649894369
0.0004275 584.492963678533
0.000428 584.156971053621
0.000428 584.156969331441
0.0004285 583.819777220716
0.0004285 583.819095676456
0.000429 583.480049829954
0.000429 583.480048332444
0.0004295 583.139565213112
0.0004295 583.138887524181
0.00043 582.797208562275
0.00043 582.797206832866
0.0004305 582.454323567002
0.0004305 582.453650634596
0.000431 582.109569733661
0.000431 582.109568003073
0.0004315 581.764280983896
0.0004315 581.763612840693
0.000432 581.417126390012
0.000432 581.417124658309
0.0004325 581.069430271432
0.0004325 581.068766950496
0.000433 580.719871332266
0.000433 580.719869599516
0.0004335 580.369764223992
0.0004335 580.369105758777
0.000434 580.01779734917
0.000434 580.01779561544
0.0004345 579.665275624835
0.0004345 579.664622049187
0.000435 579.310897219423
0.000435 579.310895484782
0.0004355 578.955957248251
0.0004355 578.955308596407
0.000436 578.599163713835
0.000436 578.599161978353
0.0004365 578.241801861731
0.0004365 578.241158168322
0.000437 577.882589597521
0.000437 577.882587861271
0.0004375 577.522802228173
0.0004375 577.522163528224
0.000438 577.161167632109
0.000438 577.161165895161
0.0004385 576.7989511081
0.0004385 576.798317437034
0.000439 576.434890577973
0.000439 576.434888840403
0.0004395 576.070032643354
0.0004395 576.069402771618
0.00044 575.702673971713
0.00044 575.702672466692
0.0004405 575.334720307297
0.0004405 575.334096269946
0.000441 574.964927647107
0.000441 574.964925904532
0.0004415 574.594532125079
0.0004415 574.593913244811
0.000442 574.222301224011
0.000442 574.222299481069
0.0004425 573.84946050742
0.0004425 573.848846821259
0.000443 573.47478783079
0.000443 573.474786087561
0.0004435 573.099498350772
0.0004435 573.098889896142
0.000444 572.722380370081
0.000444 572.722378626645
0.0004445 572.344638564156
0.0004445 572.344035378883
0.000445 571.965071758294
0.000445 571.965070014733
0.0004455 571.584874071577
0.0004455 571.584276193891
0.000446 571.202854928036
0.000446 571.202853184432
0.0004465 570.820197814454
0.0004465 570.819605282988
0.000447 570.435722830554
0.000447 570.435721086992
0.0004475 570.05060275408
0.0004475 570.050015607873
0.000448 569.66366843821
0.000448 569.663666694776
0.0004485 569.276081874111
0.0004485 569.275500152609
0.000449 568.886684746986
0.000449 568.886683003766
0.0004495 568.49662818308
0.0004495 568.496051926134
0.00045 568.104764779006
0.00045 568.104763036088
0.0004505 567.71223471694
0.0004505 567.711663964809
0.000451 567.317901585096
0.000451 567.317899842572
0.0004515 566.922894541633
0.0004515 566.922329334987
0.000452 566.526088247372
0.000452 566.526086505331
0.0004525 566.128600755691
0.0004525 566.128041135609
0.000453 565.729317881846
0.000453 565.729316140381
0.0004535 565.329346492863
0.0004535 565.328792500834
0.000454 564.927583641073
0.000454 564.927581900278
0.0004545 564.525124924769
0.0004545 564.524576602695
0.000455 564.12087871682
0.000455 564.120876976791
0.0004555 563.715929263591
0.0004555 563.715386653785
0.000456 563.309196342768
0.000456 563.309194603602
0.0004565 562.901752764783
0.0004565 562.90121590997
0.000457 562.492529797239
0.000457 562.492528059035
0.0004575 562.082588729815
0.0004575 562.082057673136
0.000458 561.670872405958
0.000458 561.670870668815
0.0004585 561.258430508952
0.0004585 561.257905293962
0.000459 560.84421754484
0.000459 560.84421580886
0.0004595 560.429271504057
0.0004595 560.428752174722
0.00046 560.012558642814
0.00046 560.0125569081
0.0004605 559.595105171424
0.0004605 559.594591772129
0.000461 559.175889184667
0.000461 559.175887451325
0.0004615 558.755925024645
0.0004615 558.755417600188
0.000462 558.334202713932
0.000462 558.334200982067
0.0004625 557.911724637507
0.0004625 557.911223233105
0.000463 557.487492835797
0.000463 557.487491105516
0.0004635 557.062497646921
0.0004635 557.062002308204
0.000464 556.635753220045
0.000464 556.635751491459
0.0004645 556.208237755879
0.0004645 556.207748528895
0.000465 555.778977604036
0.000465 555.778975877255
0.0004655 555.348938736445
0.0004655 555.348455667658
0.000466 554.917159795708
0.000466 554.917158070846
0.0004665 554.484594432779
0.0004665 554.484117569071
0.000467 554.050293676617
0.000467 554.050291953788
0.0004675 553.615198764187
0.0004675 553.614728152857
0.000468 553.178373205009
0.000468 553.178371484328
0.0004685 552.740745728215
0.0004685 552.740281416977
0.000469 552.301392418917
0.000469 552.301390700503
0.0004695 551.86122940376
0.0004695 551.860771440746
0.00047 551.419345439302
0.00047 551.419343723274
0.0004705 550.976643954224
0.0004705 550.976192387983
0.000471 550.532226473218
0.000471 550.532224759698
0.0004715 550.0869836307
0.0004715 550.086538510197
0.000472 549.640029817011
0.000472 549.640028106122
0.0004725 549.191695374255
0.0004725 549.191255584012
0.000473 548.740326393058
0.000473 548.740325218237
0.0004735 548.288118148924
0.0004735 548.287686565147
0.000474 547.834209398486
0.000474 547.834207687687
0.0004745 547.379451340931
0.0004745 547.379026390751
0.000475 546.922998670723
0.000475 546.92299696301
0.0004755 546.465688784243
0.0004755 546.465270518177
0.000476 546.00668971662
0.000476 546.006688012124
0.0004765 545.546825500198
0.0004765 545.546413969172
0.000477 545.083936418437
0.000477 545.083935267509
0.0004775 544.61941861321
0.0004775 544.619014654276
0.000478 544.153223442488
0.000478 544.15322173909
0.0004785 543.686146881285
0.0004785 543.685749802897
0.000479 543.217399301953
0.000479 543.217397602262
0.0004795 542.747762370083
0.0004795 542.747372224243
0.00048 542.276460293744
0.00048 542.276458597897
0.0004805 541.804260882066
0.0004805 541.803877721172
0.000481 541.330402283179
0.000481 541.330400591318
0.0004815 540.855638345902
0.0004815 540.855262222746
0.000482 540.379221263548
0.000482 540.379219575814
0.0004825 539.90189082
0.0004825 539.90152178777
0.000483 539.422913359665
0.000483 539.422911676204
0.0004835 538.943014496087
0.0004835 538.942652608362
0.000484 538.461474831454
0.000484 538.461473152411
0.0004845 537.9790057028
0.0004845 537.978651013551
0.000485 537.494902077563
0.000485 537.494900403086
0.0004855 537.009860909325
0.0004855 537.009513472911
0.000486 536.523191639014
0.000486 536.523189969255
0.0004865 536.035576729059
0.0004865 536.035236600227
0.000487 535.546340202886
0.000487 535.546338537996
0.0004875 535.056149923309
0.0004875 535.05581715719
0.000488 534.564344606024
0.000488 534.564342946157
0.0004885 534.071577405016
0.0004885 534.071252057124
0.000489 533.577201838785
0.000489 533.577200184097
0.0004895 533.081856242518
0.0004895 533.081538368749
0.00049 532.584909048812
0.00049 532.584907399462
0.0004905 532.086983663341
0.0004905 532.086673319969
0.000491 531.587463544841
0.000491 531.587461900989
0.0004915 531.086957058019
0.0004915 531.086654301695
0.000492 530.584862800542
0.000492 530.58486116235
0.0004925 530.081773983953
0.0004925 530.081478871698
0.000493 529.577104458381
0.000493 529.577102826014
0.0004935 529.071432169288
0.0004935 529.071144758496
0.000494 528.564186333528
0.000494 528.564184707152
0.0004945 528.055929516832
0.0004945 528.055649865266
0.000495 527.546106417782
0.000495 527.546104797565
0.0004955 527.035264108005
0.0004955 527.03499227379
0.000496 526.522862883532
0.000496 526.522861269646
0.0004965 526.009434206808
0.0004965 526.009170248432
0.000497 525.494454087752
0.000497 525.494452480369
0.0004975 524.978438263835
0.0004975 524.978182240145
0.000498 524.460878576018
0.000498 524.460876975313
0.0004985 523.942274920306
0.0004985 523.942026890504
0.000499 523.422135086559
0.000499 523.42213349271
0.0004995 522.900943012134
0.0004995 522.900703035771
0.0005 522.37822255434
0.0005 522.378220967526
0.0005005 521.854441574017
0.0005005 521.854209710994
0.000501 521.329140115168
0.000501 521.329138535569
0.0005015 520.802769843563
0.0005015 520.802546154125
0.000502 520.274887109828
0.000502 520.274885537629
0.0005025 519.745927265443
0.0005025 519.745711810173
0.000503 519.215463088253
0.000503 519.21546152364
0.0005035 518.683913495567
0.0005035 518.683706335385
0.000504 518.150867813716
0.000504 518.150866256876
0.0005045 517.616728405297
0.0005045 517.616529601454
0.000505 517.08088992533
0.000505 517.080888471687
0.0005055 516.543072616138
0.0005055 516.542883367096
0.000506 516.003778293729
0.000506 516.003776751606
0.0005065 515.463373676434
0.0005065 515.46319292678
0.000507 514.921500803531
0.000507 514.92149926981
0.0005075 514.378509337833
0.0005075 514.378337149619
0.000508 513.834058397425
0.000508 513.834056872303
0.0005085 513.288480563551
0.0005085 513.288316999134
0.000509 512.741452157222
0.000509 512.741450640898
0.0005095 512.193288554784
0.0005095 512.193133676823
0.00051 511.643683404882
0.00051 511.643681897558
0.0005105 511.092934755052
0.0005105 511.092788626504
0.000511 510.540753706875
0.000511 510.540752208754
0.0005115 509.987420854579
0.0005115 509.98728353869
0.000512 509.432664878557
0.000512 509.432663389846
0.0005125 508.876748794678
0.0005125 508.876620354979
0.000513 508.319418988583
0.000513 508.319417509491
0.0005135 507.760920772173
0.0005135 507.760801272475
0.000514 507.20101836333
0.000514 507.201016894067
0.0005145 506.639939243837
0.0005145 506.639828748223
0.000515 506.077465591347
0.000515 506.077464132126
0.0005155 505.513806930848
0.0005155 505.513705503666
0.000516 504.948763527823
0.000516 504.948762078859
0.0005165 504.382526823272
0.0005165 504.382434529132
0.000517 503.814915299079
0.000517 503.814913860589
0.0005175 503.246102184558
0.0005175 503.246019088323
0.000518 502.675924307071
0.000518 502.675922879277
0.0005185 502.104536556063
0.0005185 502.10446272284
0.000519 501.531794233925
0.000519 501.531792817049
0.0005195 500.957833761578
0.0005195 500.957769256715
0.00052 500.382529046474
0.00052 500.382527640739
0.0005205 499.805997911887
0.0005205 499.805942800962
0.000521 499.22813300082
0.000521 499.228131606454
0.0005215 498.649033409335
0.0005215 498.648987758152
0.000522 498.068610646914
0.000522 498.068609264146
0.0005225 497.486944952412
0.0005225 497.48690882699
0.000523 496.902453878277
0.000523 496.902453264606
0.0005235 496.31604968328
0.0005235 496.316026345521
0.000524 495.728350125838
0.000524 495.728348766347
0.0005245 495.139391734703
0.0005245 495.139378089875
0.000525 494.549149904345
0.000525 494.54914855726
0.0005255 493.957641087851
0.0005255 493.957637202078
0.000526 493.364860141948
0.000526 493.364858807508
0.0005265 492.770804071477
0.0005265 492.770810011058
0.000527 492.175487327314
0.000527 492.175486005757
0.0005275 491.578887335147
0.0005275 491.578903166541
0.000528 490.98103827221
0.000528 490.981036963781
0.0005285 490.381897853836
0.0005285 490.381923643652
0.000529 489.781520116116
0.000529 489.78151882106
0.0005295 489.179842932536
0.0005295 489.179878747527
0.00053 488.576940330829
0.00053 488.576939049393
0.0005305 487.972730210871
0.0005305 487.972776117925
0.000531 487.367306725083
0.000531 487.367305457517
0.0005315 486.760567667719
0.0005315 486.760623733844
0.000532 486.15262744917
0.000532 486.152626195725
0.0005325 485.543363625828
0.0005325 485.543429918147
0.000533 484.932910999559
0.000533 484.93290976049
0.0005335 484.321126756444
0.0005335 484.321203342182
0.000534 483.708166223525
0.000534 483.708164999089
0.0005345 483.092898331208
0.0005345 483.092986970069
0.000535 482.476361997525
0.000535 482.476360845884
0.0005355 481.858478751259
0.0005355 481.85857800677
0.000536 481.239448861126
0.000536 481.239447667202
0.0005365 480.619063921711
0.0005365 480.619173687779
0.000537 479.997545446695
0.000537 479.997544268239
0.0005375 479.374664041959
0.0005375 479.374784385805
0.000538 478.750662311982
0.000538 478.750661149259
0.0005385 478.12528979886
0.0005385 478.125420787745
0.000539 477.498810331613
0.000539 477.498809184891
0.0005395 476.870952255924
0.0005395 476.871093957137
0.00054 476.242000759146
0.00054 476.241999628695
0.0005405 475.611662857879
0.0005405 475.611815338726
0.000541 474.980245231631
0.000541 474.980244117722
0.0005415 474.34704125275
0.0005415 474.347205773132
0.000542 473.711045707202
0.000542 473.711045558706
0.0005425 473.073653109357
0.0005425 473.073831425313
0.000543 472.435216959347
0.000543 472.435215881228
0.0005435 471.795372336165
0.0005435 471.795561660313
0.000544 471.154499427621
0.000544 471.154498366976
0.0005445 470.512210453779
0.0005445 470.512410852681
0.000545 469.868907681517
0.000545 469.868906638624
0.0005455 469.22418128764
0.0005455 469.224392827796
0.000546 468.578455750278
0.000546 468.578454725419
0.0005465 467.930572375423
0.0005465 467.930796707426
0.000547 467.281391298697
0.000547 467.281390471188
0.0005475 466.63077292733
0.0005475 466.631009081666
0.000548 465.979189945231
0.000548 465.979188958562
0.0005485 465.326161552391
0.0005485 465.326409058893
0.000549 464.672183953623
0.000549 464.672182985895
0.0005495 464.013859016821
0.0005495 464.014124543685
0.00055 463.353391120174
0.00055 463.35339087772
0.0005505 462.691468698209
0.0005505 462.691747864783
0.000551 462.028647666453
0.000551 462.02864674336
0.0005515 461.363652252449
0.0005515 461.363944687699
0.000552 460.697475125317
0.000552 460.697474399304
0.0005525 460.029387725319
0.0005525 460.029694038214
0.000553 459.498130366178
0.000553 459.498081648705
0.0005535 459.093417487932
0.0005535 459.093327255337
0.000554 458.668691524265
0.000554 458.668698339291
0.0005545 458.171434953423
0.0005545 458.171450257474
0.000555 457.672776291939
0.000555 457.672775269963
0.0005555 457.172681315563
0.0005555 457.172705381855
0.000556 456.671196901813
0.000556 456.671195891127
0.0005565 456.16826348269
0.0005565 456.168296410728
0.000557 455.663945575125
0.000557 455.663944576046
0.0005575 455.158165918281
0.0005575 455.158207808607
0.000558 454.651006868368
0.000558 454.651005881218
0.0005585 454.142373272604
0.0005585 454.142424226799
0.000559 453.63236552841
0.000559 453.632364553515
0.0005595 453.12087038992
0.0005595 453.120930510605
0.00056 452.608006499766
0.00056 452.60800553746
0.0005605 452.093642315821
0.0005605 452.093711706662
0.000561 451.577914931999
0.000561 451.57791398262
0.0005615 451.060674304689
0.0005615 451.0607530704
0.000562 450.542076187245
0.000562 450.542075251138
0.0005625 450.0219518273
0.0005625 450.022040073643
0.000563 449.500475847884
0.000563 449.500474925399
0.0005635 448.977460578548
0.0005635 448.977558412337
0.000564 448.453099724326
0.000564 448.453098815821
0.0005645 447.927186485308
0.0005645 447.927294014411
0.000565 447.39993386295
0.000565 447.399932968788
0.0005655 446.871115714437
0.0005655 446.871233047774
0.000566 446.34096455416
0.000566 446.340963674709
0.0005665 445.8092346809
0.0005665 445.80936192845
0.000567 445.276178340592
0.000567 445.276177476229
0.0005675 444.741530056046
0.0005675 444.741667328843
0.000568 444.205562025449
0.000568 444.205561176557
0.0005685 443.667988776012
0.0005685 443.668136186145
0.000569 443.129102680977
0.000569 443.129101847943
0.0005695 442.588598050265
0.0005695 442.588755710881
0.00057 442.046787657076
0.00057 442.046786840295
0.0005705 441.503345370287
0.0005705 441.503513395589
0.000571 440.958604590056
0.000571 440.958603789931
0.0005715 440.4122185184
0.0005715 440.412397023646
0.000572 439.864541411528
0.000572 439.864540628467
0.0005725 439.315205576724
0.0005725 439.315394678226
0.000573 438.764586357437
0.000573 438.764585591855
0.0005735 438.212294936284
0.0005735 438.212494751408
0.000574 437.658727977234
0.000574 437.658727229554
0.0005745 437.103475306255
0.0005745 437.103685953415
0.000575 436.546955143195
0.000575 436.546954413845
0.0005755 435.988735723345
0.0005755 435.988957322004
0.000576 435.429257059873
0.000576 435.429256349292
0.0005765 434.868065561326
0.0005765 434.868298231991
0.000577 434.305623273704
0.000577 434.305622582333
0.0005775 433.741454540703
0.0005775 433.741698404923
0.000578 433.176043682739
0.000578 433.176043011029
0.0005785 432.608892738528
0.0005785 432.609147918887
0.000579 432.040508546535
0.000579 432.040507894945
0.0005795 431.470370598351
0.0005795 431.470637218467
0.00058 430.899008496179
0.00058 430.899007865173
0.0005805 430.325878940325
0.0005805 430.326157124842
0.000581 429.751534544454
0.000581 429.751533934505
0.0005815 429.175408971439
0.0005815 429.175698846023
0.000582 428.598078096155
0.000582 428.598077507744
0.0005825 428.018952295904
0.0005825 428.019253987238
0.000583 427.436877906371
0.000583 427.436878242782
0.0005835 426.852545399051
0.0005835 426.852863643163
0.000584 426.267033563662
0.000584 426.267033025152
0.0005845 425.679697205204
0.0005845 425.680027571584
0.000585 425.0911928326
0.000585 425.091192317277
0.0005855 424.500850215082
0.0005855 424.501192832972
0.000586 423.909350239398
0.000586 423.909349747775
0.0005865 423.315998281582
0.0005865 423.316353281183
0.000587 422.721499864948
0.000587 422.721499397543
0.0005875 422.125135715437
0.0005875 422.125503227899
0.000588 421.527636253553
0.000588 421.527635810892
0.0005885 420.928257296407
0.0005885 420.928637453816
0.000589 420.32775442418
0.000589 420.327754006798
0.0005895 419.725358284606
0.0005895 419.725751219976
0.00059 419.120184396128
0.00059 419.120184905056
0.0005905 418.512163418458
0.0005905 418.512575259746
0.000591 417.903051586232
0.000591 417.903051228774
0.0005915 417.292017422023
0.0005915 417.292442363475
0.000592 416.679905693499
0.000592 416.679905363166
0.0005925 416.065857963917
0.0005925 416.066296140487
0.000593 415.450745334256
0.000593 415.450745031614
0.0005935 414.833683035384
0.0005935 414.834134582866
0.000594 414.215568771411
0.000594 414.215568497033
0.0005945 413.595491173174
0.0005945 413.595956228183
0.000595 412.974212330062
0.000595 412.974212174741
0.0005955 412.34855016201
0.0005955 412.349035173045
0.000596 411.721875012644
0.000596 411.721874804864
0.0005965 411.093209066838
0.0005965 411.093707906222
0.000597 410.463544206009
0.000597 410.463544028448
0.0005975 409.831875032264
0.0005975 409.832387837997
0.000598 409.199221213933
0.000598 409.199221067195
0.0005985 408.564549580632
0.0005985 408.565076491428
0.000599 407.928907863188
0.000599 407.928907747887
0.0005995 407.291234845559
0.0005995 407.291776000827
0.0006 406.652606598022
};

\nextgroupplot[
    x label style={at={(axis description cs:0.5,-0.05)},anchor=north},
    y label style={at={(axis description cs:-0.005,.5)},rotate=0,anchor=south},
tick align=outside,
tick pos=left,
x grid style={white!69.0196078431373!black},
xlabel={Время},
xmin=-2.995e-05, xmax=0.00062895,
xtick style={color=black},
xtick={-0.0002,0,0.0002,0.0004,0.0006,0.0008},
xticklabels={−0.0002,0.0000,0.0002,0.0004,0.0006,0.0008},
y grid style={white!69.0196078431373!black},
ylabel={$T_0$},
ymin=6479.26632350849, ymax=11335.4072063217,
ytick style={color=black},
ytick={0,1000,...,15000}
%yticklabels={6000,8000,10000,12000}
]
\addplot [semithick, color1]
table {%
0 6700
1e-06 7191.86300239168
2e-06 7334.1572785691
3e-06 7457.56177785594
4e-06 7579.94073636473
5e-06 7701.41078198279
6e-06 7822.05604464933
7e-06 7941.94614163
8e-06 8031.63296493371
9e-06 8081.54189438689
1e-05 8130.91207555382
1.1e-05 8179.7882602184
1.2e-05 8228.20633205155
1.3e-05 8276.19891146368
1.4e-05 8323.78946684979
1.5e-05 8371.00266744895
1.6e-05 8417.85688464771
1.7e-05 8464.36753198453
1.8e-05 8510.55254787014
1.9e-05 8556.42694249224
2e-05 8602.00583912594
2.1e-05 8647.29898576612
2.2e-05 8692.31596409568
2.3e-05 8737.06643538517
2.4e-05 8781.55867199379
2.5e-05 8825.79841274002
2.6e-05 8869.79317640288
2.7e-05 8913.55028504734
2.8e-05 8957.07417683084
2.9e-05 9000.37358078353
3e-05 9043.47430034342
3.1e-05 9086.40109706972
3.2e-05 9129.16016827082
3.3e-05 9171.75591864541
3.4e-05 9200.35303569761
3.5e-05 9222.52405578116
3.6e-05 9244.53764691205
3.7e-05 9266.39674262161
3.8e-05 9288.10401058358
3.9e-05 9309.66196115472
4e-05 9331.07311742951
4.1e-05 9352.34051740841
4.2e-05 9373.46649877271
4.3e-05 9394.45314504318
4.4e-05 9415.30244968789
4.5e-05 9436.01635113186
4.6e-05 9456.59685437002
4.7e-05 9477.04572486726
4.8e-05 9497.36463940278
4.9e-05 9517.55520636099
5e-05 9537.61896925408
5.1e-05 9557.5574201457
5.2e-05 9577.37233755863
5.3e-05 9597.06566914032
5.4e-05 9616.63875337531
5.5e-05 9636.09284244433
5.6e-05 9655.42914061078
5.7e-05 9674.64892064007
5.8e-05 9693.75344389874
5.9e-05 9712.74383892594
6e-05 9731.62132996069
6.1e-05 9750.38692242273
6.2e-05 9769.04159415455
6.3e-05 9787.58643972214
6.4e-05 9806.02235925872
6.5e-05 9824.35020443172
6.6e-05 9842.5707968253
6.7e-05 9860.68492923362
6.8e-05 9878.69336688692
6.9e-05 9896.59684861476
7e-05 9914.39608795011
7.1e-05 9932.0917741781
7.2e-05 9949.68457333272
7.3e-05 9967.17512914462
7.4e-05 9984.56406394288
7.5e-05 10001.8521095561
7.6e-05 10016.1923047402
7.7e-05 10027.886807243
7.8e-05 10039.5027821286
7.9e-05 10051.0408920958
8e-05 10062.5016667301
8.1e-05 10073.8855898458
8.2e-05 10085.1931596169
8.3e-05 10096.4248943608
8.4e-05 10107.58122835
8.5e-05 10118.6625818709
8.6e-05 10129.66939514
8.7e-05 10140.6020992887
8.8e-05 10151.461084825
8.9e-05 10162.2467329955
9e-05 10172.9594161973
9.1e-05 10183.5994982598
9.2e-05 10194.1673763599
9.3e-05 10204.6634708696
9.4e-05 10215.0881238314
9.5e-05 10225.4416661933
9.6e-05 10235.7244216026
9.7e-05 10245.9367066331
9.8e-05 10256.0788310047
9.9e-05 10266.1510977959
0.0001 10276.1538036479
0.000101 10286.0872389631
0.000102 10295.9517017275
0.000103 10305.7475094136
0.000104 10315.4749370778
0.000105 10325.1342514083
0.000106 10334.7257137521
0.000107 10344.2495802769
0.000108 10353.7061021278
0.000109 10363.0955255792
0.00011 10372.4180921818
0.000111 10381.674038905
0.000112 10390.8635982746
0.000113 10399.9869985066
0.000114 10409.0444636366
0.000115 10418.0362136453
0.000116 10426.9624645801
0.000117 10435.8234286732
0.000118 10444.6193144559
0.000119 10453.3503317925
0.00012 10462.0167506503
0.000121 10470.6187925847
0.000122 10479.1566584188
0.000123 10487.630610262
0.000124 10496.040842041
0.000125 10504.3875371697
0.000126 10512.6708761743
0.000127 10520.8910367812
0.000128 10529.0481940018
0.000129 10537.1425202156
0.00013 10545.17418525
0.000131 10553.1433564589
0.000132 10561.0501987984
0.000133 10568.8948749004
0.000134 10576.677545145
0.000135 10584.3983677295
0.000136 10592.0574987369
0.000137 10599.655092202
0.000138 10607.191300175
0.000139 10614.666272785
0.00014 10622.0801582999
0.000141 10629.4331031865
0.000142 10636.7252521677
0.000143 10643.9567482787
0.000144 10651.1277329217
0.000145 10658.238362627
0.000146 10665.2887948712
0.000147 10672.2791663157
0.000148 10679.2096118872
0.000149 10686.0802651353
0.00015 10692.8912582789
0.000151 10699.6427222516
0.000152 10706.3347877152
0.000153 10712.96760533
0.000154 10719.5413155667
0.000155 10726.0560441686
0.000156 10732.5119158077
0.000157 10738.9090541239
0.000158 10745.2475817634
0.000159 10751.5276204151
0.00016 10757.7492908479
0.000161 10763.9127129449
0.000162 10770.018005739
0.000163 10776.0652874455
0.000164 10782.0546754959
0.000165 10787.9862865691
0.000166 10793.8602366233
0.000167 10799.676640926
0.000168 10805.4356140843
0.000169 10811.1372700736
0.00017 10816.781722266
0.000171 10822.3690834584
0.000172 10827.8994658989
0.000173 10833.3729813138
0.000174 10838.789740933
0.000175 10844.1498555156
0.000176 10849.453435374
0.000177 10854.7005903983
0.000178 10859.8914300797
0.000179 10865.0260635331
0.00018 10870.10459952
0.000181 10875.12714647
0.000182 10880.0938125023
0.000183 10885.0047054464
0.000184 10889.8599328628
0.000185 10894.6596020627
0.000186 10899.4038201274
0.000187 10904.0926939276
0.000188 10908.7263316122
0.000189 10913.3048547065
0.00019 10917.8283712081
0.000191 10922.2969961108
0.000192 10926.7108436584
0.000193 10931.0700192606
0.000194 10935.3746282412
0.000195 10939.6247758536
0.000196 10943.8205672964
0.000197 10947.9621077285
0.000198 10952.0495022839
0.000199 10956.0828560857
0.0002 10960.0622742608
0.000201 10963.9878619531
0.000202 10967.8597243373
0.000203 10971.6779666318
0.000204 10975.4426941119
0.000205 10979.1540121218
0.000206 10982.8120260873
0.000207 10986.4168415279
0.000208 10989.9685640679
0.000209 10993.4672994486
0.00021 10996.9131535388
0.000211 11000.306243337
0.000212 11003.6471228067
0.000213 11006.9362770532
0.000214 11010.1738257919
0.000215 11013.3598700165
0.000216 11016.494521744
0.000217 11019.5778733329
0.000218 11022.6100231824
0.000219 11025.5910655339
0.00022 11028.5210907762
0.000221 11031.4001911722
0.000222 11034.228476141
0.000223 11037.0060412587
0.000224 11039.7329777762
0.000225 11042.4093773118
0.000226 11045.0353318582
0.000227 11047.6109337896
0.000228 11050.1362785078
0.000229 11052.6114667991
0.00023 11055.0365925363
0.000231 11057.4117494272
0.000232 11059.7370315929
0.000233 11062.0125335736
0.000234 11064.2383503347
0.000235 11066.4145772715
0.000236 11068.5413102153
0.000237 11070.6186454378
0.000238 11072.6466796567
0.000239 11074.6255100401
0.00024 11076.5552342119
0.000241 11078.4359562209
0.000242 11080.267784385
0.000243 11082.0508176995
0.000244 11083.7851551778
0.000245 11085.4708963132
0.000246 11087.1081410833
0.000247 11088.696989954
0.000248 11090.237543883
0.000249 11091.7299043237
0.00025 11093.1741732287
0.000251 11094.5704530536
0.000252 11095.91884676
0.000253 11097.2194578191
0.000254 11098.4723902148
0.000255 11099.6777484468
0.000256 11100.8356375336
0.000257 11101.9461630156
0.000258 11103.0094309578
0.000259 11104.0255479524
0.00026 11104.9946211218
0.000261 11105.9167581209
0.000262 11106.7920671399
0.000263 11107.6206569063
0.000264 11108.4026366875
0.000265 11109.138116293
0.000266 11109.8272060766
0.000267 11110.4700169385
0.000268 11111.0666603272
0.000269 11111.6172482415
0.00027 11112.1218932324
0.000271 11112.5807084051
0.000272 11112.9938074203
0.000273 11113.3613044963
0.000274 11113.68331441
0.000275 11113.9599524993
0.000276 11114.1913354534
0.000277 11114.3775810114
0.000278 11114.5188058185
0.000279 11114.6151269993
0.00028 11114.6666622482
0.000281 11114.6735298302
0.000282 11114.6358485823
0.000283 11114.5537379144
0.000284 11114.4273178101
0.000285 11114.2567088281
0.000286 11114.0420321023
0.000287 11113.7834103799
0.000288 11113.4809677549
0.000289 11113.1348275898
0.00029 11112.7451137706
0.000291 11112.3119507601
0.000292 11111.835463598
0.000293 11111.3157779021
0.000294 11110.7530198679
0.000295 11110.1473162696
0.000296 11109.4987944604
0.000297 11108.8075823723
0.000298 11108.0738085172
0.000299 11107.2976019861
0.0003 11106.47909245
0.000301 11105.6184101599
0.000302 11104.7156859466
0.000303 11103.7710512209
0.000304 11102.7846379737
0.000305 11101.7565787757
0.000306 11100.6870067775
0.000307 11099.5760557093
0.000308 11098.4238598811
0.000309 11097.2305541819
0.00031 11095.9962740798
0.000311 11094.7211556219
0.000312 11093.4053354335
0.000313 11092.0489507181
0.000314 11090.6521392568
0.000315 11089.215039408
0.000316 11087.7377901068
0.000317 11086.2205308648
0.000318 11084.6634017689
0.000319 11083.0665434814
0.00032 11081.4300972392
0.000321 11079.7542048529
0.000322 11078.0390087065
0.000323 11076.2846526091
0.000324 11074.4912913018
0.000325 11072.6590726318
0.000326 11070.7881415738
0.000327 11068.8786436701
0.000328 11066.9307250294
0.000329 11064.9445323266
0.00033 11062.9202128009
0.000331 11060.8579142556
0.000332 11058.7577850571
0.000333 11056.6199741334
0.000334 11054.4446309734
0.000335 11052.2319056258
0.000336 11049.981948698
0.000337 11047.6949118674
0.000338 11045.3709539355
0.000339 11043.0102297445
0.00034 11040.6128922642
0.000341 11038.1790950163
0.000342 11035.7089920733
0.000343 11033.2027380571
0.000344 11030.6604881374
0.000345 11028.0824067367
0.000346 11025.4686632859
0.000347 11022.8194153302
0.000348 11020.1348204798
0.000349 11017.4150383072
0.00035 11014.66023688
0.000351 11011.8705764227
0.000352 11009.0462276754
0.000353 11006.1873646854
0.000354 11003.2941593365
0.000355 11000.3668004762
0.000356 10997.4057735232
0.000357 10994.4116555228
0.000358 10991.3846301635
0.000359 10988.324868311
0.00036 10985.2325412815
0.000361 10982.1078208396
0.000362 10978.9508791961
0.000363 10975.7618890063
0.000364 10972.5410233673
0.000365 10969.2884558164
0.000366 10966.0043603289
0.000367 10962.6889113154
0.000368 10959.3422836203
0.000369 10955.9646525191
0.00037 10952.5561937161
0.000371 10949.1170833424
0.000372 10945.6474979536
0.000373 10942.147614527
0.000374 10938.6176104596
0.000375 10935.0576635659
0.000376 10931.4679520747
0.000377 10927.8486546277
0.000378 10924.1999502761
0.000379 10920.5220184785
0.00038 10916.8150397205
0.000381 10913.0792041521
0.000382 10909.314696779
0.000383 10905.5216994034
0.000384 10901.7004038332
0.000385 10897.8509963344
0.000386 10893.9736592881
0.000387 10890.0685754592
0.000388 10886.1359279935
0.000389 10882.1759004152
0.00039 10878.1886766242
0.000391 10874.1744408927
0.000392 10870.1333778629
0.000393 10866.065672544
0.000394 10861.9715103092
0.000395 10857.8510768926
0.000396 10853.7045583867
0.000397 10849.5321412389
0.000398 10845.3340122491
0.000399 10841.1103585659
0.0004 10836.8613676843
0.000401 10832.587227442
0.000402 10828.2881260168
0.000403 10823.9642519229
0.000404 10819.6157940084
0.000405 10815.2429414515
0.000406 10810.8458837576
0.000407 10806.424810756
0.000408 10801.9799125966
0.000409 10797.5113797466
0.00041 10793.0194029872
0.000411 10788.5041734102
0.000412 10783.9658824146
0.000413 10779.4047217034
0.000414 10774.8208832799
0.000415 10770.2145594443
0.000416 10765.5859427903
0.000417 10760.9352262015
0.000418 10756.2626028479
0.000419 10751.5682661823
0.00042 10746.8524099368
0.000421 10742.1152281188
0.000422 10737.3569150078
0.000423 10732.5776651515
0.000424 10727.7776733622
0.000425 10722.9571347127
0.000426 10718.1162445329
0.000427 10713.2551984059
0.000428 10708.3741921641
0.000429 10703.4734235493
0.00043 10698.5531090752
0.000431 10693.6134505508
0.000432 10688.654644797
0.000433 10683.6768888621
0.000434 10678.680380018
0.000435 10673.6653157558
0.000436 10668.6318937817
0.000437 10663.5803120132
0.000438 10658.5107685748
0.000439 10653.4234617935
0.00044 10648.3185950132
0.000441 10643.1963866854
0.000442 10638.0570373862
0.000443 10632.9007462205
0.000444 10627.7277124724
0.000445 10622.5381356006
0.000446 10617.3322152338
0.000447 10612.1101511668
0.000448 10606.8721433553
0.000449 10601.618391912
0.00045 10596.3490971016
0.000451 10591.0644593362
0.000452 10585.764679171
0.000453 10580.4499572992
0.000454 10575.1204945475
0.000455 10569.7764918714
0.000456 10564.4181503503
0.000457 10559.0456711826
0.000458 10553.6592556811
0.000459 10548.2591052677
0.00046 10542.8454214688
0.000461 10537.4184059103
0.000462 10531.9782603124
0.000463 10526.5251864846
0.000464 10521.0593863207
0.000465 10515.5810617934
0.000466 10510.0904149497
0.000467 10504.5876479049
0.000468 10499.0729628379
0.000469 10493.546561986
0.00047 10488.0086476389
0.000471 10482.4594221339
0.000472 10476.8990878504
0.000473 10471.3278588214
0.000474 10465.7459791271
0.000475 10460.1536544882
0.000476 10454.5510873013
0.000477 10448.9384833489
0.000478 10443.3160901844
0.000479 10437.6841249153
0.00048 10432.0427898195
0.000481 10426.3922871551
0.000482 10420.7328191541
0.000483 10415.0645880166
0.000484 10409.3877959047
0.000485 10403.7026449367
0.000486 10398.0093371808
0.000487 10392.3080746489
0.000488 10386.5990592906
0.000489 10380.8824929869
0.00049 10375.158577544
0.000491 10369.4275146867
0.000492 10363.6895060522
0.000493 10357.9447531839
0.000494 10352.1934575245
0.000495 10346.4358204098
0.000496 10340.6720430619
0.000497 10334.9023265829
0.000498 10329.126871948
0.000499 10323.3458799986
0.0005 10317.559551436
0.000501 10311.7680868143
0.000502 10305.9716865334
0.000503 10300.1705508326
0.000504 10294.3648797831
0.000505 10288.5548738131
0.000506 10282.740751182
0.000507 10276.9227250969
0.000508 10271.1009947667
0.000509 10265.2757591937
0.00051 10259.447217166
0.000511 10253.6155672501
0.000512 10247.7810077838
0.000513 10241.9437368687
0.000514 10236.1039523626
0.000515 10230.2618518718
0.000516 10224.4176327438
0.000517 10218.5714920597
0.000518 10212.7236266262
0.000519 10206.8742329679
0.00052 10201.0235073198
0.000521 10195.171645619
0.000522 10189.3188434972
0.000523 10183.4653000819
0.000524 10177.6112583943
0.000525 10171.7569266643
0.000526 10165.9024988997
0.000527 10160.0481687555
0.000528 10154.1941295255
0.000529 10148.3405741343
0.00053 10142.4876951287
0.000531 10136.6356846696
0.000532 10130.7847345232
0.000533 10124.9350360528
0.000534 10119.0867802103
0.000535 10113.2401723957
0.000536 10107.3954199232
0.000537 10101.5527125467
0.000538 10095.7122392974
0.000539 10089.8741887479
0.00054 10084.0387490035
0.000541 10078.2061076934
0.000542 10072.3764622226
0.000543 10066.5500484629
0.000544 10060.7270565161
0.000545 10054.9076716117
0.000546 10049.0920784509
0.000547 10043.2804729522
0.000548 10037.4730575435
0.000549 10031.6700158756
0.00055 10025.8715769521
0.000551 10020.0779964159
0.000552 10014.2894690691
0.000553 10007.818280526
0.000554 9999.3728126531
0.000555 9990.92619359381
0.000556 9982.48035453993
0.000557 9974.03558772283
0.000558 9965.59218530459
0.000559 9957.15043936807
0.00056 9948.71064190698
0.000561 9940.27308481569
0.000562 9931.83805987897
0.000563 9923.40585876152
0.000564 9914.9767729974
0.000565 9906.55109397927
0.000566 9898.12911294747
0.000567 9889.71112097895
0.000568 9881.29740897604
0.000569 9872.88826765504
0.00057 9864.48398753463
0.000571 9856.08485892415
0.000572 9847.69117191169
0.000573 9839.30321635192
0.000574 9830.92128185392
0.000575 9822.54565776863
0.000576 9814.17663317623
0.000577 9805.81449687335
0.000578 9797.45953735999
0.000579 9789.11204282632
0.00058 9780.77230113929
0.000581 9772.44059982899
0.000582 9764.11722607485
0.000583 9755.80247313669
0.000584 9747.49670169818
0.000585 9739.20021386354
0.000586 9730.91329490058
0.000587 9722.63622964067
0.000588 9714.36930246353
0.000589 9706.11279728186
0.00059 9697.86700364881
0.000591 9689.63228632231
0.000592 9681.40895412789
0.000593 9673.19728839006
0.000594 9664.99756986915
0.000595 9656.81007934174
0.000596 9648.63515532314
0.000597 9640.47312988047
0.000598 9632.32428064133
0.000599 9624.18888456608
};
\end{groupplot}

\end{tikzpicture}

\end{figure}

